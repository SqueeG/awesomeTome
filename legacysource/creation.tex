\section{Generating Your Stats}

The first step of character creation is to generate the six ability scores that your character will have. There are two basic methods for doing this: Rolling and Point Buy. Check with your Game Master to see what method you should use to make your character.

\subsection{Rolling}

Rolling for your ability scores involves rolling a number of d6's and assigning the result to the desired ability.
\listone
	\bolditem{3d6:} Roll and total 3d6 of six times.
	\bolditem{4d6 Drop Lowest:} Roll 4d6 and and drop the lowest dice six times.
	\bolditem{4d6, Reroll Ones, Drop Lowest:} Roll 4d6, reroll any ones, and drop the lowest dice six times.
\end{list}
\vspace*{10pt}
Once you've generated six totals you can assign them to the desired ability score.  Some GM's may use different numbers of dice to in order to generate ability scores.

\subsection{Point Buy}

\setlength{\intextsep}{-5pt}
\begin{wraptable}{O}{2in}
\caption{Point Buy Chart}
\begin{tabular}{c|c c c|c}
Score & Cost & & Score & Cost \\
9  & 1       & & 14 & 6       \\
10 & 2       & & 15 & 8       \\
11 & 3       & & 16 & 10      \\
12 & 4       & & 17 & 13      \\
13 & 5       & & 18 & 16      \\
\multicolumn{4}{c}{} \\
\multicolumn{4}{c}{} \\
\end{tabular}
\end{wraptable}
\setlength{\intextsep}{12pt}

The other method for generating ability scores is to `buy' a certain score for each of your ability scores by `spending' a pool of points in order to determine your starting ability scores.  All racial modifiers are applied after you have bought scores for every attribute.  To use this method each of your scores starts out at 8, and you can buy a score with your points according to the the table. A `low\textendash power' campaign may use as few as 15 points, a `challenging' campaign 22 points, a `tough' campaign 28 points, and a `high\textendash powered' campaign may use 32 or more.  If the Game Master decides to use point buy, he ultimately decides how many points characters receive to buy their ability scores.

\section{Choose a Race}

The second step after generating and assigning your ability scores is to choose what race you want your character to be.  Your race determines your size, speed, type, and forms of vision.  Your race may also give you bonuses to certain ability scores, skills, grant you bonus feats, or give you many other sorts of bonuses and abilities.

\textbf{Languages: }Every Race has languages indicated as `Automatic' and `Bonus.'  Every character of a given race can speak and write in the listed Automatic Languages, for every point of Intelligence bonus that character posesses above +0 you may select one additional language from the list of Bonus Languages.

Chapter three contains many races to choose from.  Your GM may allow races from other sources.

\section{Choose an Alignment}

Alignment is a broad categorization that encompases your characters outlook on morality.  There are nine alignments derived from the axes of ``Law and Chaos" and of ``Good and Evil". Listed here are the default interpretations of alignment. Your GM or gaming group may have selected to treat ``Good and Evil" or ``Law and Chaos" in a way other than the defaults presented here, however, and the following may not apply (see the chapter on Alignment options for more information).
\vspace*{10pt}
\listone
	\bolditem{Lawful Good:} A lawful good character acts as a good person is expected or required to act. She combines a commitment to oppose evil with the discipline to fight relentlessly. She tells the truth, keeps her word, helps those in need, and speaks out against injustice. A lawful good character hates to see the guilty go unpunished. 
	\bolditem{Neutral Good:} A neutral good character does the best that a good person can do. He is devoted to helping others. He works with kings and magistrates but does not feel beholden to them.
	\bolditem{Chaotic Good:} A chaotic good character acts as his conscience directs him with little regard for what others expect of him. He makes his own way, but he�s kind and benevolent. He believes in goodness and right but has little use for laws and regulations. He hates it when people try to intimidate others and tell them what to do. He follows his own moral compass, which, although good, may not agree with that of society. 
	\bolditem{Lawful Neutral:} A lawful neutral character acts as law, tradition, or a personal code directs her. Order and organization are paramount to her. She may believe in personal order and live by a code or standard, or she may believe in order for all and favor a strong, organized government. 
	\bolditem{Neutral:} A neutral character does what seems to be a good idea. She doesn�t feel strongly one way or the other when it comes to good vs. evil or law vs. chaos. Most neutral characters exhibit a lack of conviction or bias rather than a commitment to neutrality. Such a character thinks of good as better than evil�after all, she would rather have good neighbors and rulers than evil ones. Still, she�s not personally committed to upholding good in any abstract or universal way. Some neutral characters, on the other hand, commit themselves philosophically to neutrality. They see good, evil, law, and chaos as prejudices and dangerous extremes. They advocate the middle way of neutrality as the best, most balanced road in the long run. 
	\bolditem{Chaotic Neutral:} A chaotic neutral character follows his whims. He is an individualist first and last. He values his own liberty but doesn�t strive to protect others� freedom. He avoids authority, resents restrictions, and challenges traditions. A chaotic neutral character does not intentionally disrupt organizations as part of a campaign of anarchy. To do so, he would have to be motivated either by good (and a desire to liberate others) or evil (and a desire to make those different from himself suffer). A chaotic neutral character may be unpredictable, but his behavior is not totally random. He is not as likely to jump off a bridge as to cross it. 
	\bolditem{Lawful Evil:}  A lawful evil villain methodically takes what he wants within the limits of his code of conduct without regard for whom it hurts. He cares about tradition, loyalty, and order but not about freedom, dignity, or life. He plays by the rules but without mercy or compassion. He is comfortable in a hierarchy and would like to rule, but is willing to serve. He condemns others not according to their actions but according to race, religion, homeland, or social rank. He is loath to break laws or promises. This reluctance comes partly from his nature and partly because he depends on order to protect himself from those who oppose him on moral grounds.
	\bolditem{Neutral Evil:} A neutral evil villain does whatever she can get away with. She is out for herself, pure and simple. She sheds no tears for those she kills, whether for profit, sport, or convenience. She has no love of order and holds no illusion that following laws, traditions, or codes would make her any better or more noble. On the other hand, she doesn�t have the restless nature or love of conflict that a chaotic evil villain has. 
	\bolditem{Chaotic Evil:} A chaotic evil character does whatever his greed, hatred, and lust for destruction drive him to do. He is hot-tempered, vicious, arbitrarily violent, and unpredictable. If he is simply out for whatever he can get, he is ruthless and brutal. If he is committed to the spread of evil and chaos, he is even worse. Thankfully, his plans are haphazard, and any groups he joins or forms are poorly organized. Typically, chaotic evil people can be made to work together only by force, and their leader lasts only as long as he can thwart attempts to topple or assassinate him. 
\end{list}
\vspace*{10pt}
Your character begins the game as the alignment of your choice, though the GM may decide that it changes through the course of play because of your actions.
 
\section{Choose a Class}

Your class is the primary method by which your character advances.  You start off with one level in a class of your choice.  Every time you gain a level in a class you gain all the features indicated in that classes table for that level.  This includes Base Attack Bonus, bonuses to your Saves, Hit Dice, Skill Points, and other class features.  Chapter 4 includes all the base classes you can take.

When you gain a new level you can take the next level of a class you already have levels in, or you can get the features from the first level of a new class.  The total of all the levels you have taken is your character level.

At first level you gain quadruple normal skill points for the class and get the maximum hit points possible for you classes hit die.

At higher levels you may begin qualifying for prestige classes.  You gain levels in prestige classes the same way you do base classes with the exception that your character must meet the listed prerequisites for the class in order to begin taking levels.

\section{Skills}

Every class gives you a certain number of skill points that you can assign to skills.  Each point buys you one rank of a skill and every rank adds a +1 bonus to rolls involving the skill.  The maximum number points that you can invest in a class skill is your character level + 3, and the maximum ranks you can invest in a cross-class skill is half of your class skill maximum.

\section{Feats}

At first level, third level, and every third level thereafter your character gains a feat.  Feats represent abilities your character has but are not directly related to your class. Many, but not all, feats scale to one of your characters features. [Combat] feats scale to your base attack bonus, [Skill] feats scale to ranks in a specific skill, and [Magic] feats scale to the higest level spell you can cast, other types of feats may scale to other attributes.  [General] and [Metamagic] feats do not scale to any attribute.  Some feats have requirements that must be met in order to take them.

\section{Equipment}

Your character starts with a certain amount of gold that can be used to buy equipment that it will start with.  Later in the game you will gain access to powerful magic items and whatnot.
