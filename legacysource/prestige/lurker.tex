\classname{Lurker in the Swarm}
\vspace*{-8pt}
\quot{``You cannot know what we know. Your terror smells so\ldots\ delicious\ldots\ to\ldots\ bees."}

Through the ages, terrible necromantic practices have flourished in the dark and forgotten places of the world, but none has been so insidious as the practice of producing Blood Honey. Composed of the dying essences of sentients and produced by undead bees in necromantically-charged obsidian hives, Blood Honey has the remarkable qualities restoring vigor and youth to any that consume it. Kingdoms have fallen and lives have been lost due to the addiction for this substance as the Lurkers in the Swarm have peddled it among the powerful and elite in exchange for dark favors.

The Lurkers in the Swarm are masters of the art of summoning and controlling undead swarms of bees, adding their minions' powers to their own. Each maintains a hidden lair for their obsidian hives, commanding their undead bees to produce blood honey. Often, they conduct raids on the lairs of other Lurkers, each attempting to corner the market on Blood Honey and loot the stores of others. When cabals of Lurkers form, terrible plans are hatched and atrocities are committed as these necromancers have the patience of an immortal and the alien insights of an undead insect.

\ability{Prerequisites:}{}
\listprereq
\itemability{Skills:}{Knowledge (Nature) 9 ranks. Handle Animal 4 ranks}
\itemability{Spellcasting:}{Must be able to cast \spell{summon swarm} and \spell{vampiric touch} as arcane spells.}
\end{list}\vspace{8pt}

\ability{Hit Die:}{d4}

\ability{Class Skills:}{The Lurker in the Hive's class skills (and the key ability for each skill) are Concentration (Con), Craft (Int), Diplomacy (Cha), Handle Animal (Cha), Knowledge (all skills, taken individually) (Int), Listen (Wis), Profession (Wis), Spot(Wis), Search (Int), and Spellcraft (Int).}

\ability{Skills/Level:}{2 + Intelligence Bonus}

\begin{table}[tbh]
\begin{small}
\begin{tabular}{lp{1.9cm}p{0.7cm}p{0.7cm}p{0.7cm}p{6cm}l}
Level  &Base Attack Bonus &Fort Save &Ref Save &Will Save &Special &Spellcasting\\
1st &+0 &+0 &+0 &+2 &Speech of the Queen, Swarms of Bees &+1 spellcaster level\\
2nd &+1 &+0 &+0 &+3 &Queen's Right &+1 spellcaster level\\
3rd &+1 &+1 &+1 &+3 &Gather the Blood Honey &+1 spellcaster level\\
4th &+2 &+1 &+1 &+4 &Husks of the Swarm &+1 spellcaster level\\
5th &+2 &+1 &+1 &+4 &Harvest of the Bitter Nectar &+1 spellcaster level\\
6th &+3 &+2 &+2 &+5 &The Plague that Swallows the Sun &+1 spellcaster level\\
7th &+3 &+2 &+2 &+5 &The Sight that is Shared &+1 spellcaster level\\
8th &+4 &+2 &+2 &+6 &Greater Plagues &+1 spellcaster level\\
9th &+4 &+3 &+3 &+6 &Spirits of the Hive &+1 spellcaster level\\
10th &+5 &+3 &+3 &+7 &Hive Storm &+1 spellcaster level\\
\end{tabular}
\end{small}
\end{table}


\ability{Weapon and Armor Proficiency:}{The Lurker in the Swarm gains no proficiency with armor or weapons.}

\ability{Spellcasting:}{Every level, the Lurker in the Swarm casts spells (including gaining any new spell slots and spell knowledge) as if she had also gained a level in a spellcasting class she had previous to gaining that level.}

\ability{Speech of the Queen (Ex):}{The Lurker in the Swarm can communicate with vermin as if they were capable of communicating in a language she understands. She can likewise communicate with any creature that probably should have been classified as a vermin, subject to the DM's whim. She is fully able to speak to and understand a Hellwasp Swarm, a Phase Spider, or a Gelugon, for example. With respect to such creatures, the Lurker in the Swarm can use her ranks in Handle Animal or Diplomacy to influence their opinion, as well as using her Intelligence bonus instead of her Charisma bonus.}

\ability{Swarms of Bees:}{When the Lurker in the Swarm casts \spell{summon swarm}, she instead gets a swarm of bees rather than a swarm of bats, rats, or spiders. This is considered a Wasp Swarm (FF, p. 172).}

\ability{Queen's Right (Ex):}{At 2nd level, the Lurker in the Swarm can control the movements of the Swarms she summons. Any swarms resulting from a casting of summon swarm, or other conjurations on her part will be able to move normally on her turn, and do so under the direction of the Lurker. In addition, the Lurker in the Swarm no longer takes damage while in the space occupied by a swarm.}

\ability{Gather the Blood Honey:}{At 3rd level, the Lurker in the Swarm gains the ability to produce the notorious Blood Honey. Composed of the necromantically extracted essence of dying sentients and produced by undead bees, this blood red honey has the ability to heal wounds and restore vigor. When consumed, each dose of Blood Honey has the effects of a restoration and a regenerate spell.

Blood Honey is created when the Lurker's swarms kill a creature with at least 5 HD and an Int of 5 or better and then spends one hour in an obsidian hive. Only one dose can be produced each day by any one swarm.}

\ability{Husks of the Swarm:}{The Lurker in the Swarm gains such power over death and bees that at 4th level she can summon a swarm of Animate Bee Corpses: treat as a Bloodfiend Locust Swarm (FF, p. 170) with the Corpse Template (BoVD, p. 185).}

\ability{Harvest of the Bitter Nectar (Su):}{At 5th level, the Lurker in the Swarm gains the ability to hold onto her youth indefinitely by producing the Royal Jelly form of Blood Honey out of the blood and energy she harvests from living intelligent creatures caught in by her Animate Bees. After her swarms kill a creature with at least 10 HD using negative levels she can make a dose of magical honey sufficient to restore a single character to the Young Adult Age Category. Age penalties are removed, but mental attribute bonuses for her true age do not change. This effect lasts for one month, and then the Lurker must consume another dose of Royal Jelly or else she returns to the true age (and dies if she has surpassed her maximum age). It gradually becomes more difficult to benefit from Royal Jelly. For every year a Lurker maintains her youth in this fashion, it requires an extra dose each month to retain the benefit (but never exceeds 30 doses for a month).}

\ability{The Plague that Swallows the Sun:}{A lurker in the Swarm learns the spell insect plague at 6th level even though it is most likely not on her list. The spell is still a 5th level spell, though the Lurker's Swarms of Bees, Queen's Right, and Husks of the Swarm abilities all apply (allowing the Lurker to produce swarms of bees or animated bee corpses which move at her direction). At 8th level, the lurker learns the creeping doom spell, which again can take the form of various bees and move at her command without needing to expend actions.}

\ability{The Sight that is Shared (Su):}{At 7th level, the Lurker in the Swarm is capable of sharing her senses with all bees within long range (400' + 40' per caster level) of herself, whether or not line of effect exists between the Lurker and the bees. The Lurker can make a Search, Spot, or listen check as if she shared the location of any bees in that area, and anything noticed by the Lurker (or any of the bees) is automatically noticed by the Lurker and all of the bees. The Lurker in the Swarm can see through swarms of bees as if they did not block vision at all. The Lurker can also replicate animal messenger at will, though the messenger can and must be a bee.}

\ability{Spirits of the Hive:}{At 9th level, any time the Lurker in the Swarm summons bees, she may opt to replace summoned bees with Ephemeral Swarms (of bees). The constituents of the swarms are fine creatures, but otherwise conform to the listing of the Ephemeral Swarm (MM3, p. 50).}

\ability{Hive Storm(Su):}{At 10th level, the Lurker in the Swarm contains an improbable number of bees within her own body, and can surround herself with a swarm of them at any time as a move action (these bees are considered a Hellwasp swarm (MM, p. 238) while they are surrounding her). The swarm moves with the Lurker, and while they may be dispersed as normal, the Lurker can replace the swarm at any time with a move action.

In addition, the Lurker in the Swarm is immune to the Distraction ability of swarms.}
