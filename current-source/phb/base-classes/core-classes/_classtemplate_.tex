\classentry{ <-class name-> }
<- choose one of the following to indicate the BAB progression of the class: \goodbab \modebab \poorbab ->
<- choose one of the following to indicate the Fort save progression of the class: \goodfor \poorfor ->
<- choose one of the following to indicate the Ref save progression of the class: \goodref \poorref ->
<- choose one of the following to indicate the Will save progression of the class: \goodwil \poorwil ->
\quot{`` <-class quote-> ''}

\desc{ <-class description. you should write one. two, maybe three paragraphs even. When you want to start a new paragraph, type \newline before it starts.-> }

\playingaclass{ <-breif advice on how to play the class should go here.-> }

\alignment{ <-allowed class alignments. probably "Any" because alignments are dumb.-> }

\races{ <-allowed races. again, probably "Any" because racial restrictions on classes are dumb-> }

\startinggold{ <-starting moneys, like "gold6d4x10 gp (150 gold)"-> }

\startingage{ <-starting age, often written as a class reference like "As Rogue."-> }

\hitdie{ <-this should be really easy, "d#"-> }

\classskills{ <-a big pile of class skils, normally in the form of "Balance (Dex)" separated by commas. it ends with ".", because why not?-> }

\skillpoints{ <-do I really need to tell you what to put here? because I can. it's a number}

\begin{classtable}{ <-this part is kind of complicated. if you want any columns after Special, include them here preceded by an &, like this "& Death Attack & Whatever". and if you don't want any of those, level this blank, like "{}"-> }
\levelone{ <-class features for this level. if you want anything in a custom column defined above, put it after an "&", like "Stuff & Other thing"-> }
\leveltwo{ <-class features for this level. if you want anything in a custom column defined above, put it after an "&", like "Stuff & Other thing"-> }
\levelthree{ <-class features for this level. if you want anything in a custom column defined above, put it after an "&", like "Stuff & Other thing"-> }
\levelfour{ <-class features for this level. if you want anything in a custom column defined above, put it after an "&", like "Stuff & Other thing"-> }
\levelfive{ <-class features for this level. if you want anything in a custom column defined above, put it after an "&", like "Stuff & Other thing"-> }
\levelsix{ <-class features for this level. if you want anything in a custom column defined above, put it after an "&", like "Stuff & Other thing"-> }
\levelseven{ <-class features for this level. if you want anything in a custom column defined above, put it after an "&", like "Stuff & Other thing"-> }
\leveleight{ <-class features for this level. if you want anything in a custom column defined above, put it after an "&", like "Stuff & Other thing"-> }
\levelnine{ <-class features for this level. if you want anything in a custom column defined above, put it after an "&", like "Stuff & Other thing"-> }
\levelten{ <-class features for this level. if you want anything in a custom column defined above, put it after an "&", like "Stuff & Other thing"-> }
\leveleleven{ <-class features for this level. if you want anything in a custom column defined above, put it after an "&", like "Stuff & Other thing"-> }
\leveltwelve{ <-class features for this level. if you want anything in a custom column defined above, put it after an "&", like "Stuff & Other thing"-> }
\levelthirteen{ <-class features for this level. if you want anything in a custom column defined above, put it after an "&", like "Stuff & Other thing"-> }
\levelfourteen{ <-class features for this level. if you want anything in a custom column defined above, put it after an "&", like "Stuff & Other thing"-> }
\levelfifteen{ <-class features for this level. if you want anything in a custom column defined above, put it after an "&", like "Stuff & Other thing"-> }
\levelsixteen{ <-class features for this level. if you want anything in a custom column defined above, put it after an "&", like "Stuff & Other thing"-> }
\levelseventeen{ <-class features for this level. if you want anything in a custom column defined above, put it after an "&", like "Stuff & Other thing"-> }
\leveleighteen{ <-class features for this level. if you want anything in a custom column defined above, put it after an "&", like "Stuff & Other thing"-> }
\levelnineteen{ <-class features for this level. if you want anything in a custom column defined above, put it after an "&", like "Stuff & Other thing"-> }
\leveltwenty{ <-class features for this level. if you want anything in a custom column defined above, put it after an "&", like "Stuff & Other thing"-> }
\end{classtable}

<- if you need a custom extra table, you can drop it here by updating the following format. or you can skip it and ask for help, because this shit can be complicated....

\begin{floatingfigure}[r]{3.9in}
\rowcolors{1}{colorone}{colortwo}
\noindent\begin{tabular}{lllllllllllllllll}
\header & \multicolumn{7}{c}{Spells Per Day} & & &\multicolumn{7}{c}{Spells Known}\\
   &0 &1 &2 &3 &4 &5 &6 &  &   &0 &1 &2 &3 &4 &5 &6\\
1  &2 &- &- &- &- &- &- &  &1  &4 &- &- &- &- &- &-\\
2  &3 &0 &- &- &- &- &- &  &2  &5 &2 &- &- &- &- &-\\
3  &3 &1 &- &- &- &- &- &  &3  &6 &3 &- &- &- &- &-\\
4  &3 &2 &0 &- &- &- &- &  &4  &6 &3 &2 &- &- &- &-\\
5  &3 &3 &1 &- &- &- &- &  &5  &6 &4 &3 &- &- &- &-\\
6  &3 &3 &2 &- &- &- &- &  &6  &6 &4 &3 &- &- &- &-\\
7  &3 &3 &2 &0 &- &- &- &  &7  &6 &4 &4 &2 &- &- &-\\
8  &3 &3 &3 &1 &- &- &- &  &8  &6 &4 &4 &3 &- &- &-\\
9  &3 &3 &3 &2 &- &- &- &  &9  &6 &4 &4 &3 &- &- &-\\
10 &3 &3 &3 &2 &0 &- &- &  &10 &6 &4 &4 &4 &2 &- &-\\
11 &3 &3 &3 &3 &1 &- &- &  &11 &6 &4 &4 &4 &3 &- &-\\
12 &3 &3 &3 &3 &2 &- &- &  &12 &6 &4 &4 &4 &3 &- &-\\
13 &3 &3 &3 &3 &2 &0 &- &  &13 &6 &4 &4 &4 &4 &2 &-\\
14 &3 &3 &3 &3 &3 &1 &- &  &14 &6 &4 &4 &4 &4 &3 &-\\
15 &3 &3 &3 &3 &3 &2 &- &  &15 &6 &4 &4 &4 &4 &3 &-\\
16 &3 &3 &3 &3 &3 &2 &0 &  &16 &6 &5 &4 &4 &4 &4 &2\\
17 &3 &3 &3 &3 &3 &3 &1 &  &17 &6 &5 &5 &4 &4 &4 &3\\
18 &3 &3 &3 &3 &3 &3 &2 &  &18 &6 &5 &5 &5 &4 &4 &3\\
19 &3 &3 &3 &3 &3 &3 &3 &  &19 &6 &5 &5 &5 &5 &4 &4\\
20 &3 &3 &3 &3 &3 &3 &3 &  &20 &6 &5 &5 &5 &5 &5 &4\\
\end{tabular}
\end{floatingfigure}

and if you don't need it, delete this shit. please. -> 

\startclassfeatures

\proficiencies{ <-the proficiencies that the class is granted. refer to altered equipment if necessary, or leave blank. or make up something funny.-> }

\classfeature{ <-class feature name. with stuff like "(Ex)" if appropriate.-> }{ <-class feature description. should be complete rules text, because that's what goes here.-> }

<- repeat the above as necessary.-> 

<-if you need a bulletted list in one of them, you can use the following:

\listone
    \item \ability{ <-ability name-> }{ <-ability description-> }
    <-repeat the above as required, one per bullet point->
\end{list}
\vspace{8pt}

this is probably contained in the ability description part of a class feature.->