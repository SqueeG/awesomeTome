\afterpage{\newcommand{\armorcat}[1]{\cellcolor{headercolor}\textbf{#1}& \cellcolor{headercolor}AC& \cellcolor{headercolor}Max  Dex& \cellcolor{headercolor}ACP& \cellcolor{headercolor}Cost & \cellcolor{headercolor}Weight}

\newcommand{\armortablerow}[2]{#1 & #2\\}

\begin{table}
\caption{Armor Types}
{\tabulinesep=1mm
\rowcolors{1}{colorone}{colortwo}
\begin{tabu} to \linewidth {X[3l] c X[c] c l c   |   X[3l] c X[c] c l c}
\armortablerow{\armorcat{Non-\linebreak Armors}}{\armorcat{Medium\linebreak Armor}} \hline
\armortablerow{Camouflaged \linebreak Clothing & 0 & - & 0 & 1gp & 5 lb}{Gith Armor & +5 & +4 & -2 & 900gp & 25 lb}
\armortablerow{Functional \linebreak Clothing & 0 & - & 0 & 1gp & 5 lb}{Animal Spirit Armor & +4 & +3 & -3 & 1,000gp & 30 lb}
\armortablerow{Thieves \linebreak Clothing & 0 & - & 0 & 1gp & 5 lb}{Dragonscale Shirt\textsuperscript{1} & +7 & +5 & -4 & 3,400gp & 30 lb}
\armortablerow{Travelers \linebreak Clothing & 0 & - & 0 & 1gp & 5 lb}{Mithril Suit\textsuperscript{1} & +7 & +5 & -2 & 5,000gp & 25 lb}
\armortablerow{Fancy \linebreak Clothing & 0 & +6 & -1 & 30gp & 5 lb}{Adamantine Breastplate\textsuperscript{1} & +8 & +3 & -6 & 5,000gp & 60 lb}
\armortablerow{\armorcat{Light \linebreak Armor}}{\armorcat{Heavy \linebreak Armor}} \hline
\armortablerow{Cord \linebreak Armor & +2 & - & 0 & 20gp & 10 lb}{Hoplite\linebreak Armor & +6 & +1 & -9 & 250gp & 60 lb}
\armortablerow{Winter Clothes & +2 & +4 & -4 & 30gp & 25 lb}{Half Plate & +7 & +2 & -5 & 800gp & 45 lb}
\armortablerow{Leather\linebreak Armor & +2  & +7  & 0 & 45gp & 10 lb}{Coral Armor & +6 & +2 & -3 & 850gp & 40 lb}
\armortablerow{Still Suit & +2 & +5 & -3 & 50gp & 25 lb}{Great Armor & +7 & +2 & -7 & 1,000gp & 50 lb}
\armortablerow{Wicker \linebreak Armor & +3 & +7  & -1 & 55gp & 10 lb}{Full Plate & +8 & +1 & -6 & 1,100gp & 50 lb}
\armortablerow{Studded Leather & +3 & +6 & -1 & 65gp & 16 lb}{Silk Steel\linebreak Armor\textsuperscript{1} & +7 & +3 & -4 & 3,400gp & 50 lb}
\armortablerow{Chain Shirt & +4 & +5 & -2 & 100gp & 25 lb}{Dragonscale Suit\textsuperscript{1} & +9 & +4 & -5 & 5,500gp & 40 lb}
\armortablerow{Brigadine & +4 & +6 & -2 & 110gp & 25 lb}{Adamantine Carapace\textsuperscript{1} & +11 & +2 & -9 & 9,000gp & 75 lb}
\armortablerow{Gray \linebreak Armor & +4 & - & 0 & 1,000gp & 15 lb}{\armorcat{Shields}}\tabucline{7-12}
\armortablerow{Ironskin Leather\textsuperscript{1}& +5 & +5 & -2 & 3,300 gp & 15 lb}{Buckler & +1 & - & -1 & 15gp & 5 lb}
\armortablerow{Mithril Shirt\textsuperscript{1} & +5 & +6 & -1 & 4,000 gp & 15 lb}{Wooden Shield & +1 & - & 0 & 15gp & 5 lb}
\armortablerow{\armorcat{Medium \linebreak Armor}}{Steel Shield & +2 & - & -1 & 50gp & 10 lb}\tabucline{1-6}
\armortablerow{Hide & +3 & +4 & -4 & 15gp & 25 lb}{Dragonscale Shield\textsuperscript{1} & +3 & - & -1 & 950gp & 15 lb}
\armortablerow{Scale Mail & +4	 & +3 & -5 & 50gp & 35 lb}{Mithril Shield\textsuperscript{1} & +2 & - & 0 & 1,020gp & 5 lb}
\armortablerow{Ringmail & +4 & +6 & -2 & 75gp & 30 lb}{Adamantine & +3 & - & -1 & 2,000gp & 25 lb}
\armortablerow{Chainmail & +5 & +3 & -3 & 150gp & 40 lb}{\armorcat{Great \linebreak Shields}}\tabucline{7-12}
\armortablerow{Lamellar & +5 & +4 & -4 & 190gp & 40 lb}{Tower Shield & +4 & - & -10 & 100gp & 45 lb}
\armortablerow{Elaborate Gown & +1 & +3 & -5 & 300gp & 15 lb}{Kite Shield & +4 & - & -5 & 120gp & 35 lb}
\armortablerow{Lobster Mail & +5 & +2 & -5 & 350gp & 40 lb}{Bone Wall & +3 & - & -10 & 150gp & 30 lb}
\armortablerow{Breastplate & +6 & +2 & -4 & 450gp & 40 lb}{Kappa Shell & +3 & - & -12 & 500gp & 30 lb}
\armortablerow{Chitin\linebreak Carapace & +6 & +4 & -3 & 500gp & 30 lb}{& & & & & }
\hline\multicolumn{12}{l}{\textsuperscript{1} This armor is already masterwork, and cannot be improved further.}\\ \hline
\end{tabu}}
\end{table}}

\subsection{Armor Qualities}
To wear heavier armor effectively, a character can select the Armor Proficiency feats, but most classes are automatically proficient with the armors that work best for them. Characters who are not proficient with an armor or shield incur additional penalties when wearing them. Armor and shields can take damage from some types of attacks. Here is the format for armor entries (given as column headings on the table below).

\subsection{Category}
Armors and shields come in different categories. Nonarmor, light armor, medium armor, and heavy armor for armors, and shields and great shields for shields. Characters who are wearing armor that they do not have the appropriate proficiency with take additional penalties for wearing them (see Armor Class Penalties below). All characters are proficient with nonarmors.

\subsection{Cost}
The cost of the armor for Small or Medium humanoid creatures. See Armor for Unusual Creatures, below, for armor prices for other creatures.

\subsection{Armor Class (AC) Bonus}
Each armor grants an armor bonus to AC, while shields grant a shield bonus to AC. The armor bonus from a suit of armor doesn't stack with other effects or items that grant an armor bonus. Similarly, the shield bonus from a shield doesn't stack with other effects that grant a shield bonus.

\subsection{Maximum Dex Bonus (Max Dex)}
This number is the maximum Dexterity bonus to AC that this type of armor allows. Heavier armors limit mobility, reducing the wearer's ability to dodge blows. This restriction doesn't affect any other Dexterity-related abilities.

Even if a character's Dexterity bonus to AC drops to 0 because of armor, this situation does not count as losing a Dexterity bonus to AC.

Your character's encumbrance (the amount of gear he or she carries) may also restrict the maximum Dexterity bonus that can be applied to his or her Armor Class.

Shields do not affect a character's maximum Dexterity bonus.

\subsection{Armor Check Penalty (ACP)}
Any armor heavier than leather hurts a character's ability to use some skills. An armor check penalty number is the penalty that applies to Balance, Climb, Escape Artist, Hide, Jump, Move Silently, Sleight of Hand, or Tumble checks by a character wearing a certain kind of armor. Double the normal armor check penalty is applied to Swim checks. If a character casts arcane spells in armor which has an Armor Check Penalty, their spells with somatic components have a 5\% chance of failing for every point of armor check penalty. A character's encumbrance (the amount of gear carried, including armor) may also apply an armor check penalty.

If a character is wearing armor and using a shield, both armor check penalties apply.

If a character has a current armor check penalty, from his armor and shield combined, that is greater than their Base Attack Bonus, they move at reduced speed (\sfrac{2}{3}~Speed). If the Armor Check Penalty is 4 points greater than their Base Attack Bonus, they cannot take the Run action. If the Armor Check Penalty is ten points greater than their Base Attack Bonus, they can only stagger around, and only gain a standard action each round instead of a full-round action.

\subsection{Nonproficient with Armor Worn}
A character who wears armor and/or uses a shield with which he or she is not proficient treats that armor as if it had an Armor Check Penalty 4 greater than normal.

\subsection{Sleeping in Armor}
A character who sleeps in medium or heavy armor is automatically fatigued the next day. He or she takes a -2 penalty on Strength and Dexterity and can't charge or run. Sleeping in light armor does not cause fatigue.

\subsection{Weight}
This column gives the weight of the armor sized for a Medium wearer. Armor fitted for Small characters weighs half as much, and armor for Large characters weighs twice as much.

\subsection{Individual Armor Entries}
In addition to the qualities given on the table, every armor and shield provides a special benefit unique to it, given below.

\armorentry{Adamantine Breastplate}{Medium Armor}{Made of one of the hardest and most durable known metals, this breastplate provides excellent protection.}{You gain DR equal to your BaB/Adamantine.}

\armorentry{Adamantine Carapace}{Heavy Armor}{Made of one of the hardest and most durable known metals, this armor completely encases you.}{You gain DR equal to your BaB/Adamantine}

\armorentry{Adamantine Shield}{Shield}{A target shield constructed of pure Adamantine, it is nearly indestructible and can be placed between your important bits and deadly weapons.}{As an Immediate action you may force an opponent to reroll a successful attack against you.}

\armorentry{Animal Spirit Armor}{Medium Armor}{Fashioned of the skin of an angry beast, this armor still carries its spirit and will lend you its strength.}{Whilst Charging all your attacks deal +1d6 damage. If you have additional attacks from a high BaB this damage increases by +1d6 per extra attack you have gained.}

\armorentry{Bone Wall}{Great Shield}{A seemingly random assortment of bones collected into a large shield.}{A Bone Wall provides a bonus to saves against Death Effects and Necromancy spells equal to its Shield bonus}

\armorentry{Breastplate}{Medium Armor}{A solid steel armor underlayed with chainmail, it protects all of your vital bits.}{Whenever you suffer lethal physical damage you can convert an amount of that damage equal to your BAB into nonlethal damage.}

\armorentry{Brigadine}{Light Armor}{A sleeveless leather shirt with small steel plates riveted to the fabric. Like a medieval flak jacket, these plates protect your organs from stabbing weapons.}{Brigandine prevents an amount of Piercing damage per attack equal to your BAB.}

\armorentry{Buckler}{Shield}{A small shield strapped to the wrist or forearm used for parrying.}{A buckler provides no bonuses while you are denied your Dexterity bonus to AC. You may use a weapon with the hand using the Buckler, but doing so causes you to suffer a -1 penalty to attack rolls using this hand (including two handed weapons).}

\armorentry{Chain Shirt}{Light Armor}{Interwoven links of the finest steel cover your torso.}{A Chain shirt prevents an amount of Slashing damage per attack equal to your BAB.}

\armorentry{Chainmail}{Medium Armor}{Interwoven links of the finest steel cover your entire body.}{Chainmail prevents an amount of Slashing damage per attack equal to your BAB.}

\armorentry{Chitin Carapace}{Medium Armor}{Made out of an Ankheg or something, it's amazingly light and makes you look like a crazy mantis man when you wear it.}{You gain a climb speed equal to your land speed.}

\armorentry{Coral Armor}{Heavy Armor}{Made of living Coral, this armor is as dangerous to your opponents as it is protective.}{All Coral Armor counts as having been made with Armor Spikes. Enemies you damage with the Coral spikes are poisoned (DC 10 + \half your Level + Con Bonus), initial and secondary damage of 1d3 Dex.}

\armorentry{Cord Armor}{Light Armor}{A series of knots wrapped about your person protect you from incoming attacks.}{As an Immediate Action, you can replace the AC for one attack made against you with the result of a tumble check. You may use this ability after an attack has hit you, but not after damage has been rolled.}

\armorentry{Dragonscale Shield}{Shield}{Fashioned from a single dragon scale, this shield deflects dragon breath and other similar attacks.}{Whilst carrying this shield you may use an Immediate Action to gain Evasion for 1 round.}

\armorentry{Dragonscale Shirt}{Medium Armor}{Forged from the hide of a dragon, this armor provides a warding effect against its still living brethren.}{You gain Energy Resistance to one Energy type [chosen when the armor is made] equal to the Armor Bonus the shirt provides.}

\armorentry{Dragonscale Suit}{Heavy Armor}{This armor forged from the hide of a dragon provides excellent protection from swords and flames alike.}{You gain Energy Resistance to one Energy type equal to the Armor Bonus the suit provides (inc. any Enhancement bonus).}

\armorentry{Elaborate Gown}{Medium Armor}{It's a big frilly dress, or a bulky robe, or something else that's expensive and hard to move in.}{You gain a +2 bonus to Intimidate and Perform checks. You can hide weapons of a size up to your own inside your outfit with a normal Sleight of hand check.}

\armorentry{Full Plate}{Heavy Armor}{Plates of steel, flexible chain armor, and leather straps encase your entire body.}{Whenever you suffer lethal physical damage you may convert an amount of that damage equal to your BAB into nonlethal damage.}

\armorentry{Gith Armor}{Medium Armor}{The Gith have mastered the techniques of manipulating Astral Driftmetal and the chaotic stuff of Limbo to make a reasonably lightweight, yet oddly protective garment.}{You do not suffer any Arcane Spell Failure in this armor if you are proficient with it.}

\armorentry{Gray Armor}{Light Armor}{Made from slaadskin, gray armor shifts to anticipate attacks with the power of Limbo.}{You always act in the surprise round of any combat.}

\armorentry{Great Armor}{Heavy Armor}{Large, bulky armor typically worn by Samurai, Great Armor consists of many bands of lacquered iron with great shoulder pads known as O-sode.}{Great Armor includes arm panels which provide a +2 Shield bonus to AC}

\armorentry{Half Plate}{Heavy Armor}{A mixture of rigid plates and flexible chain, Half Plate combines protection with flexibility.}{Whenever you suffer lethal physical damage you may convert an amount of that damage equal to your BAB into nonlethal damage.}

\armorentry{Hide Armor}{Medium Armor}{You wear some other creature as a hat.}{Animals will not attack you unless you attack or otherwise threaten them first, and you cannot be detected by Scent.}

\armorentry{Hoplite Armor}{Heavy Armor}{Oldschool armor made of bronze and layered on in thick sheets all over the vitals.}{Hoplite Armor provides Medium Fortification as an Ex ability whilst worn.}

\armorentry{Ironskin Armor}{Light Armor}{The skin of creatures resistant to nonmagical weapons must be cut using enchanted tools, however the resulting armor is both light and hard as steel.}{You gain DR equal to your BaB/Magic and Piercing.}

\armorentry{Kappa Shell}{Great Shield}{Fitting on the back of a character like the Kappa's Shell it is named after, the Kappa Shell provides decent protection for units moving across a dangerous battle field.}{You may use both your hands while using this shield, but your attacks suffer a -2 penalty.}

\armorentry{Kite Shield}{Great Shield}{Shaped metal shields fit easily over a rider and his steed, allowing good protection for mounted troops.}{Your mount gains any benefit you gain from this shield (AC, feat bonuses etc.).}

\armorentry{Lamellar Armor}{Medium Armor}{Small metal or leather plates that are linked together to protect you, it provides a smooth surface that causes attacks to slide away rather than pierce the wearer.}{Prevents an amount of Piercing damage per attack equal to your BaB.}

\armorentry{Leather Armor}{Light Armor}{Made from cured cowhide, it's cheap, but does its job.}{You gain fire resistance 5. This fire resistance does not protect against environmental effects.}

\armorentry{Lobster Mail}{Medium Armor}{A carapace of a deep aquatic design. The engineering style is reminiscent of that of Kwalish, but much sleeker and individualized.}{Whilst underwater this armor allows you to breathe water, and its ACP is reduced to 0 if you are proficient with it.}

\armorentry{Mithril Shield}{Shield}{It's as strong as titanium and as light as titanium. Who are we fooling? This is a great shield.}{Attacks against you by enemies that are not flanking you have a 20\% miss chance.}

\armorentry{Mithril Shirt}{Light Armor}{The links that make up this chain shirt are woven much tighter and made of mithril.}{You gain DR equal to your BaB/Bludgeoning}

\armorentry{Mithril Suit}{Medium Armor}{A full body covering of light metal. Very shiny, and nearly skin tight, the mithril suit is surprisingly protective.}{You gain DR equal to your BaB/Bludgeoning.}

\armorentry{Ringmail}{Medium Armor}{Steel rings are woven onto a leather armor backing to provide extra reinforcement.}{Ringmail prevents the first 10 points of damage from a critical hit or sneak attack damage.}

\armorentry{Scale Mail}{Medium Armor}{Many small armor scales attached to each other in overlapping rows over a backing of leather, its rigid surface provides better protection from blunt attacks than mail.}{Prevents an amount of Bludgeoning damage per attack equal to your BAB.}

\armorentry{Silk Steel Armor}{Heavy Armor}{Made using an ancient bugbear technique, overlapping plates of steel are held apart by layers of silk and the entire carapace slides virtually without sound. While bulky, this black armor is remarkably stealthy.}{Whenever you suffer lethal physical damage you may convert an amount of that damage equal to your BAB into nonlethal damage. The ACP of this armor is ignored when using Move Silently.}

\armorentry{Steel Shield}{Shield}{It can be round or square or shaped like something in particular. It's not important, the key is that it's between you and sharp objects and it's made out of steel.}{Whilst there are no enemies within your natural reach you count as having cover from ranged attacks.}

\armorentry{Still Suit}{Light Armor}{A watertight suit envelops your whole body, recycling all of your excretions and protecting you from the heat.}{Whilst wearing a Still Suit you do not have to make Fortitude checks for hot environments and use half the daily allowance of water.}

\armorentry{Studded Leather Armor}{Light Armor}{Leather armor that has been adorned with metal studs to help protect your vitals.}{You may ignore the first 10 points of damage from each critical hit or sneak attack damage.}

\armorentry{Tower Shield}{Great Shield}{Giant pieces of wood or metal, tower shields offer tremendous protection, but cannot be effectively used while mounted.}{You may claim \half Cover. This must be declared at the start of your action and lasts until the start of your next action, but your own attacks suffer a -2 to hit penalty during this time.}

\armorentry{Wicker Armor}{Light Armor}{This armor is made of woven willow branches, and makes you look a bit like furniture with wings.}{Whilst wearing Wicker Armor you are immune to falling damage and gain a +3 circumstance bonus to Jump checks.}

\armorentry{Winter Clothes}{Light Armor}{Thick clothing that keeps you warm.}{You do not have to make Fortitude checks for cold environments whilst wearing Winter Clothing and you gain Cold resistance 5.}

\armorentry{Wooden Shield}{Shield}{Made of wood and held together with bands of steel or strips of leather, a wooden shield makes up in shock absorbance what it loses in resilience.}{Whilst there are no enemies within your natural reach you count as having cover from ranged attacks.}

\subsection{Masterwork Armor}

You can purchase or craft masterwork versions of armor or 
shields. Such a well-made item functions like the normal version, except that its 
armor check penalty is lessened by 1. 

A masterwork suit of armor or shield costs an extra 150 gp over and above the normal 
cost for that type of armor or shield.

The masterwork quality of a suit of armor or shield never provides a bonus on attack 
or damage rolls, even if the armor or shield is used as a weapon.

All magic armors and shields are automatically considered to be of masterwork quality.

You can't add the masterwork quality to armor or a shield after it is created; 
it must be crafted as a masterwork item.

\subsection{Armor For Unusual Creatures}
\begin{wraptable}{r}{.5\linewidth}
\caption{Nonstandard Armors}
\rowcolors{1}{colorone}{colortwo}
\centering
{\tabulinesep=1mm
\begin{tabu}to \linewidth {X[l] X[c] X[c] X[c] X[c]}
\header\textbf{Size} & \multicolumn{2}{c}{\textbf{Humanoid}} & \multicolumn{2}{c}{\textbf{Non-Humanoid}}\\ 
\header & \textit{Cost} & \textit{Weight} & \textit{Cost} & \textit{Weight} \\ \hline
Tiny or smaller\textsuperscript{1} & x\sfrac{1}{2} & x\sfrac{1}{10} & x1  & x\sfrac{1}{10}\\
Small & x1 & x\sfrac{1}{2} & x1 & x\sfrac{1}{2}\\
Medium & x1 & x1 & x2 & x1\\
Large & x2 & x2 & x4 & x2\\
Huge & x4 & x5 & x8 & x5\\
Gargantuan& x8 & x8 & x16 & x8\\
Colossal & x16 & x12 & x32 & x12\\ \hline
\multicolumn{5}{l}{\textsuperscript{1}Divide armor bonus by 2.}\\ \hline
\end{tabu}
}
\end{wraptable}

Armor and shields for unusually big creatures, unusually little creatures, and nonhumanoid creatures have different costs and weights from those given on the armors table. Refer to the appropriate line on Table: Nonstandard Armors and apply the multipliers to cost and weight for the armor type in question.

\subsection{Getting Into And Out Of Armor}

The time required to don armor depends on its type, see the Table: Donning and Removing Armor.

\textbf{Don:} This column tells how long it takes a character to put the armor 
on. (One minute is 10 rounds.) Readying (strapping on) a shield is only a move 
action.

\textbf{Don Hastily:} This column tells how long it takes to put the armor on in 
a hurry. The armor check penalty and armor bonus for hastily donned armor are each 
1 point worse than normal. 

\textbf{Remove:} This column tells how long it takes to get the armor off. Loosing 
a shield (removing it from the arm and dropping it) is only a move action.

~\\*
\begin{table}[b]
\caption{Donning and Removing Armor}
{\tabulinesep=1mm\centering
\rowcolors{1}{colorone}{colortwo}
\begin{tabu}to \textwidth {X l l l}
\header\textbf{Armor Type} & \textbf{Don} & \textbf{Don Hastily} & \textbf{Remove}\\ \hline
Shield or Great Shield & 1 move action & \sfrac{n}{a} & 1 move action\\
Nonarmor & 5 rounds & 1 full-round action & 1 full-round action\\
Light Armor & 1 minute & 5 rounds & 1 minute\textsuperscript{1}\\
Medium Armor & 4 minutes\textsuperscript{1} & 1 minute & 1 minute\textsuperscript{1}\\
Heavy Armor & 4 minutes\textsuperscript{2} & 4 minutes\textsuperscript{1} & 1d4+1 minutes\textsuperscript{1}\\ \hline
\multicolumn{4}{p{\textwidth}}{\textsuperscript{1} If the character has some help, cut this time in half. A single character doing nothing else can help one or two adjacent characters. Two characters can't help each other don armor at the same time.}\\
\multicolumn{4}{p{\textwidth}}{\textsuperscript{2} The wearer must have help to don this armor. Without help, it can be donned only hastily.}\\ \hline
\end{tabu}}
\end{table}