\section{[Combat] Feats} \label{feats:combat}

	\begin{multicols}{2}

	\combatfeat{Blind Fighting}{[Combat]}
{You don't have to see to kill.}
{You may reroll your miss chances caused by concealment.}
{While in darkness, you may move your normal speed without difficulty.}
{You have Blindsense out to 60', this allows you to know the location of all creatures within 60'.}
{You have Tremorsense out to 120', this allows you to ``see" anything within 120' that is touching the earth.}
{You cannot be caught flat footed.}
\combatfeat{Blitz}{[Combat]}
{You go all out and try to achieve goals in a proactive manner.}
{While charging, you may opt to lose your Dexterity Bonus to AC for one round, but inflicting an extra d6 of damage if you hit.}
{You may go all out when attacking, adding your Base Attack Bonus to your damage, but provoking an Attack of Opportunity.}
{Bonus attacks made in a Full Attack for having a high BAB are made with a -2 penalty instead of a -5 penalty.}
{Every time you inflict damage upon an opponent with your melee attacks, you may immediately use an Intimidate attempt against that opponent as a bonus action.}
{You may make a Full Attack action as a Standard Action.}
\combatfeat{Combat Looting}{[Combat]}
{You can put things into your pants in the middle of combat.}
{You may sheathe or store an object as a free action.}
{You get a +3 bonus to disarm attempts. Picking up objects off the ground does not provoke an attack of opportunity.}
{As a Swift action, you may take a ring, amulet/necklace, headband, bracer, or belt from an opponent you have successfully grappled. You may pick up an item off the ground in the middle of a move action.}
{If you are grappling with an opponent, you may activate or deactivate their magic items with a successful Use Magic Device check. You may make Appraise checks as a free action.}
{You can take 10 on Use Magic Device and Sleight of Hand checks.}
\combatfeat{Combat School}{[Combat]}
{You are a member of a completely arbitrary fighting school that has a number of recognizable signature fighting moves.}
{First, name your fighting style (such as ``Hammer and Anvil Technique'' or ``Crescent Moon Style'', or ``Way of the Lightning Mace''). This fighting style only works with a small list of melee weapons that you have to describe the connectedness to the DM in a half-way believable way. Now, whenever you are using that technique in melee combat, you gain a +2 bonus on attack rolls.}
{Your immersion in your technique gives you great martial prowess, you gain a +2 to damage rolls in melee combat.}
{When you strike your opponent with the signature moves of your fighting school in melee, they must make a Fortitude Save (DC 10 + 1/2 your level + your Strength bonus) or become dazed for one round. If they succeed on this save they are immune to further dazing attempts for one round.}
{You may take 10 on attack rolls while using your special techniques. The DC to disarm you of a school-appropriate weapon is increased by 4.}
{You may add +5 to-hit on any one attack you make after the first each turn. If you hit an opponent twice in one round, all further attacks this round against that opponent are made with The Edge.}
\spellentry{Command}

Enchantment (Compulsion) [Language-Dependent, Mind-Affecting]

\textbf{Level:} Clr 1

\textbf{Components:} V

\textbf{Casting Time:} 1 standard action

\textbf{Range:} Close (25 ft. + 5 ft./2 levels)

\textbf{Target:} One living creature

\textbf{Duration:} 1 round

\textbf{Saving Throw:} Will negates

\textbf{Spell Resistance:} Yes

You give the subject a single command, which it obeys to the best of its ability 
at its earliest opportunity. You may select from the following options.

\textit{Approach:} On its turn, the subject moves toward you as quickly and directly 
as possible for 1 round. The creature may do nothing but move during its turn, 
and it provokes attacks of opportunity for this movement as normal.

\textit{Drop:} On its turn, the subject drops whatever it is holding. It can't 
pick up any dropped item until its next turn.

\textit{Fall:} On its turn, the subject falls to the ground and remains prone for 
1 round. It may act normally while prone but takes any appropriate penalties.

\textit{Flee:} On its turn, the subject moves away from you as quickly as possible 
for 1 round. It may do nothing but move during its turn, and it provokes attacks 
of opportunity for this movement as normal.

\textit{Halt:} The subject stands in place for 1 round. It may not take any actions 
but is not considered helpless.

If the subject can't carry out your command on its next turn, the spell automatically 
fails.


\combatfeat{Danger Sense}{[Combat]}
{Maybe Spiders tell you what's up. You certainly react to danger with uncanny effectiveness.}
{You get a +3 bonus on Initiative checks.}
{For the purpose of Search, Spot, and Listen, you are always considered to be ``actively searching". You also get Uncanny Dodge.}
{You may take 10 on Listen, Spot, and Search checks.}
{You may make a Sense Motive check (opposed by your opponent's Bluff check) immediately whenever any creature approaches within 60' of you with harmful intent. If you succeed, you know the location of the creature even if you cannot see it.}
{You are never surprised and always act on the first round of any combat.}
\combatfeat{Elusive Target}{[Combat]}
{You are very hard to hit when you want to be.}
{You gain a +2 Dodge bonus to AC.}
{Your opponents do not gain flanking or higher ground bonuses against you.}
{Your opponents do not inflict extra damage from the power attack option.}
{Diverting Defense -- As an immediate action, you may redirect an attack against you to any creature in your threatened range, friend or foe. You may not redirect an attack to the creature making the attack.}
{As an immediate action, you may make an attack that would normally hit you miss instead.}
\combatfeat{Expert Tactician}{[Combat]}
{You benefit your allies so good they remember you long time.}
{You gain a +4 bonus when flanking instead of the normal +2 bonus. Your allies who flank with you gain the same advantage.}
{You may feint as an Immediate action.}
{As a move action, you may make any 5' square adjacent to yourself into difficult ground.}
{For determining flanking with your allies, you may count your location as being 5' in any direction from your real location.}
{You ignore Cover bonuses less than full cover.}
\combatfeat{Ghost Hunter}{[Combat]}
{You smack around those folks in the spirit world.}
{Your attacks have a 50\% chance of striking incorporeal opponents even if they are not magical.}
{You can hear incorporeal and ethereal creatures as if they lacked those traits (note that shadows and the like rarely bother to actively move silently).}
{You can see invisible and ethereal creatures as if they lacked those traits.}
{Your attacks count as if you had the Ghost Touch property on your weapons.}
{Any Armor or shield you use benefits from the Ghost Touch quality.}
\combatfeat{Giant Slayer}{[Combat]}
{Everyone has a specialty. Yours is miraculously finding ways to stab creatures in the face when it seems improbable that you would be able to reach that high.}
{When you perform a grab on grapple maneuver, you do not provoke an attack of opportunity.}
{You gain a +4 Dodge bonus to your AC and Reflex Saves against attacks from any creature with a longer natural reach than your own.}
{You have The Edge against any creature you attack that is larger than you. Also, an opponent using the Improved Grab ability on you provokes an attack of opportunity from you. You may take this attack even if you do not threaten a square occupied by your opponent.}
{When you attempt to trip an opponent, you may choose whether your opponent resists with Strength or Dexterity.}
{When involved in an opposed bull rush, grapple, or trip check as the attacker or defender, you may negate the size modifier of both participants. You may not choose to negate the size modifier of only one character.}
\combatfeat{Great Fortitude}{[Combat]}
{You are so tough. Your belly is like a prism.}
{You gain a +3 bonus to your Fortitude Saves.}
{You die at -20 instead of -10.}
{You gain 1 hit point per level.}
{You gain DR of 5/-.}
{You are immune to the fatigued and exhausted conditions. If you are already immune to these conditions, you gain 1 hit point per level for each condition you were already immune to.}
\combatfeat{Horde Breaker}{[Combat]}
{You kill really large numbers of people.}
{You gain a number of extra attacks of opportunity each round equal to your Dexterity Bonus (if positive).}
{Whenever you drop an opponent with a melee attack, you are entitled to a bonus ``cleave" attack against another opponent you threaten. You may not take a 5' step or otherwise move before taking this bonus attack. This cleave attack is considered an attack of opportunity.}
{You may take a bonus 5' step every time you are entitled to a cleave attack, which you may take either before or after the attack.}
{You may generate an aura of fear on any opponents within 10' of yourself whenever you drop an opponent in melee. The save DC is 10 + the Hit Dice of the dropped creature.}
{Opponents you have the Edge against provoke an attack of opportunity from you by moving into your threatened area or attacking you.}
\combatfeat{Hunter}{[Combat]}
{You can move around and shoot things with surprising effectiveness.}
{The penalties for using a ranged weapon from an unstable platform (such as a ship or a moving horse) are halved.}
{Shot on the Run -- you may take a standard action to attack with a ranged weapon in the middle of a move action, taking some of your movement before and some of your movement after your attack. That still counts as your standard and move action for the round.}
{You suffer no penalties for firing from unstable ground, a running steed, or any of that.}
{You may take a full round action to take a double move and make a single ranged attack from any point during your movement.}
{You may take a full round action to run a full four times your speed and make a single ranged attack from any point during your movement. You retain your Dexterity modifier to AC while running.}
\combatfeat{Insightful Strike}{[Combat]}
{You Hack people down with inherent awesomeness.}
{You may use your Wisdom Modifier in place of your Strength Modifier for your melee attack rolls.}
{Your attacks have The Edge against an opponent who has a lower Wisdom and Dexterity than your own Wisdom, regardless of relative BAB.}
{Your melee attacks have a doubled critical threat range.}
{You make horribly telling blows. The extra critical multiplier of your melee attacks is doubled (x2 becomes x3, x3 becomes x5, and x4 becomes x7).}
{Any Melee attack you make is considered to be made with a magic weapon that has an enhancement bonus equal to your Wisdom Modifier (if positive).}
\combatfeat{Iron Will}{[Combat]}
{You are able to grit your teeth and shake off mental influences.}
{You gain a +3 bonus to your Willpower saves.}
{You gain the slippery mind ability of a Rogue.}
{If you are stunned, you are dazed instead.}
{You do not suffer penalties from pain and fear.}
{You are immune to compulsion effects.}
\combatfeat{Juggernaut}{[Combat]}
{You are an unstoppable Juggernaut.}
{You may be considered one size category larger for the purposes of any size dependant roll you make (such as a Bull Rush, Overrun, or Lift action).}
{You do not provoke an attack of opportunity for entering an opponent's square.}
{You gain a +4 bonus to attack and damage rolls to destroy objects. You may shatter a Force Effect by inflicting 30 damage on it.}
{When you successfully bullrush or overrun an opponent, you automatically Trample them, inflicting damage equal to a natural slam attack for a creature of your size.}
{You gain the Rock Throwing ability of any standard Giant with a strength equal to or less than yourself.}
\combatfeat{Lightning Reflexes}{[Combat]}
{You are fasty McFastFast. It helps keep you alive.}
{You gain a +3 bonus to your Reflex saves.}
{You gain Evasion, if you already have Evasion, that stacks to Improved Evasion.}
{You may make a Balance Check in place of your Reflex save.}
{You gain a +3 bonus to your Initiative.}
{When you take the Full Defense Action, add your level to your AC.}
\combatfeat{Mage Slayer}{[Combat]}
{You have trained long and hard to kill magic users. Maybe you hate them, maybe you just noticed that most of the really dangerous creatures in the world use magic.}
{You gain Spell Resistance of 5 + Character Level.}
{Damage you inflict is considered ``ongoing damage" for the purposes of concentration checks made before the beginning of your next round. All your attacks in a round are considered the same source of continuing damage.}
{Creatures cannot cast defensively within your threat range.}
{Your attacks ignore Deflection bonuses to AC.}
{When a creature uses a [Teleportation] effect within medium range of yourself, you may choose to be transported as well. This is not an action.}
\combatfeat{Murderous Intent}{[Combat]}
{You stab people in the face.}
{You may make a Coup de Grace as a standard action.}
{When you kill an opponent, you gain a +2 Morale Bonus to your attack and damage rolls for 1 minute.}
{Once per round, you may take an attack of opportunity against an opponent who is denied their Dexterity bonus to AC.}
{You may take a Coup de Grace action against opponents who are stunned.}
{You may take a Coup de Grace action against opponents who are dazed.}
\combatfeat{Phalanx Fighter}{[Combat]}
{You fight well in a group.}
{You may take attacks of opportunity even while flat footed.}
{Any Dodge bonus to AC you gain is also granted to any adjacent allies for as long as you benefit from the bonus and your ally remains adjacent.}
{Charging is an action that provokes an attack of opportunity from you. This attack is considered to be a ``readied attack" if it matters for purposes like setting against a charge.}
{You may attack with a reach weapon as if it was not a reach weapon. Thus, a medium creature would normally threaten at 5' and 10' with a reach weapon.}
{You may take an Aid Another action once per round as a free action. You provide double normal bonuses from this effect.}
\combatfeat{Point Blank Shot}{[Combat]}
{You are crazy good using a ranged weapon in close quarters.}
{When you are within 30' of your target, your attacks with a ranged weapon gain a +3 bonus to-hit.}
{You add your base attack bonus to damage with any ranged attack within the first range increment.}
{You do not provoke an attack of opportunity when you make a ranged attack.}
{When armed with a Ranged Weapon, you may make attacks of opportunity against opponents who provoke them within 30' of you. Movement within this area does not provoke an attack of opportunity.}
{With a Full Attack action, you may fire a ranged weapon once at every single opponent within the first range increment of your weapon. You gain no additional attacks for having a high BAB. Make a single attack roll for the entire round, and compare to the armor class of each opponent within range.}
\combatfeat{Sniper}{[Combat]}
{Your shooting is precise and dangerous.}
{Your range increments are 50\% longer than they would ordinarily be. Any benefit of being within 30' of an opponent is retained out to 60'.}
{You do not suffer a -4 penalty when firing a ranged weapon into melee and never hit an unintended target in close combats or grapples.}
{Your ranged attacks ignore Cover Bonuses (total cover still bones you).}
{Opponents struck by your ranged attacks do not automatically know what square your attack came from, and must attempt to find you normally.}
{Any time you hit an opponent with a ranged weapon, it is counted as a critical threat. If your weapon already had a 19-20 threat range, increase its critical multiplier by 1.}
\combatfeat{Subtle Cut}{[Combat]}
{You cut people so bad they have to ask you about it later.}
{Any time you damage an opponent, that damage is increased by 1.}
{As a standard action, you can make a weapon attack that also reduces a creature's movement rate. For every 5 points of damage this attack does, reduce the creature's movement by 5'. This penalties lasts until the damage is healed.}
{As a standard action, you may make a weapon attack that also does 2d4 points of Dexterity damage.}
{Any weapon attack that you make at this level acts as if the weapon had the wounding property.}
{As a standard action, you may make an attack that dazes your opponent. This effect lasts one round, and has a DC of 10 + half your level + your Intelligence bonus.}
\combatfeat{Two Weapon Fighting}{[Combat]}
{When armed with two weapons, you fight with two weapons rather than picking and choosing and fighting with only one. Kind of obvious in retrospect.}
{You suffer no penalty for doing things with your off-hand. When you make an attack or full-attack action, you may make a number of attacks with your off-hand weapon equal to the number of attacks you are afforded with your primary weapon.}
{While armed with two weapons, you gain an extra Attack of Opportunity each round for each attack you would be allowed for your BAB, these extra attacks of opportunity must be made with your off-hand.}
{You gain a +2 Shield Bonus to your armor class when fighting with two weapons and not flat footed.}
{You may feint as a Swift action.}
{While fighting with two weapons and not flat footed you may add the enhancement bonus of either your primary or your off-hand weapon to your Shield Bonus to AC.}
\combatfeat{Weapon Finesse}{[Combat]}
{You are incredibly deft with a sword.}
{You may use your Dexterity Modifier instead of your Strength modifier for calculating your melee attack bonus.}
{Your special attacks are considered to have the Edge when you attack an opponent with a Dexterity modifier smaller than yours, even if your Base Attack Bonus is not larger.}
{You may use your Dexterity modifier in place of your Strength modifier when attempting to trip an opponent.}
{You may use your Dexterity modifier in place of your Strength modifier for calculating your melee damage.}
{Once per turn, when an opponent is struck, you may take an attack of opportunity on that opponent.}
\combatfeat{Weapon of Righteous Destruction}{[Combat]}
{Your hands make whatever is being held by them holy and on fire. For some reason this doesn't make them melt or burn up.}
{Whatever weapon you are wielding is considered Magical (+\sfrac{1}{3} bonus/level) in addition to any other properties that it has. Your unarmed attacks, even if not proficient, count for this effect.}
{The above, plus Flaming.}
{The above, Holy instead of Flaming.}
{The above, plus Sun weapon, Fort save. (BoG)}
{The above, plus Vorpal weapon (BoG).}
\combatfeat{Whirlwind}{[Combat]}
{You are just as dangerous to everyone around you as to anyone around you.}
{As a full round action, you may make a whirlwind attack - you may make a single attack against each opponent you can reach. Roll one attack roll and compare to each available opponent's AC individually.}
{You gain a +3 bonus to Balance checks.}
{When you make a whirlwind attack, you may also take a regular move action. You may make a single attack against each opponent you can reach at any point during your movement. Roll one attack roll and compare to each available opponent's AC individually, as normal.}
{Until your next round after making a whirlwind attack, you may take an attack of opportunity against any opponent that enters your threatened area.}
{When you make a whirlwind attack, you may also take a double move action as if you had charged. You overrun any creature in your path and may make a single attack against each opponent you can reach at any point during your movement. Roll one attack roll and compare to each available opponent's AC individually, as normal.}
\combatfeat{Zen Archery}{[Combat]}
{You are very calm about shooting people in the face. That's a good place to be.}
{You may use your Wisdom Modifier in place of your Dexterity Modifier on ranged attack rolls.}
{Any opponent you can hear is considered an opponent you can see for purposes of targeting them with ranged attacks.}
{If you cast a Touch Spell, you can deliver it with a ranged weapon (though you must hit with a normal attack to deliver the spell).}
{As a Full Round Action, you may make one ranged attack with a +20 Insight bonus to hit.}
{As a Full Round Action, you may make one ranged attack with a +20 Insight bonus to hit. If this attack hits, your attack is automatically upgraded to a critical threat. If the threat range of your weapon is 19-20, your critical multiplier is increased by one.}

	\end{multicols}

	\section{[Skill] Feats}

	\begin{multicols}{2}

	\skillfeat{Acquirer's Eye}{[Skill:Appraise]}
{You know what you want, even if other people have it right now.}
{You gain +3 to your Appraise checks.}
{You automatically know if something is ordinary, masterwork, or magic when looking at it.}
{You can discover the properties of a magic item, including how to activate it (if appropriate) and how many charges are left (if it has them), with a successful Appraise check (DC item's caster level + 10) and 10 minutes of work.}
{Once per round as a free action, you can examine a magic item and attempt an Appraise check (DC item's caster level + 20) to determine its properties, including its functions, how to activate those functions (if necessary), and how many charges it has left (if it has charges).}
{You know what the most valuable piece of treasure is in any collection, such as the most valuable magic item an enemy is wearing or the most valuable object in a dragon's horde, just by looking at the collection. You automatically recognize an artifact when looking at it.}
\skillfeat{Acrobatic}{[Skill:Tumble]}
{You can totally flip out and kill someone with your gymnastic prowess.}
{You gain a +3 bonus to Tumble checks.}
{When using the Combat Expertise option, your dodge bonus to AC increases by +1. This further increases by +1 for every ten ranks of Tumble you have (+2 at 14, +3 at 24, and so on).}
{If an opponent attempts to bull-rush, overrun, or trample you, if you succeed on Tumble check of DC 25 + their base attack bonus, their movement continues in a straight line to the maximum allowed by their speed, you remain where you were, and you don't suffer from the effects of their bull-rush, overrun, or trample. If you fail, you provoke an attack of opportunity from that enemy.}
{If you succeed on a DC 40 Tumble check, you can move 10 feet when taking a 5-foot step.}
{If you succeed on a Tumble check against a DC of 30 + an opponent's base attack bonus, an action that would normally provoke an attack of opportunity doesn't.}
\skillfeat{Alertness}{[Skill:Listen]}
{Your ears are so sharp you probably wouldn't miss your eyes.}
{You gain a +3 bonus to Listen checks.}
{You can make a Listen check once a round as a free action. You don't take penalties for distractions on your Listen checks.}
{You gain Blindsense to 60 feet. You don't take penalties for ambient noise, such as loud winds. Divide any distance penalties you take on Listen checks by two.}
{You gain Blindsight to 120 feet.}
{You can hear through magical silence and similar effects, but you take a -20 penalty on your check. Divide any distance penalties you take on Listen checks by five.}
\skillfeat{Animal Affinity}{[Skill:Handle Animal]}
{You're one of those people animals just won't leave alone for no apparent reason.}
{You gain the wild empathy ability, with your check equal to your character level plus your Charisma modifier plus any other applicable bonuses. If you already have wild empathy, or later gain it from another source, you gain a +3 bonus on Handle Animal checks.}
{You can handle an animal as a free action, and push it as a move action.}
{You gain the benefits of \spell{speak with animals} permanently as an extraordinary ability. The DCs for you to rear and train creatures are halved.}
{With a DC 30 Handle Animal check, you can use a mass version of \spell{charm animal} as a spell-like ability, with save DC equal to 10 + \half your character level + your Cha modifier and effective caster level equal to your bonus on Handle Animal checks.}
{You can summon animals to your aid. Choose an animal with a CR equal to or less than your character level, and make a Handle Animal check at a DC of 25 + your character level. If you succeed, you summon a number of animals depending on how much the animal's CR is less than your character level for an hour. You can't use this ability again until any animals you've summoned with it have unsummoned or you've dismissed them.

\vspace{2pt}
\begin{tabular}{l l|l l}
	\textbf{CR} &\textbf{Appearing} &\textbf{CR} &\textbf{Appearing}\\
	Level - 1 &1   &Level - 11 &15+3d10\\
	Level - 2 &1d3 &Level - 12 &40\\
	Level - 3 &1d4 &Level - 13 &50\\
	Level - 4 &1d6 &Level - 14 &60\\
	Level - 5 &1d8 &Level - 15 &80\\
	Level - 6 &1d10 &Level - 16 &100\\
	Level - 7 &2d6 &Level - 17 &150\\
	Level - 8 &3d6 &Level - 18 &200\\
	Level - 9 &3d10 &Level - 19 &300 \\
	Level - 10 &10+3d6 &Level - 20 &450
\end{tabular}}
%\listone
%	\bolditem{CR;}{Number Appearing}
%	\bolditem{Level - 1;}{1}
%	\bolditem{Level - 2;}{1d3}
%	\bolditem{Level - 3;}{1d4}
%	\bolditem{Level - 4;}{1d6}
%	\bolditem{Level - 5;}{1d8}
%	\bolditem{Level - 6;}{1d10}
%	\bolditem{Level - 7;}{2d6}
%	\bolditem{Level - 8;}{3d6}
%	\bolditem{Level - 9;}{3d10}
%	\bolditem{Level - 10;}{10+3d6}
%	\bolditem{Level - 11;}{15+3d10}
%	\bolditem{Level - 12;}{40}
%	\bolditem{Level - 13;}{50}
%	\bolditem{Level - 14;}{60}
%	\bolditem{Level - 15;}{80}
%	\bolditem{Level - 16;}{100}
%	\bolditem{Level - 17;}{150}
%	\bolditem{Level - 18;}{200}
%	\bolditem{Level - 19;}{300}
%\end{list}
\skillfeat{Army of Demons}{[Celestial] [Fiend] [Leadership] [Skill:Knowledge (The Planes)]}
{You have an army of planar crazy crap.}
{You have a Command Rating equal to your Knowledge (The Planes) ranks divided by five (round up).}
{You can muster a group of followers. Your leadership score is your ranks in Knowledge: Planes plus your Charisma mod. These followers can and must be outsiders.}
{Your followers swell in number to that of an army.}
{You own a planar stronghold.}
{Your allies gain a +2 morale bonus to all saving throws if they can see you and you are within medium range.}
\label{comm:feat:battlefieldsurgeon}\skillfeat{Battlefield Surgeon [Skill]}{
You like to cut people open with a saw. But it's good for them. Seriously.}{
Heal ranks:}{
You gain +3 to your Heal checks.}{
You can make first aid, treat poison, and treat wound checks as move actions.}{
For every 5 points your Heal check exceeds the DC for long term care, your patients recover another +100\% faster. For instance, if your Heal check result is 23, your patients would heal at thrice the normal rate.}{
If you operate on a patient for a minute, they regain hit points equal to your Heal check result. You also may, instead of healing hit point damage, cure any condition that heal could, reattach severed limbs, or repair ruined organs, if you succeed on a DC 30 check. Patients under your long-term care heal permanent ability drain as if it was ability damage.}{
With one hour of work, 25,000 gp worth of materials (which are consumed in the process), and a DC 40 Heal check, you can restore a creature that died within the last twenty-four hours to life. The subject's soul must be free and willing to return for the effect to work.}

\hypertarget{feat:bureaucrat}{}\skillfeat{Bureaucrat [Skill] [Leadership]}{
You have a functioning guild that makes stuff for you and gives you money}{
Appraise ranks:}{
You draw an income for working as an administrator, getting 1 GP/week per rank in Appraise.}{
You can muster a group of followers. Your leadership score is your ranks in Appraise plus your Intelligence modifier. These followers all have profession and craft skills.}{
You get your own Stronghold.}{
You get a +2 bonus to profit checks.}{
Your guild goes planar, your number of followers swell to the size of an army and their ranks start filling up with producers and managers from other planes of existence.}
\skillfeat{Combat Casting}{[Skill:Concentration]}
{Having a sword sticking out of your chest doesn't noticeably impede your ability to do\ldots well, just about anything.}
{You gain +3 to your Concentration checks.}
{You can take 10 on Concentration checks and caster level checks.}
{You may maintain concentration on a spell as a move action (DC 25 + spell level). If you beat the DC by 10 or more, you can maintain concentration as a swift action. If you fail your check, you lose concentration.}
{If you would be nauseated, you're sickened instead.}
{All Concentration DCs are halved for you.}
\skillfeat{Con Artist}{[Skill:Bluff]}
{You can fool some of the people, all of the time.}
{You gain a +3 bonus to Bluff checks.}
{Magic effects that would detect your lies or force you to speak the truth must succeed on a caster level check with DC equal to 10 plus your ranks in Bluff or fail.}
{Divination magic used on you detects a false alignment of your choice. You can present false surface thoughts to \spell{detect thoughts} and similar effects, changing your apparent Intelligence score (and thus your apparent mental strength) by as much as 10 points and can place any thought in your ``surface thoughts" to be read by such spells or effects.}
{If you beat someone's Sense Motive check by 25, you can instill a \spell{suggestion} in them, as the spell. This suggestion lasts for one hour for each of your character levels.}
{You are protected from all spells and effects that detect or read emotions or thoughts, as by \spell{mind blank}.}
\skillfeat{Cryptographer}{[Skill:Decipher Script]}
{You're good at reading things no one intended you to.}
{You gain +3 to your Decipher Script checks.}
{You can decipher a written spell (like a scroll) without using \spell{read magic}, if you succeed on a Decipher Script check of DC 20 + the spell's level. You can try once per day on any particular written spell.}
{You don't trigger written magic traps (like \spell{explosive runes} or \spell{symbols}) by reading them. You can disable them with Decipher Script as if you were using Disable Device. You can read the material hidden by a \spell{secret page} with a DC 25 Decipher Script check.}
{When you cast a spell from a scroll, the spell's save DC is equal to 10 + the spell's level + your Intelligence modifier + any other applicable bonuses, and its caster level is equal to your character level, plus other applicable bonuses.}
{Reading text using Decipher Script is a free action for you. You may disable written magical traps as a swift action, and you can cast 5th-level or lower spells from scrolls as a swift action.}
\skillfeat{Deft Fingers}{[Skill:Sleight of Hand]}
{Your amazing manual dexterity is the talk of princes and princesses.}
{You gain a +3 bonus on your Sleight of Hand checks.}
{If you draw a hidden weapon and attack with it in the same round, your opponent loses their Dexterity bonus to AC against your first attack with that weapon that round. This ability can only be used once per round.}
{You can make an adjacent creature or object your size or smaller ``disappear" with your legerdemain. If you succeed on a DC 30 Sleight of Hand check as a standard action, your target can make a Hide check, or you can make the Hide check for them or it. As usual, you can hide larger creatures or objects by taking a -20 cumulative penalty for each size category larger they are than you.}
{With a DC 30 Sleight of Hand check, you can use \spell{shrink item} as a spell-like ability.}
{With a DC 40 Sleight of Hand check, you can use \spell{teleport object} as a spell-like ability. You can also retrieve items placed in the Ethereal Plane using \spell{teleport object}. With a DC 40 Sleight of Hand check, you can use \spell{instant summons} as a spell-like ability without requiring \spell{arcane mark}, but you may only designate one item at a time.}
\skillfeat{Detective}{[Skill:Gather Information]}
{You're good at finding things out just by conversing with townsfolk.}
{You gain a +3 bonus on your Gather Information checks.}
{Your ability to pick up on the social context aids you in establishing rapport. After succeeding on a Gather Information check, you gain a +2 bonus to Knowledge checks, Sense Motive checks, and checks for Cha-based skills in the same milieu.}
{With 2d6 hours of research, you can study a specific topic, such as a particular location or a well-known local monster, and substitute a Gather Information check for any Knowledge checks pertaining to the topic. You need access to local informants, a library, scholars, or other appropriate sources to use this ability.}
{You can gain the benefits of \spell{legend lore} with a DC 30 Gather Information check. If you have the person or thing at hand, or are in the place, this takes a day; otherwise, it consumes the time as normal for \spell{legend lore}. You need access to individuals or resources with relevant knowledge to use this ability.}
{With a DC 40 Gather Information check and 1d4+1 days of talking to people, you can either find an answer to any question you can pose in ten words or less, or find out where you need to go to get the answer. You need access to individuals or resources with relevant knowledge to use this ability.}
\label{comm:feat:dreadfuldemeanor}\skillfeat{Dreadful Demeanor [Skill]}{
People know you're a badass motherfvcker the instant you enter the room.}{
Intimidate ranks:}{
You gain +3 to your Intimidate checks.}{
You can demoralize an opponent as a move action.}{
Opponents you've demoralized remain \condition{shaken} until they lose sight of you.}{
Opponents who would be \condition{panicked} because of your fear effects are \condition{cowered} instead for the duration of the effect.}{
Any time you confirm a critical hit in melee, your target is \spell{cowered} until they lose sight of you. This is a fear effect.}
\skillfeat{Expert Counterfeiter}{[Skill:Forgery]}
{You aren't a common forger, you're an \textit{}artiste.}
{You gain a +3 bonus to Forgery checks.}
{When creating a forgery, you roll twice and take the better result.}
{In situations where you can present a legal document of some sort, you can substitute a Forgery check for a Bluff, Diplomacy, or Intimidate check.}
{You can purchase items with counterfeit bills of exchange, falsified credit vouchers, and the like. You can acquire any item available through the gold economy in this method. Normally, your counterfeits are so good they don't provoke suspicion, but if someone examines them, they must still beat you in an opposed Forgery check to recognize they're not the real thing.}
{You can duplicate a scroll with eight hours of work and a Forgery check against DC 35 + the spell's level. The duplicate functions in all manners like the original scroll. You must have appropriate materials on hand for scribing the scroll, and if the spell requires XP or expensive material components, you must provide the requisite components or make up the XP cost in materials.}
\skillfeat{Ghost Step}{[Skill:Move Silently]}
{You might as well be incorporeal for all the noise you make.}
{You gain +3 to your Move Silently checks.}
{Anyone attempting to use Survival to track you must beat you in an opposed check against Move Silently.}
{Creatures with blindsense, blindsight, tremorsense, or similar abilities do not automatically detect your presence, but must succeed on a Listen check, opposed by your Move Silently check, to notice you.}
{With success on a DC 30 Move Silently check as a standard action, you can control ambient sounds within 30 feet of yourself for a round. You can specifically duplicate any effect from \spell{control sound} (XPH), \spell{silence}, or \spell{ventriloquism}, and in general can make sound you've heard come from any part of the area, displace sounds in the area, or suppress any sounds or sounds. Also, if you take a -10 DC penalty on your Move Silently check, anyone within 30 feet of you can substitute your check result for their own.}
{You're so quiet that people don't even remember you when you're standing right next to them. Your opponents count as flat-footed whenever you attack them.}
\skillfeat{Investigator}{[Skill:Search]}
{You have an eye for detail and so much patience that going through a 100' by 100' room inch-by-inch doesn't even try it.}
{You can use Search to find traps like a character with trapfinding. If you already have that ability, you gain +3 to your Search checks. Search is always a class skill for you.}
{You can Search a 10' by 10' area with a full-round action.}
{You automatically sense any active magic effects in an area you search. If you succeed on a DC 20 Search check, you can determine their number, strength, and school, as if using \spell{detect magic}.}
{You can Search objects or areas within 30 feet of yourself. You can make a Search check as a swift action.}
{You have an intuitive sense for hidden things. Anytime something that someone has hidden is within 60 feet of you, you know it; if there are multiple things, you know how many. However, you must still make Search checks as normal to locate them.}
\skillfeat{Item Master}{[Skill:Use Magic Device]}
{You make magic items do things you want.}
{You gain a +3 bonus to Use Magic Device checks.}
{You don't suffer mishaps with magic items.}
{When rolling Use Magic Device checks or random effects from magic items, you may roll twice and take the better result.}
{With a swift action and a successful Use Magic Device check against a DC of 30 + the item's caster level, you can gain the benefits of a slotted magic item without needing to have a slot available (for instance, a third ring on your finger) for one round.}
{When you activate a wand or staff, you can substitute a spell slot instead of using a charge. The spell slot must be one you have not used for the day, though you may lose a prepared spell to emulate a wand charge (you may not lose prepared spells from your school of specialty, if any). The spell slot lost must be equal to or higher in level than the spell stored in the wand, including any level-increasing metamagic enhancements. When using spell trigger, spell completion, or other consumable magic items, if you succeed on a Use Magic Device check of 40 + the caster level of the item as a swift action, the item or charges thereof are not consumed.}
\skillfeat{Leadership}{[Leadership] [Skill:Diplomacy]}
{You convince people that obeying you is a good career move.}
{You can awe even strangers and enemies into following your orders. With a DC 20 Diplomacy check, you can use \spell{command} as a spell-like ability, with save DC equal to 10 + \half your character level + your Cha modifier.}
{Your natural talent for leaderships attracts followers. Your leadership score is equal to your ranks in Diplomacy plus your Charisma modifier.}
{You persuade someone that you are so awesome that they should follow you around all the time, acquiring a cohort. A cohort is an intelligent and loyal creature with a CR at least 2 less than your character level. Cohorts gain levels when you do.}
{Your natural majesty stirs guilt in those who refuse your demands. With a DC 30 Diplomacy check, you can use \spell{geas} as a spell-like ability, but it offers a Will save at DC 10 + \half your character level + your Cha modifier.}
{You command the loyalty of armies\ldots even opposing ones. With a DC 40 Diplomacy check, you can use \spell{greater command} as a spell-like ability, with save DC equal to 10 + \half your character level + your Cha modifier and effective caster level equal to your bonus on Diplomacy checks.}
\skillfeat{Legendary Wrangler}{[Skill:Use Rope]}
{No one can tell where you end and your ropes begin.}
{You gain a +3 bonus to Use Rope checks and proficiency with the bolas, net, and whip.}
{You can use a rope as if it was a bolas or whip, and you can substitute your ranks in Use Rope for your Base Attack Bonus for combat maneuvers made with it. You can also use it as a net, replacing the normal DC 20 Escape Artist check for someone entangled with it with your Use Rope check. You can throw a grappling hook, tie a knot, tie a special knot, or tie a rope around yourself one-handed as a move action. You don't provoke attacks of opportunity for using Use Rope.}
{You can use a rope, whip, grappling hook, or similar item to manipulate any item within 30 feet of yourself as easily as if it was in your hands; you can also make disarm, entangling (as if with a net), and trip attempts with it. You can move around on ropes and similar structures, like webs, as easily as you can on the ground.}
{With a DC 30 Use Rope check, you can use \spell{animate rope} as a spell-like ability; you can use any ability you can with an ordinary rope with an animated rope.}
{You can manipulate items out to 60 feet with ropes and similar items. You can use ropes for the grab on and hold down grapple maneuvers. When using combat maneuvers with ropes, you can replace the relevant check (disarm, grapple, trip, etc.) with a Use Rope check.}
\skillfeat{Magical Aptitude}{[Skill:Spellcasting]}
{You're crazy good at manipulating magic.}
{You gain a +3 bonus on Spellcraft checks.}
{When counterspelling, you may use a spell of the same school that is one or more spell levels higher than the target spell.}
{You can dismiss a spell as a free action. You can redirect a spell as a move action, if it normally requires a standard action, or a swift action, if it normally takes a move action. You gain a +3 bonus on dispel checks.}
{You can counter a spell as an immediate action.}
{You automatically know which spells or magic effects are active on upon any individual object you see, as if you had \spell{greater arcane sight} active on yourself.}
\skillfeat{Many-Faced}{[Skill:Disguise]}
{You change identities so often even you don't remember what you look like anymore.}
{You gain +3 to your Disguise checks.}
{When creating a disguise, you roll twice and take the better result.}
{You can use \spell{Nystul's magic aura} as a spell-like ability at will, with a caster level equal to your character level and a save DC of 10 + \half your character level + your Cha modifier.}
{You can create a disguise as a full-round action, but you take a -10 penalty to your Disguise check. You can't be under direct observation while doing this, but you can use Bluff to create a diversion to allow you to change guises, as for the Hide skill.}
{You can choose an appearance that anyone viewing you with scrying or other divination magic sees instead of your ``real" appearance. Even someone who benefits from \spell{true seeing} must succeed on a caster level check (DC 11 + your ranks in Disguise) to penetrate the illusion.}
\skillfeat{Master of Terror}{[Leadership] [Skill:Intimidate]}
{You scare people so bad they follow you around hoping you won't hurt them.}
{Whenever you use Intimidate in combat, it affects everyone within 30 feet of you.}{
You gain followers. Your leadership score is equal to your ranks in Intimidate plus your Charisma modifier.}
{You gain a cohort who enjoys frightening your underlings almost as much as you do. A cohort is an intelligent and loyal creature with a CR at least 2 less than your character level. Cohorts gain levels when you do.}
{You gain the frightful presence ability. When you speak or attack, enemies within 30 feet of you must succeed on a Will save (DC 10 + \half your character level + your Cha modifier) or become shaken for 5d6 rounds. An opponent that succeeds on its saving throw is immune to your frightful presence for 24 hours.}
{Your opponents take a -2 morale penalty to saving throws if they can see you and you are within medium range (based on your character level).}
\skillfeat{Natural Empath}{[Skill:Sense Motive]}
{You read people like books.}
{You gain a +3 bonus to Sense Motive checks.}
{You can quickly size up potential opponents. If you succeed on a Sense Motive check as a free action, opposed by their Bluff, you can tell if they're an even match (their CR equals your character level), an easy challenge (their CR is 1-3 less than your level), irrelevant (their CR is 4 or more less than your level), stronger (their CR is 1-3 higher than your level), or overwhelmingly powerful (their CR is 4 or more higher than your level). You can use this ability once on a particular creature every 24 hours.}
{If you succeed on a Sense Motive check, opposed by Bluff, you know your opponent's alignment. If you beat their Bluff by 20 or more, you can read their surface thoughts, as if during the third round of \spell{detect thoughts}.}
{You have an uncanny intuition for when people are interested in you. Any time someone uses a remote spell or effect, like \spell{scrying}, to examine you, you know you're under observation and if you make a Sense Motive check that beats their Bluff check, you know some details about them: if you've met them before, you recognize them, but if not, you get a basic idea of their reasons for their interest in you. Similarly, if you use Sense Motive on someone influenced by an enchantment effect, you can find out who created the effect with a Sense Motive check opposed by the controller's Bluff, getting the same information.}
{You know what people are going to do before they do. Any time someone you're aware of attacks you, make a Sense Motive check opposed by their Bluff: if you succeed, you get a free surprise round.}
\skillfeat{Persuasive}{[Skill:Diplomacy]}
{When you tell you people something that contradicts the evidence of their own eyes, they believe you.}
{You gain a +3 bonus to Diplomacy checks.}
{Your words can stop fights before they start. Any creature that can hear you speak must make a Will save (DC 10 + \half your character level + your Cha modifier) or it can't attack you directly; however, you aren't protected from its area or effect spells, or similar abilities. Any creature that succeeds on its save is immune to this ability for 24 hours. You may use nonattack spells or otherwise act, but if you attack the creature or its allies, it may attack you. This is a mind-affecting, language-dependent charm effect.}
{You can fascinate creatures with your silver tongue. You can affect as many HD of creatures as your bonus on Diplomacy checks; any creature that fails a Will save (DC 10 + \half your character level + your Cha modifier) becomes fascinated. If you use this ability in combat, each target gains a +2 bonus on its saving throw. If the spell affects only a single creature not in combat at the time, the saving throw has a penalty of -2. While a subject is fascinated by this spell, it reacts as though it were two steps more friendly in attitude, allowing you to make a single request of an affected creature. The request must be brief and reasonable. Even after the spell ends, the creature retains its new attitude toward you, but only with respect to that particular request. A creature that fails its saving throw does not remember that you enspelled it.}
{You can influence even hostile creatures into talking things over with you. With a DC 30 Diplomacy check, you can use a language-dependent version of \spell{charm monster} as a spell-like ability, with save DC equal to 10 + \half your character level + your Cha modifier; this is a mind-affecting charm effect.}
{You can convince an entire group of enemies to listen to you. If you succeed on a DC 40 Diplomacy check, your \spell{charm monster} ability improves to \spell{mass charm monster}, with a caster level equal to your bonus on Diplomacy checks.}
\skillfeat{Professional Luddite}{[Skill:Disable Device]}
{You've learned to break machines because you're an antitechnology fanatic -- or maybe you just work for the local protection racket.}
{You can use Disable Device on magic traps like a character with trapfinding. If you already have that ability, you gain +3 to your Disable Device checks. Disable Device is always a class skill for you.}
{You can use your Dexterity modifier instead of your Intelligence modifier for Disable Device checks. Darkness and blindness do not hinder your ability to disable devices.}
{You can reduce the amount of time required to disable a device. For each multiple of 10 you beat the required DC, you can decrease the time required from 2d4 rounds to 1d4 rounds to 1 round to a standard action to a move-equivalent action to a free action.}
{You can use Disable Device to end any persistent effect or area spell effect as if it was a magic trap, but the DC is 25 + twice the spell's level.}
{As an attack action, you can disable magic items. You must succeed on a melee touch attack roll for attended objects. Make a Disable Device check against a DC of 15 + the item's caster level: if your check succeeds, the item must make a Will save against a DC of 10 + \half your character level or be turned into a normal item, and even if it saves, its magical properties are suppressed for 1d4 rounds.}
\skillfeat{Sharp-Eyed}{[Skill:Spot]}
{Nothing escapes your sight.}
{You gain a +3 bonus to Spot checks.}
{You can make a Spot check once a round as a free action. You don't take penalties for distractions on your Spot checks.}
{As a move action, you can make a Spot check against a DC of an opponent's Armor Class: if you succeed, you can ignore their Armor and Natural Armor bonus to AC for the next attack you make against them. If you accept a -20 penalty to your check, you can attempt this check as a swift action. Divide any distance penalties you take on Spot checks by two.}
{If you beat an opponent's Hide check with a Spot check at a -10 penalty, you can ignore concealment. If you beat their Hide check at a -30 penalty, you can ignore total concealment.}
{You can see through solid objects, but you take a -20 penalty on your Spot check for each 5'. Divide any distance penalties you take on Spot checks by five.}
\skillfeat{Slippery Contortionist}{[Skill:Escape Artist]}
{Your childhood nickname was ``Greasy the Pig," but now people call you ``The Great Hamster."}
{You gain +3 to your Escape Artist checks.}
{While squeezing into a space at least half as wide as your normal space, you may move your normal speed and you take no penalty to your attack rolls or AC for squeezing.}
{You can squeeze through a tight space or an extremely tight space as a full-round action, but you take a -10 penalty to your Escape Artist check. Opponents grappling you don't get positive size modifiers added to their grapple bonus when you use Escape Artist to try to break their hold.}
{If you succeed on a DC 30 Escape Artist check, you can ignore magical effects that impede movement as if you were under the effects of \spell{freedom of movement} for one round; this is not an action. You can also slip through a \spell{wall of force} or similar barrier with a DC 40 check.}
{You can make an Escape Artist check instead of a saving throw for any effect that would keep you from taking actions. (This does not help against effects that don't allow a saving throw.)}
\skillfeat{Steady Stance}{[Skill:Balance]}
{You can fight just about anywhere.}
{You gain a +3 bonus to your Balance checks.}
{If an effect would knock you prone, if you succeed on a DC 20 Balance check, you remain standing.}
{If your opponent is balancing, you gain a +3 dodge bonus to AC against their attacks unless they succeed at beating you in an opposed Balance check.}
{All Balance DCs are halved for you.}
{You never suffer any impairment or damage from anything you're standing on, whether it's molten lava, a cloud, or even another creature. Ambient conditions, such as lighting or weather, can still impair you.}
\skillfeat{Stealthy}{[Skill:Hide]}
{If someone sees you, you have to kill them.}
{You gain a +3 bonus to your Hide checks.}
{You can Hide as a free action after attacking, and snipe with melee attacks (or ranged attacks from closer than 10').}
{A constant \spell{nondetection effect} protects you and your equipment, with an effective caster level equal to your ranks in Hide.}
{You can attempt to Hide even when under direct observation, but you take the usual -20 penalty to your check.}
{Even opponents who can see you have trouble locating you. If they succeed at beating your Hide check with Spot (and thus can see you), they have a 50\% concealment miss chance when attacking you, which decreases by 5\% for each point they beat your Hide DC.}
\skillfeat{Swim Like a Fish}{[Skill:Swim]}
{You're at least as home in the water as you are on land.}
{You gain +3 to your Swim checks.}
{You gain a swim speed equal to your base land speed, with the attendant benefits. You don't take armor check penalties to your Swim checks.}
{You can breathe water, and you can attack through water as if under the effects of \spell{freedom of movement}.}
{While under water, you can substitute Swim checks for Reflex saves, and you gain a +4 bonus to attack and damage rolls.}
{As a swift action, you can add your ranks in Swim as a dodge bonus to your Armor Class while under water.}
\input{feats/combat/skill/
\input{feats/combat/skill/
\input{feats/combat/skill/
\input{feats/combat/skill/
\input{feats/combat/skill/

	\end{multicols}

\section{[Celestial] and [Fiend] Feats} \label{feats:outsider}

A feat with the [Celestial] or [Fiend] tag can only be taken by a creature who is an Outsider. For this purpose, any creature from any upper plane is a Celestial regardless of its alignment, while any creature from any lower plane is a Fiend regardless of its alignment. Further, any elemental or outsider with a Good alignment is a Celestial regardless of its plane of origin., while any elemental or outsider with an Evil alignment is a Fiend regardless of its plane of origin. The abilities granted by feats with the [Celestial] or [Fiend] tag are Extraordinary abilities unless otherwise stated. A Celestial does not gain a Fiendish trait from taking a [Celestial] feat that also has the [Fiend] tag.

	\begin{multicols}{2}

	\descfeat{Apprenticeship}
{New Mentor Types for the Lower Planes:}
While the race of a mentor is usually irrelevant, some mentors draw their knowledge and experience solely from their racial heritage and the magical radiations of their home plane. To choose one of these extraplanar mentors, the character must have at least 2 ranks in Knowledge (planes).\vspace*{\baselineskip}

	\shortability{Devil:}{A devil mentor is a powerful baatezu from the Nine Hells of Baator that has decided to share its knowledge with a worthy apprentice. An apprentice of this mentor gains an innate understanding of infernal contracts, and may use Knowledge(planes) to influence the attitude of any native of a plane that is aligned to law and evil, or any subject of a calling spell.}
	 \vspace*{-\baselineskip}\listone \item Knowledge (Planes) \item Knowledge (any one)\end{list} \vspace*{\baselineskip}

	\shortability{Demon:}{A demon mentor is a powerful tanar'ri from the Abyss, and it has forced his apprentices to hide a portion of his power. Once per month, you may use one of its spell-like abilities of a 2nd level effect or less.}
	 \vspace*{-\baselineskip}\listone \item Bluff \item Knowledge (Planes)\end{list} \vspace*{\baselineskip}

	\shortability{Yugoloth:}{While utterly evil, Yugoloth mentors are honorable in their own way and have been known to train apprentices in the dark arts. Apprentices of these fiends learn the true nature of evil, and may choose to count as evil for the prerequisites of feats or prestige classes, and for magical effects like spells or magic items.}
	 \vspace*{-\baselineskip}\listone \item Knowledge (Planes) \item Diplomacy\end{list} \vspace*{\baselineskip}

	\shortability{Demondand:}{A demondand mentor is a powerful fiend from Carceri. An apprentice of this mentor learns the arts of punishment at the hands of these extraplanar jailors, and may use Intimidate to influence a creature's attitude to Helpful by accepting a +10 to the DC of the check.}
	 \vspace*{-\baselineskip}\listone \item Knowledge (Planes) \item Intimidate\end{list} \vspace*{\baselineskip}
\descfeat{Attune Domain}{You incorporate the workings of a divine domain into your magic.}
	\shortability{Prerequisite:}{Caster level 1+, Must follow a god or philosophy consistent with the chosen domain.}
	\shortability{Benefit:}{Choose a domain when this feat is selected. Every spell from that domain is considered to be on your spell-list for any spellcasting classes you happen to have. These spells are considered to be spells known at the level they appear in the chosen domain. These spells are cast (and prepared, if appropriate) as normal for your class.}
	\shortability{Special:}{You may select this feat multiple times, its effects do not stack. Each time you may select a new domain, so long as your chosen god or philosophy can incorporate all of them. As usual, your DM must approve any god or philosophy. You may not have more than three attuned domains or spheres together.}
\descfeat{Attune Sphere}{You incorporate the workings of a sphere into your magic.}
	\shortability{Prerequisite:}{Caster level 1+, Must have bled from a wound inflicted by an outsider with access to the chosen sphere.}
	\shortability{Benefit:}{Choose a sphere when this feat is selected. Every spell from that sphere is considered to be on your spell-list for any spellcasting classes you happen to have. You are considered to know each of those spells at the level they appear in the chosen sphere. These spells are cast (and prepared, if appropriate) as normal for your class.}
	\shortability{Special:}{You may select this feat multiple times, its effects do not stack. Each time you may select a new sphere. You may not have more than three attuned domains or spheres together.}
\descfeat{Blood War Sorcerer}
{As a battle magician in the Blood War, you've learned killing arts that would amaze common spellcasters.}
	\shortability{Prerequisite:}{Blood War Squaddie, Caster level 5, must have fought in the Blood war for one year.}
	\shortability{Benefit:}{Each time one of your spells successfully damages a creature with Spell Resistance, they take a cumulative -1 penalty to SR. This penalty is reduced by 5 for every day of rest, and can be otherwise healed as ability damage. In addition, you may cast any spell that requires you to be of a fiendish race.}
\descfeat{Blood War Squaddie}
	{Due to your time during the Blood War, you've been tainted, honed, and hardened by the horrors you've seen.}
	\shortability{Prerequisite:}{Knowledge(planes) 2, must have fought in the Blood War for one year.}
	\shortability{Benefit:}{You are immune to fear, and actually gain a +2 Morale bonus to hit, damage, and saves when exposed to an enemy's fear effect (this bonus lasts one minute). In addition, you may treat any fiendish Exotic weapons as martial weapons.}
	\shortability{Special:}{This can only be taken at 1st level.}

	\begin{small}
	\begin{tabular}{ll}
	Spell   &CR\\
	Summon Monster I    &2\\
	Summon Monster II   &4\\
	Summon Monster III  &6\\
	Summon Monster IV   &8\\
	Summon Monster V    &10\\
	Summon Monster VI   &12\\
	Summon Monster VII  &14\\
	Summon Monster VIII &16\\
	Summon Monster IX   &18\\
	\end{tabular}
	\end{small}
\featname{Breath Weapon [Celestial], [Fiend]}
	\shortability{Prerequisites:}{Character level 6.}
	\shortability{Benefits:}{Choose a spell-like ability you possess with a duration of Instantaneous: this ability can be used as a Supernatural Breath Weapon with an area equal to a 10' per spell level of the spell-like ability used. Each use of this ability expends one use of the spell-like ability. Each time this breath weapon is used, it cannot be used again for 1d4 rounds.}
\descfeat{Broker of the Infernal}{Dues to the study of the Infernal laws, you have learned to harness the powers of True Names in your summoning magic.}
	\shortability{Prerequisites:}{Knowledge (Planes) 10, must be able to cast a spell of the [calling] subtype}
	\shortability{Benefits:}{When you possess the True Name of a creature, you may summon it with a Summon Monster spell. The version of the summon monster spell used must equal half their CR, as shown below. For all effects, this spell is a summoning spell, and functions as if the creature were a summoned monster, but if killed the creature is dead as normal and cannot be summoned again until it is returned from the dead.}
\descfeat{Carrier [Fiend]}
{You are a carrier of a dangerous disease, though you are immune to its effects}
	\shortability{Prerequisite:}{Must have one level of a Fiend class.}
	\shortability{Benefit:}{When you gain this ability choose a disease with a DC equal to the DC your disease would have(Half HD + Con mod). You disease does ability damage or special effects equal to the disease chosen. Once chosen, your disease type does not change, but your disease DC will increase when your HD or Con modifier increase. Unlike a normal disease, this is a supernatural disease, and its initial effects occur immediately.}
\descfeat{Constricting Fiend [Fiend]}
{Your legs merge into a long tail, and you gain the ability to squeeze the life from your foes.}
	\shortability{Prerequisites:}{Character level 6.}
	\shortability{Benefits:}{On a successful Grapple check, you can choose to do a 4d6 Constricting attack as a normal attack. Due to you change in form and body type, you can only use nonstandard-sized armor.}
\descfeat{Craft of the Soulstealer}
{By studying stolen souls, you have learned to fully tap their power for your magical creations.}
	\shortability{Prerequisites:}{Three or more item creation feats, caster level 6.}
	\shortability{Benefits:}{When creating magic items, you can bind a soul into the item by adding the actual receptacle of the soul into the item. In many cases, this is a gemstone that is added as decoration. A single soul is worth GP equal to its CR square, times 100 for magic item creation purposes, and is worth 1/5th of that value in XP. Only one soul may be added to an item, and any extra gold or XP provided by the soul above the cost of the item is wasted. Also, if the creature whose soul was taken had spell-like abilities, these spells may be used as prerequisites for the item's creation. Any item created by this art radiates the alignment of the soul inside the item, and it also radiates strong evil. If the receptacle containing the soul is removed from the item, the item is destroyed and the soul is released.}
\descfeat{Devour the Soul [Fiend]}
{As a fiend, you gain nourishment from devouring souls.}
	\shortability{Prerequisite:}{Must have one level of a Fiend class.}
	\shortability{Benefit:}{Each time a soul is consumed (either a receptacle or petitioner), you regain HPs equal to 10 times its CR, and heal ability damage or drain equal to its CR. Souls eaten in this fashion cannot be restored from the dead until you are killed.}
\descfeat{Dominions of the Infernal}
{When you call, armies of those you have defeated are forced to answer in service.}
	\shortability{Prerequisite:}{Must have the signature \spell{summon} ability of the great Fiendish Houses; must have a Leadership score.}
	\shortability{Benefit:}{If you successfully \spell{summon} a fiend with a CR less than your level, more than one creature may appear. The weaker the creatures are, the more are summoned.}

	\begin{small}
	\begin{tabular}{ll}%
	CR  &Number Appearing\\
	Level-2   &d2\\
	Level-3   &d3\\
	Level-4   &d4\\
	Level-5   &d6\\
	Level-6   &d8\\
	Level-7   &2d6\\
	Level-8   &2d10\\
	Level-9   &3d10\\
	Level-10  &7d6\\
	Level-11  &3d20\\
	Level-12  &7d12\\
	Level-13  &d100\\
	Level-14  &6d20\\
	Level-15  &25d6\\
	Level-16  &10d20\\
	Level-17  &40d6\\
	Level-18  &60d6\\
	Level-19  &80d6\\
	\end{tabular}
	\end{small}
\descfeat{Elemental Aura [Celestial], [Fiend]}
{Your close relationship with primal elemental forces has manifested in a damaging aura.}
	\shortability{Prerequisites:}{Character level 7, must have a subtype granting immunity to a form of elemental damage.}
	\shortability{Benefits:}{Choose one of your elemental subtypes granting immunity to a form of elemental damage. You radiate a damaging aura that does 4d6 of elemental damage of that type to any creature within 10' of you at the beginning of your turn.}
\descfeat{Essence Gourmand [Fiend]}
{Even among soul-eating fiends, you are a accomplished eater.}
	\shortability{Prerequisite:}{Must have one level of a Fiend class, Devour the Soul.}
	\shortability{Benefit:}{Whenever you devour a soul, you gain knowledge of your victim's personal history and important memories (not skills, levels, feats, etc), in addition to the normal effects. You also may cure any one status effect.}
\descfeat{Extra Arms [Celestial], [Fiend]}
{You have more arms than normal.}
	\shortability{Prerequisite:}{Character level 6 (per extra pair).}
	\shortability{Benefit:}{You have two extra humanoid arms. Each arm has your full strength and dexterity.}
	\shortability{Special:}{You may take this feat more than once, its effects stack. You must have a minimum of 6 levels for each iteration of this feat (so a 12th level character may have 2 sets of extra arms).}
\descfeat{Extra Summons [Celestial], [Fiend]}
{You may use your Summoning ability two extra times each day}
	\shortability{Prerequisite:}{Must have the signature \spell{summon} ability of the great Angelic or Fiendish Houses.}
	\shortability{Benefit:}{Your \spell{summon} ability may be used two extra times each day (the ability is normally usable once each day, so it could be used for 3 separate chances to conjure an ally.)}
\descfeat{Fiend Cabalist}
{You were trained in the mystic arts by a powerful fiend, and your magical power stems from a dark source.}
	\shortability{Prerequisite:}{Caster level 1}
	\shortability{Benefit:}{You gain Knowledge(planes) as a class skill, and all of your spells gain the [evil] subtype, and spells you cast and magic items you create radiate an evil aura equal to the strength of their normal magical aura. Any spells you cast that already have the Evil subtype gain a +4 caster level.}
	\shortability{Special:}{This can only be taken at 1st level.}
\descfeat{Fiendish Invisibility [Fiend]}
{You cannot be seen.}
	\shortability{Prerequisite:}{Character level 6}
	\shortability{Benefit:}{You are naturally invisible, as with the spell improved invisibility.}
	\shortability{Special:}{Fiendish Invisibility always has a flaw, something that will allow your character to be seen. Examples include:} \vspace*{-\baselineskip}
	\listone
	    \item\ability{Invisible in Light:}{If you are ever in shadowy illumination, you are visible.}
	    \item\ability{Visible by Breath:}{You are only invisible if you hold your breath for 3 rounds first. When you next exhale, you become visible again.}
	    \item\ability{Invisible on Stone:}{Your character is invisible when touching the ground. While standing on worked floors or flying, you can be seen.}
	\end{list}\vspace{\baselineskip}
\descfeat{Greater Teleport [Celestial], [Monstrous]}
{The extraplanar blood running through your veins allows you to use the signature travel methods of the outer planes.}
	\shortability{Prerequisite:}{Outsider, character level 5+}
	\shortability{Benefit:}{You may use \spell{greater teleport} at will as a spell-like ability. You may only transport yourself and 50 pounds of carried items.}
\descfeat{Harmless Form [Celestial], [Fiend]}
{You can assume the likeness of a mortal.}
	\shortability{Prerequisites:}{Character level 4}
	\shortability{Benefits:}{You can Change Shape into a medium-sized Humanoid appearance. You can use this ability to Disguise yourself as other people, and it gives a +10 to Disguise checks as normal. When using this ability, your reflection in mirrors is of your true form.}
\descfeat{Heighten Spell-like Ability [Celestial], [Fiend]}
{You can treat your spell-like abilities as more powerful spell effects.}
	\shortability{Benefit:}{When you use a spell-like ability, you may use it as if it were of a higher than normal spell level. You may not raise a spell-like ability to in this fashion to an effective spell-level higher than half your character level.}
\descfeat{Hellscarred}
{Having had your mind or body twisted by the essence of a fiend, you have gained some sensitivity and immunity to their power.}
	\shortability{Prerequisites:}{Must have failed a saving throw to a spell or effect associated with a fiend, and cannot be a fiend or have any feats with the [Fiend] subtype.}
	\shortability{Benefits:}{If you ever fail a save to a Special attack, Special Quality, or spell of a fiend, you may a reroll that save (once per save). This ability may be used a number of times per day equal to your Charisma modifier (minimum 1).
	In addition, you may cast detect fiends as a spell like ability at will. This spell functions as enlarged detect magic (120 foot range), but it only detects the presence of fiends and their magical effects. This effect also cannot determine school of magic, but instead will indicate the race of the fiend (baazetu, tanar'ri, yugoloth, or demondand subtype if appropriate).}
\descfeat{Huge Size [Celestial], [Fiend]}
{Your size increases to Huge.}
	\shortability{Prerequisites:}{Character level 10}
	\shortability{Benefits:}{If your size would normally be Large without this feat, it increases to Huge (with all the usual changes).}
\featname{Large Size [Elemental]}
As the Outsider feat of the same name, except that, optionally, two points of the Strength boost and one point of the Natural Armor increase for the size increase may be traded to remove the reduction to Dexterity. You may do this trade twice, for a final total of STR +4, Dex +2, Natural Armor +0.



\descfeat{Memories of Death}
{You retain your memories perfectly after you are slain and brought back from the dead.}
	\shortability{Prerequisite:}{Must be a native to the Prime Material Plane}
	\shortability{Benefit:}{When you die and are returned back from the dead by any means, you do not lose a level, any XP, or Constitution. Any other penalties associated with returning to life (such as being exhausted or waking up in a new body) are unchanged. Note that this means that you have flawless intelligence as to the alignment of whoever brought you back from the dead.}
\descfeat{Pincers [Fiend]}
{Two of your hands are converted into pincers.}
	\shortability{Benefit:}{Each Pincer is a natural weapon, and attacks made with the Pincer are considered to have the Improved Grab ability.}
\descfeat{Poison Sacs [Fiend]}
{One of your natural weapons is envenomed.}
	\shortability{Prerequisite:}{Must have one level of a Fiend class.}
	\shortability{Benefit:}{When you gain this ability choose any poison in Dungeons and Dragons with a DC equal or less to the DC your poison would have(Half HD + Con mod). You poison does ability damage or special effects equal to the poison chosen. Once chosen, your poison type does not change, but your poison DC will increase when your HD or Con modifier increase.}
\descfeat{Product of Celestial Dalliance}
{One of your recent ancestors was a Celestial Outsider or from a good-aligned plane. Maybe your parents play it off as a virgin birth, maybe your dad became a Saint.}
	\shortability{Benefits:}{You may take any [Celestial] feat. Additionally, you gain Resistance 5 to Acid, Cold, and  Electricity; the [Angel], [Archon], [Eladrin], or [Guardinal] subtype; and a Smite Evil attack usable at will that does bonus damage equal to \half of your strength modifier.}
	\shortability{Special:}{Can only be taken at 1st level.}
\descfeat{Product of Infernal Dalliance}
{One of your recent ancestors mated with an infernal creature, and now the tainted blood of a Lower Planar creature flows in your veins. Though you can resist the call of your evil heritage, it manifests itself in an inheritance of fiendish power.}
	\shortability{Benefits:}{You may take any feat with the [Fiend] subtype. In addition, you radiate faint evil, have either two claws or one bite natural weapon, and have Cold Resistance 5 or Fire Resistance 5. When this feat is gained, you also gain the [Baazetu], [Tanar'ri], [Yugoloth], or [Demondand] subtype.}
	\shortability{Special:}{Can only be taken at 1st level.}
\descfeat{Slime Trail [Fiend]}
{Your body secretes a slick mucus that dries quickly in contact with air, but you've learned to use this to your advantage.}
	\shortability{Prerequisites:}{Character level 2.}
	\shortability{Benefits:}{Your square counts as if the spell grease has been cast in it, and this effect ends when you leave a square and renews itself at the end of your turn. You are immune to this grease effect. You also gain a +4 bonus any checks to escape a Grapple.}
\descfeat{Sphere Focus [Monstrous]}{You can draw on the power of a specific Sphere more easily.}
	\shortability{Prerequisite:}{Access to at least one Sphere}
	\shortability{Benefit:}{Select a Sphere that you know.  The DC of any saving throw against spell-like abilities from that Sphere increases by 1.}
	\shortability{Special:}{You may select this feat multiple times.  Its effects do not stack.  Each time you take this feat, it applies to a different Sphere.}
\descfeat{Spines of Fury [Fiend]}
{Spines cover your body, and you may fire these spine at your enemies.}
	\shortability{Prerequisites:}{Character level 3.}
	\shortability{Benefits:}{You may fire up to two of your body's protruding spines per round as a standard action. You are proficient in these spines, and they have the same game effects as daggers. You may also remove them and use them as daggers, and they count as your natural weapons for purposes of damage reduction and spell effects. You body has a number of spines equal to twice your character level, and regenerate these amounts after one day of rest.}
\descfeat{Sting of the Scorpion [Fiend]}
{You have a viciously barbed tail that carries a lethal poison.}
	\shortability{Benefit:}{You have a stinger as a natural weapon that carries a poison that inflicts initial and secondary damage of 1d6 Con. The save DC is Constitution based. You may only inject a number of doses of poison per day equal to your Con bonus.}

\featname{Stolen Breath [Fiend]}
	\shortability{Prerequisites:}{Character level 3.}
	\shortability{Benefits:}{On a successful grapple check, your opponent may not speak or breathe for one round in addition to any normal effects of a successful Grapple check.}
\descfeat{Stoning Gaze [Fiend]}
{Your gaze petrifies the living and leaves them as statue to decorate your domain as a warning to others.}
	\shortability{Prerequisite:}{Character level 9}
	\shortability{Benefit:}{Once per round as a Free Action, you must designate one living creature within 60 feet of you. If that creature meets your gaze before your next turn, it must make a Fortitude save or be permanently transformed into Stone as by a stone to flesh spell. The save DC is Charisma based. The effects of this feat are a Supernatural Ability.}

\descfeat{Supernatural Virulence [Fiend]}
{Your poison is as much magical as it is biological.}
	\shortability{Prerequisite:}{Must have a poisonous natural weapon.}
	\shortability{Benefit:}{Choose one of your spell-like abilities of 3rd level or lower. Any time you successfully poison a victim, they are also targeted by this spell-like ability as if this effect was cast (this expends one use of the ability). While poisoned with your venom, the victim cannot be affected by your spell-like ability again.}
\descfeat{Wings of Evil [Fiend]}
{You have sinister bat-like wings growing from your back.}
	\shortability{Prerequisite:}{Character level 5.}
	\shortability{Benefit:}{You have a fly speed double that of your normal ground speed. You have good maneuverability, and you must be able to flap your wings to stay aloft (meaning that it requires very specialized armor or cloaks to permit flight).}
	\shortability{Special:}{If you would prefer to have insectile wings or feathered wings instead, you can do that. The maneuverability and speed are unchanged. Once the look and feel of the wings is selected it cannot be changed.}
\featname{Wings of Good [Celestial]}
	\shortability{Benefits:}{You gain wings and a fly speed equal to double your base land speed with good maneuverability. You must be able to flap your wings to stay aloft (meaning that it requires very specialized armor or cloaks to permit flight). These must be feathery or energy-based.}

	\end{multicols}

\section{[Elemental] Feats} \label{feats:elemental}

A feat with the [Elemental] tag can only be taken by a creature who is from an inner plane or any Elemental or Outsider with an elemental subtype. If the feat has another similar tag (such as [Celestial], [Fey], or [Fiend]), a creature who fulfills the criteria for the other tag may waive this requirement. The abilities granted by feats with the [Elemental] tag are Extraordinary abilities unless otherwise stated.

	\begin{multicols}{2}

		\descfeat{Abode of Earth [Elemental]}{You are at home within the earth.}
\shortability{Prerequisites:}{(Earth) subtype, Burrow speed, Character level 3+}
\shortability{Benefits:}{Your Burrow speed improves by 10' or to a minimum of 30', and you may burrow through rock.  You may leave a tunnel or leave the earth behind you undisturbed, as you choose.  If you leave the earth undisturbed, there is no sign of your passage unless you are in a square adjacent to a surface, except to creatures with Tremorsense or who make the Perception check to hear you.  The Perception check is not made more difficult by the earth you are in, just by distance through it.  Other rocks and earthen walls do interfere as normal.}



\descfeat{Adept Flyer [Elemental]}{You are a natural flyer.}
\shortability{Prerequisites:}{(Air) subtype, Fly speed, Character level 5+}
\shortability{Benefits:}{Your Fly speed improves to twice your base land speed (minimum 60'), or +10', whichever is more, and your maneuverability improves to Perfect.  Your Fly speed improves by 20' for every five character levels you gain beyond 5th.}



\descfeat{Binding Growth [Elemental]}{You grow on people.}
\shortability{Prerequisite:}{Wood Elemental Creature (e.g., Psuedoelemental Being (Wood) feat)}
\shortability{Benefit:}{After pinning or lifting a creature for a round, you may attempt to grow a Binding Growth on them with another grapple check against a DC of 10 + their Grapple modifier.  Once you do so, they are bound, losing their Dexterity bonus to AC and their ability to take physical actions other than try to escape, until they break the bonds.  The bonds can be broken by others with a slashing melee weapon capable of doing 5 + your hit dice points of damage against AC 5+your Constitution modifier, but a miss hurts the bound creature, or by a Strength check (DC 15 + your Constitution modifier) or Escape Artist check (DC 10 + your hit dice + your Constitution modifier).  Even once broken, they remain on for 1d4 rounds, entangling the bound creature.}



\descfeat{Ice Trail [Elemental]}{You leave a trail of ice wherever you go.}
\shortability{Prerequisite:}{Character Level 3+, (Cold) Subtype}
\shortability{Benefit:}{Your square counts as if it had the \emph{Grease} spell cast on it, except that the slick is made of ice and has the (Cold) descriptor.  Any square you leave has this effect on it, lasting until the end of your next turn.  You never slip on ice, making you immune to this effect.}



\descfeat{Infusion of Elemental Essence}{You have been infused with the power of one of the elemental planes, granting you an affinity for that element and a small degree of magical power.}
\shortability{Benefit:}{You may take any feat with the [Elemental] subtype that you qualify for; additionally, choose an elemental subtype (Air, Earth, Fire, or Water), and you may take [Elemental] feats as though you had that subtype. You also gain Resistance 10 to an energy type dependent on your element:

\begin{list}{}{\itemspace}
    \item Air or Earth: Acid or Electricity
    \item Fire: Fire
    \item Water: Acid or Cold
\end{list}
You may select this feat only once.}



\featname{Large Size [Elemental]}
As the Outsider feat of the same name, except that, optionally, two points of the Strength boost and one point of the Natural Armor increase for the size increase may be traded to remove the reduction to Dexterity. You may do this trade twice, for a final total of STR +4, Dex +2, Natural Armor +0.



\descfeat{Primal Armor [Elemental]}{Your body deflects blows off of itself.}
\shortability{Benefit:}{You gain impenetrable Damage Reduction equal to half your character level, rounded up (1/- at first level, 2/- at 3rd, 5/- at 9th, 10/- at 19th).}



\descfeat{Primal Fortification [Elemental], [Racial]}{Your body has become even more impenetrable.}
\shortability{Prerequisite:}{Elemental-Bodied, Hardiness of the Elements}
\shortability{Benefit:}{You gain immunity to Critical Hits.  You also cannot be flanked, as your undiffentiated body has no clear front or back.}



\descfeat{Psuedoelemental Being [Racial]}{You are a psuedoelemental being, with rare and unique powers.}
\shortability{Prerequisite:}{Elemental-Bodied.}
\shortability{Benefits:}{Instead of picking a normal elemental type as an elemental-bodied, select one of the following other planes: Ice, Magma, Shadow, or Wood.  You gain benefits as follows for the type you've picked:
\begin{list}{}{\itemspace}
    \item Ice: You gain the (Cold) subtype, a 30' base land speed, a 30' swim speed, and +2 to Str.  Your melee attacks do 1d6 bonus Cold damage.  You speak Aquan and Auran.
    \item Magma: You gain both the (Earth) and (Fire) subtypes. Your base land speed is 20', and you gain +2 Str. Otherwise you gain the full benefits of both elements.
    \item Shadow: You have a 30' base land speed and a Fly speed of 10', with good maneuverability, and gain +2 Dex. You are \emph{invisible} in any lighting less than bright light. You speak Common. Despite your affiliation to the Plane of Shadow instead of to the Inner Planes, you still qualify for [Elemental] feats.
    \item Wood: You have no elemental subtype, and gain +2 Con and a 10' Climb speed. You gain Regeneration 0, penetrable by Fire and Slashing weapons, which improves to Regeneration equal to your level in areas of natural daylight or equivalent brightness (such as a \emph{daylight} spell). You only gain natural healing if you spend at least 8 hours/day in such brightness. You count as a Plant, in addition to an Elemental, for all effects relating to type. You speak your choice of Sylvan or Treant, and any Elemental language.
\end{list}
Other Dual-element types than Magma, such as Ooze (Water and Earth), Smoke (Air and Fire), Vapor (Water and Air), and so on are possible.}
\shortability{Special:}{This feat can only be taken at 1st level.}



\featname{Stolen Breath [Fiend]}
	\shortability{Prerequisites:}{Character level 3.}
	\shortability{Benefits:}{On a successful grapple check, your opponent may not speak or breathe for one round in addition to any normal effects of a successful Grapple check.}
\descfeat{Tremorsense [Elemental]}{Your close connection to your home element gives you Tremorsense.}
\shortability{Prerequisites:}{(Earth) or (Water) Subtype, Character level 6+}
\shortability{Benefits:}{You gain Tremorsense out to 120'.  You gain Blindsight out to 30' against any creature you can Tremorsense.  If you have the (Water) subtype and not the (Earth) subtype, your Tremorsense works at its full range in liquids, but only to half range and you do not gain Blindsight through solids.}



\descfeat{Touch of Shadow [Elemental]}{Your shadowy touch can bypass armor.}
\shortability{Prerequisite:}{Shadow Elemental Creature (Psuedoelemental Being (Shadow), Shadow Genasi, or similar), Natural weapon, Character Level 3+}
\shortability{Benefit:}{You may choose to make natural weapon attacks as touch attacks.  Such attacks use your Dexterity bonus to hit, instead of Strength, and do not gain Strength to damage.}



\descfeat{Uncanny Flexibility [Elemental], [Racial]}{Your body, being made of a material other than flesh, bends in directions and places that flesh neither can  nor should.}
\shortability{Prerequisite:}{Airbodied, Firebodied, or Waterbodied; or Psuedoelemental Being (Magma or Shadow).}
\shortability{Benefit:}{You gain a +4 bonus to Escape Artist checks, and Escape Artist is always a class skill for you.  You can compress your head to about half area for purposes of slipping through tight spaces, and may attempt to slip manacles, ropes, webs, nets, grapplers, and similar bonds as a free action.}



\descfeat{Unstoppable Force [Elemental], [Racial]}{You cannot be stopped.}
\shortability{Prerequisites:}{Elemental-Bodied, Hardiness of the Elements}
\shortability{Benefit:}{You become immune to paralysis and stunning.}

%\end{multicols}


	\end{multicols}

\section{[Necromantic] Feats} \label{feats:necromantic}

\bolded{Necromantic Creation Feats:}

Any feat with the [Necromantic] tag is a necromantic creation feat. This means that it is merely one part of the dark tradition of necromancy; other means such as necromancy spells or other effects can create these undead, but this an easy path for the serious Necromancer. One trait shared by these feats is that each feat has a separate control pool for the undead it creates. For example, if a necromancer has the Path of Blood feat and the A Feast Unknown feat, he may control up to his unmodified charisma modifier in vampires or vampire spawn in addition to controlling up to his unmodified charisma modifier in ghouls. It is a move action to give commands any one undead creature. Any undead controlled by this feat cannot create undead or use the Spawn Undead ability.

The rituals are inexpensive, but require the flesh and blood of intelligent creatures as well as living creatures or fresh corpses as subjects. Any additional costs or conditions are listed in the individual feat. These rites take 1 hour per CR of the creature created, and can only be performed at night or in a location that has never been touched by the sun (such as a deep cave). The maximum CR of an undead creature created with these rites is two less than the creator's Character level.

Materials to create any undead always cost at least 25 gp per hit die. Creating undead by these method generally requires at least an hour.\\


	\begin{multicols}{2}

		\descfeat{A Feast Unknown [Necromantic]}
{You have partaken of the feast most foul and count yourself a king among the ghouls.}
\shortability{Prerequisite:}{You must have consumed the rapidly cooling flesh of an intelligent mortal creature. Must be evil.}
\shortability{Benefit:}{You can create Ghouls or Ghasts from any dying person (at -1 to -9 hps). Any undead you create have the Scent special quality. In addition, any time you completely consume the flesh of a sentient creature, you regain 5 hps per HD.\\
You automatically control up to your Charisma modifier in undead created by this feat, but no undead can have a CR greater than two less your Character level.}


%\descfeat{Blood Painter}
{By painting magical diagrams out of your own blood, you can spontaneously cast spells using only your own life energy. This is especial use to casters who prepare spells, or to casters who have run out of spells.}
\shortability{Prerequisite:}{Path of Blood, Caster level 5, Spellcraft 4 ranks}
\shortability{Benefit:}{At any time, a caster with this feat can cast any spell he knows by painting a magical diagram on a flat 10' by 10' surface. This takes one minute per spell level, and deals two points of Constitution damage per spell level to the caster (or loses a like amount of Blood Pool if he has one). If the caster's current Con or Blood Pool is less than double the spell's level, the spell cannot be cast.\\
Any spells cast with this feat are Supernatural effects.}


%\descfeat{Body Assemblage [Necromantic]}
{The discarded husks of life are nothing more than a building material to you.}
\shortability{Prerequisite:}{Caster Level 1, ability to cast 1st level spells of the Necromancy school.}
\shortability{Benefit:}{You may create skeletons and zombies that serve you alone.  You automatically control up to your unmodified Charisma modifier in undead created by this feat, but no undead can have a CR greater than two less than your Character level.}
\shortability{Special:}{A first or second level character can create undead less than their own CR, but each undead creature counts as two for control purposes.}


%\descfeat{Boneblade Master}
{You have mastered the alchemic processes needed to create boneblades, as well as their use in combat.}
\shortability{Prerequisite:}{Craft(alchemy) 4, Craft 4 (scrimshaw)}
\shortability{Benefit:}{You are considered to be proficient in the use of any weapon made from the special material Boneblade, and you may craft weapons out of Boneblade. In addition, you are considered to have the Improved Critical feat for any boneblade weapons you use in melee combat.\\
You also gain a +2 to Initiative checks.}


%\descfeat{Child Necromancer}
{An obsession with death and experimentation with necromancy early in your childhood perverted your body and blossoming magical talent. As a result, your body never aged past childhood, and you are an adult in a child's body, magically powerful but physically weak.}
\shortability{Prerequisite:}{Caster level 1, must know at least one necromancy spell of each spell level you can cast.}
\shortability{Benefit:}{All Necromancy spells you cast are at +4 caster level, and you gain the effects of \hyperlink{feat:weaponfinesse}{Weapon Finesse} for all Necromancy touch attack spells you use (if you desire). You have -4 Strength, and appear to be a child despite your actual age category (this does not prevent penalties or bonuses from advancing in age categories, or stop the aging process). You are one size category smaller than normal for your race (do not further adjust ability modifiers). If you are a spontaneous caster, you may permanently exchange any spell known for any Necromancy spell you possess in written or scroll form. If you are a preparation caster, you may learn any Necromancy spell you possess in written or scroll form from any list, and you may not select Necromancy as a restricted school. These Necromancy spells may be from any list, can be exchanged at any time, and once gained are cast as spells of your spellcasting class. These spells remain as spells known even if you later lose this feat.}
\shortability{Special:}{This feat can only be taken at 1st level. If circumstances ever cause a character to no longer meet the prerequisites of this feat, they may choose any metamagic feat they qualify for to permanently replace this feat.}


%\descfeat{Devil Preparation}
{By learning dark culinary techniques, you have learned to consume the flesh of devils, demons, and other infernals, absorbing their taint and some of their power.}
\shortability{Prerequisite:}{A Feast Unknown, Character level 10, must have eaten the flesh of a Devil or Demon.}
\shortability{Benefit:}{You gain the ability to cast one spell from the Half-Fiend template per day as a spell-like ability (limited by your HD on the Half-fiend chart). In addition, all spells from the Evil Domain are considered spells known for you, you gain a +2 to Intimidate checks, and you can choose to count as a Tanari or Baazetu for the effects of spells, magic items, or prerequisites for feats or prestige classes.}


%\descfeat{Fairy Eater}
{By consuming the flesh of fairies, you have absorbed a fraction of their magic.}
\shortability{Prerequisite:}{A Feast Unknown, must have eaten the flesh of a creature with the Fey type.}
\shortability{Benefit:}{All figments and glamers you cast have their duration extended by two rounds. In addition, all spells from the Trickery Domain are considered spells known for you, you gain a +4 to Disguise checks, and you can choose to count as a Fey for the effects of spells, magic items, or prerequisites for feats or prestige classes.}


%\descfeat{Feed the Dark Gods [Necromantic]}
{You have attracted the attention of dark gods and demon lords, and they are willing to grant dark life to your creations in exchange for pain and power.}
\shortability{Prerequisite:}{Any two necromantic feats, Character level 7, 10 ranks in Knowledge(Religion)}
\shortability{Benefit:}{You may create any undead creature through the art of sacrifice. For every CR of the creature you wish to create, you must sacrifice one sentient soul (Int of 5 or better) and 500 gp. For example, if you wish to create a CR 8 Slaughterwight, you must sacrifice eight sentients and 4,000 gp. You cannot create any undead with a CR greater than two less than your Character level.\\
You automatically control up to your unmodified Charisma modifier in undead created by this feat, but no undead can have a CR greater than two less than your character level.}


%\descfeat{Ghost Cut Technique}
{Study of the ephemeral essence of incorporeal undead has taught you combat techniques that transcend the limitations of the flesh.}
\shortability{Prerequisite:}{Whispers of the Otherworld}
\shortability{Benefit:}{Each day, you can use the spell \spell{wraithstrike} as swift action spell-like ability a number of time equal to half your character level.\\
You also gain a +2 to initiative checks, a +4 to Move Silently checks, and Lifesight as a Special Quality.}


%\descfeat{Heavenly Desserts}
{By gorging on the sweet flesh of angels, you have digested a portion of their divine essence.}
\shortability{Prerequisite:}{A Feast Unknown, Character level 10, must have eaten the flesh of an Angel, Archon, Eladrin, or Deva.}
\shortability{Benefit:}{You gain the ability to cast one spell from the Half-Celestial template per day as a spell-like ability (limited by your HD on the Half-Celestial chart). In addition, all spells from the Gluttony Domain are considered spells known for you, you gain a +2 to Diplomacy checks, and you can choose to count as Good for the effects of spells or magic items.}


%\descfeat{Sleep of the Ages}
{Your mastery of ancient mummification techniques has revealed a secret technique for sleeping away the ages.}
\shortability{Prerequisite:}{Character level 8, Wrappings of the Ages, you must remove all of your internal organs and place them within canoptic jars during a magic ritual}
\shortability{Benefit:}{By arranging focuses worth 1,000 GP in a ritual manner and wrapping yourself in the funeral arrangements of a mummy, you can initiate the Sleep of the Ages. Until your focuses are disturbed, you will stay in suspended animation. In this state, you do not age, breath, need to eat, or are subject to any effect requiring a Fort Save.\\
As a side effect of learning this technique, you remove all of your internal organs and place them within canoptic jars during a magical ritual. This process does not harm you, and from this point onward you are no longer subject to critical hits or sneak attacks. Having your organs in canoptic jars has no other game effect, but if they are destroyed you no longer gain the effects of this feat (your organs magically return to your body and you must remove them again to regain the use of this feat.)}





\end{multicols}

\section{[Undead] Feats}

The powers of the undead are legendary, in part because they are so varied. A feat with the [Undead] tag can only be selected or used by a character who is undead.\\

\begin{multicols}{2}


%\descfeat{The Path of Blood [Necromantic]}
{You have learned the dark and selfish rites that create vampires, the legendary immortal blood drinkers of the night.}
\shortability{Prerequisite:}{Character level 5}
\shortability{Benefit:}{You can create Vampires and Vampire Spawn. Your unintelligent undead heal fully at the next sunset following them killing a living creature with a piercing or slashing attack. A spellcaster with this feat has access to any spell with a [blood] component. You automatically control up to your unmodified Charisma modifier in undead created by this feat, but no undead can have a CR greater than two less than your Character level.}


%\descfeat{Whispers of the Otherworld[Necromantic]}
{You have learned the tricks of torturing a soul past the veil of life, and into the shadow of death.}
\shortability{Prerequisite:}{Character level 4}
\shortability{Benefit:}{You may create incorporeal undead. In addition, any undead you create have a +2 to Initiative, +4 to Move Silently checks, and Lifesight as a Special Quality.\\
You automatically control up to your Charisma modifier in undead created by this feat, but no undead can have a CR greater than two less than your character level.}


%\descfeat{Wrappings of the Ages [Necromantic]}
{The ancient secrets by which unlife can be sustained in mummification have been unearthed.}
\shortability{Prerequisite:}{Character level 8}
\shortability{Benefit:}{You can create mummies. In addition, any undead you create has their natural armor increase by +3. Also, any time your undead rest (take no actions) in an enclosed space that has never been touched by the sun, the location counts as a Tomb for them as long as they inhabit it (see New Rules). In all other ways, the area is not a Tomb.\\
You automatically control up to your Charisma modifier in undead created by this feat, but no undead can have a CR greater than two less than your character level.}



%\descfeat{Enervating Touch [Undead]}
{Your undead nature allows you to drain the life out of living victims.}
\shortability{Benefit:}{Your unarmed attacks and natural weapons inflict one negative level. The DC to remove that negative level is Charisma based. You gain 5 temporary hit points every time you inflict a negative level on an intelligent creature in this way.}


%\descfeat{Control Spawn [Undead]}
{Your victims serve you eternally in death.}
\shortability{Prerequisite:}{Enervating Touch}
\shortability{Benefit:}{When a creature dies from the negative levels you inflict and rises as a Wight, it comes under your control. At any one time, you may control a number of Wights in this manner equal to your Charisma modifier (minimum of one). If you create additional Wights, you choose which spawn you lose control over.}


%\descfeat{Paralyzing Touch [Undead]}
{The touch of your clawed hand freezes the lifeblood of the hardiest of mortals.}
\shortability{Prerequisite:}{Ghoul}
\shortability{Benefit:}{Your unarmed strikes and natural attacks cause paralysis for one minute unless your victim makes a Fortitude save. This save is Charisma based.}

\end{multicols}



	\end{multicols}

\section{[Undead] Feats}

The powers of the undead are legendary, in part because they are so varied. A feat with the [Undead] tag can only be selected or used by a character who is undead.\\

	\begin{multicols}{2}

	\descfeat{A Feast Unknown [Necromantic]}
{You have partaken of the feast most foul and count yourself a king among the ghouls.}
\shortability{Prerequisite:}{You must have consumed the rapidly cooling flesh of an intelligent mortal creature. Must be evil.}
\shortability{Benefit:}{You can create Ghouls or Ghasts from any dying person (at -1 to -9 hps). Any undead you create have the Scent special quality. In addition, any time you completely consume the flesh of a sentient creature, you regain 5 hps per HD.\\
You automatically control up to your Charisma modifier in undead created by this feat, but no undead can have a CR greater than two less your Character level.}


%\descfeat{Blood Painter}
{By painting magical diagrams out of your own blood, you can spontaneously cast spells using only your own life energy. This is especial use to casters who prepare spells, or to casters who have run out of spells.}
\shortability{Prerequisite:}{Path of Blood, Caster level 5, Spellcraft 4 ranks}
\shortability{Benefit:}{At any time, a caster with this feat can cast any spell he knows by painting a magical diagram on a flat 10' by 10' surface. This takes one minute per spell level, and deals two points of Constitution damage per spell level to the caster (or loses a like amount of Blood Pool if he has one). If the caster's current Con or Blood Pool is less than double the spell's level, the spell cannot be cast.\\
Any spells cast with this feat are Supernatural effects.}


%\descfeat{Body Assemblage [Necromantic]}
{The discarded husks of life are nothing more than a building material to you.}
\shortability{Prerequisite:}{Caster Level 1, ability to cast 1st level spells of the Necromancy school.}
\shortability{Benefit:}{You may create skeletons and zombies that serve you alone.  You automatically control up to your unmodified Charisma modifier in undead created by this feat, but no undead can have a CR greater than two less than your Character level.}
\shortability{Special:}{A first or second level character can create undead less than their own CR, but each undead creature counts as two for control purposes.}


%\descfeat{Boneblade Master}
{You have mastered the alchemic processes needed to create boneblades, as well as their use in combat.}
\shortability{Prerequisite:}{Craft(alchemy) 4, Craft 4 (scrimshaw)}
\shortability{Benefit:}{You are considered to be proficient in the use of any weapon made from the special material Boneblade, and you may craft weapons out of Boneblade. In addition, you are considered to have the Improved Critical feat for any boneblade weapons you use in melee combat.\\
You also gain a +2 to Initiative checks.}


%\descfeat{Child Necromancer}
{An obsession with death and experimentation with necromancy early in your childhood perverted your body and blossoming magical talent. As a result, your body never aged past childhood, and you are an adult in a child's body, magically powerful but physically weak.}
\shortability{Prerequisite:}{Caster level 1, must know at least one necromancy spell of each spell level you can cast.}
\shortability{Benefit:}{All Necromancy spells you cast are at +4 caster level, and you gain the effects of \hyperlink{feat:weaponfinesse}{Weapon Finesse} for all Necromancy touch attack spells you use (if you desire). You have -4 Strength, and appear to be a child despite your actual age category (this does not prevent penalties or bonuses from advancing in age categories, or stop the aging process). You are one size category smaller than normal for your race (do not further adjust ability modifiers). If you are a spontaneous caster, you may permanently exchange any spell known for any Necromancy spell you possess in written or scroll form. If you are a preparation caster, you may learn any Necromancy spell you possess in written or scroll form from any list, and you may not select Necromancy as a restricted school. These Necromancy spells may be from any list, can be exchanged at any time, and once gained are cast as spells of your spellcasting class. These spells remain as spells known even if you later lose this feat.}
\shortability{Special:}{This feat can only be taken at 1st level. If circumstances ever cause a character to no longer meet the prerequisites of this feat, they may choose any metamagic feat they qualify for to permanently replace this feat.}


%\descfeat{Devil Preparation}
{By learning dark culinary techniques, you have learned to consume the flesh of devils, demons, and other infernals, absorbing their taint and some of their power.}
\shortability{Prerequisite:}{A Feast Unknown, Character level 10, must have eaten the flesh of a Devil or Demon.}
\shortability{Benefit:}{You gain the ability to cast one spell from the Half-Fiend template per day as a spell-like ability (limited by your HD on the Half-fiend chart). In addition, all spells from the Evil Domain are considered spells known for you, you gain a +2 to Intimidate checks, and you can choose to count as a Tanari or Baazetu for the effects of spells, magic items, or prerequisites for feats or prestige classes.}


%\descfeat{Fairy Eater}
{By consuming the flesh of fairies, you have absorbed a fraction of their magic.}
\shortability{Prerequisite:}{A Feast Unknown, must have eaten the flesh of a creature with the Fey type.}
\shortability{Benefit:}{All figments and glamers you cast have their duration extended by two rounds. In addition, all spells from the Trickery Domain are considered spells known for you, you gain a +4 to Disguise checks, and you can choose to count as a Fey for the effects of spells, magic items, or prerequisites for feats or prestige classes.}


%\descfeat{Feed the Dark Gods [Necromantic]}
{You have attracted the attention of dark gods and demon lords, and they are willing to grant dark life to your creations in exchange for pain and power.}
\shortability{Prerequisite:}{Any two necromantic feats, Character level 7, 10 ranks in Knowledge(Religion)}
\shortability{Benefit:}{You may create any undead creature through the art of sacrifice. For every CR of the creature you wish to create, you must sacrifice one sentient soul (Int of 5 or better) and 500 gp. For example, if you wish to create a CR 8 Slaughterwight, you must sacrifice eight sentients and 4,000 gp. You cannot create any undead with a CR greater than two less than your Character level.\\
You automatically control up to your unmodified Charisma modifier in undead created by this feat, but no undead can have a CR greater than two less than your character level.}


%\descfeat{Ghost Cut Technique}
{Study of the ephemeral essence of incorporeal undead has taught you combat techniques that transcend the limitations of the flesh.}
\shortability{Prerequisite:}{Whispers of the Otherworld}
\shortability{Benefit:}{Each day, you can use the spell \spell{wraithstrike} as swift action spell-like ability a number of time equal to half your character level.\\
You also gain a +2 to initiative checks, a +4 to Move Silently checks, and Lifesight as a Special Quality.}


%\descfeat{Heavenly Desserts}
{By gorging on the sweet flesh of angels, you have digested a portion of their divine essence.}
\shortability{Prerequisite:}{A Feast Unknown, Character level 10, must have eaten the flesh of an Angel, Archon, Eladrin, or Deva.}
\shortability{Benefit:}{You gain the ability to cast one spell from the Half-Celestial template per day as a spell-like ability (limited by your HD on the Half-Celestial chart). In addition, all spells from the Gluttony Domain are considered spells known for you, you gain a +2 to Diplomacy checks, and you can choose to count as Good for the effects of spells or magic items.}


%\descfeat{Sleep of the Ages}
{Your mastery of ancient mummification techniques has revealed a secret technique for sleeping away the ages.}
\shortability{Prerequisite:}{Character level 8, Wrappings of the Ages, you must remove all of your internal organs and place them within canoptic jars during a magic ritual}
\shortability{Benefit:}{By arranging focuses worth 1,000 GP in a ritual manner and wrapping yourself in the funeral arrangements of a mummy, you can initiate the Sleep of the Ages. Until your focuses are disturbed, you will stay in suspended animation. In this state, you do not age, breath, need to eat, or are subject to any effect requiring a Fort Save.\\
As a side effect of learning this technique, you remove all of your internal organs and place them within canoptic jars during a magical ritual. This process does not harm you, and from this point onward you are no longer subject to critical hits or sneak attacks. Having your organs in canoptic jars has no other game effect, but if they are destroyed you no longer gain the effects of this feat (your organs magically return to your body and you must remove them again to regain the use of this feat.)}





\end{multicols}

\section{[Undead] Feats}

The powers of the undead are legendary, in part because they are so varied. A feat with the [Undead] tag can only be selected or used by a character who is undead.\\

\begin{multicols}{2}


%\descfeat{The Path of Blood [Necromantic]}
{You have learned the dark and selfish rites that create vampires, the legendary immortal blood drinkers of the night.}
\shortability{Prerequisite:}{Character level 5}
\shortability{Benefit:}{You can create Vampires and Vampire Spawn. Your unintelligent undead heal fully at the next sunset following them killing a living creature with a piercing or slashing attack. A spellcaster with this feat has access to any spell with a [blood] component. You automatically control up to your unmodified Charisma modifier in undead created by this feat, but no undead can have a CR greater than two less than your Character level.}


%\descfeat{Whispers of the Otherworld[Necromantic]}
{You have learned the tricks of torturing a soul past the veil of life, and into the shadow of death.}
\shortability{Prerequisite:}{Character level 4}
\shortability{Benefit:}{You may create incorporeal undead. In addition, any undead you create have a +2 to Initiative, +4 to Move Silently checks, and Lifesight as a Special Quality.\\
You automatically control up to your Charisma modifier in undead created by this feat, but no undead can have a CR greater than two less than your character level.}


%\descfeat{Wrappings of the Ages [Necromantic]}
{The ancient secrets by which unlife can be sustained in mummification have been unearthed.}
\shortability{Prerequisite:}{Character level 8}
\shortability{Benefit:}{You can create mummies. In addition, any undead you create has their natural armor increase by +3. Also, any time your undead rest (take no actions) in an enclosed space that has never been touched by the sun, the location counts as a Tomb for them as long as they inhabit it (see New Rules). In all other ways, the area is not a Tomb.\\
You automatically control up to your Charisma modifier in undead created by this feat, but no undead can have a CR greater than two less than your character level.}



%\descfeat{Enervating Touch [Undead]}
{Your undead nature allows you to drain the life out of living victims.}
\shortability{Benefit:}{Your unarmed attacks and natural weapons inflict one negative level. The DC to remove that negative level is Charisma based. You gain 5 temporary hit points every time you inflict a negative level on an intelligent creature in this way.}


%\descfeat{Control Spawn [Undead]}
{Your victims serve you eternally in death.}
\shortability{Prerequisite:}{Enervating Touch}
\shortability{Benefit:}{When a creature dies from the negative levels you inflict and rises as a Wight, it comes under your control. At any one time, you may control a number of Wights in this manner equal to your Charisma modifier (minimum of one). If you create additional Wights, you choose which spawn you lose control over.}


%\descfeat{Paralyzing Touch [Undead]}
{The touch of your clawed hand freezes the lifeblood of the hardiest of mortals.}
\shortability{Prerequisite:}{Ghoul}
\shortability{Benefit:}{Your unarmed strikes and natural attacks cause paralysis for one minute unless your victim makes a Fortitude save. This save is Charisma based.}

\end{multicols}



	\end{multicols}