\section{Types of Leadership}
A character moving into a leadership position can be a natural way for a character to advance as a game progresses. There are several ways a character can do this.

\ability{Cohort}{Having a cohort is very similar to having a secondary character. You have direct control over your cohort, just as you have control over your own character. Your cohort, if you have one, always has a CR that is 2 less than your character's level. A cohort increases in power, gaining a level or hit dice, when you gain a level. You may only ever have one cohort at a time, if a source would grant you more than one cohort, you instead gain a companion. Cohorts are usually gained by taking [Leadership] feats.}

\ability{Companion}{Companions play supporting roles to your own character. You have direct control of your companions as long as they are with your character. If for some reason one of your companions is separated from your character, the companion continues to act in your best interest, typically following any orders you have given it. Your companions always have a CR equal to your level -4 or to \sfrac{1}{2} of your level (round down), whichever is greater. Companions are usually gained through class features.}

\ability{Followers}{Your followers are people that take orders from you, possibly as members of an organization you lead or have a leadership position within. Instead of keeping track of all of your followers individually, they are represented by your Leadership Score (see below). By spending points of your Leadership Score (and temporarily reducing it) you can have your followers accomplish tasks for you.}

\section{Leadership Score}

Your Leadership Score represents the amount of resources and manpower your followers can produce. Your Leadership Score has a maximum value equal to your level, and is replenished by an amount equal to your Charisma bonus each week (minimum 1). You only have one Leadership Score, if a source would grant you more than one Leadership Score, your score is instead increased by 4. Below are the tasks you can accomplish by spending your Leadership Score.

When you first gain a leadership score, you should select a location to serve as the base for your organization. The location of your base affects the DC of and the the time it takes perform certain actions.

%\begin{wraptable}{o}{.6\linewidth}
\begin{table}[h!]
\centering
\caption{Uses of Leadership}
\rowcolors{1}{colorone}{colortwo}
\begin{tabu}to \linewidth{X X}
\header Action & Leadership Cost \\ \hline
Gather Intelligence & 1\\
Labor & Varies, see text\\
Personal Retinue & Equal to EL of Followers\\
Provide Service & Varies\\
Skill Check Equivalent & 1 per +5 Bonus\\ \hline
\end{tabu}
\end{table}
%\end{wraptable}

\ability{Gather Intelligence}{You can use your followers to find out about a person place or thing. If successful you learn a single peice of information about the object of your investigation at the end of the week, and an additional peice of information for every 5 points by which your check exceeds the DC. On a failed check, you gain a piece of information that turns out to be false. The base DC for this check is 10, modified by the conditions shown on the table below. You have a bonus to the check equal to you level.
	\begin{awesomelist}
		\item \ability{Person}{A successful check indicates that you learn one thing of your choice about a person: their current location, where they plan to be during the next week, if there is currently a plot against that person, the truth about one rumor about the person.}
		\item \ability{Place}{A successful check allows you to learn one thing about the place: presence of people or monsters and a general indication of how dangerous the area is, the number of a specific group of people that you already know the presence of in the area, the presence or absence of ambush points.}
		\item \ability{Thing}{A successful check lets you learn one of the following about an object: the history of the object, the previous owner of the object, the function of the object, the activation method for the object (if magical).}
	\end{awesomelist}
	
\begin{table}[h!]
\centering
\caption{Gather Intelligence DC Modifiers}
\rowcolors{1}{colorone}{colortwo}
\begin{tabu}to \linewidth{X[3] X || X[3] X || X[3] X}
\header Person is... & DC & Place is... & DC & Thing is... & DC \\ \hline
Friendly  & -5 & Friendly Territory & -5 & In Your Posession & -5 \\
An Enemy  & +5 & Hostile Territory  & +5 & Not Specific      & +5 \\
Secretive & +5 & Remote             & +5 & Medium or Major   & +5 \\
In Hiding & +5 & Guarded            & +5 & An Artifact       & +10 \\ \hline
\end{tabu}
\end{table}
}

\ability{Labor}{You can have your followers work to build or craft something for one week. The amount of work done is based on how much of your leadership score you spend, and is equivalent to a given number of people working for one week. This cost recurs weekly if you have your followers continue to labor. If the task would normally require some sort of skill check to complete, you must arrange for that to be done separately.}

%\begin{wraptable}{o}{.4\linewidth}
\begin{table}[h!]
\centering
\caption{Cost of Labor}
\rowcolors{1}{colorone}{colortwo}
\begin{tabu}{l l}
\header Points Spent & Number of Laborers \\ \hline
1 & 1 \\
2 & 3 \\
3 & 5 \\
4 & 10 \\
5 & 25 \\
6 & 55 \\
7 & 110 \\
8 & 225 \\
9 & 450 \\
10 & 900 \\
11 & 1,800 \\
12 & 3,750 \\
13 & 7,500 \\
14 & 15,000 \\
15 & 30,000 \\
16 & 60,000 \\
17 & 125,000 \\
18 & 250,000 \\
19 & 500,000 \\
20 & 1,000,000 \\
+1 & x2 \\ \hline
\end{tabu}
\end{table}
%\end{wraptable}

\ability{Personal Retinue}{You can take followers with you on adventures. The cost to do this is equal to the encounter level (EL) of the followers you bring with you, up to a maximum of \sfrac{1}{2} of your level. The followers remain with you for one week, and the cost recurs for each week they remain with you. If some of them are killed or otherwise rendered unfit for service, a one-time penalty equal to half of the cost to bring them is incurred. If \emph{all} of them are killed or otherwise rendered unfit for service, the penalty is instead equal to the full cost to bring them. This can temporarily reduce your Leadership Score to a negative value.}

\ability{Skill Check Equivalent}{You can have your followers perform some task that is equivalent to a skill check. The skill may be any relevant skill, and the bonus is equal to your character level. For every additional point you spend, the check is made at an additional +5 bonus, up to a maximum of double your character level.}

\subsection{Followers and Location}

When you first gain your leadership score, you should select a place to serve as the base of operations for your organization. Tasks take additional time to start based on how distant the task is to take place from your base. This amount of time does not necessarily equate to the time it takes for your followers to physically reach the location, but could be the time it takes to get in touch with with local contacts or contractors.

\begin{table}[h!]
\centering
\caption{Distance and Time}
\rowcolors{1}{colorone}{colortwo}
\begin{tabu}{l l}
\header Distance to Task & Extra Time Required \\ \hline
Local & No Extra Time \\
Neighboring Province (100 mi.) & 1 Week \\
Distant Province (250 mi.) &  2 Weeks \\
Neighboring Country (500 mi.) & 1 Month \\
Distant Country (1000 mi.) & 2 Months \\
Another Continent (3000 mi) & 3 Months \\
Another Plane & 4 Months \\ \hline
\end{tabu}
\end{table}

\subsection{Rushing Your Followers}

By spending an extra 50\% of the cost of a use of your leadership (round up), you can reduce the time it takes to complete the task from a number of weeks, to an equivalent number of days.

\section{Replacing Cohorts and companions}

Sometimes followers, companions, and even cohorts might die. Or maybe the one you have just isn't cutting it anymore. Often the requirements involve some amount of in-game time passing, and this assumes that the character spends some minimal amount of time between adventures looking for suitable replacements. Unless there are extenuating circumstances, the DM should try to follow these guidelines or work out another suitable means of replacing the minion. The requirements to replace them are as follows.

\ability{Replacing a Cohort}{A cohort can be replaced any time you gain a level, or after some period of in-game time (typically a month). Your DM may also allow you to immediately replace your cohort with an appropriate NPC that is already in the campaign.}

\ability{Replacing a Companion}{Typically the steps necessary to replace a companion are given by the feat or class feature that granted it. If no method of replacement is specified, a companion can usually be replaced after one week of in-game time.}

\section{Converting Followers}

Sometimes you might want to convert a specific NPC into a follower. You can try to turn an NPC that is friendly to you with at least one minute of effort and a successful Charisma check with a DC of 10 + the NPC's CR. If the NPC is already the follower of another character the DC is increased by the Charisma bonus of the character they are already a follower of. A failure carries no penalty, but you cannot try to convert the same NPC again.