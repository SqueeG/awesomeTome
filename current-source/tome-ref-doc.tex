\documentclass[10pt,twoside,onecolumn,openany,final]{memoir}
\setstocksize{11in}{8.5in}

\usepackage[toc,lot,lof]{multitoc}
\usepackage[top=.5in, bottom=.5in, left=.75in, right=.75in]{geometry}
\usepackage{graphicx} \graphicspath{{./images/}}
\usepackage{mdwlist}
\usepackage{microtype} \DisableLigatures{encoding = *, family = *}
\usepackage{multicol}
\usepackage{textcomp}
\usepackage[normalem]{ulem}
\usepackage{wrapfig}
\usepackage{xtab}
\usepackage{enumerate}
\usepackage{phonetic}
\usepackage{bbding}
\usepackage{linearb}

\usepackage{cypriot}

\usepackage{tipa}
\usepackage{xfrac}
\usepackage{appendix}
\usepackage{xparse}
\usepackage{letltxmacro}
\usepackage{makeidx} \makeindex
\usepackage[table,dvipsnames]{xcolor}
\definecolor{offyellow}{RGB}{255,255,128}
\definecolor{links}{RGB}{200,0,50}
\usepackage{placeins}
\usepackage{floatflt}
\usepackage{anyfontsize}
\usepackage{mdframed}
\usepackage{colortbl}
\usepackage{tabularx}
\usepackage{tabu}
\usepackage{longtable}
\usepackage{afterpage}
\usepackage{caption}

%\usepackage[bf, big, raggedright]{titlesec}

\usepackage{amsmath}

%% Font
\usepackage[T1]{fontenc}
\usepackage[bitstream-charter]{mathdesign}
\usepackage{aurical}

\usepackage[colorlinks=true,linkcolor=blue,urlcolor=links,pdfstartview={XYZ null null 1.00},bookmarksdepth=2]{hyperref}


%%%%%%%%%%%%%%%%%%%%%%%%%
%%%% End of Import %%%%%%
%%%%%%%%%%%%%%%%%%%%%%%%%



%%%%%%%%%%%%%%%%%%%%%%%%%%%%%%%%%%%%%%%%%%%%%%%%%%
%%%%%%%%%%%%%%%%%%%%%%%%%%%%%%%%%%%%%%%%%%%%%%%%%%
%%% General Font Display
%%%%%%%%%%%%%%%%%%%%%%%%%%%%%%%%%%%%%%%%%%%%%%%%%%
%%%%%%%%%%%%%%%%%%%%%%%%%%%%%%%%%%%%%%%%%%%%%%%%%%

\renewcommand*{\familydefault}{\sfdefault}
%% sets default text to sans-serif, so text doesn't flip back to serif in some environments.



%%%%%%%%%%%%%%%%%%%%%%%%%%%%%%%%%%%%%%%%%%%%%%%%%%
%%%%%%%%%%%%%%%%%%%%%%%%%%%%%%%%%%%%%%%%%%%%%%%%%%
%%% Sectioning Display
%%%%%%%%%%%%%%%%%%%%%%%%%%%%%%%%%%%%%%%%%%%%%%%%%%
%%%%%%%%%%%%%%%%%%%%%%%%%%%%%%%%%%%%%%%%%%%%%%%%%%






%%%%%%%%%%%%%%%%%%%%%%%%%%%%%%%%%%%%%%%%%%%%%%%%%%
%%%%%%%%%%%%%%%%%%%%%%%%%%%%%%%%%%%%%%%%%%%%%%%%%%
%%% Revised Commands
%%%%%%%%%%%%%%%%%%%%%%%%%%%%%%%%%%%%%%%%%%%%%%%%%%
%%%%%%%%%%%%%%%%%%%%%%%%%%%%%%%%%%%%%%%%%%%%%%%%%%
\makeatletter

%fiddles with how chapter titles are displayed
\renewcommand{\@makechapterhead}[1]{%
\vspace*{0 pt}{%
\raggedright \fontsize{32}{32} \selectfont \bfseries%
\ifnum \value{secnumdepth}>-1%
  \if@mainmatter \vspace{-8pt} {\fontsize{20}{20} \selectfont Chapter \thechapter:}\\[8pt]%
  \fi%
\fi
\hspace{0.65cm} #1\par\nobreak\vspace{20 pt}%
}}

%makes paragraphs show up closer together
\renewcommand{\paragraph}{%
\@startsection{paragraph}{4}%
{\z@}{1.0ex \@plus 1ex \@minus 0.2ex}{-1em} % wtf is an 'ex' anyways?
{\normalfont\normalsize\bfseries}%
}

%lets multicolumn have the proper background colors as defined by rowcolors
\let\oldmc\multicolumn
\newcommand{\mcinherit}{% Activate \multicolumn inheritance
  \renewcommand{\multicolumn}[3]{%
    \oldmc{##1}{##2}{\ifodd\rownum \@oddrowcolor\else\@evenrowcolor\fi ##3}%
  }}

\makeatother

%add labels within sections, subsections, and subsubsections
\LetLtxMacro{\oldsection}{\section}
\renewcommand{\section}[1]{\oldsection{#1}\label{sec:#1}}

\LetLtxMacro{\oldsubsection}{\subsection}
\renewcommand{\subsection}[1]{\oldsubsection{#1}\label{sec:#1}}

\LetLtxMacro{\oldsubsubsection}{\subsubsection}
\renewcommand{\subsubsection}[1]{\oldsubsubsection{#1}\label{sec:#1}}

%only put chapters and sections into the TOC
\setcounter{secnumdepth}{1}

%makes a subsubsection start off indented.
\setlength{\beforesubsubsecskip}{-\beforesubsubsecskip}



%%%%%%%%%%%%%%%%%%%%%%%%%%%%%%%%%%%%%%%%%%%%%%%%%%
%%%%%%%%%%%%%%%%%%%%%%%%%%%%%%%%%%%%%%%%%%%%%%%%%%
%%% Table Formatting
%%%%%%%%%%%%%%%%%%%%%%%%%%%%%%%%%%%%%%%%%%%%%%%%%%
%%%%%%%%%%%%%%%%%%%%%%%%%%%%%%%%%%%%%%%%%%%%%%%%%%
\newcolumntype{L}[1]{>{\raggedright\let\newline\\\arraybackslash\hspace{0pt}}m{#1}} %New type of column 'L' that is ragged-right, behaves like a paragraph, and allows manual definition of width like a 'p' column.
\newcolumntype{C}[1]{>{\centering\let\newline\\\arraybackslash\hspace{0pt}}m{#1}}  %New type of column 'C' that is centered, behaves like a paragraph, and allows manual definition of width like a 'p' column.
\newcolumntype{R}[1]{>{\raggedleft\let\newline\\\arraybackslash\hspace{0pt}}m{#1}}  %New type of column 'R' that is ragged-left, behaves like a paragraph, and allows manual definition of width like a 'p' column.
\newcommand{\header}{\rowcolor{headercolor}}
%when inserted in a row, makes that row the color headercolor

\global\tabulinesep=1mm


%%%%%%%%%%%%%%%%%%%%%%%%%%%%%%%%%%%%%%%%%%%%%%%%%%
%%%%%%%%%%%%%%%%%%%%%%%%%%%%%%%%%%%%%%%%%%%%%%%%%%
%%% List Formatting
%%%%%%%%%%%%%%%%%%%%%%%%%%%%%%%%%%%%%%%%%%%%%%%%%%
%%%%%%%%%%%%%%%%%%%%%%%%%%%%%%%%%%%%%%%%%%%%%%%%%%

\let\olditemize\itemize
\renewcommand{\itemize}{
  \olditemize
  \setlength{\itemsep}{1pt}
  \setlength{\parskip}{0pt}
  \setlength{\parsep}{0pt}
}
%fixes itemize spacing

\let\olddescription\description
\renewcommand{\description}{
  \olddescription
  \setlength{\itemsep}{1pt}
  \setlength{\parskip}{0pt}
  \setlength{\parsep}{0pt}
}
%fixes description spacing

\let\oldenumerate\enumerate
\renewcommand{\enumerate}{
  \oldenumerate
  \setlength{\itemsep}{1pt}
  \setlength{\parskip}{0pt}
  \setlength{\parsep}{0pt}
}
%fixes enumerate spacing

\newcommand{\descability}[2]{\item[#1:] #2}


%%%%%%%%%%%%%%%%%%%%%%%%%%%%%%%%%%%%%%%%%%%%%%%%%%
%%%%%%%%%%%%%%%%%%%%%%%%%%%%%%%%%%%%%%%%%%%%%%%%%%
%%% New Commands
%%%%%%%%%%%%%%%%%%%%%%%%%%%%%%%%%%%%%%%%%%%%%%%%%%
%%%%%%%%%%%%%%%%%%%%%%%%%%%%%%%%%%%%%%%%%%%%%%%%%%


%%%%%%%%%%%%%%%%%%%%%%%%
%%Basic Formatting
%%%%%%%%%%%%%%%%%%%%%%%%

\newcommand{\originallineskip}{\baselineskip}
 %A command that is equal to the original \baselineskip of the doc, in case we change it for a section and want to change it back later

\newcommand{\ability}[2]{\textbf{#1:} #2} 
%The \ability{#1}{#2} command from legacy-source. Should rarely be directly used, changes to this will cascade into other new commands that use its functionality

\newcommand{\shortability}[2]{\noindent\textbf{#1} #2\\}
%A specialized version of the \ability command

\newcommand{\itemspace}{\setlength{\itemsep}{-1mm}\setlength{\topsep}{-1mm} }
%A command from legacy-source for compatabilty

\newenvironment{awesomelist}{\begin{list}{$\bullet$}{\itemspace}}{\end{list}\vspace{8pt}}

\newcommand{\listone}{\begin{list}{$\bullet$}{\itemspace}}

\newcommand{\listtwo}{\begin{list}{$\triangleright$}{\itemspace}}
%A type of list from legacy sorce

\newcommand{\listnum}{\begin{list}{\textbf{\arabic{counter}}:}{\usecounter{counter}}}

\newcommand{\spell}[1]{\emph{\MakeLowercase{#1}}}
%makes spell name lowercase italics.

\setlength{\parindent}{12pt}
%sets the indentation of all paragraphs in the work

\newcommand{\quot}[1]{
	\vspace{-8pt}
	\noindent\emph{#1}\medskip}
%Displays a flavor quote.}

\newcommand{\half}[0]{\ensuremath{\sfrac{1}{2}} }

\newcommand{\third}[0]{\ensuremath{\sfrac{1}{3}} }

\newcommand{\fourth}[0]{\ensuremath{\sfrac{1}{4}} }

\newcommand{\ex}{(Ex)}
\newcommand{\sla}{(Sp)}
\newcommand{\su}{(Su)}

\newcommand{\condition}[1]{\emph{#1}}

%%%%%%%%%%%%%%%%%%%%%%%%
%%Logic
%%%%%%%%%%%%%%%%%%%%%%%%
\newcommand{\testempty}{\empty}
\newcommand{\isempty}{\empty}
%Two commands that can be compared to one another for \ifx logic tests. \isempty should never be changed. If \testempty holds a value of anything but empty, the test should return false.

\newcounter{counter}




%%%%%%%%%%%%%%%%%%%%%%%%
%%Colors
%%%%%%%%%%%%%%%%%%%%%%%%
\colorlet{colorone}{white}
\colorlet{colortwo}{gray!15}
\colorlet{headercolor}{gray!50}
\colorlet{tablecolorone}{gray!40}
\colorlet{tablecolortwo}{gray!20}


%%%%%%%%%%%%%%%%%%%%%%%%
%%Sectioning
%%%%%%%%%%%%%%%%%%%%%%%%
\newcommand{\classentry}[1]{\section{#1} \label{class:#1} \renewcommand{\class}{#1} \index{#1 (class)} \renewcommand{\testempty}{\isempty}}
%\newcommand{\classentry}[1]{\newpage \section{#1} \label{class:#1} \renewcommand{\class}{#1} \index{#1 (class)} \renewcommand{\testempty}{\isempty}}
%Starts a <new page>, creates a section with the name of the class (#1), sets \class to be the name of the class, indexes the class.

\newcommand{\raceentry}[2]{\newpage\renewcommand{\race}{#1}\section{#1} \label{race:#1}\vspace{-1em}\textit{#2}\newline}
%\newcommand{\raceentry}[1]{\oldsection{#1}\index{#1 (race)}\label{race:#1}}

\newcommand{\Requirements}{\oldsubsubsection*{Requirements}}

\newcommand{\Basics}{\oldsubsubsection*{Basics}}

\newcommand{\ClassFeatures}{\oldsubsubsection*{Class Features}}

\newcommand{\skillentry}[2]{\oldsubsection[#1]{#1 #2}\index{#1 (skill)}\label{skill:#1}}





%%%%%%%%%%%%%%%%%%%%%%%%
%%Unsorted Commands
%%%%%%%%%%%%%%%%%%%%%%%%
\newcommand{\tagline}[1]{\vspace{-6pt} \textit{#1} \medskip}

\newcommand{\gameterm}[1]{#1\index{#1}}

\NewDocumentCommand\featentry{m+g}{%
  \IfNoValueTF{#2}
    {\oldsubsubsection[#1]{#1 [General]}\label{feat:#1}}%no second arg, general feat
    {\oldsubsubsection[#1]{#1 [#2]}\label{feat:#1}}%second arg, special type of feat
}

\newcommand{\spellentry}[1]{\oldsubsubsection{#1}\label{spell:#1}}

\NewDocumentCommand\linkrace{m+g}{%
  \IfNoValueTF{#2}
    {\hyperref[race:#1]{#1}}%no second arg, display is same as link
    {\hyperref[race:#1]{#2}}%second arg, link to first and display second
}
\NewDocumentCommand\linkclass{m+g}{%
  \IfNoValueTF{#2}
    {\hyperref[class:#1]{#1}}%no second arg, display is same as link
    {\hyperref[class:#1]{#2}}%second arg, link to first and display second
}
\NewDocumentCommand\linkskill{m+g}{%
  \IfNoValueTF{#2}
    {\hyperref[skill:#1]{#1}}%no second arg, display is same as link
    {\hyperref[skill:#1]{#2}}%second arg, link to first and display second
}
\NewDocumentCommand\linkfeat{m+g}{%
  \IfNoValueTF{#2}
    {\hyperref[feat:#1]{#1}}%no second arg, display is same as link
    {\hyperref[feat:#1]{#2}}%second arg, link to first and display second
}
\NewDocumentCommand\linkspell{m+g}{%
  \IfNoValueTF{#2}
    {\hyperref[spell:#1]{#1}}%no second arg, display is same as link
    {\hyperref[spell:#1]{#2}}%second arg, link to first and display second
}
\NewDocumentCommand\linkcondition{m+g}{%
  \IfNoValueTF{#2}
    {\hyperref[condition:#1]{#1}}%no second arg, display is same as link
    {\hyperref[condition:#1]{#2}}%second arg, link to first and display second
}
\NewDocumentCommand\linkability{m+g}{%
  \IfNoValueTF{#2}
    {\hyperref[ability:#1]{#1}}%no second arg, display is same as link
    {\hyperref[ability:#1]{#2}}%second arg, link to first and display second
}
\NewDocumentCommand\linksec{m+g}{%
  \IfNoValueTF{#2}
    {\hyperref[sec:#1]{#1}}%no second arg, display is same as link
    {\hyperref[sec:#1]{#2}}%second arg, link to first and display second
}

\begin{document}

%%%%%%%%%%%%%%%%%%%%%%%%%%%%%%%%%%%%%%%%%%%%%%%%%%
%%%%%%%%%%%%%%%%%%%%%%%%%%%%%%%%%%%%%%%%%%%%%%%%%%
%%% Title Page
%%%%%%%%%%%%%%%%%%%%%%%%%%%%%%%%%%%%%%%%%%%%%%%%%%
%%%%%%%%%%%%%%%%%%%%%%%%%%%%%%%%%%%%%%%%%%%%%%%%%%
\thispagestyle{empty}
\begin{center}
\textsc{\Large}\\[0.25cm]
\rule{\linewidth}{0.5mm} \\[0.70cm]
\fontsize{30}{30} \selectfont Tome Reference Document\\[.30cm]
\fontsize{16}{18} \selectfont \guillemotleft{} For that game we all known and love \guillemotright{}\\
\rule{\linewidth}{0.5mm} \\[0.6cm]
%\includegraphics[clip,trim=5cm 2cm 9cm 1cm,width=\linewidth]{OldBookArt--MapImages-173.jpg}
\vfill
{\large \textit{This material is Open Game Content, and is licensed for public use under the terms of the Open Game License v1.0a.}\\
\today}
\end{center}

\pagebreak
\sffamily
\pagestyle{plain}
\raggedbottom

%%%%%%%%%%%%%%%%%%%%%%%%%%%%%%%%%%%%%%%%%%%%%%%%%%
%%%%%%%%%%%%%%%%%%%%%%%%%%%%%%%%%%%%%%%%%%%%%%%%%%
%%% Table of Contents
%%%%%%%%%%%%%%%%%%%%%%%%%%%%%%%%%%%%%%%%%%%%%%%%%%
%%%%%%%%%%%%%%%%%%%%%%%%%%%%%%%%%%%%%%%%%%%%%%%%%%
\renewcommand{\contentsname}{Table of Contents}
\setcounter{tocdepth}{1}
\tableofcontents

%%%%%%%%%%%%%%%%%%%%%%%%%%%%%%%%%%%%%%%%%%%%%%%%%%
%%%%%%%%%%%%%%%%%%%%%%%%%%%%%%%%%%%%%%%%%%%%%%%%%%
%%% Main Content %%%
%%%%%%%%%%%%%%%%%%%%%%%%%%%%%%%%%%%%%%%%%%%%%%%%%%
%%%%%%%%%%%%%%%%%%%%%%%%%%%%%%%%%%%%%%%%%%%%%%%%%%

%% Primary Chapters Here

\clearpage

%%%%%%%%%%%%%%%%%%%%%%%%%%%%%%%%%%%%%%%%%%%%%%%%%%
%%%%%%%%%%%%%%%%%%%%%%%%%%%%%%%%%%%%%%%%%%%%%%%%%%
\chapter{The Basics}\label{chapter:Basics}
%%%%%%%%%%%%%%%%%%%%%%%%%%%%%%%%%%%%%%%%%%%%%%%%%%
%%%%%%%%%%%%%%%%%%%%%%%%%%%%%%%%%%%%%%%%%%%%%%%%%%

Wherever possible, I have tried to stick to simply presenting the SRD as is. However, for a document produced by a professional corporation, the SRD has a lot of typos and word omissions that needed to be fixed. I've fixed those sorts of minor problems whenever I've seen them, and I've also organized the various sections a little better, which includes moving some auxiliary feats, domains, and spells into the main section.

The Psionic, Epic, and Divine parts of the SRD have been left out entirely; I don't plan to include them ever. For the Monster section, the current plan is to only include the monsters that have a listed purchase price or appear on a \textit{Summon} list. This makes it a little more player oriented I guess, but the MC should be able to find suitable monster entries all over the web with a \href{https://www.google.com/search?q=SRD monsters}{Simple Search}.

I've also added some entirely new \linksec{Character Creation} and \linksec{Character Advancement} rules so that players can play the game using just this PDF if they want. These aren't in the SRD, but they're pretty self-evident things if you've played the game before. You only really need to refer to them if you're new to the game.

%%%%%%%%%%%%%%%%%%%%%%%%%%%%%%%%%%%%%%%%%%%%%%%%%%
\section{The Core Mechanic}
%%%%%%%%%%%%%%%%%%%%%%%%%%%%%%%%%%%%%%%%%%%%%%%%%%
Whenever you attempt an action that has some chance 
of failure, you roll a twenty-sided die (d20). To determine if your character succeeds 
at a task you do this:
\begin{itemize*}
\item Roll a d20.
\item Add any relevant modifiers.
\item Compare the result to a target number.
\end{itemize*}
If the result equals or exceeds the target number, your character succeeds. If 
the result is lower than the target number, you fail.

%%%%%%%%%%%%%%%%%%%%%%%%%%%%%%%%%%%%%%%%%%%%%%%%%%
\section{Dice}
%%%%%%%%%%%%%%%%%%%%%%%%%%%%%%%%%%%%%%%%%%%%%%%%%%

Dice rolls are described with expressions such as "3d4+3", which means "roll 
three four-sided dice and add 3" (resulting in a number between 6 and 15). The 
first number tells you how many dice to roll (adding the results together). The 
number immediately after the "d" tells you the type of die to use. Any number 
after that indicates a quantity that is added or subtracted from the result.

\textbf{d\%:} Percentile dice work a little differently. You generate a number 
between 1 and 100 by rolling two different ten-sided dice. One (designated before 
you roll) is the tens digit. The other is the ones digit. Two 0s represent 100.

%%%%%%%%%%%%%%%%%%%%%%%%%%%%%%%%%%%%%%%%%%%%%%%%%%
\section{Rounding Fractions}
%%%%%%%%%%%%%%%%%%%%%%%%%%%%%%%%%%%%%%%%%%%%%%%%%%

In general, if you wind up with a fraction, round down, even if the fraction is 
one-half or larger.

\textit{Exception:} Certain rolls, such as damage and hit points, have a minimum 
of 1.

%%%%%%%%%%%%%%%%%%%%%%%%%%%%%%%%%%%%%%%%%%%%%%%%%%
\section{Multiplying}
%%%%%%%%%%%%%%%%%%%%%%%%%%%%%%%%%%%%%%%%%%%%%%%%%%

Sometimes a rule makes you multiply a number or a die roll. As long as you're applying 
a single multiplier, multiply the number normally. When two or more multipliers 
apply to any abstract value (such as a modifier or a die roll), however, combine 
them into a single multiple, with each extra multiple adding 1 less than its value 
to the first multiple. Thus, a double (x2) and a double (x2) 
applied to the same number results in a triple (x3, because 2 
+ 1 = 3).


When applying multipliers to real-world values (such as weight or distance), normal 
rules of math apply instead. A creature whose size doubles (thus multiplying its 
weight by 8) and then is turned to stone (which would multiply its weight by a 
factor of roughly 3) now weighs about 24 times normal, not 10 times normal. Similarly, 
a blinded creature attempting to negotiate difficult terrain would count each square 
as 4 squares (doubling the cost twice, for a total multiplier of x4), 
rather than as 3 squares (adding 100\% twice). 

%%%%%%%%%%%%%%%%%%%%%%%%%%%%%%%%%%%%%%%%%%%%%%%%%%
\section{Ability Scores}
%%%%%%%%%%%%%%%%%%%%%%%%%%%%%%%%%%%%%%%%%%%%%%%%%%

The table below shows the ability modifier associated with an ability score of a given value, as well as the bonus spells that a caster with that ability score as their spellcasting stat gains. Regardless of ability score, a spellcaster never gains bonus 0th level spell slots.

\begin{table}[htb]
\rowcolors{1}{white}{offyellow}
\caption{Ability Modifiers and Bonus Spells}
\centering
\begin{tabular}{*{11}{c}}
\textbf{Stat} & \textbf{Mod} & \textbf{1st} & \textbf{2nd} & \textbf{3rd} & \textbf{4th} & \textbf{5th} & \textbf{6th} & \textbf{7th} & \textbf{8th} & \textbf{9th}\\
1 & -5 & -- & -- & -- & -- & -- & -- & -- & -- & -- \\
2-3 & -4 & -- & -- & -- & -- & -- & -- & -- & -- & -- \\
4-5 & -3 & -- & -- & -- & -- & -- & -- & -- & -- & -- \\
6-7 & -2 & -- & -- & -- & -- & -- & -- & -- & -- & -- \\
8-9 & -1 & -- & -- & -- & -- & -- & -- & -- & -- & -- \\
10-11 & +0 & 0 & 0 & 0 & 0 & 0 & 0 & 0 & 0 & 0\\
12-13 & +1 & 1 & 0 & 0 & 0 & 0 & 0 & 0 & 0 & 0\\
14-15 & +2 & 1 & 1 & 0 & 0 & 0 & 0 & 0 & 0 & 0\\
16-17 & +3 & 1 & 1 & 1 & 0 & 0 & 0 & 0 & 0 & 0\\
18-19 & +4 & 1 & 1 & 1 & 1 & 0 & 0 & 0 & 0 & 0\\
20-21 & +5 & 2 & 1 & 1 & 1 & 1 & 0 & 0 & 0 & 0\\
22-23 & +6 & 2 & 2 & 1 & 1 & 1 & 1 & 0 & 0 & 0\\
24-25 & +7 & 2 & 2 & 2 & 1 & 1 & 1 & 1 & 0 & 0\\
26-27 & +8 & 2 & 2 & 2 & 2 & 1 & 1 & 1 & 1 & 0\\
28-29 & +9 & 3 & 2 & 2 & 2 & 2 & 1 & 1 & 1 & 1\\
30-31 & +10 & 3 & 3 & 2 & 2 & 2 & 2 & 1 & 1 & 1\\
32-33 & +11 & 3 & 3 & 3 & 2 & 2 & 2 & 2 & 1 & 1\\
34-35 & +12 & 3 & 3 & 3 & 3 & 2 & 2 & 2 & 2 & 1\\
36-37 & +13 & 4 & 3 & 3 & 3 & 3 & 2 & 2 & 2 & 2\\
38-39 & +14 & 4 & 4 & 3 & 3 & 3 & 3 & 2 & 2 & 2\\
40-41 & +15 & 4 & 4 & 4 & 3 & 3 & 3 & 3 & 2 & 2\\
42-43 & +16 & 4 & 4 & 4 & 4 & 3 & 3 & 3 & 3 & 2\\
44-45 & +17 & 5 & 4 & 4 & 4 & 4 & 3 & 3 & 3 & 3\\
46-47 & +18 & 5 & 5 & 4 & 4 & 4 & 4 & 3 & 3 & 3\\
48-49 & +19 & 5 & 5 & 5 & 4 & 4 & 4 & 4 & 3 & 3\\
50-51 & +20 & 5 & 5 & 5 & 5 & 4 & 4 & 4 & 4 & 3\\
\end{tabular}
\end{table}

%%%%%%%%%%%%%%%%%%%%%%%%%
\subsection{Ability Modifiers}\index{Ability Modifer}
%%%%%%%%%%%%%%%%%%%%%%%%%

Each ability, after changes made because of race, has a modifier ranging from -5 
to +5. Table 1.1: Ability Modifiers and Bonus Spells shows the modifier for each score. 
It also shows bonus spells, which you'll need to know about if your character is 
a spellcaster.

The modifier is the number you apply to the die roll when your character tries 
to do something related to that ability. You also use the modifier with some numbers that aren't die rolls. 
A positive modifier is called a bonus, and a negative modifier is called a penalty.

%%%%%%%%%%%%%%%%%%%%%%%%%
\subsection{Abilities and Spellcasters}\index{Bonus Spells}
%%%%%%%%%%%%%%%%%%%%%%%%%

The ability that governs bonus spells depends on what type of spellcaster your 
character is: Intelligence for wizards; Wisdom for clerics, druids, paladins, and 
rangers; or Charisma for sorcerers and bards. In addition to having a high ability 
score, a spellcaster must be of high enough class level to be able to cast spells 
of a given spell level. (See the class descriptions for details.)

%%%%%%%%%%%%%%%%%%%%%%%%%
\subsection{The Abilities}
%%%%%%%%%%%%%%%%%%%%%%%%%

Each ability partially describes your character and affects some of his or her 
actions.

When an ability score changes, all attributes associated with that score change 
accordingly. A character does not retroactively get additional skill points for 
previous levels if she increases her intelligence.

%%%
\subsubsection{Strength (Str)}\index{Strength}
%%%

Strength measures your character's muscle and physical power. This ability is especially 
important for fighters, barbarians, paladins, rangers, and monks because it helps 
them prevail in combat. Strength also limits the amount of equipment your character 
can carry.

You apply your character's Strength modifier to:
\begin{itemize}
\item Melee attack rolls.
\item Damage rolls when using a melee weapon or a thrown weapon (including a sling). 
(\textit{Exceptions:} Off-hand attacks receive only one-half the character's Strength 
bonus, while two-handed attacks receive one and a half times the Strength bonus. 
A Strength penalty, but not a bonus, applies to attacks made with a bow that is 
not a composite bow.)
\item \linkskill{Climb}, \linkskill{Jump}, and \linkskill{Swim} checks. These are the skills that have Strength as their 
key ability.
\item Strength checks (for breaking down doors and the like).
\end{itemize}

%%%
\subsubsection{Dexterity (Dex)}\index{Dexterity}
%%%

Dexterity measures hand-eye coordination, agility, reflexes, and balance. This 
ability is the most important one for rogues, but it's also high on the list for 
characters who typically wear light or medium armor (rangers and barbarians) or 
no armor at all (monks, wizards, and sorcerers), and for anyone who wants to be 
a skilled archer.

You apply your character's Dexterity modifier to:
\begin{itemize}
\item Ranged attack rolls, including those for attacks made with bows, crossbows, throwing 
axes, and other ranged weapons.
\item Armor Class (AC), provided that the character can react to the attack.
\item Reflex saving throws, for avoiding fireballs and other attacks that you can escape 
by moving quickly.
\item \linkskill{Balance}, \linkskill{Escape Artist}, \linkskill{Hide}, \linkskill{Move Silently}, \linkskill{Open Lock}, \linkskill{Ride}, \linkskill{Sleight of Hand}, 
\linkskill{Tumble}, and \linkskill{Use Rope} checks. These are the skills that have Dexterity as their 
key ability.
\end{itemize}

%%%
\subsubsection{Constitution (Con)}\index{Constitution}
%%%

Constitution represents your character's health and stamina. A Constitution bonus 
increases a character's hit points, so the ability is important for all classes.

You apply your character's Constitution modifier to:
\begin{itemize}
\item Each roll of a Hit Die (though a penalty can never drop a result below 1, that 
is, a character always gains at least 1 hit point each time he or she advances 
in level).
\item Fortitude saving throws, for resisting poison and similar threats.
\item \linkskill{Concentration} checks. Concentration is a skill, important to spellcasters, that 
has Constitution as its key ability.
\end{itemize}
If a character's Constitution score changes enough to alter his or her Constitution 
modifier, the character's hit points also increase or decrease accordingly.

%%%
\subsubsection{Intelligence (Int)}\index{Intelligence}
%%%

Intelligence determines how well your character learns and reasons. This ability 
is important for wizards because it affects how many spells they can cast, how 
hard their spells are to resist, and how powerful their spells can be. It's also 
important for any character who wants to have a wide assortment of skills.

You apply your character's Intelligence modifier to:
\begin{itemize}
\item The number of languages your character knows at the start of the game.
\item The number of skill points gained each level. (But your character always gets at 
least 1 skill point per level.)
\item \linkskill{Appraise}, \linkskill{Craft}, \linkskill{Decipher Script}, \linkskill{Disable Device}, \linkskill{Forgery}, \linkskill{Knowledge}, \linkskill{Search}, and 
\linkskill{Spellcraft} checks. These are the skills that have Intelligence as their key ability.
\end{itemize}
A wizard gains bonus spells based on her Intelligence score. The minimum Intelligence 
score needed to cast a wizard spell is 10 + the spell's level. 

An animal has an Intelligence score of 1 or 2. A creature of humanlike intelligence 
has a score of at least 3.

%%%
\subsubsection{Wisdom (Wis)}\index{Wisdom}
%%%

Wisdom describes a character's willpower, common sense, perception, and intuition. 
While Intelligence represents one's ability to analyze information, Wisdom represents 
being in tune with and aware of one's surroundings. Wisdom is the most important 
ability for clerics and druids, and it is also important for paladins and rangers. 
If you want your character to have acute senses, put a high score in Wisdom. Every 
creature has a Wisdom score.

You apply your character's Wisdom modifier to:
\begin{itemize}
\item Will saving throws (for negating the effect of charm person and other spells).
\item \linkskill{Heal}, \linkskill{Listen}, \linkskill{Profession}, \linkskill{Sense Motive}, \linkskill{Spot}, and \linkskill{Survival} checks. These are the 
skills that have Wisdom as their key ability.
\end{itemize}
Clerics, druids, paladins, and rangers get bonus spells based on their Wisdom scores. 
The minimum Wisdom score needed to cast a cleric, druid, paladin, or ranger spell 
is 10 + the spell's level.

%%%
\subsubsection{Charisma (Cha)}\index{Charisma}
%%%

Charisma measures a character's force of personality, persuasiveness, personal 
magnetism, ability to lead, and physical attractiveness. This ability represents 
actual strength of personality, not merely how one is perceived by others in a 
social setting. Charisma is most important for paladins, sorcerers, and bards. 
It is also important for clerics, since it affects their ability to turn undead. 
Every creature has a Charisma score.

You apply your character's Charisma modifier to:
\begin{itemize}
\item \linkskill{Bluff}, \linkskill{Diplomacy}, \linkskill{Disguise}, \linkskill{Gather Information}, \linkskill{Handle Animal}, \linkskill{Intimidate}, \linkskill{Perform}, 
and \linkskill{Use Magic Device} checks. These are the skills that have Charisma as their key 
ability.
\item Checks that represent attempts to influence others. 
\item Turning checks for clerics and paladins attempting to turn zombies, vampires, and 
other undead.
\end{itemize}
Sorcerers and bards get bonus spells based on their Charisma scores. The minimum 
Charisma score needed to cast a sorcerer or bard spell is 10 + the spell's level.

%%%%%%%%%%%%%%%%%%%%%%%%%%%%%%%%%%%%%%%%%%%%%%%%%%
\section{Ability Score Generation}
%%%%%%%%%%%%%%%%%%%%%%%%%%%%%%%%%%%%%%%%%%%%%%%%%%

Your ability scores are generated randomly at the start of the game so that they range from 3 to 18 before applying racial modifiers (an average human has a 10 or 11). Many ways exist to do this. The problem with rolling is that one player might get stats that are simply better than another player's, which isn't fair. The problem with some sort of point-buy system is that players will always max out their primary stat first at the expense of other stats, and the resulting stats look very inorganic. Spellcasters particularly benefit from point-buy systems, and they're the ones who generally shoot off the charts in terms of power level in the first place, and so we don't want that.

As a result, the suggested method for stat generation is as follows:
\begin{itemize}
\item Each player rolls 4d6 dice for each stat, dropping the lowest die from the total. (Player Characters are above average, so we give them slightly better odds than just 3d6.) Do this six times to get a number for each stat.
\item If the resulting stat set doesn't have a single stat of at least 13, reroll it.
\item Also, If the resulting stat set doesn't have a total ability modifier of at least +1, reroll it.
\item Each player does this so that they have a stat set, then all players can pick any of the stat sets that were rolled.
\item Assign each stat one of the numbers from the set and proceed with the rest of Character Creation.
\end{itemize}
This way, you get "organic" style stat sets that aren't always just 18 and then a few 14s, but you also don't give any single player an unfair advantage.

%%%%%%%%%%%%%%%%%%%%%%%%%%%%%%%%%%%%%%%%%%%%%%%%%%
\section{Character Creation}
%%%%%%%%%%%%%%%%%%%%%%%%%%%%%%%%%%%%%%%%%%%%%%%%%%

To create a complete character there are several steps that you need to follow:

\begin{itemize}
\item Select Race (\hyperref[chapter:Races]{Chapter 2})
\item Select Class (\hyperref[chapter:Classes]{Chapter 3})
\item Determine Ability Scores (\linksec{Ability Score Generation}{above})
\item Select Class Features (not all classes have feature options at first level)
\item Assign Skill Points (\hyperref[chapter:Skills]{Chapter 4})
\item Select Feat (\hyperref[chapter:Feats]{Chapter 5})
\item Purchase Starting Equipment (\hyperref[chapter:Equipment]{Chapter 6})
\item Note Character Details (\hyperref[chapter:Description]{Chapter 7})
\end{itemize}

%%%%%%%%%%%%%%%%%%%%%%%%%%%%%%%%%%%%%%%%%%%%%%%%%%
\section{Character Advancement}
%%%%%%%%%%%%%%%%%%%%%%%%%%%%%%%%%%%%%%%%%%%%%%%%%%

As you adventure and such your \gameterm{MC} (short for "\gameterm{Mister Cavern}") will describe the world around the characters of the other players, and also play the role of all the Non-Player Characters (NPCs\index{NPC}) that you meet. At the end of each play session your characters will earn \gameterm{Experience Points} (\gameterm{XP}) based on what you've done. These are used to gain additional levels, which lets you take on stronger threats and generally have larger scale adventures.

You need 1,000 experience points times your current \gameterm{Character Level} to advance to the next level. Each time you gain a level, spend the appropriate number of points and then follow these steps:

\begin{itemize}
\item Choose the class you want to level up in. One of your existing classes goes up by one level, or you can add a new class at level 1 by following the rules for Multiclassing.
\item Increase Base Attack Bonus and Base Save Bonus based on the new class level you gained.
\item Roll for additional hit points based on the class level you gained. Remember that you add your Constitution Modifier to the roll.
\item If your total level is now a multiple of 4 (4th, 8th, 12th, etc), increase a stat of your choosing by 1 point.
\item Assign additional skill points from your new class level. If you gained a stat point during the previous step and used it to increase Intelligence, then you gain additional skill points (if any) during this level.
\item If your total level is now a multiple of 3 (3rd, 6th, 9th, etc) then you gain a new Feat.
\item Add any new class features that you gained to your sheet, and update your old class features that may have improved (such as gaining additional spells per day, or increased sneak attack damage)
\end{itemize}

Characters earn experience points for getting things done. Things that are level-appropriate, and usually things of a questly nature. This does include fighting foes, but it also includes stealing items, swaying minds, and generally having an important effect on the world around you. If the players take on threats and challenges that are above their level they get more experience, and if they take on things beneath them then they get less experience. 

Generally, a threat of CR X will give 75*X experience points to each player that participated. For each level a challenge is above or below the level of a character, increase or decrease the amount of experience that that character earns by 10\%. Whenever possible, the group as a whole should be kept at the same level. It sucks being even one level behind the people around you for an extended period. Remember that a Cohort from the Leadership feat is only 2 levels behind their leader, each level is important in this game. If a player is lower level than the rest of the group, they won't be able to do as much and they can easily become frustrated at the fact that everyone around them is doing bigger and better things.

Groups can usually face threats below their level with ease. Increasing the number of foes will usually keep things even (double the number for each 2 levels below the group level). A group generally can't take on a threat more than 2 levels above their own if they want anything more than a pyrrhic victory. A threat more than 4 levels above their own might quickly turn into a Total Party Kill.
%%%%%%%%%%%%%%%%%%%%%%%%
%%Race Chapter Formatting
%%%%%%%%%%%%%%%%%%%%%%%%
\newcommand{\race}{placeholder}

\newcommand{\racedescription}[1]{\indent\ability{Physical Description}{#1}}
\newcommand{\racepersonality}[1]{\indent\ability{Personality}{#1}}
\newcommand{\racesociety}[1]{\indent\ability{Society}{#1}}
\newcommand{\racealignment}[1]{\indent\ability{Alignment}{#1}}

\newcommand{\type}[1]{\ability{Type}{#1}\\ }
\newcommand{\size}[1]{\ability{Size}{#1}\\ }
\newcommand{\speed}[1]{\ability{Speed}{#1 feet}\\ }
\newcommand{\scores}[1]{\ability{Racial Ability Score Modifiers}{#1}\\ }
\newcommand{\racialtraits}[1]{~\\*\ability{\race ~Special Abilities}{#1}\\ }
\newcommand{\racetrait}[2]{\newline\indent\ability{#1}{#2} }
\newcommand{\senses}[1]{\ability{Senses}{#1}\\ }
\newcommand{\autolanguages}[1]{\ability{Automatic Languages}{#1}\\ }
\newcommand{\bonuslanguages}[1]{\ability{Bonus Languages}{#1}\\ }
\newcommand{\favoredclasses}[1]{\ability{Favored Classes}{#1}\\ }
\newcommand{\male}[4]{Male &#1 &#2 &#3 &#4\\ }
\newcommand{\female}[4]{Female &#1 &#2 &#3 &#4\\ }

\newcommand{\racedatastart}{
\noindent
\begin{minipage}[t]{\linewidth}
\vspace{-.5em}
\begin{multicols}{2}
}

\newcommand{\racedataend}{\
\end{multicols}
\end{minipage}
}

\newenvironment{racetable}
{
\tabulinesep=1mm
\renewcommand\arraystretch{1.4}
\noindent
\begin{tabu} to \linewidth {X}
\header\textbf{\race ~Racial Traits} \\ 
\hline
\end{tabu}
\rowcolors{1}{colortwo}{colorone}
\begin{tabu} to \linewidth {X [1, l]}
}{
\hline
\end{tabu}
}

\newcommand{\agetable}[4]{
\columnbreak
\renewcommand\arraystretch{1.4}
\tabulinesep=1mm
\noindent
\begin{tabu} to \linewidth {X}
\header\textbf{\race ~Starting Age} \\ \hline
\end{tabu}
\rowcolors{1}{colortwo}{colorone}
\begin{tabu} to \linewidth {X X X X}
\textbf{Adulthood:} &\textbf{Simple:} &\textbf{Moderate:} &\textbf{Complex:} \\
#1 Years &#2 &#3 &#4 \\ \hline
\end{tabu}
}

\newenvironment{heightweighttable}
{
\tabulinesep=1mm
\renewcommand\arraystretch{1.4}
\noindent
\begin{tabu} to \linewidth {X}
\header\textbf{\race ~Height and Weight} \\ \hline
\end{tabu}
\vspace{-1pt}
\rowcolors{1}{colortwo}{colorone}
\begin{tabu} to \linewidth {X X X X X}
\textbf{Gender} &\textbf{Base Height} &\textbf{Height Mod.} &\textbf{Base Weight} &\textbf{Weight Mod.} \\
}{
\hline
\end{tabu}
}

\newenvironment{raceleft}
{
\vspace{0pt}
\begin{minipage}[t]{0.5\linewidth}
}{
\end{minipage}
}

\newenvironment{raceright}
{\begin{minipage}[t]{0.5\linewidth}
\vspace{0pt}
}{
\end{minipage}
}

\newenvironment{racebox}
{
\vspace{0pt}
%\nointerlineskip
\begin{minipage}{\textwidth}
}{
\end{minipage}
}


\chapter{Races}

%\begin{racebox}
\raceentry{Aasimar}{``My ancestors were more beautiful than you can imagine."}

Aasimar are humans that have a beautiful outsider, usually but not always a celestial, somewhere in their ancestry.

\racedescription{Aasimar look like especially beautiful humans, though they sometimes bear vestiges of their ancestry that denote them as being different (strangely colored eyes, silver-blonder or white hair, slightly `off' facial features).}

\racepersonality{Though mostly human, an aasimar's immortal heritage influences their mental development. Aasimar tend toward more extreme personalities, being especially quiet and introspective or particularly loud and boisterous. Most aasimar are very opinionated, and have strongly held beliefs.}

\racesociety{Aasimar are typically born and raised in human societies, and gain the same customs of that culture}

\racealignment{Most aasimar are the descendants of celestials, and tend towards the good alignments. Rarely, an aasimar might instead have an infernal heritage, being the descendant of an erinyes or succubus. Such aasimar instead tend towards an evil alignment.}

\racedatastart
\begin{racetable}
\type{Outsider (Native and Human Subtype)}
\size{Medium}
\scores{+2 Wisdom, +2 Charisma}
\speed{30}
\senses{Standard}
\autolanguages{Common}
\bonuslanguages{Abyssal, Aquan, Auran, Celestial, Formian, Ignan, Slaad, Sylvan, Terran.}
\favoredclasses{Paladin and Sorcerer}
\end{racetable}

\vspace{\baselineskip}
\agetable{20}{+1d6}{+2d6}{+3d6}

\vspace{\baselineskip}
\begin{heightweighttable}
\male{4' 7"}{+2d8}{90 lb.}{x(2d4)}
\female{4' 5"}{+2d8}{80 lb.}{x(2d4)}
\end{heightweighttable}
\racedataend

\racialtraits{
\racetrait{Inner Light \sla}{An Aasimar with a Charisma of at least 10 may cast \spell{light} once per day, with a caster level equal to their character level.}
\racetrait{Keen Senses}{+2 bonus to Spot, and Listen checks.}
}
%\end{racebox}
%\begin{racebox}
\raceentry{Drow}{``Time to die for the Spider Queen."}

No description until we either obtain or write one.

\racedescription{NYW}

\racepersonality{NYW}

\racesociety{NYW}

\racealignment{NYW}

\racedatastart
\begin{racetable}
\type{Humanoid (Elf Subtype)}
\size{Medium}
\scores{+2 Dexterity, -2 Constitution}
\speed{30}
\senses{Darkvision 120'}
\autolanguages{Elvish}
\bonuslanguages{Abyssal, Beholder, Common, Draconic, Drow Sign Language, Dwarvish, Gnome, Kuo-Toa, Terran, Undercommon}
\favoredclasses{Cleric and Wizard}
\end{racetable}

\vspace{\baselineskip}
\agetable{20}{+1d6}{+2d6}{+3d6}

\vspace{\baselineskip}
\begin{heightweighttable}
\male{4' 7"}{+2d8}{90 lb.}{x(2d4)}
\female{4' 5"}{+2d8}{80 lb.}{x(2d4)}
\end{heightweighttable}
\racedataend

\racialtraits{
\racetrait{Daylight Sensitivity}{While in brightly lit surroundings (such as a daylight spell), a Drow suffers a -2 penalty to attack rolls and precision-based skill checks.}
\racetrait{Innate Magic}{Drow with a Charisma of at least 10 may cast deeper darkness (duration 4 hours), and fairie fire as spell-like abilities with a caster level equal to their character level once per day each.}
\racetrait{Magic Resistant}{+2 bonus to saving throws against spells and spell-like abilities.}
\racetrait{Skill Bonus}{+2 bonus to Spot and Listen checks.}
\racetrait{Elven Trance}{Drow never sleep and are immune to sleep effects. Drow must still perform their 4 hour daily trance to stay coherent and rested.}
\racetrait{Interesting Times}{Drow live an exceedingly interesting life and every Drow has proficiency with the rapier and an exotic ranged weapon of their choice.}
}
%\end{racebox}

\raceentry{Dwarf}
\quot{``I remember that...''}

\listone
		\item Medium Size
		\item 20' movement
		\item Humanoid Type (Dwarf Subtype)
		\item +2 Constitution, -2 Charisma
		\item Dwarves can move up to their full speed even when wearing medium or heavy armor or when carrying a medium or heavy load
		\item Darkvision: Dwarves can see up to 60 feet in the dark.
		\item Stonecunning: This ability grants a dwarf a +2 racial bonus on Search checks to notice unusual stonework, such as sliding walls, stonework traps, new construction (even when built to match the old), unsafe stone surfaces, shaky stone ceilings, and the like. Something that isn’t stone but that is disguised as stone also counts as unusual stonework. A dwarf who merely comes within 10 feet of unusual stonework can make a Search check as if he were actively searching, and a dwarf can use the Search skill to find stonework traps as a rogue can. A dwarf can also intuit depth, sensing his approximate depth underground as naturally as a human can sense which way is up.
		\item Weapon Familiarity: Dwarves may treat dwarven waraxes and dwarven urgroshes as martial weapons, rather than exotic weapons.
		\item Stability: A dwarf gains a +4 bonus on ability checks made to resist being bull rushed or tripped when standing on the ground (but not when climbing, flying, riding, or otherwise not standing firmly on the ground).
		\item +2 racial bonus on saving throws against poison.
		\item +2 racial bonus on saving throws against spells and spell-like effects.
		\item +1 racial bonus on attack rolls against orcs and goblinoids.
		\item +4 dodge bonus to Armor Class against monsters of the giant type. Any time a creature loses its Dexterity bonus (if any) to Armor Class, such as when it’s caught flat-footed, it loses its dodge bonus, too.
		\item +2 racial bonus on Appraise checks that are related to stone or metal items.
		\item +2 racial bonus on Craft checks that are related to stone or metal.
		\item Favored class: Fighter
		\item Automatic Languages: Common and Dwarven.
		\item Bonus Languages: Giant, Gnome, Goblin, Orc, Terran, and Undercommon.
\end{list}

\raceentry{Elf}
\quot{``You shall never harm my beautiful trees!''}

\listone
		\item Medium Size
		\item Humanoid Type (Elf Subtype)
		\item 30' movement
		\item +2 Dexterity, -2 Constitution
		\item Low Light Vision: Elves can see twice as far as a human in poor lighting.
		\item Weapon Proficiency: Elves are proficient with the longsword, rapier, longbow (including composite longbow), and shortbow (including composite shortbow).
		\item +2 racial bonus on Listen, Search, and Spot checks. An elf who merely passes within 5 feet of a secret or concealed door is entitled to a Search check to notice it as if she were actively looking for it.
		\item Favored Class: Wizard.
		\item Automatic Languages: Common and Elven. 
		\item Bonus Languages: Draconic, Gnoll, Gnome, Goblin, Orc, and Sylvan.
\end{list}
\raceentry{Feytouched}
\quot{``All my life, I have never fit in. Not in town, not in the forest. In some integral fashion I am unlike those around me, and I believe it is my fate to live and die alone."}

\listone
		\item Medium Size
    \item Fey Type
    \item 30' movement
    \item Low-Light Vision: Feytouched can twice as far as a human in poor lighting.
    \item +2 Dexterity, +2 Charisma, -2 Constitution. Feytouched are graceful and those which are not beautiful are terrifying, but they are fragile like flowers.
    \item Immunity to [Compulsion] Effects
    \item Magic Affinity: Every Feytouched is different, and marked by the signature magics of the fey in a different manner. Every Feytouched has one spell that can be used once per day as a spell-like ability. This spell is chosen at 1st level and cannot be changed. Any 1st level Illusion or Enchantment spell from the Sorcerer/Wizard list is fair game, and the save DC is Charisma-based.
    \item Favored Class: Bard
    \item Feytouched speak Common and Sylvan. Bonus Languages may be selected from the following list:
      Aquan, Auran, Elvish, Draconic, Dwarvish, Druidic, Goblin, Gnoll, Gnome, Halfling.
\end{list}
\begin{racebox}
\raceentry{Gnome}{``What's that you say little mole? Kobolds in the well!?"}
\begin{multicols}{2}

\begin{racetable}
\type{Humanoid (Gnome subtype)}
\size{Small}
\scores{+2 Constitution, --2 Strength}
\speed{20}
\senses{Low Light Vision}
\autolanguages{Common and Gnome}
\bonuslanguages{Draconic, Dwarven, Elven, Giant, Goblin, and Orc. In addition, a gnome can speak with a burrowing mammal (a badger, fox, rabbit, or the like).}
\favoredclasses{Bard}
\end{racetable}

Nothing until we find or write it.

\racedescription{NYW}

\racepersonality{NYW}

\racesociety{NYW}

\racealignment{NYW}

\racialtraits{
\racetrait{Weapon Familiarity}{Gnomes may treat gnome hooked hammers as martial weapons rather than exotic weapons.}
\racetrait{+2 racial bonus on saving throws against illusions.}
\racetrait{Add +1 to the Difficulty Class for all saving throws against illusion spells cast by gnomes. This adjustment stacks with those from similar effects.}
\racetrait{+1 racial bonus on attack rolls against kobolds and goblinoids.}
\racetrait{+4 dodge bonus to Armor Class against monsters of the giant type.}
\racetrait{+2 racial bonus on Listen checks.}
\racetrait{+2 racial bonus on Craft (alchemy) checks.}
\racetrait{Spell-Like Abilities: 1/day—speak with animals (burrowing mammal only, duration 1 minute). A gnome with a Charisma score of at least 10 also has the following spell-like abilities: 1/day—dancing lights, ghost sound, prestidigitation. Caster level 1st; save DC 10 + gnome’s Cha modifier + spell level.} % I'm pretty sure there's a much better way to format this, say by using the \sla used for aasimar. That Cha req goofs it up.
}


%\columnbreak

\vspace{\baselineskip}
\agetable{40}{+4d6}{+6d6}{+9d6}

\vspace{\baselineskip}
\begin{heightweighttable}
\male{3' 0"}{+2d4}{40 lb.}{x(1)}
\female{2' 10"}{+2d4}{35 lb.}{x(1)}
\end{heightweighttable}
\end{multicols}
\end{racebox}
\raceentry{Goblin}
\quot{``You weren't hired to think. You were hired because you have opposable thumbs."}

\listone
    \item Small Size
    \item 30' movement (despite small size).
    \item Humanoid Type (Goblinoid subtype)
    \item Darkvision: Goblins can see up to 60 feet in the dark.
    \item +2 Dexterity, -2 Strength, -2 Charisma
    \item +4 bonus to Move Silently and Ride checks.
    \item Bonus Feat: Mounted Combat
    \item Goblins benefit from an ancient pact with the Worgs, and every Goblin receives a +2 bonus to any Bluff, Diplomacy, Handle Animal, Sense Motive, or Survival check made with respect to a Worg.
    \item Favored Classes: Rogue and Wizard
    \item Automatic Languages: Common, Goblin
    \item Bonus Languages: Draconic, Elvish, Dwarvish, Giant, Gnoll, Infernal, Orcish, Undercommon, and Worg.
\end{list}
\raceentry{Half-Elf}{``I don't fit in anywhere, please, listen to me cry.''}

\listone
		\item Medium Size
		\item 30' Movement
		\item Humanoid Type
		\item Low-Light Vision: Half-Elves can see twice as humans in poor lighting.
		\item Immunity to sleep spells and similar magical effects, and a +2 racial bonus on saving throws against enchantment spells or effects.
		\item +1 racial bonus on Listen, Search, and Spot checks.
		\item +2 racial bonus on Diplomacy and Gather Information checks.
		\item Elven Blood: For all effects related to race, a half-elf is considered an elf.
		\item Favored Class: Any
		\item Automatic Languages: Common and Elven.
		\item Bonus Languages: Any (other than secret languages, such as Druidic).
\end{list}
\raceentry{Halfling}{``Where are we going Mr. Frodo?''}

\listone
		\item Small Size
		\item 20' movement
		\item +2 Dexterity, -2 Strength
		\item +2 racial bonus on Climb, Jump, Listen, and Move Silently checks.
		\item +1 racial bonus on all saving throws.
		\item +2 morale bonus on saving throws against fear: This bonus stacks with the halfling’s +1 bonus on saving throws in general.
		\item +1 racial bonus on attack rolls with thrown weapons and slings.
		\item Favored Class: Rogue
		\item Automatic Languages: Common and Halfling.
		\item Bonus Languages: Dwarven, Elven, Gnome, Goblin, and Orc.
\end{list}		

\raceentry{Half-Orc}
\quot{``I don't fit in anywhere, but you may be surprised to know that this dagger fits all kinds of places."}

\listone
    \item Medium Size
    \item 30' movement
    \item Humanoid Type (Orc and Human subtype)
    \item Darkvision: Half-Orcs can see up to 60 feet in the dark.
    \item +2 Strength
    \item +2 to Intimidate, Gather Information, and Survival checks.
    \item Favored Classes: Assassin and Barbarian
    \item Automatic Languages: Orc, Common
    \item Bonus Languages: Any.
\end{list}
\raceentry{Hobgoblin}
\quot{``That's some tough talk from a man who wears a basket on his head."}

\listone
    \item Medium Size
    \item 30' movement.
    \item Humanoid Type (Goblinoid subtype)
    \item Darkvision: Hobgoblins can see up to 60 feet in the dark.
    \item +2 Dexterity, +2 Constitution
    \item +4 bonus to Move Silently checks.
    \item Favored Classes: Fighter and Samurai
    \item Automatic Languages: Common, Goblin
    \item Bonus Languages: Draconic, Elvish, Dwarvish, Giant, Gnoll, Ignan, Infernal, Orcish.
\end{list}
\raceentry{Human}
\quot{``Yeah, I'm pretty normal.''}

\listone
	\item Medium Size
	\item 30' movement.
	\item Humanoid Type (Human subtype)
	\item 1 extra feat at 1st level.
	\item 4 extra skill points at 1st level and 1 extra skill point at each additional level.
	\item Favored Class: Any. When determining whether a multiclass human takes an experience point penalty, his or her highest-level class does not count.
	\item Automatic Language: Common. 
	\item Bonus Languages: Any (other than secret languages, such as Druidic). See the Speak Language skill.
\end{list}
\raceentry{Kobold}
\quot{``Aieeeeeeeee!''}

\listone
		\item Small Size
		\item 30' movement (despite small size)
		\item Humanoid Type (Reptilian subtype)
		\item Darkvision: Kobolds can see up to 60 feet in the dark.
		\item -4 Strength, +2 Dexterity, -2 Constitution
		\item Racial Skills: A kobold character has a +2 racial bonus on Craft (trapmaking), Profession (miner), and Search checks.
		\item +1 natural armor bonus.
		\item Light sensitivity: Kobolds are dazzled in bright sunlight or within the radius of a daylight spell. 
		\item Favored Class: Sorcerer.
		\item Automatic Languages: Draconic.
		\item Bonus Languages: Common, Undercommon.
\end{list}
\raceentry{Orc}
\quot{``Waaarrrggghhhh!"}

\listone
    \item Medium Size
    \item 30' movement.
    \item Humanoid Type (Orc subtype)
    \item Darkvision: Orcs can see up to 60 feet in the dark.
    \item +4 Strength, -2 Intelligence, -2 Charisma, -2 Wisdom
    \item Daylight Sensitivity: While in brightly lit surroundings (such as a daylight spell), an Orc suffers the dazzled condition and is thus at a -1 penalty to attack rolls and precision-based skill checks.
    \item +2 bonus to saving throws vs. Poison and Disease.
    \item Immunity to ingested poisons.
    \item +2 to Jump and Survival checks.
    \item Favored Classes: Barbarian and Cleric
    \item Automatic Languages: Orc, Common
    \item Bonus Languages: Dwarvish, Elvish, Giant, Gnoll, Goblin, Sylvan, Undercommon.
\end{list}
\raceentry{Tiefling}
\quot{``My ancestors were more evil than you will ever know, but let's see how I compare.''}

\listone
    \item Medium Size
    \item 30' movement.
    \item Outsider Type (Native and Human subtype)
    \item Darkvision: Tieflings can see up to 60 feet in the dark.
    \item +2 Dexterity, +2 Intelligence, -2 Charisma
    \item Tieflings with a Charisma of at least 10 may cast darkness as a spell-like ability with a caster level equal to their character level once per day.
    \item +2 bonus to Bluff, Hide, and Move Silently checks.
    \item Favored Classes: Rogue and True Fiend
    \item Automatic Languages: Common
    \item Bonus Languages: Abyssal, Aquan, Auran, Celestial, Formian, Ignan, Slaad, Sylvan, Terran.
\end{list}
%%%%%%%%%%%%%%%%%%%%%%%%
%%Class Chapter Formatting
%%%%%%%%%%%%%%%%%%%%%%%%

\newcommand{\class}{placeholder}
%Holds the class's name, as defined by \classentry

\newenvironment{classpreamble}{
%\centering
%\rowcolors{1}{colorone}{colortwo}
%\begin{tabu} to \textwidth {X}
}{
%\end{tabu}
}

\newcommand{\desc}[1]{\noindent#1}

\newcommand{\playingaclass}[1]{\indent\ability{Playing a \class}{#1}}

\newcommand{\alignment}[1]{\indent\ability{Alignment}{#1}}

\newcommand{\races}[1]{\indent\ability{Races}{#1}}

\newcommand{\startinggold}[1]{\indent\ability{Starting Gold}{#1}}

\newcommand{\startingage}[1]{\indent\ability{Starting Age}{#1}}

\newcommand{\hitdie}[1]{\indent\ability{Hit Die}{#1}}

\newcommand{\classskills}[1]{\indent\ability{Class Skills}{The {\class}'s class skills (and the key ability for each skill) are #1}}

\newcommand{\skillpoints}[1]{\indent\ability{Skill Points per Level}{#1 + Intelligence Bonus}}


%\newcommand{\desc}[1]{ #1 \\}
%\newcommand{\playingaclass}[1]{\selectfont\ability{Playing a \class : }{#1}\\}
%\newcommand{\hitdie}[1]{\ability{Hit Die: }{#1}\\}
%\newcommand{\alignment}[1]{\ability{Alignment: }{#1}\\}
%\newcommand{\races}[1]{\ability{Races: }{#1}\\}
%\newcommand{\startinggold}[1]{\ability{Starting Gold: }{#1}\\}
%\newcommand{\startingage}[1]{\ability{Starting Age: }{#1}\\}  
%\newcommand{\skillpoints}[1]{\ability{Skill Points per Level: }{#1 + Intelligence Bonus}\\}
%\newcommand{\classskills}[1]{\ability{Class Skills: }{The {\class}'s class skills (and the key ability for each skill) are #1}\\}


\newcommand{\startclassfeatures}{
 \vspace{0.5em}\smallskip\noindent All of the following are class features of the \class ~class.}
%place before actual class features entries.

\newcommand{\proficiencies}[1]{
 \ability{Weapon and Armor Proficiencies}{The \class ~is proficient with #1}}
%Displays proficiencies with minimal input, implimentation looks like \proficiencies{the proficiencies}

\newcommand{\classfeature}[2]{
  \ability{#1}{#2}}
%No functional difference from \ability currently

%%%%Class Table Commands

\newcommand{\gbab}{\empty}
\newcommand{\mbab}{\empty}
\newcommand{\fort}{\empty}
\newcommand{\refl}{\empty}
\newcommand{\will}{\empty}
%Creates new commands for use in \ifx statements for formatting purposes.

\newcommand{\goodbab}{\renewcommand{\gbab}{\empty}\renewcommand{\mbab}{a}}
\newcommand{\modebab}{\renewcommand{\gbab}{a}\renewcommand{\mbab}{\empty}}
\newcommand{\poorbab}{\renewcommand{\gbab}{a}\renewcommand{\mbab}{a}}
%A set of commands to tell LaTeX what BAB progression the class has. Only one should be called per class.

\newcommand{\goodfor}{\renewcommand{\fort}{\empty}}
\newcommand{\poorfor}{\renewcommand{\fort}{a}}
%A set of commands to tell LaTeX what Fortitude progression the class has. Only one should be called per class.

\newcommand{\goodref}{\renewcommand{\refl}{\empty}}
\newcommand{\poorref}{\renewcommand{\refl}{a}}
%A set of commands to tell LaTeX what Reflex progression the class has. Only one should be called per class.

\newcommand{\goodwil}{\renewcommand{\will}{\empty}}
\newcommand{\poorwil}{\renewcommand{\will}{a}}
%A set of commands to tell LaTeX what Will progression the class has. Only one should be called per class.

\newenvironment{classtable}[1]
{
\begin{table}[tpb]
\centering
\rowcolors{1}{colorone}{colortwo}
\begin{tabu} to \textwidth {p{.275in} l p{0.275in} p{0.275in} p{0.275in} X l l l l} 
\rowcolor{headercolor} Level & Base Attack & Fort. & Ref. & Will & Special #1 \\
}{
\hline
\end{tabu}
\end{table}
}
%A a new environment that sets up the class tables. Include the \level commands between \begin{classtable}.

\newenvironment{minorcastingclasstable}
{
%\table[htb]
%\center
\centering
\rowcolors{1}{colorone}{colortwo}
\begin{tabu}to \textwidth{p{.275in}lp{0.275in}p{0.275in}p{0.275in}Xccccccc}
\rowcolor{headercolor} & & & & & &\multicolumn{7}{c}{Spells Per Day (By Level)} \\
\rowcolor{headercolor} Level & Base Attack & Fort. & Ref. & Will & Special &0&1&2&3&4&5&6\\
}{
\hline
\end{tabu}
%\endcenter
%\endtable
}
%A a new environment similar to classtable, but with columns for a minor (zero through six) spell slot progression.

\newenvironment{fullcastingclasstable}
{
\table[htb]
\center
\rowcolors{1}{colorone}{colortwo}
\begin{tabu}to \textwidth{p{.275in}lp{0.275in}p{0.275in}p{0.275in}Xcccccccccc}
\rowcolor{headercolor} & & & & & &\multicolumn{10}{c}{Spells Per Day (By Level)} \\
\rowcolor{headercolor} Level & Base Attack & Fort. & Ref. & Will & Special &0&1&2&3&4&5&6&7&8&9\\
}{
\hline
\end{tabu}
\endcenter
\endtable
}
%Another environment for class tables, this one for full (0 through 9) spell slot progression.

\newcommand{\levelone}[1]{
\hline
1st  & \ifx\gbab\isempty +1 \else\ifx\mbab\isempty +0 \else +0 \fi \fi
	 & \ifx\fort\isempty +2 \else +0 \fi
	 & \ifx\refl\isempty +2 \else +0 \fi
	 & \ifx\will\isempty +2 \else +0 \fi
	 & #1 \\}
%A command that declares a table row within the class feature table.

\newcommand{\leveltwo}[1]{
2nd  & \ifx\gbab\isempty +2 \else\ifx\mbab\isempty +1 \else +1 \fi \fi
	 & \ifx\fort\isempty +3 \else +0 \fi
	 & \ifx\refl\isempty +3 \else +0 \fi
	 & \ifx\will\isempty +3 \else +0 \fi
	 & #1 \\}
%A command that declares a table row within the class feature table.

\newcommand{\levelthree}[1]{
3rd  & \ifx\gbab\isempty +3 \else\ifx\mbab\isempty +2 \else +1 \fi \fi
	 & \ifx\fort\isempty +3 \else +1 \fi
	 & \ifx\refl\isempty +3 \else +1 \fi
	 & \ifx\will\isempty +3 \else +1 \fi
	 & #1 \\}
%A command that declares a table row within the class feature table.

\newcommand{\levelfour}[1]{
4th  & \ifx\gbab\isempty +4 \else\ifx\mbab\isempty +3 \else +2 \fi \fi
	 & \ifx\fort\isempty +4 \else +1 \fi
	 & \ifx\refl\isempty +4 \else +1 \fi
	 & \ifx\will\isempty +4 \else +1 \fi
	 & #1 \\}
%A command that declares a table row within the class feature table.

\newcommand{\levelfive}[1]{
5th  & \ifx\gbab\isempty +5 \else\ifx\mbab\isempty +3 \else +2 \fi \fi
	 & \ifx\fort\isempty +4 \else +1 \fi
	 & \ifx\refl\isempty +4 \else +1 \fi
	 & \ifx\will\isempty +4 \else +1 \fi
	 & #1 \\}
%A command that declares a table row within the class feature table.
	 
\newcommand{\levelsix}[1]{
6th  & \ifx\gbab\isempty +6/+1 \else\ifx\mbab\isempty +4 \else +3 \fi \fi
	 & \ifx\fort\isempty +5 \else +2 \fi
	 & \ifx\refl\isempty +5 \else +2 \fi
	 & \ifx\will\isempty +5 \else +2 \fi
	 & #1 \\}
%A command that declares a table row within the class feature table.

\newcommand{\levelseven}[1]{
7th  & \ifx\gbab\isempty +7/+2 \else\ifx\mbab\isempty +5 \else +3 \fi \fi
	 & \ifx\fort\isempty +5 \else +2 \fi
	 & \ifx\refl\isempty +5 \else +2 \fi
	 & \ifx\will\isempty +5 \else +2 \fi
	 & #1 \\}
%A command that declares a table row within the class feature table.
	 
\newcommand{\leveleight}[1]{
8th  & \ifx\gbab\isempty +8/+3 \else\ifx\mbab\isempty +6/+1 \else +4 \fi \fi
	 & \ifx\fort\isempty +6 \else +2 \fi
	 & \ifx\refl\isempty +6 \else +2 \fi
	 & \ifx\will\isempty +6 \else +2 \fi
	 & #1 \\}
%A command that declares a table row within the class feature table.
	 
\newcommand{\levelnine}[1]{
9th  & \ifx\gbab\isempty +9/+4 \else\ifx\mbab\isempty +6/+1 \else +4 \fi \fi
	 & \ifx\fort\isempty +6 \else +3 \fi
	 & \ifx\refl\isempty +6 \else +3 \fi
	 & \ifx\will\isempty +6 \else +3 \fi
	 & #1 \\}
%A command that declares a table row within the class feature table.
	 
\newcommand{\levelten}[1]{
10th & \ifx\gbab\isempty +10/+5 \else\ifx\mbab\isempty +7/+2 \else +5 \fi \fi
	 & \ifx\fort\isempty +7 \else +3 \fi
	 & \ifx\refl\isempty +7 \else +3 \fi
	 & \ifx\will\isempty +7 \else +3 \fi
	 & #1 \\}
%A command that declares a table row within the class feature table.
	 
\newcommand{\leveleleven}[1]{
11th & \ifx\gbab\isempty +11/+6/+6 \else\ifx\mbab\isempty +8/+3 \else +5 \fi \fi
	 & \ifx\fort\isempty +7 \else +3 \fi
	 & \ifx\refl\isempty +7 \else +3 \fi
	 & \ifx\will\isempty +7 \else +3 \fi
	 & #1 \\}
%A command that declares a table row within the class feature table.
	 
\newcommand{\leveltwelve}[1]{
12th & \ifx\gbab\isempty +12/+7/+7 \else\ifx\mbab\isempty +9/+4 \else +6/+1 \fi \fi
	 & \ifx\fort\isempty +8 \else +4 \fi
	 & \ifx\refl\isempty +8 \else +4 \fi
	 & \ifx\will\isempty +8 \else +4 \fi
	 & #1 \\}
%A command that declares a table row within the class feature table.
	 
\newcommand{\levelthirteen}[1]{
13th & \ifx\gbab\isempty +13/+8/+8 \else\ifx\mbab\isempty +9/+4 \else +6/+1 \fi \fi
	 & \ifx\fort\isempty +8 \else +4 \fi
	 & \ifx\refl\isempty +8 \else +4 \fi
	 & \ifx\will\isempty +8 \else +4 \fi
	 & #1 \\}
%A command that declares a table row within the class feature table.
	 
\newcommand{\levelfourteen}[1]{
14th & \ifx\gbab\isempty +14/+9/+9 \else\ifx\mbab\isempty +10/+5 \else +7/+2 \fi \fi
	 & \ifx\fort\isempty +9 \else +4 \fi
	 & \ifx\refl\isempty +9 \else +4 \fi
	 & \ifx\will\isempty +9 \else +4 \fi
	 & #1 \\}
%A command that declares a table row within the class feature table.
	 
\newcommand{\levelfifteen}[1]{
15th & \ifx\gbab\isempty +15/+10/+10 \else\ifx\mbab\isempty +11/+6/+6 \else +7/+2 \fi \fi
	 & \ifx\fort\isempty +9 \else +5 \fi
	 & \ifx\refl\isempty +9 \else +5 \fi
	 & \ifx\will\isempty +9 \else +5 \fi
	 & #1 \\}
%A command that declares a table row within the class feature table.
	 
\newcommand{\levelsixteen}[1]{
16th & \ifx\gbab\isempty +16/+11/+11/+11 \else\ifx\mbab\isempty +12/+7/+7 \else +8/+3 \fi \fi
	 & \ifx\fort\isempty +10 \else +5 \fi
	 & \ifx\refl\isempty +10 \else +5 \fi
	 & \ifx\will\isempty +10 \else +5 \fi
	 & #1 \\}
%A command that declares a table row within the class feature table.
	 
\newcommand{\levelseventeen}[1]{
17th & \ifx\gbab\isempty +17/+12/+12/+12 \else\ifx\mbab\isempty +12/+7/+7 \else +8/+3 \fi \fi
	 & \ifx\fort\isempty +10 \else +5 \fi
	 & \ifx\refl\isempty +10 \else +5 \fi
	 & \ifx\will\isempty +10 \else +5 \fi
	 & #1 \\}
%A command that declares a table row within the class feature table.
	 
\newcommand{\leveleighteen}[1]{
18th & \ifx\gbab\isempty +18/+13/+13/+13 \else\ifx\mbab\isempty +13/+8/+8 \else +9/+4 \fi \fi
	 & \ifx\fort\isempty +11 \else +6 \fi
	 & \ifx\refl\isempty +11 \else +6 \fi
	 & \ifx\will\isempty +11 \else +6 \fi
	 & #1 \\}
%A command that declares a table row within the class feature table.
	 
\newcommand{\levelnineteen}[1]{
19th & \ifx\gbab\isempty +19/+14/+14/+14 \else\ifx\mbab\isempty +14/+9/+9 \else +9/+4 \fi \fi
	 & \ifx\fort\isempty +11 \else +6 \fi
	 & \ifx\refl\isempty +11 \else +6 \fi
	 & \ifx\will\isempty +11 \else +6 \fi
	 & #1 \\}
%A command that declares a table row within the class feature table.
	 
\newcommand{\leveltwenty}[1]{
20th & \ifx\gbab\isempty +20/+15/+15/+15 \else\ifx\mbab\isempty +15/+10/+10 \else +10/+5 \fi \fi
	 & \ifx\fort\isempty +12 \else +6 \fi
	 & \ifx\refl\isempty +12 \else +6 \fi
	 & \ifx\will\isempty +12 \else +6 \fi
	 & #1 \\}
%A command that declares a table row within the class feature table.

\newmdenv[hidealllines=true,backgroundcolor=gray!20]{optionbox}

\newcommand{\option}[1]{
  \renewmdenv[hidealllines=true,backgroundcolor=colorone]{optionbox}
   \begin{optionbox}\noindent{#1}\end{optionbox}
   ~\\*
}

\newenvironment{optional}{
\colorlet{colortwo}{white}
\colorlet{colorone}{gray!15}
}

%%%%%%%%%%%%%%%%%%%%%%%%

\chapter{Classes}
\input{phb/base-classes/class-basics}
\section{Core Classes}
%%%%%%%%%%%%%%%%%%%%%%%%%%%%%%%%%%%%%%%%%%%%%%%%%%
\classentry{Assassin}
%%%%%%%%%%%%%%%%%%%%%%%%%%%%%%%%%%%%%%%%%%%%%%%%%%

%%%%%%%%%%%%%%%%%%%%%%%%%
\Requirements
%%%%%%%%%%%%%%%%%%%%%%%%%

To qualify to become an assassin, a character must fulfill all the following criteria.

\textbf{Alignment:} Any evil.

\textbf{Skills:} \linkskill{Disguise} 4 ranks, \linkskill{Hide} 8 ranks, \linkskill{Move Silently} 8 ranks.

\textbf{Special:} The character must kill someone for no other reason than to join 
the assassins.

%%%%%%%%%%%%%%%%%%%%%%%%%
\Basics
%%%%%%%%%%%%%%%%%%%%%%%%%

\textbf{Hit Die:} d6.

\textbf{Class Skills}

The assassin's class skills (and the key ability for each skill) are \linkskill{Balance} (Dex), 
\linkskill{Bluff} (Cha), \linkskill{Climb} (Str), \linkskill{Craft} (Int), \linkskill{Decipher Script} (Int), \linkskill{Diplomacy} (Cha), 
\linkskill{Disable Device} (Int), \linkskill{Disguise} (Cha), \linkskill{Escape Artist} (Dex), \linkskill{Forgery} (Int),
\linkskill{Gather Information} (Cha), \linkskill{Hide} (Dex), \linkskill{Intimidate} (Cha), \linkskill{Jump} (Str), \linkskill{Listen} (Wis), \linkskill{Move Silently} (Dex), \linkskill{Open Lock} (Dex), \linkskill{Search} (Int), \linkskill{Sense Motive} (Wis), \linkskill{Sleight of Hand} 
(Dex), \linkskill{Spot} (Wis), \linkskill{Swim} (Str), \linkskill{Tumble} (Dex), \linkskill{Use Magic Device} (Cha), and \linkskill{Use Rope} 
(Dex). 

\textbf{Skill Points at Each Level:} 4 + Int modifier.

\begin{table}[htb]
\rowcolors{1}{white}{offyellow}
\caption{The Assassin}
\centering
\begin{tabular}{*{6}{l}*{4}{c}}
\textbf{Level} & \textbf{BAB} & \textbf{Fort} & \textbf{Reflex} & \textbf{Will} & \textbf{Special} & \textbf{1st} & \textbf{2nd} & \textbf{3rd} & \textbf{4th} \\
1st & +0 & +0 & +2 & +0 & Sneak Attack +1d6, Death Attack, Poison Use, Spells & 0 & - & - & - \\
2nd & +1 & +0 & +3 & +0 & +1 save against poison, Uncanny Dodge & 1 & - & - & - \\
3rd & +2 & +1 & +3 & +1 & Sneak Attack +2d6 & 2 & 0 & - & - \\
4th & +3 & +1 & +4 & +1 & +2 save against poison & 3 & 1 & - & - \\
5th & +3 & +1 & +4 & +1 & Sneak Attack +3d6, Improved Uncanny Dodge & 3 & 2 & 0 & - \\
6th & +4 & +2 & +5 & +2 & +3 save against poison & 3 & 3 & 1 & - \\
7th & +5 & +2 & +5 & +2 & Sneak Attack +4d6 & 3 & 3 & 2 & 0 \\
8th & +6 & +2 & +6 & +2 & +4 save against poison, Hide In Plain Sight & 3 & 3 & 3 & 1 \\
9th & +6 & +3 & +6 & +3 & Sneak Attack +5d6 & 3 & 3 & 3 & 2 \\
10th & +7 & +3 & +7 & +3 & +5 save against poison & 3 & 3 & 3 & 3 \\
\end{tabular}
\end{table}

%%%%%%%%%%%%%%%%%%%%%%%%%
\ClassFeatures
%%%%%%%%%%%%%%%%%%%%%%%%%

All of the following are Class Features of the assassin prestige class.

\textbf{Weapon and Armor Proficiency:} Assassins are proficient with the crossbow 
(hand, light, or heavy), dagger (any type), dart, rapier, sap, shortbow (normal 
and composite), and short sword. Assassins are proficient with light armor but 
not with shields.

\textbf{Sneak Attack:} This is exactly like the rogue ability of the same name. 
The extra damage dealt increases by +1d6 every other level (2nd, 4th, 6th, 8th, 
and 10th). If an assassin gets a sneak attack bonus from another source the bonuses 
on damage stack.

\textbf{Death Attack:} If an assassin studies his victim for 3 rounds and then 
makes a sneak attack with a melee weapon that successfully deals damage, the sneak 
attack has the additional effect of possibly either paralyzing or killing the target 
(assassin's choice). While studying the victim, the assassin can undertake other 
actions so long as his attention stays focused on the target and the target does 
not detect the assassin or recognize the assassin as an enemy. If the victim of 
such an attack fails a Fortitude save (DC 10 + the assassin's class level + the 
assassin's Int modifier) against the kill effect, she dies. If the saving throw 
fails against the paralysis effect, the victim is rendered helpless and unable 
to act for 1d6 rounds plus 1 round per level of the assassin. If the victim's saving 
throw succeeds, the attack is just a normal sneak attack. Once the assassin has 
completed the 3 rounds of study, he must make the death attack within the next 
3 rounds.

If a death attack is attempted and fails (the victim makes her save) or if the 
assassin does not launch the attack within 3 rounds of completing the study, 3 
new rounds of study are required before he can attempt another death attack.

\textbf{Poison Use:} Assassins are trained in the use of poison and never risk 
accidentally poisoning themselves when applying poison to a blade.

\textbf{Spells:} Beginning at 1st level, an assassin gains the ability to cast 
a number of arcane spells. To cast a spell, an assassin must have an Intelligence 
score of at least 10 + the spell's level, so an assassin with an Intelligence of 
10 or lower cannot cast these spells. Assassin bonus spells are based on Intelligence, 
and saving throws against these spells have a DC of 10 + spell level + the assassin's 
Intelligence bonus. When the assassin gets 0 spells per day of a given spell level 
he gains only the bonus spells he would be entitled to based on his Intelligence 
score for that spell level.

The assassin's spell list appears below. An assassin casts spells just as a bard 
does.

Upon reaching 6th level, at every even-numbered level after that (8th and 10th), 
an assassin can choose to learn a new spell in place of one he already knows. The 
new spell's level must be the same as that of the spell being exchanged, and it 
must be at least two levels lower than the highest-level assassin spell the assassin 
can cast. An assassin may swap only a single spell at any given level, and must 
choose whether or not to swap the spell at the same time that he gains new spells 
known for that level.

\begin{table}[htb]
\rowcolors{1}{white}{offyellow}\mcinherit
\caption{Assassin Spells Known}
\centering
\begin{tabular}{l*{4}{c}}
\textbf{Level} & \textbf{1st} & \textbf{2nd} & \textbf{3rd} & \textbf{4th}\\
1st & 2\textsuperscript{1} & - & - & -\\
2nd & 3 & - & - & -\\
3rd & 3 & 2\textsuperscript{1} & - & -\\
4th & 4 & 3 & - & -\\
5th & 4 & 3 & 2\textsuperscript{1} & -\\
6th & 4 & 4 & 3 & -\\
7th & 4 & 4 & 3 & 2\textsuperscript{1}\\
8th & 4 & 4 & 4 & 3\\
9th & 4 & 4 & 4 & 3\\
10th & 4 & 4 & 4 & 4\\
\multicolumn{5}{p{7cm}}{\textsuperscript{1}Provided the assassin has sufficient Intelligence to have a bonus spell of this level.}\\
\end{tabular}
\end{table}

\textbf{Save Bonus against Poison:} The assassin gains a natural saving throw bonus 
to all poisons gained at 2nd level that increases by +1 for every two additional 
levels the assassin gains.

\textbf{Uncanny Dodge (Ex):} Starting at 2nd level, an assassin retains his Dexterity 
bonus to AC (if any) regardless of being caught flat-footed or struck by an invisible 
attacker. (He still loses any Dexterity bonus to AC if immobilized.)

If a character gains uncanny dodge from a second class the character automatically 
gains improved uncanny dodge (see below).

\textbf{Improved Uncanny Dodge (Ex):} At 5th level, an assassin can no longer be 
flanked, since he can react to opponents on opposite sides of him as easily as 
he can react to a single attacker. This defense denies rogues the ability to use 
flank attacks to sneak attack the assassin. The exception to this defense is that 
a rogue at least four levels higher than the assassin can flank him (and thus sneak 
attack him).

If a character gains uncanny dodge (see above) from a second class the character 
automatically gains improved uncanny dodge, and the levels from those classes stack 
to determine the minimum rogue level required to flank the character.

\textbf{Hide in Plain Sight (Su):} At 8th level, an assassin can use the Hide skill 
even while being observed. As long as he is within 10 feet of some sort of shadow, 
an assassin can hide himself from view in the open without having anything to actually 
hide behind.

He cannot, however, hide in his own shadow.

%%%
\subsubsection{Assassin Spell List}
%%%

Assassins choose their spells from the following list:

\textbf{1st Level:} \linkspell{Disguise Self}, \linkspell{Detect Poison}, \linkspell{Feather Fall}, \linkspell{Ghost Sound}, \linkspell{Jump}, \linkspell{Obscuring Mist}, \linkspell{Sleep}, \linkspell{True Strike}.

\textbf{2nd Level:} \linkspell{Alter Self}, \linkspell{Cat's Grace}, \linkspell{Darkness}, \linkspell{Fox's Cunning}, \linkspell{Illusory Script}, \linkspell{Invisibility}, \linkspell{Pass Without Trace}, \linkspell{Spider Climb}, \linkspell{Undetectable Alignment}.

\textbf{3rd Level:} \linkspell{Deep Slumber}, \linkspell{Deeper Darkness}, \linkspell{False Life}, \linkspell{Magic Circle Against Good}, \linkspell{Misdirection}, \linkspell{Nondetection}.

\textbf{4th Level:} \linkspell{Clairaudience/Clairvoyance}, \linkspell{Dimension Door}, \linkspell{Freedom of Movement}, \linkspell{Glibness}, \linkspell{Greater Invisibility}, \linkspell{Locate Creature}, \linkspell{Modify Memory}, \linkspell{Poison}.


\classname{Barbarian} \label{class:barbarian}
\vspace{-8pt}
\quot{"My name is Sharptooth of the Wolf Tribe. Your women, lands, and riches are mine."}

\ability{Playing a Barbarian:}{Playing a Barbarian is actually very easy. In general, you hit things, and they fall down. A Barbarian's action in almost any circumstance can plausibly be "I hit it with my great axe!" As such, a Barbarian character can be a good method to introduce a new player to the game or kill some orcs when you've had a few glasses of brew.}

\ability{Alignment:}{Every alignment has its share of Barbarians, however more Barbarians are of Chaotic alignment than of Lawful Alignment.}

\ability{Races:}{Anybody can become a barbarian, and in areas with little in the way of civilization, a lot of people do.}

\ability{Starting Gold:}{4d6x10 gp (140 gold)}

\ability{Starting Age:}{As Barbarian.}

\ability{Hit Die:}{d12}

\ability{Class Skills:}{The Barbarian's class skills (and the key ability for each skill) are Balance (Dex), Climb (Str), Hide (Dex), Intimidate (Cha), Jump (Str), Knowledge: Nature (Int), Listen (Wis), Move Silently (Dex), Sense Motive (Wis), Spot (Wis), Survival (Wis), and Swim (Str).}

\ability{Skills/Level:}{4 + Intelligence Bonus}

\begin{table}[htb]
\begin{small}
\begin{tabular}{lp{3cm}p{0.7cm}p{0.7cm}p{0.7cm}l}
Level  &Base Attack Bonus &Fort Save &Ref Save &Will Save &Special\\
1st &+1 &+2 &+0 &+0 &Rage, Fast Healing 1\\
2nd &+2 &+3 &+0 &+0 &Rage Dice +1d6, Combat Movement +5'\\
3rd &+3 &+3 &+1 &+1 &Battle Hardened\\
4th &+4 &+4 &+1 &+1 &Rage Dice +2d6, Combat Movement +10'\\
5th &+5 &+4 &+1 &+1 &Sidestep Hazards , Fast Healing 5\\
6th &+6/+1 &+5 &+2 &+2 &Rage Dice +3d6, Combat Movement +15'\\
7th &+7/+2 &+5 &+2 &+2 &Great Blows\\
8th &+8/+3 &+6 &+2 &+2 &Rage Dice +4d6, Combat Movement +20'\\
9th &+9/+4 &+6 &+3 &+3 &Great Life\\
10th &+10/+5 &+7 &+3 &+3 &Rage Dice +5d6, Combat Movement +25', Fast Healing 10\\
11th &+11/+6/+6 &+7 &+3 &+3 &Call the Horde\\
12th &+12/+7/+7 &+8 &+4 &+4 &Rage Dice +6d6, Combat Movement +30'\\
13th &+13/+8/+8 &+8 &+4 &+4 &Watched by Totems\\
14th &+14/+9/+9 &+9 &+4 &+4 &Rage Dice +7d6, Combat Movement +35'\\
15th &+15/+10/+10 &+9 &+5 &+5 &Primal Assault, Fast Healing 15\\
16th &+16/+11/+11/+11 &+10 &+5 &+5 &Rage Dice +8d6, Combat Movement +40'\\
17th &+17/+12/+12/+12 &+10 &+5 &+5 &Savagery\\
18th &+18/+13/+13/+13 &+11 &+6 &+6 &Rage Dice +9d6, Combat Movement +45'\\
19th &+19/+14/+14/+14 &+11 &+6 &+6 &One With The Beast\\
20th &+20/+15/+15/+15 &+12 &+6 &+6 &Rage Dice +10d6, Combat Movement +50', Fast Healing 20\\
\end{tabular}
\end{small}
\end{table}

\smallskip\noindent All of the following are Class Features of the Barbarian class.


\ability{Weapon and Armor Proficiency:}{Barbarians are proficient with simple weapons, martial weapons, light armor, medium armor and with shields.}

\ability{Rage (Ex):}{When doing melee damage to a foe or being struck by a foe, a Barbarian may choose to enter a Rage as an immediate action. While Raging, a Barbarian gains a +2 morale bonus to hit and damage in melee combat and may apply any Rage Dice he has to his melee damage rolls. He also gains a +2 to saves, a -2 to AC, and he gains DR X/-- with "X" being equal to half his Barbarian level +2 (rounded down). For example, a 1st level Barbarian has DR 3/-- while Raging and a 10th level Barbarian has DR 7/-- while Raging. While Raging, a Barbarian may not cast spells, activate magic items, use spell-like abilities, or drop his weapons or shield. Rage lasts until he has neither struck an enemy for three consecutive rounds nor suffered damage from an enemy for three consecutive rounds. He may voluntarily end a Rage as a full-round action.}

\ability{Fast Healing:}{Barbarians shrug off wounds that would cripple a lesser man, and have learned to draw upon deep reserves of energy and stamina. At 1st level, they gain Fast Healing 1. At 5th level this becomes Fast Healing 5, Fast Healing 10 at 10th level, Fast Healing 15 at 15th level, and Fast Healing 20 at 20th level. This healing only applies while he is not raging. \smallskip

If a Barbarian ever multiclasses, he permanently loses this ability. A multiclass character does not gain this ability.  A character with 4 or more levels of Barbarian gains this ability even if multiclassed.}

\ability{Rage Dice:}{While Raging, a Barbarian may add these dice of damage to each of his melee attacks. These dice are not multiplied by damage multipliers, and are not applied to any bonus attacks beyond those granted by Base Attack Bonus. These dice are not sneak attack dice, and do not count as sneak attack dice for the prerequisites of prestige classes or feats.}

\ability{Combat Movement:}{While Raging, a Barbarian moves faster in combat, and may add his Combat Movement to his speed when he takes a move action to move.}

\ability{Battle Hardened:}{At 3th level, a Raging Barbarian's mind has been closed off from distractions by the depths of his bloodlust and battle fury. While Raging, he may use his Fortitude Save in place of his Will Save. If he is under the effects of a compulsion or fear effect, he may act normally while Raging as if he was inside a \spell{protection from evil} effect.}

\ability{Sidestep Hazards (Ex):}{At 5th level, a Raging Barbarian learns to sidestep hazards with an intuitive and primal danger sense. While Raging, he may use his Fortitude Save in place of his Reflex Save.}

\ability{Great Blows (Ex):}{At 7th level, a Raging Barbarian's melee attacks are Great Blows. Any enemy struck by the Barbarian's melee or thrown weapon attacks must make a Fort Save or be stunned for one round. No enemy can be targeted by this ability more than once a round, and the save DC for this ability is 10 + half the Barbarian's HD + his Constitution modifier.}

\ability{Great Life (Ex):}{While Raging, a 9th level Barbarian is immune to nonlethal damage, death effects, stunning, critical hits, negative levels, and ability damage (but not ability drain).}

\ability{Call the Horde (Ex):}{An 11th level Barbarian becomes a hero of his people. He gains the Command feat as a bonus feat, but his followers must be Barbarians. In campaigns that do not use Leadership feats, he instead gains a +2 unnamed bonus to all saves.}

\ability{Watched by Totems (Ex):}{At 13th level, a Barbarian may immediately reroll any failed save. He may do this no more than once per failed save.}

\ability{Primal Assault (Ex):}{At 15th level, a Raging Barbarian may choose to radiate an effect similar to an \spell{antimagic field} when he enters a Rage, with a caster level equal to his HD. Unlike a normal antimagic field, this effect does not suppress magic effects on him or the effects of magic items he is wearing or holding.}

\ability{Savagery (Ex):}{At 17th level, a Raging Barbarian may take a full round action to make a normal melee attack that has an additional effect similar to a \spell{mordenkainen's disjunction}. Unlike a normal \spell{mordenkainen's disjunction}, this effect only targets a single item or creature struck.}

\ability{One With The Beast:}{At 19th level, a Barbarian no longer needs to be in a Rage to use any Barbarian ability.}

%\input{Bard}
\classentry{Cleric}
\modebab
\goodfor
\poorref
\goodwil
\quot{``Fear my righteous shining holy beacon of... righteousness?''}

\desc{\class s are the holy (or unholy) warriors, standing fast against the darkness (or light). They are also made of cheese, and thus a prime target for minmaxxers.}

\playingaclass{The \class ~class can fit many different playstyles, but all \class s should have a high Wisdom score.}

\alignment{A \class 's alignment must be within one step of his deity's (that is, it may be one step away on either the lawful-chaotic axis or the good-evil axis, but not both). A \class ~may not be neutral unless his deity's alignment is also neutral.}

\races{Any.}

\startinggold{5d4x10gp (125 Gold)}

\startingage{ <-starting age, often written as a class reference like "As Rogue."-> }

\hitdie{d8}

\classskills{Concentration (Con), Craft (Int), Diplomacy (Cha), Heal (Wis), Knowledge (arcana) (Int), Knowledge (history) (Int), Knowledge (religion) (Int), Knowledge (the planes) (Int), Profession (Wis), and Spellcraft (Int).}

\ability{Domains and Class Skills}{A \class ~who chooses the Animal or Plant domain adds Knowledge (nature) (Int) to the \class ~class skills listed above. A \class ~who chooses the Knowledge domain adds all Knowledge (Int) skills to the list. A \class ~who chooses the Travel domain adds Survival (Wis) to the list. A \class ~who chooses the Trickery domain adds Bluff (Cha), Disguise (Cha), and Hide (Dex) to the list. See Deity, Domains, and Domain Spells, below, for more information.}

\skillpoints{2}

\begin{fullcastingclasstable}
\levelone{Turn or Rebuke Undead  &3 &1 &- &- &- &- &- &- &- &- }
\leveltwo{&4 &2 &- &- &- &- &- &- &- &- }
\levelthree{&4 &2 &1 &- &- &- &- &- &- &- }
\levelfour{&5 &3 &2 &- &- &- &- &- &- &- }
\levelfive{&5 &3 &2 &1 &- &- &- &- &- &- }
\levelsix{&5 &3 &3 &2 &- &- &- &- &- &- }
\levelseven{&6 &4 &3 &2 &1 &- &- &- &- &- }
\leveleight{&6 &4 &3 &3 &2 &- &- &- &- &- }
\levelnine{&6 &4 &4 &3 &2 &1 &- &- &- &- }
\levelten{&6 &4 &4 &3 &3 &2 &- &- &- &- }
\leveleleven{&6 &5 &4 &4 &3 &2 &1 &- &- &- }
\leveltwelve{&6 &5 &4 &4 &3 &3 &2 &- &- &- }
\levelthirteen{&6 &5 &5 &4 &4 &3 &2 &1 &- &- }
\levelfourteen{&6 &5 &5 &4 &4 &3 &3 &2 &- &- }
\levelfifteen{&6 &5 &5 &5 &4 &4 &3 &2 &1 &- }
\levelsixteen{&6 &5 &5 &5 &4 &4 &3 &3 &2 &- }
\levelseventeen{&6 &5 &5 &5 &5 &4 &4 &3 &2 &1 }
\leveleighteen{&6 &5 &5 &5 &5 &4 &4 &3 &3 &2 }
\levelnineteen{&6 &5 &5 &5 &5 &5 &4 &4 &3 &3 }
\leveltwenty{&6 &5 &5 &5 &5 &5 &4 &4 &4 &4 }
\end{fullcastingclasstable}

%Theres no way the table would fit with the '+1's in the spells per day section, so I took it out, the text is pretty clear that you get an extra spell slot for your domain.  Feel free to break it into two tables if you think its best.

\startclassfeatures

\proficiencies{all types of armor (light, medium, and heavy), and with shields (except tower shields).

\smallskip\noindent A \class ~who chooses the War domain receives the Weapon Focus feat related to his deity's weapon as a bonus feat. He also receives the appropriate Martial Weapon Proficiency feat as a bonus feat, if the weapon falls into that category.}

\classfeature{Aura (Ex)}{A \class ~of a chaotic, evil, good, or lawful deity has a particularly powerful aura corresponding to the deity's alignment (see the detect evil spell for details). \class s who don't worship a specific deity but choose the Chaotic, Evil, Good, or Lawful domain have a similarly powerful aura of the corresponding alignment.}

\classfeature{Spells}{A \class ~casts divine spells, which are drawn from the \class ~spell list. However, his alignment may restrict him from casting certain spells opposed to his moral or ethical beliefs; see Chaotic, Evil, Good, and Lawful Spells, below. A \class ~must choose and prepare his spells in advance (see below). To prepare or cast a spell, a \class ~must have a Wisdom score equal to at least 10 + the spell level. The Difficulty Class for a saving throw against a \class 's spell is 10 + the spell level + the \class 's Wisdom modifier. Like other spellcasters, a \class ~can cast only a certain number of spells of each spell level per day. His base daily spell allotment is given on Table: The \class. In addition, he receives bonus spells per day if he has a high Wisdom score. 

\smallskip\noindent A \class ~also gets one domain spell of each spell level he can cast, starting at 1st level. When a \class ~prepares a spell in a domain spell slot, it must come from one of his two domains (see Deities, Domains, and Domain Spells, below). \class smeditate or pray for their spells. Each \class ~must choose a time at which he must spend 1 hour each day in quiet contemplation or supplication to regain his daily allotment of spells. 

\smallskip\noindent Time spent resting has no effect on whether a \class ~can prepare spells. A \class ~may prepare and cast any spell on the \class ~spell list, provided that he can cast spells of that level, but he must choose which spells to prepare during his daily meditation.}

\classfeature{Deity, Domains, and Domain Spells}{A \class 's deity influences his alignment, what magic he can perform, his values, and how others see him. A \class ~chooses two domains from among those belonging to his deity. A \class ~can select an alignment domain (Chaos, Evil, Good, or Law) only if his alignment matches that domain. If a \class ~is not devoted to a particular deity, he still selects two domains to represent his spiritual inclinations and abilities. The restriction on alignment domains still applies. Each domain gives the \class ~access to a domain spell at each spell level he can cast, from 1st on up, as well as a granted power. The \class ~gets the granted powers of both the domains selected. With access to two domain spells at a given spell level, a \class ~prepares one or the other each day in his domain spell slot. If a domain spell is not on the \class ~spell list, a \class ~can prepare it only in his domain spell slot.}

\classfeature{Spontaneous Casting}{A good \class ~(or a neutral \class ~of a good deity) can channel stored spell energy into healing spells that the \class ~did not prepare ahead of time. The \class ~can ``lose'' any prepared spell that is not a domain spell in order to cast any cure spell of the same spell level or lower (a cure spell is any spell with ``cure'' in its name). An evil \class ~(or a neutral \class ~of an evil deity), can't convert prepared spells to cure spells but can convert them to inflict spells (an inflict spell is one with ``inflict'' in its name). A \class ~who is neither good nor evil and whose deity is neither good nor evil can convert spells to either cure spells or inflict spells (player's choice). Once the player makes this choice, it cannot be reversed. This choice also determines whether the \class ~turns or commands undead (see below).}

\classfeature{Chaotic, Evil, Good, and Lawful Spells}{A \class ~can't cast spells of an alignment opposed to his own or his deity's (if he has one). Spells associated with particular alignments are indicated by the chaos, evil, good, and law descriptors in their spell descriptions.}

\classfeature{Turn or Rebuke Undead (Su)}{Any \class, regardless of alignment, has the power to affect undead creatures by channeling the power of his faith through his holy (or unholy) symbol (see Turn or Rebuke Undead). A good \class ~(or a neutral \class ~who worships a good deity) can turn or destroy undead creatures. An evil \class ~(or a neutral \class ~who worships an evil deity) instead rebukes or commands such creatures. A neutral \class ~of a neutral deity must choose whether his turning ability functions as that of a good \class ~or an evil \class. Once this choice is made, it cannot be reversed. This decision also determines whether the \class ~can cast spontaneous cure or inflict spells (see above). A \class ~may attempt to turn undead a number of times per day equal to 3 + his Charisma modifier. A \class ~with 5 or more ranks in Knowledge (religion) gets a +2 bonus on turning checks against undead.}

\classfeature{Bonus Languages}{A \class s bonus language options include Celestial, Abyssal, and Infernal (the languages of good, chaotic evil, and lawful evil outsiders, respectively). These choices are in addition to the bonus languages available to the character because of his race.}

\classfeature{Ex-\class ~s}{A \class ~who grossly violates the code of conduct required by his god loses all spells and class features, except for armor and shield proficiencies and proficiency with simple weapons. He cannot thereafter gain levels as a \class ~of that god until he atones (see the atonement spell description).}

%\input{Druid}
\classname{Fighter} \label{class:fighter}
\vspace{-8pt}
\quot{"I've seen this kind of fire-breathing chicken-demon before. We're going to need more rope. Also a bigger cart."}

\desc{The Fighter is a versatile combatant who is able to actively disrupt the activities of his enemies. Fighters represent plucky heroes and grizzled veterans, but they always appear to surmount impossible odds. Which means in retrospect that the odds weren't all that impossible. At least, not for someone with a Fighter's talents.}

\ability{Playing a Fighter:}{Fighters are often handed to beginning players in order to help them learn the ropes. This is a cruel practice that dates back to when the Fighter was explicitly a weak class that players were forced to play to the (quit proximate) death if for whatever reason they didn't roll well enough on their stats to play a real character. The Fighter described here is not the hazing ritual of old, but it is a more complicated character than many others, being the non-magical equivalent to the Wizard. Beginning characters should probably be given a Barbarian, Conduit, or Rogue character to introduce them to the game mechanics of D\&D.}

\desc{A Fighter has an answer for virtually any circumstance and a great deal of adaptability and flexibility, and benefits greatly from being played by a player who actually knows how far a Roper's strands or a Beholder's rays reach. The Fighter character is archetypically a character who uses her opponent's limitations against them, and it really slows down play if the player needs to have those limitations explained during combat. As such, a full classed Fighter is recommended for experienced players of the game.}

\desc{That being said, a level or two of Fighter can give some breadth and resilience to almost any martial build, and makes a good multiclassing dip even (sometimes especially) for inexperienced players.}

\ability{Alignment:}{Every alignment has its share of Fighters, however more Fighters are of Lawful alignment than of Chaotic Alignment.}

\ability{Races:}{Every humanoid race has warriors, but actual Fighters are rarer in societies that don't value logistics and planning. So while there are many Fighters among the Hobgoblins, Dwarves, and Fire Giants, a Fighter is rarely seen among the ranks of the Orcs, Gnomes, or Ogres.}

\ability{Starting Gold:}{6d6x10 gp (210 gold)}

\ability{Starting Age:}{As Fighter.}

\ability{Hit Die:}{d10}

\ability{Class Skills:}{The Fighter's class skills (and the key ability for each skill) are Balance (Dex), Bluff (Cha), Climb (Str), Craft (Int), Diplomacy (Cha), Escape Artist (Dex), Handle Animal (Cha), Intimidate (Cha), Jump (Str), Knowledge (all skills individually) (Int), Listen (Wis), Move Silently (Dex), Profession (Wis), Ride (Dex), Sense Motive (Wis), Spot (Wis), Survival (Wis), Swim (Str), Tumble (Dex), and Use Rope (Dex).}

\ability{Skills/Level:}{6 + Intelligence Bonus}

\begin{table}[tbh]
\begin{small}
\begin{tabular}{lp{3cm}p{0.7cm}p{0.7cm}p{0.7cm}l}
Level  &Base Attack Bonus &Fort Save &Ref Save &Will Save &Special\\
1st &+1 &+2 &+2 &+2 &Weapons Training, Combat Focus\\
2nd &+2 &+3 &+3 &+3 &Bonus Feat\\
3rd &+3 &+3 &+3 &+3 &Problem Solver, Pack Mule\\
4th &+4 &+4 &+4 &+4 &Bonus Feat\\
5th &+5 &+4 &+4 &+4 &Logistics Mastery, Active Assualt\\
6th &+6/+1 &+5 &+5 &+5 &Bonus Feat\\
7th &+7/+2 &+5 &+5 &+5 &Forge Lore, Improved Delay\\
8th &+8/+3 &+6 &+6 &+6 &Bonus Feat\\
9th &+9/+4 &+6 &+6 &+6 &Foil Action\\
10th &+10/+5 &+7 &+7 &+7 &Bonus Feat\\
11th &+11/+6/+6 &+7 &+7 &+7 &Lunging Attacks\\
12th &+12/+7/+7 &+8 &+8 &+8 &Bonus Feat\\
13th &+13/+8/+8 &+8 &+8 &+8 &Array of Stunts\\
14th &+14/+9/+9 &+9 &+9 &+9 &Bonus Feat\\
15th &+15/+10/+10 &+9 &+9 &+9 &Greater Combat Focus\\
16th &+16/+11/+11/+11 &+10 &+10 &+10 &Bonus Feat\\
17th &+17/+12/+12/+12 &+10 &+10 &+10 &Improved Foil Action\\
18th &+18/+13/+13/+13 &+11 &+11 &+11 &Bonus Feat\\
19th &+19/+14/+14/+14 &+11 &+11 &+11 &Intense Focus, Supreme Combat Focus\\
20th &+20/+15/+15/+15 &+12 &+12 &+12 &Bonus Feat\\
\end{tabular}
\end{small}
\end{table}


\smallskip\noindent All of the following are Class Features of the Fighter class.

\ability{Weapon and Armor Proficiency:}{Fighters are proficient with all simple and Martial Weapons. Fighters are proficient with Light, Medium, and Heavy Armor and with Shields and Great Shields.}

\ability{Weapons Training (Ex):}{Fighters train obsessively with armor and weapons of all kinds, and using a new weapon is easy and fun. By practicing with a weapon he is not proficient with for a day, a Fighter may permanently gain proficiency with that weapon by succeeding at an Intelligence check DC 10 (you may not take 10 on this check).}

\ability{Combat Focus (Ex):}{A Fighter is at his best when the chips are down and everything is going to Baator in a handbasket. When the world is on fire, a Fighter keeps his head better than anyone. If the Fighter is in a situation that is stressful and/or dangerous enough that he would normally be unable to "take 10" on skill checks, he may spend a Swift Action to gain Combat Focus. A Fighter may end his Combat Focus at any time to reroll any die roll he makes, and if not used it ends on its own after a number of rounds equal to his Base Attack Bonus.}

\ability{Problem Solver (Ex):}{A Fighter of 3rd level can draw upon his intense and diverse training to respond to almost any situation. As a Swift action, he may choose any [Combat] feat he meets the prerequisites for and use it for a number of rounds equal to his base attack bonus. This ability may be used once per hour.}

\ability{Pack Mule (Ex):}{Fighters are used to long journeys with a heavy pack and the use of a wide variety of weaponry and equipment. A 3rd level Fighter suffers no penalties for carrying a medium load, and may retrieve stored items from his person without provoking an attack of opportunity.}

\ability{Logistics Mastery (Ex):}{Fighters are excellent and efficient logisticians. When a Fighter reaches 5th level, he gains a bonus to his Command Rating equal to one third his Fighter Level.}

\ability{Active Assault (Ex):}{A 5th level Fighter can flawlessly place himself where he is most needed in combat. He may take a 5 foot step as an immediate action. This is in addition to any other movement he takes during his turn, even another 5 foot step.}

\ability{Forge Lore:}{A 7th level Fighter can produce magical weapons and equipment as if he had a Caster Level equal to his ranks in Craft.}

\ability{Improved Delay (Ex):}{A Fighter of 7th level may delay his action in one round without compromising his Initiative in the next round. In addition, a Fighter may interrupt another action with his delayed action like it was a readied action (though he does not have to announce his intentions before hand).}

\ability{Foil Action (Ex):}{A 9th level Fighter may attempt to monkeywrench any action an opponent is taking. The Fighter may throw sand into a beholder's eye, bat aside a key spell component, or strike a weapon hand with a thrown object, but the result is the same: the opponent's action is wasted, and any spell slots, limited ability uses, or the like used to power it are expended. A Fighter must be within 30 feet of his opponent to use this ability, and must hit with a touch attack or ranged touch attack. Using Foil Action is an Immediate action. At 17th level, Foil Action may be used at up to 60 feet.}

\ability{Lunging Attacks (Ex):}{The battlefield is an extremely dangerous place, and 11th level Fighters are expected to hold off Elder Elementals, Hezrous, and Hamatulas. Fighters of this level may add 5 feet to the reach of any of their weapons.}

\ability{Array of Stunts (Ex):}{A 13th level Fighter may take one extra Immediate Action between his turns without sacrificing a Swift action during his next turn.}

\ability{Greater Combat Focus (Ex):}{At 15th level, a Fighter may voluntarily expend his Combat Focus as a non-action to suppress any status effect or ongoing spell effect on himself for his Base Attack Bonus in rounds.}

\ability{Intense Focus (Ex):}{A 19th level Fighter may take an extra Swift Action each round (in addition to the extra Immediate Action he can take from Array of Stunts).}

\ability{Supreme Combat Focus (Ex):}{A 19th level Fighter may expend his Combat Focus as a non-action to take 20 on any die roll. He must elect to use Supreme Combat Focus before rolling the die.}

\classentry{Knight}
\goodbab
\poorfor
\poorref
\goodwil
\quot{``Do you hear me you big lizard? You unhand that young man this instant!''}

\desc{Knights are more than a social position; in fact many knights don't have any social standing at all. These knight errants uphold the values of honor, and make a name for themselves adventuring.}

\playingaclass{A Knight has the potential to dish out tremendous damage to a single opponent, and it is tempting to think of them as monster killers. However, it is best to realize in advance that the Knight does not often realize their tremendous damage output. The threat of the Knight's Designate Opponent ability is just that -- a threat. A Knight excels at defensive tasks, and attacking a Knight is often one of the least effective options an opponent might exercise.

So by making it be a logical combat action for your opponents to attack your party's defensive expert, you've really contributed a lot to the party. A Knight can take a lot of the heat off the rest of the party. So don't get frustrated if enemies constantly interrupt your Designate Opponent action -- that's the whole point. A Knight's role is to protect others, and the best way you can do that is to provide a legitimate threat to your opponents.}

\alignment{Many Knights are Lawful. But not all of them. You have to maintain your code of conduct, but plenty of Chaotic creatures can do that too.}

\races{Knights require a fairly social background to receive their training. After all, a solitary creature generally has little use for honor. As such, while Knights often spend tremendous amounts of time far from civilization, they are almost exclusively recruited from the ranks of races that are highly urban in nature.}

\startinggold{6d6x10 gp (210 gold)}

\startingage{ <-starting age, often written as a class reference like "As Rogue."-> }

\hitdie{d12}

\classskills{ Climb (Str), Craft (Int), Diplomacy (Cha), Handle Animal (Cha), Intimidate (Cha), Jump (Str), Knowledge (History, Nobility, and Geography) (Int), Listen (Wis), Perform (Cha), Ride (Dex), Sense Motive (Wis), Spot (Wis), and Swim (Str).}

\skillpoints{4}

\begin{classtable}{}
\levelone{Designate Opponent, Mounted Combat, Code of Conduct}
\leveltwo{Damage Reduction}
\levelthree{Energy Resistance, Speak to Animals}
\levelfour{Immunity to Fear, Knightly Spirit}
\levelfive{Command}
\levelsix{Defend Others, Quick Recovery}
\levelseven{Bastion of Defense, Draw Fire}
\leveleight{Mettle, Spell Shield}
\levelnine{Sacrifice}
\levelten{Knightly Order}
\end{classtable}

\startclassfeatures

\proficiencies{simple weapons and Martial Weapons. Knights are proficient with Light, Medium, and Heavy Armor, Shields and Great Shields.}

\classfeature{Designate Opponent (Ex)}{As a Swift Action, a Knight may mark an opponent as their primary foe. This foe must be within medium range and be able to hear the Knight's challenge. If the target creature inflicts ay damage on the Knight before the Knight's next turn, the attempt fails. Otherwise, any attacks the Knight uses against the opponent during her next turn inflict an extra d6 of damage for each Knight level. This effect ends at the end of her next turn, or when she has struck her opponent a number of times equal to the number of attacks normally allotted her by her Base Attack Bonus. \smallskip

\emph{Example: Vayn is a 6th level Knight presently benefiting from a haste spell, granting her an extra attack during a Full Attack action. On her turn she designates an Ettin as her primary opponent, and the Ettin declines to attack her during the ensuing turn. When her next turn comes up, she uses a Full Attack and attacks 3 times. The first two hits inflict an extra 6d6 of damage, and then she designates the Ettin as her opponent again. It won't soon ignore her!}}

\classfeature{Mounted Combat}{A Knight gains Mounted Combat as a bonus feat at 1st level. If she already has Mounted Combat, she may gain any Combat feat she meets the prerequisites for instead.}

\classfeature{Code of Conduct}{A Knight must fight with honor even when her opponents do not. Indeed, a Knight subscribes to honor to a degree far more than that which is strictly considered necessary by other honorable characters. Actions which even hint at the appearance of impropriety are anathema to the Knight:

\begin{awesomelist}
    \item A Knight must not accept undue assistance from allies even in combat. A Knight must refuse bonuses from Aid Another actions.
    \item A Knight must refrain from the use poisons of any kind, even normally acceptable poisons such as blade toxins.
    \item A Knight may not voluntarily change shape, whether she is impersonating a specific creature or not.
    \item A Knight may not sell Magic Items.
\end{awesomelist}

A Knight who fails to abide by her code of conduct loses the ability to use any of her Knightly abilities which require actions until she atones.}

\classfeature{Damage Reduction (Ex)}{A Knight trains to suffer the unbearable with chivalry and grace. At 2nd level, she gains Damage Reduction of X/-, where X is half her Knight level, rounded down.}

\classfeature{Energy Resistance (Ex)}{A Knight may protect herself from energy types that she expects. As a Swift Action, a 3rd level Knight may grant herself Energy Resistance against any energy type she chooses equal to her Knight Level plus her Shield Bonus. This energy resistance lasts until she spends a Swift Action to choose another Energy type or her Shield bonus is reduced.}

\classfeature{Speak to Animals (Ex)}{A Knight can make herself understood by beasts. Her steed always seems to be able to catch the thrust of anything she says. A 3rd level Knight gains a bonus to any of her Ride and Handle Animal checks equal to half her Knight Level. In addition, there is no limit to how many tricks she can teach a creature, and her her Handle Animal checks are not penalized for attempting to get a creature to perform a trick it does not know.}

\classfeature{Immunity to Fear (Ex)}{At 4th level, a Knight becomes immune to [Fear] effects.}

\classfeature{Knightly Spirit (Ex)}{As a Move Equivalent Action, a 4th level Knight may restore any amount of attribute damage or drain that she has suffered.}

\classfeature{Command}{A Knight gains Command as a bonus feat at level 5.}

\classfeature{Defend Others (Ex)}{A 6th level Knight may use her own body to defend others. Any ally adjacent to the Knight gains Evasion, though she does not.}

\classfeature{Quick Recovery (Ex)}{If a 6th level Knight is stunned or dazed during her turn, that condition ends at the end of that turn. \smallskip

\emph{Example: Vayn is hit by a mindblast and would be stunned for 7 turns. She misses her next action and then shakes off the condition ready to fight.}}

\classfeature{Bastion of Defense (Ex)}{A 7th level Knight can defend others with great facility. All adjacent allies except the Knight gain a +2 Dodge bonus to their Armor Class and Reflex Saves.}

\classfeature{Draw Fire (Ex)}{A 7th level Knight can exploit the weaknesses of unintelligent opponents. With a Swift Action, she may pique the interest of any mindless opponent within medium range. That creature must make a Willpower Save (DC 10 + \half\  Hit Dice + Constitution Modifier) or spend all of its actions moving towards or attacking the Knight. This effect ends after a number of rounds equal to the Knight's class level.}

\classfeature{Mettle (Ex)}{An 8th level Knight who succeeds at a Fortitude Partial or Willpower Partial save takes no effect as if she had immunity. \smallskip

\emph{For example, if Vayn was hit with an inflict wounds spell and made her saving throw, she would take no damage instead of the partial effect in the spell description (half damage in this case).}}

\classfeature{Spell Shield (Ex)}{An 8th level Knight gains Spell Resistance of 5 + her character level. This Spell Resistance is increased by her shield bonus to AC if she has one.}

\classfeature{Sacrifice (Ex)}{As an immediate action, a 9th level Knight may make herself the target of an attack or targeted effect that targets any creature within her reach.}

\classfeature{Knightly Order}{What is a powerful Knight without a descriptive adjective? Upon reaching 10th level, a Knight must join or found a Knightly order. From this point on, she may ignore one of the prerequisites for joining a Knightly Order prestige class. In addition, becoming a member of an order has special meaning for a 10th level Knight, and she gains an ability related to the order she joins. Some sample orders are listed below:}

\begin{awesomelist}
    \item \ability{Angelic Knight}{The Angelic Knights are a transformational order that attempts to live by the precepts of the upper planes. An Angelic Knight gains wings that allow her to fly at double her normal speed with perfect maneuverability. Also an Angelic Knight benefits from protection from evil at all times.}
    \item \ability{Bane Knight}{The Bane Knights stand for running around burning the countryside with extreme burning. Bane Knights are immune to fire and do not have to breathe. In addition, a Bane Knight may set any unattended object on fire with a Swift Action at up to Medium Range.}
    \item \ability{Chaos Knight}{Chaos Knights stand for madness and Giant Frog. With the powers of Giant Frog, they can Giant Frog. Also their natural armor bonus increases by +5 and they are immune to sleep effects.}
    \item \ability{Dragon Knight}{Dedicated to the Platinum Dragon, the Dragon Knights serve love and justice in equal measure as dishes to those who need them. A Dragon Knight gains a +5 bonus to Sense Motive and any armor she wears has its enhancement bonus increased to +5 (it also gains a platinum sheen in the process, and as a side effect a Dragon Knight is never dirty for more than a few seconds).}
    \item \ability{Elemental Knight}{The Elemental Knights may be dedicated to a particular element, or somehow dedicated to all of them. An Elemental Knight can planeshift at will to any Inner plane or the Prime Material plane. Also, she is immune to stunning and always benefits from attune form when on any Inner Plane.}
    \item \ability{Fey Knight}{Using the powers of the Sprites, the Fey Knight has many fairy strengths. Firstly, she gains DR 10/Iron. Also, any of her attacks may do non-lethal damage at any time if this is desired. Also she never ages and does not need to drink.}
    \item \ability{Great Knight}{Clad in opulent armor, the Great Knight cares only for her own power. The Great Knight gains a +4 bonus on Disarm or Sunder tests, and gains a +4 Profane bonus to her Strength.}
    \item \ability{Hell Knight}{Forged in the sulphurous clouds of Baator, the Hell Knight is bathed in an evil radiance. In addition to being granted a ceremonial weapon made of green steel, a Hell Knight gains the coveted see in darkness ability of the Baatorians. Also, she has an inherent ability to know what every creature within 60' of finds most repugnant.}
    \item \ability{Imperial Knight}{The great Empire needs champions able to unswervingly support its interests, and the Imperial Knight is one of the best. She may impose a zone of truth at will as a Supernatural ability, and all of her attacks are Lawfully aligned. Also, she continuously benefits from \spell{magic circle against chaos}.}
\end{awesomelist}


%%%%%%%%%%%%%%%%%%%%%%%%%%%%%%%%%%%%%%%%%%%%%%%%%%
\classentry{Monk}
%%%%%%%%%%%%%%%%%%%%%%%%%%%%%%%%%%%%%%%%%%%%%%%%%%

\textbf{Alignment:} Any lawful.

\textbf{Hit Die:} d8.

\textbf{Class Skills}

The monk's class skills (and the key ability for each skill) are \linkskill{Balance} (Dex), 
\linkskill{Climb} (Str), \linkskill{Concentration} (Con), \linkskill{Craft} (Int), \linkskill{Diplomacy} (Cha), \linkskill{Escape Artist} (Dex), 
\linkskill{Hide} (Dex), \linkskill{Jump} (Str), \linkskill{Knowledge} (arcana) (Int), \linkskill{Knowledge} (religion) (Int), \linkskill{Listen} 
(Wis), \linkskill{Move Silently} (Dex), \linkskill{Perform} (Cha), \linkskill{Profession} (Wis), \linkskill{Sense Motive} (Wis), 
\linkskill{Spot} (Wis), \linkskill{Swim} (Str), and \linkskill{Tumble} (Dex).

\textbf{Skill Points at 1st Level:} (4 + Int modifier) x4.

\textbf{Skill Points at Each Additional Level:} 4 + Int modifier.

\begin{table}[htb]
\rowcolors{1}{white}{offyellow}
\caption{The Monk}
\centering
\begin{tabular}{*{7}{l}}
\textbf{Level} & \textbf{BAB} & \textbf{Fort} & \textbf{Reflex} & \textbf{Will} & \textbf{Special} & \textbf{Speed Bonus}\\
1st & +0 & +2 & +2 & +2 & Bonus Feat, Flurry Of Blows, Unarmed Strike & +0ft\\
2nd & +1 & +3 & +3 & +3 & Bonus Feat, Evasion & +0ft\\
3rd & +2 & +3 & +3 & +3 & Still Mind & +10ft\\
4th & +3 & +4 & +4 & +4 & Ki Strike (Magic), Slow Fall 20ft & +10ft\\
5th & +3 & +4 & +4 & +4 & Purity Of Body & +10ft\\
6th & +4 & +5 & +5 & +5 & Bonus Feat, Slow Fall 30ft & +20ft\\
7th & +5 & +5 & +5 & +5 & Wholeness of Body & +20ft\\
8th & +6 & +6 & +6 & +6 & Slow Fall 40ft & +20ft\\
9th & +6 & +6 & +6 & +6 & Improved Evasion & +30ft\\
10th & +7 & +7 & +7 & +7 & Ki Strike (Lawful), Slow Fall 50ft & +30ft\\
11th & +8 & +7 & +7 & +7 & Diamond Body, Greater Flurry & +30ft\\
12th & +9 & +8 & +8 & +8 & Abundant Step, Slow Fall 60ft & +40ft\\
13th & +9 & +8 & +8 & +8 & Diamond Soul & +40ft\\
14th & +10 & +9 & +9 & +9 & Slow Fall 70ft & +40ft\\
15th & +11 & +9 & +9 & +9 & Quivering Palm & +50ft\\
16th & +12 & +10 & +10 & +10 & Ki Strike (Adamantine), Slow Fall 80ft & +50ft\\
17th & +12 & +10 & +10 & +10 & Timeless Body, Tongue of The Sun And Moon & +50ft\\
18th & +13 & +11 & +11 & +11 & Slow Fall 90ft & +60ft\\
19th & +14 & +11 & +11 & +11 & Empty Body & +60ft\\
20th & +15 & +12 & +12 & +12 & Perfect Self, Slow Fall Any Distance & +60ft\\
\end{tabular}
\end{table}

%%%%%%%%%%%%%%%%%%%%%%%%%
\ClassFeatures
%%%%%%%%%%%%%%%%%%%%%%%%%

All of the following are class features of the monk.

\textbf{Weapon and Armor Proficiency:} Monks are proficient with club, crossbow 
(light or heavy), dagger, handaxe, javelin, kama, nunchaku, quarterstaff, sai, 
shuriken, siangham, and sling.

Monks are not proficient with any armor or shields.

When wearing armor, using a shield, or carrying a medium or heavy load, a monk 
loses her AC bonus, as well as her fast movement and flurry of blows abilities.

\textbf{AC Bonus (Ex):} When unarmored and unencumbered, the monk adds her Wisdom 
bonus (if any) to her AC. In addition, a monk gains a +1 bonus to AC at 5th level. 
This bonus increases by 1 for every five monk levels thereafter (+2 at 10th, +3 
at 15th, and +4 at 20th level).

These bonuses to AC apply even against touch attacks or when the monk is flat-footed. 
She loses these bonuses when she is immobilized or helpless, when she wears any 
armor, when she carries a shield, or when she carries a medium or heavy load.

\textbf{Flurry of Blows (Ex):} When unarmored, a monk may strike with a flurry 
of blows at the expense of accuracy. When doing so, she may make one extra attack 
in a round at her highest base attack bonus, but this attack takes a -2 penalty, 
as does each other attack made that round. The resulting modified base attack bonuses 
are shown in the Flurry of Blows Attack Bonus column on Table: The Monk. This penalty 
applies for 1 round, so it also affects attacks of opportunity the monk might make 
before her next action. When a monk reaches 5th level, the penalty lessens to -1, 
and at 9th level it disappears. A monk must use a full attack action to strike 
with a flurry of blows.

When using flurry of blows, a monk may attack only with unarmed strikes or with 
special monk weapons (kama, nunchaku, quarterstaff, sai, shuriken, and siangham). 
She may attack with unarmed strikes and special monk weapons interchangeably as 
desired. When using weapons as part of a flurry of blows, a monk applies her Strength 
bonus (not Str bonus x1-1/2 or x1/2) to her damage rolls for all successful 
attacks, whether she wields a weapon in one or both hands. The monk can't use any 
weapon other than a special monk weapon as part of a flurry of blows.

In the case of the quarterstaff, each end counts as a separate weapon for the purpose 
of using the flurry of blows ability. Even though the quarterstaff requires two 
hands to use, a monk may still intersperse unarmed strikes with quarterstaff strikes, 
assuming that she has enough attacks in her flurry of blows routine to do so. 

When a monk reaches 11th level, her flurry of blows ability improves. In addition 
to the standard single extra attack she gets from flurry of blows, she gets a second 
extra attack at her full base attack bonus.

\textbf{Unarmed Strike:} At 1st level, a monk gains Improved Unarmed Strike as 
a bonus feat. A monk's attacks may be with either fist interchangeably or even 
from elbows, knees, and feet. This means that a monk may even make unarmed strikes 
with her hands full. There is no such thing as an off-hand attack for a monk striking 
unarmed. A monk may thus apply her full Strength bonus on damage rolls for all 
her unarmed strikes.

Usually a monk's unarmed strikes deal lethal damage, but she can choose to deal 
nonlethal damage instead with no penalty on her attack roll. She has the same choice 
to deal lethal or nonlethal damage while grappling.

A monk's unarmed strike is treated both as a manufactured weapon and a natural 
weapon for the purpose of spells and effects that enhance or improve either manufactured 
weapons or natural weapons.

A monk also deals more damage with her unarmed strikes than a normal person of their size would, 
as shown on Table: Monk Unarmed Damage.

\begin{table}[htb]
\rowcolors{1}{white}{offyellow}
\caption{Monk Unarmed Damage}
\centering
\begin{tabular}{l c c c}
\textbf{Level} & \textbf{Small} & \textbf{Medium} & \textbf{Large}\\
1st-3rd & 1d4 & 1d6 & 1d8 \\
4th-7th & 1d6 & 1d8 & 2d6 \\
8th-11th & 1d8 & 1d10 & 2d8 \\
12th-15th & 1d10 & 2d6 & 3d6 \\
16th-19th & 2d6 & 2d8 & 3d8 \\
20th & 2d8 & 2d10 & 4d8 \\
\end{tabular}
\end{table}

\textbf{Bonus Feat:} At 1st level, a monk may select either \linkfeat{Improved Grapple} or 
\linkfeat{Stunning Fist} as a bonus feat. At 2nd level, she may select either \linkfeat{Combat Reflexes} 
or \linkfeat{Deflect Arrows} as a bonus feat. At 6th level, she may select either \linkfeat{Improved Disarm}
or \linkfeat{Improved Trip} as a bonus feat. A monk need not have any of the prerequisites 
normally required for these feats to select them.

\textbf{Evasion (Ex):} At 2nd level or higher if a monk makes a successful Reflex 
saving throw against an attack that normally deals half damage on a successful 
save, she instead takes no damage. Evasion can be used only if a monk is wearing 
light armor or no armor. A helpless monk does not gain the benefit of evasion.

\textbf{Fast Movement (Ex):} At 3rd level, a monk gains an enhancement bonus to 
her speed, as shown on Table: The Monk. A monk in armor or carrying a medium or 
heavy load loses this extra speed.

\textbf{Still Mind (Ex):} A monk of 3rd level or higher gains a +2 bonus on saving 
throws against spells and effects from the school of enchantment.

\textbf{Ki Strike (Su):} At 4th level, a monk's unarmed attacks 
are empowered with \textit{ki}. Her unarmed attacks are treated as magic weapons 
for the purpose of dealing damage to creatures with damage reduction. Ki 
strike improves with the character's monk level. At 10th level, her unarmed attacks 
are also treated as lawful weapons for the purpose of dealing damage to creatures 
with damage reduction. At 16th level, her unarmed attacks are treated as adamantine 
weapons for the purpose of dealing damage to creatures with damage reduction and 
bypassing hardness.

\textbf{Slow Fall (Ex):} At 4th level or higher, a monk within arm's reach of a 
wall can use it to slow her descent. When first using this ability, she takes damage 
as if the fall were 20 feet shorter than it actually is. The monk's ability to 
slow her fall (that is, to reduce the effective distance of the fall when next 
to a wall) improves with her monk level until at 20th level she can use a nearby 
wall to slow her descent and fall any distance without harm.

\textbf{Purity of Body (Ex):} At 5th level, a monk gains immunity to all diseases 
except for supernatural and magical diseases.

\textbf{Wholeness of Body (Su):} At 7th level or higher, a monk can heal her own 
wounds. She can heal a number of hit points of damage equal to twice her current 
monk level each day, and she can spread this healing out among several uses.

\textbf{Improved Evasion (Ex):} At 9th level, a monk's evasion ability improves. 
She still takes no damage on a successful Reflex saving throw against attacks, 
but henceforth she takes only half damage on a failed save. A helpless monk does 
not gain the benefit of improved evasion.

\textbf{Diamond Body (Su):} At 11th level, a monk gains immunity to poisons of 
all kinds.

\textbf{Abundant Step (Su):} At 12th level or higher, a monk can slip magically 
between spaces, as if using the spell \linkspell{Dimension Door}, once per day. Her 
caster level for this effect is one-half her monk level (rounded down).

\textbf{Diamond Soul (Ex):} At 13th level, a monk gains spell resistance equal 
to her current monk level + 10. In order to affect the monk with a spell, a spellcaster 
must get a result on a caster level check (1d20 + caster level) that equals or 
exceeds the monk's spell resistance.

\textbf{Quivering Palm (Su):} Starting at 15th level, a monk can set up vibrations 
within the body of another creature that can thereafter be fatal if the monk so 
desires. She can use this quivering palm attack once a week, and she must announce 
her intent before making her attack roll. Constructs, oozes, plants, undead, incorporeal 
creatures, and creatures immune to critical hits cannot be affected. Otherwise, 
if the monk strikes successfully and the target takes damage from the blow, the 
quivering palm attack succeeds. Thereafter the monk can try to slay the victim 
at any later time, as long as the attempt is made within a number of days equal 
to her monk level. To make such an attempt, the monk merely wills the target to 
die (a free action), and unless the target makes a Fortitude saving throw (DC 10 
+ 1/2 the monk's level + the monk's Wis modifier), it dies. If the saving throw 
is successful, the target is no longer in danger from that particular quivering 
palm attack, but it may still be affected by another one at a later time.

\textbf{Timeless Body (Ex):} Upon attaining 17th level, a monk no longer takes 
penalties to her ability scores for aging and cannot be magically aged. Any such 
penalties that she has already taken, however, remain in place. Bonuses still accrue, 
and the monk still dies of old age when her time is up.

\textbf{Tongue of the Sun and Moon (Ex):} A monk of 17th level or higher can speak 
with any living creature.

\textbf{Empty Body (Su):} At 19th level, a monk gains the ability to assume an 
ethereal state for 1 round per monk level per day, as though using the spell \linkspell{Etherealness}. 
She may go ethereal on a number of different occasions during any single day, as 
long as the total number of rounds spent in an ethereal state does not exceed her 
monk level.

\textbf{Perfect Self:} At 20th level, a monk becomes a magical creature. She is 
forevermore treated as an outsider rather than as a humanoid (or whatever the monk's 
creature type was) for the purpose of spells and magical effects. Additionally, 
the monk gains damage reduction 10/magic, which allows her to ignore the first 
10 points of damage from any attack made by a nonmagical weapon or by any natural 
attack made by a creature that doesn't have similar damage reduction. Unlike other 
outsiders, the monk can still be brought back from the dead as if she were a member 
of her previous creature type.

%%%%%%%%%%%%%%%%%%%%%%%%%
\subsection{Ex-Monks}
%%%%%%%%%%%%%%%%%%%%%%%%%

A monk who becomes nonlawful cannot gain new levels as a monk but retains all monk 
abilities.

Like a member of any other class, a monk may be a multiclass character, but multiclass 
monks face a special restriction. A monk who gains a new class or (if already multiclass) 
raises another class by a level may never again raise her monk level, though she 
retains all her monk abilities.
%\input{Paladin}
%\input{Ranger}
%\input{Rogue}
%\input{Samurai}
%\input{Sorcerer}
\classentry{Wizard}
\poorbab
\poorfor
\poorref
\goodwil
\quot{``Don't make this wizard mad, don't make this wizard pissed, I can kill a hill giant with a flick of my wrist!'''}

\desc{ }

\playingaclass{Wizards primarily rely on a high Intelligence to learn and cast their spells. Some wizards also rely on having a decent Dexterity score to deliver touch attacks.}

\alignment{Though the study and practice involved in wizardry tends to attract practitioners of a lawful alignment, people of any alignment may become wizards.}

\races{Becoming a wizard takes years of study, and wizards are typically represented by races that are more long lived and by societies that are well organized. That said, people of any race may become wizards.}

\startinggold{3d4x10 gp (75 Gold)}

\startingage{Something needs to go here when we decide what even goes here.}

\hitdie{d4}

\classskills{Concentration (Con), Craft (Int), Decipher Script (Int), Knowledge (all skills, taken individually) (Int), Profession (Wis), and Spellcraft (Int).}

\skillpoints{2}

\begin{fullcastingclasstable}
\levelone{Summon familiar, Scribe Scroll &3 &1 &- &- &- &- &- &- &- &-}
\leveltwo{&4 &2 &- &- &- &- &- &- &- &-}
\levelthree{&4 &2 &1 &- &- &- &- &- &- &-}
\levelfour{&4 &3 &2 &- &- &- &- &- &- &-}
\levelfive{Bonus Feat&4 &3 &2 &1 &- &- &- &- &- &-}
\levelsix{&4 &3 &3 &2 &- &- &- &- &- &-}
\levelseven{&4 &4 &3 &2 &1 &- &- &- &- &-}
\leveleight{&4 &4 &3 &3 &2 &- &- &- &- &-}
\levelnine{&4 &4 &4 &3 &2 &1 &- &- &- &-}
\levelten{&4 &4 &4 &3 &3 &2 &- &- &- &-}
\leveleleven{&4 &4 &4 &4 &3 &2 &1 &- &- &-}
\leveltwelve{&4 &4 &4 &4 &3 &3 &2 &- &- &-}
\levelthirteen{&4 &4 &4 &4 &4 &3 &2 &1 &- &-}
\levelfourteen{&4 &4 &4 &4 &4 &3 &3 &2 &- &-}
\levelfifteen{&4 &4 &4 &4 &4 &4 &3 &2 &1 &-}
\levelsixteen{&4 &4 &4 &4 &4 &4 &3 &3 &2 &-}
\levelseventeen{&4 &4 &4 &4 &4 &4 &4 &3 &2 &1}
\leveleighteen{&4 &4 &4 &4 &4 &4 &4 &3 &3 &2}
\levelnineteen{&4 &4 &4 &4 &4 &4 &4 &4 &3 &3}
\leveltwenty{&4 &4 &4 &4 &4 &4 &4 &4 &4 &4}
\end{fullcastingclasstable}

\startclassfeatures

\proficiencies{the club, dagger, heavy crossbow, light crossbow, and quarterstaff, but not with any type of armor or shield. Armor of any type interferes with a wizard's movements, which can cause her spells with somatic components to fail.}

\classfeature{Spells}{A wizard casts arcane spells which are drawn from the sorcerer/ wizard spell list. A wizard must choose and prepare her spells ahead of time (see below).
To learn, prepare, or cast a spell, the wizard must have an Intelligence score equal to at least 10 + the spell level. The Difficulty Class for a saving throw against a wizard's spell is 10 + the spell level + the wizard's Intelligence modifier.

Like other spellcasters, a wizard can cast only a certain number of spells of each spell level per day. Her base daily spell allotment is given on Table: The Wizard. In addition, she receives bonus spells per day if she has a high Intelligence score.

Unlike a bard or sorcerer, a wizard may know any number of spells. She must choose and prepare her spells ahead of time by getting a good night's sleep and spending 1 hour studying her spellbook. While studying, the wizard decides which spells to prepare.
Bonus Languages: A wizard may substitute Draconic for one of the bonus languages available to the character because of her race.}

\classfeature{Familiar}{A wizard can obtain a familiar in exactly the same manner as a sorcerer can. See the sorcerer description and the information on Familiars below for details.}

\classfeature{Scribe Scroll}{At 1st level, a wizard gains Scribe Scroll as a bonus feat.}

\classfeature{Bonus Feats}{At 5th, 10th, 15th, and 20th level, a wizard gains a bonus feat. At each such opportunity, she can choose a metamagic feat, an item creation feat, or Spell Mastery. The wizard must still meet all prerequisites for a bonus feat, including caster level minimums.

These bonus feats are in addition to the feat that a character of any class gets from advancing levels. The wizard is not limited to the categories of item creation feats, metamagic feats, or Spell Mastery when choosing these feats.}

\classfeature{Spellbooks}{A wizard must study her spellbook each day to prepare her spells. She cannot prepare any spell not recorded in her spellbook, except for read magic, which all wizards can prepare from memory.

A wizard begins play with a spellbook containing all 0-level wizard spells (except those from her prohibited school or schools, if any; see School Specialization, below) plus three 1st-level spells of your choice. For each point of Intelligence bonus the wizard has, the spellbook holds one additional 1st-level spell of your choice. At each new wizard level, she gains two new spells of any spell level or levels that she can cast (based on her new wizard level) for her spellbook. At any time, a wizard can also add spells found in other wizards' spellbooks to her own.}

\subsubsection{School Specialization}

A school is one of eight groupings of spells, each defined by a common theme. If desired, a wizard may specialize in one school of magic (see below). Specialization allows a wizard to cast extra spells from her chosen school, but she then never learns to cast spells from some other schools.

A specialist wizard can prepare one additional spell of her specialty school per spell level each day. She also gains a +2 bonus on Spellcraft checks to learn the spells of her chosen school.

The wizard must choose whether to specialize and, if she does so, choose her specialty at 1st level. At this time, she must also give up two other schools of magic (unless she chooses to specialize in divination; see below), which become her prohibited schools.

A wizard can never give up divination to fulfill this requirement.

Spells of the prohibited school or schools are not available to the wizard, and she can't even cast such spells from scrolls or fire them from wands. She may not change either her specialization or her prohibited schools later.

The eight schools of arcane magic are abjuration, conjuration, divination, enchantment, evocation, illusion, necromancy, and transmutation.

Spells that do not fall into any of these schools are called universal spells.

\begin{awesomelist}
	\item \ability{Abjuration}{Spells that protect, block, or banish. An abjuration specialist is called an abjurer.}
	\item \ability{Conjuration}{Spells that bring creatures or materials to the caster. A conjuration specialist is called a conjurer.}
	\item \ability{Divination}{Spells that reveal information. A divination specialist is called a diviner. Unlike the other specialists, a diviner must give up only one other school.}
	\item \ability{Enchantment}{Spells that imbue the recipient with some property or grant the caster power over another being. An enchantment specialist is called an enchanter.}
	\item \ability{Evocation}{Spells that manipulate energy or create something from nothing. An evocation specialist is called an evoker.}
	\item \ability{Illusion}{Spells that alter perception or create false images. An illusion specialist is called an illusionist.}
	\item \ability{Necromancy}{Spells that manipulate, create, or destroy life or life force. A necromancy specialist is called a necromancer.}
	\item \ability{Transmutation}{Spells that transform the recipient physically or change its properties in a more subtle way. A transmutation specialist is called a transmuter.}
	\item \ability{Universal}{Not a school, but a category for spells that all wizards can learn. A wizard cannot select universal as a specialty school or as a prohibited school. Only a limited number of spells fall into this category.}
\end{awesomelist}

\subsubsection{Familiars}

A familiar is a normal animal that gains new powers and becomes a magical beast when summoned to service by a sorcerer or wizard. It retains the appearance, Hit Dice, base attack bonus, base save bonuses, skills, and feats of the normal animal it once was, but it is treated as a magical beast instead of an animal for the purpose of any effect that depends on its type. Only a normal, unmodified animal may become a familiar. An animal companion cannot also function as a familiar.

A familiar also grants special abilities to its master (a sorcerer or wizard), as given on the table below. These special abilities apply only when the master and familiar are within 1 mile of each other.

Levels of different classes that are entitled to familiars stack for the purpose of determining any familiar abilities that depend on the master's level.

\begin{table}[h]
\rowcolors{1}{colorone}{colortwo}
\begin{tabu}to \textwidth{lX}
\header Familiar & Special \\ \hline
Bat & Master gains a +3 bonus on Listen checks \\
Cat & Master gains a +3 bonus on Move Silently checks \\
Hawk & Master gains a +3 bonus on Spot checks in bright light \\
Lizard & Master gains a +3 bonus on Climb checks \\
Owl & Master gains a +3 bonus on Spot checks in shadows \\
Rat & Master gains a +2 bonus on Fortitude saves \\
Raven\textsuperscript{1} & Master gains a +3 bonus on Appraise checks	\\
Snake\textsuperscript{2} & Master gains a +3 bonus on Bluff checks \\
Toad & Master gains +3 hit points \\
Weasel & Master gains a +2 bonus on Reflex saves \\ \hline
\multicolumn{2}{l}{\textsuperscript{1} A raven familiar can speak one language of its master's choice as a supernatural ability.} \\
\multicolumn{2}{l}{\textsuperscript{2} Tiny viper.} \\ \hline
\end{tabu}
\end{table}

Use the basic statistics for a creature of the familiar's kind, but make the following changes.

\ability{Hit Dice}{For the purpose of effects related to number of Hit Dice, use the master's character level or the familiar's normal HD total, whichever is higher.}

\ability{Hit Points}{The familiar has one-half the master's total hit points (not including temporary hit points), rounded down, regardless of its actual Hit Dice.}

\ability{Attacks}{Use the master's base attack bonus, as calculated from all his classes. Use the familiar's Dexterity or Strength modifier, whichever is greater, to get the familiar's melee attack bonus with natural weapons. Damage equals that of a normal creature of the familiar's kind.}

\ability{Saving Throws}{For each saving throw, use either the familiar's base save bonus (Fortitude +2, Reflex +2, Will +0) or the master's (as calculated from all his classes), whichever is better. The familiar uses its own ability modifiers to saves, and it doesn't share any of the other bonuses that the master might have on saves.}

\ability{Skills}{For each skill in which either the master or the familiar has ranks, use either the normal skill ranks for an animal of that type or the master's skill ranks, whichever are better. In either case, the familiar uses its own ability modifiers. Regardless of a familiar's total skill modifiers, some skills may remain beyond the familiar's ability to use.}

\ability{Familiar Ability Descriptions}{All familiars have special abilities (or impart abilities to their masters) depending on the master's combined level in classes that grant familiars, as shown on the table below. The abilities given on the table are cumulative.}

\begin{awesomelist}
  \item \ability{Natural Armor Adj.}{The number noted here is an improvement to the familiar's existing natural armor bonus.}
  \item \ability{Int}{The familiar's Intelligence score.}
  \item \ability{Alertness (Ex)}{While a familiar is within arm's reach, the master gains the Alertness feat.}
  \item \ability{Improved Evasion (Ex)}{When subjected to an attack that normally allows a Reflex saving throw for half damage, a familiar takes no damage if it makes a successful saving throw and half damage even if the saving throw fails.}
  \item \ability{Share Spells}{At the master's option, he may have any spell (but not any spell-like ability) he casts on himself also affect his familiar. The familiar must be within 5 feet at the time of casting to receive the benefit. If the spell or effect has a duration other than instantaneous, it stops affecting the familiar if it moves farther than 5 feet away and will not affect the familiar again even if it returns to the master before the duration expires. Additionally, the master may cast a spell with a target of ``You'' on his familiar (as a touch range spell) instead of on himself. A master and his familiar can share spells even if the spells normally do not affect creatures of the familiar's type (magical beast).}
  \item \ability{Empathic Link (Su)}{The master has an empathic link with his familiar out to a distance of up to 1 mile. The master cannot see through the familiar's eyes, but they can communicate empathically. Because of the limited nature of the link, only general emotional content can be communicated. Because of this empathic link, the master has the same connection to an item or place that his familiar does.}
  \item \ability{Deliver Touch Spells (Su)}{If the master is 3rd level or higher, a familiar can deliver touch spells for him. If the master and the familiar are in contact at the time the master casts a touch spell, he can designate his familiar as the toucher. The familiar can then deliver the touch spell just as the master could. As usual, if the master casts another spell before the touch is delivered, the touch spell dissipates.}
  \item \ability{Speak with Master (Ex)}{If the master is 5th level or higher, a familiar and the master can communicate verbally as if they were using a common language. Other creatures do not understand the communication without magical help.}
  \item \ability{Speak with Animals of Its Kind (Ex)}{If the master is 7th level or higher, a familiar can communicate with animals of approximately the same kind as itself (including dire varieties): bats with bats, rats with rodents, cats with felines, hawks and owls and ravens with birds, lizards and snakes with reptiles, toads with amphibians, weasels with similar creatures (weasels, minks, polecats, ermines, skunks, wolverines, and badgers). Such communication is limited by the intelligence of the conversing creatures.}
  \item \ability{Spell Resistance (Ex)}{If the master is 11th level or higher, a familiar gains spell resistance equal to the master's level + 5. To affect the familiar with a spell, another spellcaster must get a result on a caster level check (1d20 + caster level) that equals or exceeds the familiar's spell resistance.}
  \item \ability{Scry on Familiar (Sp)}{If the master is 13th level or higher, he may scry on his familiar (as if casting the scrying spell) once per day.}
\end{awesomelist}

\begin{table}[h]
\rowcolors{1}{colorone}{colortwo}
\begin{tabu} to \textwidth{lllX}
\header Master Class Level & Natural Armor Adj. & Int & Special \\ \hline
1st-2nd & +1 & 6 & Alertness, improved evasion, share spells, empathic link \\
3rd-4th & +2 & 7 & Deliver touch spells \\
5th-6th & +3 & 8 & Speak with master \\
7th-8th & +4 & 9 & Speak with animals of its kind \\
9th-10th & +5 & 10 & -\\
11th-12th & +6 & 11 & Spell resistance \\
13th-14th & +7 & 12 & Scry on familiar \\
15th-16th & +8 & 13 & -\\
17th-18th & +9 & 14 &  -\\
19th-20th & +10 & 15 & -\\
\end{tabu}
\end{table}

\subsubsection{Arcane Spells and Armor}

Wizards and sorcerers do not know how to wear armor effectively.

If desired, they can wear armor anyway (though they'll be clumsy in it), or they can gain training in the proper use of armor (with the various Armor Proficiency feats light, medium, and heavy and the Shield Proficiency feat), or they can multiclass to add a class that grants them armor proficiency. Even if a wizard or sorcerer is wearing armor with which he or she is proficient, however, it might still interfere with spellcasting.

Armor restricts the complicated gestures that a wizards or sorcerer must make while casting any spell that has a somatic component (most do). The armor and shield descriptions list the arcane spell failure chance for different armors and shields.

By contrast, bards not only know how to wear light armor effectively, but they can also ignore the arcane spell failure chance for such armor. A bard wearing armor heavier than light or using any type of shield incurs the normal arcane spell failure chance, even if he becomes proficient with that armor.

If a spell doesn't have a somatic component, an arcane spellcaster can cast it with no problem while wearing armor. Such spells can also be cast even if the caster's hands are bound or if he or she is grappling (although Concentration checks still apply normally). Also, the metamagic feat Still Spell allows a spellcaster to prepare or cast a spell at one spell level higher than normal without the somatic component. This also provides a way to cast a spell while wearing armor without risking arcane spell failure. 
%\section{Additional Classes}
\classentry{Templar}

\newcommand{\vow}[6]{
\subsubsection{Vow of #1}
\quot{``#2''}
\listone
	\item \ability{First: }{#3}
	\item \ability{Second: }{#4}
	\item \ability{Third: }{#5}
\end{list}
\vspace{8pt}
\ability{Roleplaying Ideas: }{#6}
}

\newcommand{\faith}[7]{
\subsubsection{#1}
#2
\begin{list}{\textbf{\arabic{counter}}:~}{\itemspace\usecounter{counter}}
	\item #3
	\item #4
	\item #5
	\item #6
	\item #7
\end{list}
\vspace{8pt}
}

\goodbab
\goodfor
\poorref
\goodwil
\quot{``Nobody is more dangerous than he who imagines himself pure in heart, for his purity, by definition, is unassailable.''}

\begin{classpreamble}
\desc{Every religion has clerics, those tasked with performing the duties of the religion. Many also have faithful members who leave their homes to travel distant lands, spreading the word of their god or pantheon. Templars are ordained warriors tasked with spreading the faith and defending the faithful, while also beating down the foes of a deity.
\newline
Templars are the militant arm of their church and/or cause. They are often guards of sacred places, dispatched away from the temples as agents of higher powers, or simply wander to share the virtues of their philosophy and ideal with others. Initially able and zealous warriors combining martial abilities with the power of their deity, they eventually become an active sword or shield for their deity, with high levels of offensive prowess and devastating crowd control. Whether as a bodyguard or a support character, they often find themselves in the ranks of adventuring parties who can make use of the talents.
\newline
A templar generally exemplifies a particular ideology of life, and associated nomenclature may depend on the side with which he aligns himself. A good templar, for instance, might assume the title of paladin while those who embrace evil are often known as blackguards and those who serve neutrality are called gray wardens. What truly differentiates these characters are the vows that they swear to uphold.}
\playingaclass{Templars value Charisma greatly, as it allows them to better convince those they encounter of the importance of their deity and provides force to their spells. They also value Strength as it allows them to beat up those who steadfastly refuse to believe and get in the way of the templar's work. Constitution is often the third most important ability for a templar, as it allows them to stand longer in the fray.}
\alignment{Any, though a templar may only select a deity who allows worshipers of the templar's alignment. Conversely, a templar of a specific deity is limited to only those alignments which would be allowed by the deity for a follower. Templars without a patron deity may select any alignment they like.}
\races{Any. Every race that has deities has templars to spread their teachings.}
\startinggold{3d10x10 gp (165 gp).}
%\startingage{Moderate}
\hitdie{d10}
\classskills{Appraise (Int), Climb (Str), Concentration (Con), Craft (Int), Heal (Wis), Intimidate (Cha), Jump (Str), Knowledge (nobility and royalty) (religion) (Int), Listen (Wis), Ride (Dex), Sense Motive (Wis), Speak Language (None), Spellcraft (Int), Swim (Str).}
\skillpoints{4}
\end{classpreamble}

\afterpage{
\begin{minorcastingclasstable}
\levelone{Divine Vow (Once Vowed), Vow of Piety (Once Vowed)& 						2&-&-&-&-&-&-}
\leveltwo{Avenger of the Faith (Primary)& 															3&-&-&-&-&-&-}
\levelthree{Divine Vow (Once Vowed)& 																3&2&-&-&-&-&-}
\levelfour{Avenger of the Faith (Secondary)& 														3&2&-&-&-&-&-}
\levelfive{Divine Vow (Once Vowed)& 																	3&3&2&-&-&-&-}
\levelsix{Avenger of the Faith (Primary), Arms of the Faithful& 							3&3&2&-&-&-&-}
\levelseven{Divine Vow (Twice Vowed), Vow of Piety (Twice Vowed)& 				3&3&3&2&-&-&-}
\leveleight{Avenger of the Faith (Secondary), Inquisitor& 									3&3&3&2&-&-&-}
\levelnine{Divine Vow (Twice Vowed)& 																3&3&3&2&-&-&-}
\levelten{Avenger of the Faith (Primary)& 															3&3&3&3&2&-&-}
\leveleleven{Divine Vow (Twice Vowed)& 															3&3&3&3&2&-&-}
\leveltwelve{Avenger of the Faith (Secondary), Sustained by Faith& 					3&3&3&3&2&-&-}
\levelthirteen{Divine Vow (Thrice Vowed)& 															3&3&3&3&3&2&-}
\levelfourteen{Avenger of the Faith (Primary)& 													4&3&3&3&3&2&-}
\levelfifteen{Divine Vow (Thrice Vowed), Undying Faith (as raise dead)& 			4&4&3&3&3&2&-}
\levelsixteen{Avenger of the Faith (Secondary)& 													4&4&4&3&3&3&2}
\levelseventeen{Divine Vow (Thrice Vowed)& 														4&4&4&4&3&3&2}
\leveleighteen{Avenger of the Faith (Primary), Undying Faith (as resurrection)&	4&4&4&4&4&3&3}
\levelnineteen{Divine Vow (Thrice Vowed)& 														4&4&4&4&4&4&3}
\leveltwenty{Avenger of the Faith (Secondary), All Things Are Possible& 			4&4&4&4&4&4&4}
\end{minorcastingclasstable}}

\startclassfeatures

\proficiencies{simple and martial weapons, all forms of armor, and all shields.}

\classfeature{Spells:}{A templar cast divine spells, which are drawn from the list below and supplemented by their deity's domains (see Vow of Piety). His caster level for these spells is equal to his class level. The save DCs for these spells are equal to 10 + the spell's level + his Charisma modifier. A templar must have a charisma score of at least 10 + the spell's level in order to cast the spell.
\newline
A Templar know all of the spells on his class list, and may cast any of them without preparation so long as he has an appropriate spell slot available and an charisma score of at least 10 + the spell's level. His maximum available slots per day are determined by his class level (as seen on Table: The Templar), and he gains bonus slots from his charisma score.
\newline
In order to receive their spell slots, the templar must pray for 1 hour without interruption in a place free from distractions or noise. At the end of this time, he receives his spell slots. After praying, the templar cannot pray again until one whole day (24 hours) has passed.
A templar’s spells are more for utility than combat efficacy, either allowing him to better solve problems through non-violent means or enhancing his combat abilities past even their already formidable limits.}

\classfeature{Code of Conduct (Ex):}{Like any other character, a templar does what he must to uphold the duties given to him by an organization of which he is a part, even if that organization is as loose as his alignment group. But let’s face it; sometimes even the good and honorable knight may want to lie about his identity or consort with unscrupulous characters in order to root out the evil, demonic cult. And evil knights can be obsessed with battle, honor, and battling with honor. A templar is not specifically prohibited from acts that lie outside of their alignment or run counter to their deity's wishes. Many aspire to these things and most follow them, but not all do so and no templar is punished for being found slightly wanting. Templars who actively displease or betray their deity may still be stripped of their powers and dismissed, however.}

\classfeature{Divine Vow (Su):}{A templar’s code is somewhat variable; different deities and philosophies extol different virtues that a templar must try to uphold. But more than that, each templar is permitted to extol these virtues in slightly different ways. The vows a templar makes are a representation of his personal or religious code, and determine which aspects he attempts to uphold most strongly. These vows grant him extraordinary powers (the nature of which vary based on the vows he takes). These are detailed in the section on divine vows below.
\listone
	\item At 1st level the templar gains the Vow of Piety and one other rank 1 vow of their choice. At every odd-numbered class level thereafter the templar may take a new vow, but he may not advance one of his existing vows beyond rank 1 at this time.
	\item At 7th level, he reaffirms his Vow of Piety and gains a second domain. He may also reaffirm any other vow which he already possesses to gain the rank 2 ability. A vow that has been reaffirmed in this way is known as "twice vowed." Instead of reaffirming a rank 1 vow, he may instead select two new vows at rank 1. He may not advance a vow beyond rank 2 at this time.
	\item At 13th level, he may reaffirm any other vow in which he already possesses the rank 2 ability to gain the rank 3 ability. A vow that has been reaffirmed in this way is known as "thrice vowed." Instead of reaffirming a rank 2 vow, he may instead select two rank 1 vows at advance to rank 2, or may select a new vow to gain both the rank 1 and rank 2 benefits.
\end{list}}

\classfeature{Avenger of the Faith:}{A templar trains himself in multiple forms of combat, so as to serve as both the weapon and shield of their church or ideals. Starting at second level, he chooses a primary combat form (see Avenger of the Faith Styles) for which he gains the corresponding abilities at 2nd level and every four class levels thereafter. At 4th level, he chooses his secondary style, and gains the benefits thereof at each 4 class levels.}

\classfeature{Arms of the Faithful (Ex):}{At sixth level a templar gains Craft Magic Arms and Armor as a bonus feat. When crafting any magic items with this feat, they are treated as having access to the spells of the war domain in addition to those on their class list. If they already possess Craft Magic Arms and Armor, they may select another item creation feat for which they qualify.}

\classfeature{Inquisitor (Su):}{An eigth level templar can detect the alignments of any creature that he can see as a swift action. He instantly gains all information about their alignment as if he had spent three rounds concentrating on them with the appropriate spells. If the creature is warded, the templar may make a caster level check against the warding spell to gain the information if such a check is allowed by the ward. In addition, all the templar’s attacks are automatically considered aligned (good or evil, lawful or chaotic, etc. based on his alignment) for the purposes of overcoming damage reduction.}

\classfeature{Sustained by Faith (Ex):}{An eleventh level templar gains everything they need to live from their relationship with their deity. They no longer need to eat, drink, breathe, or sleep. They can still do these things if they want to of course.}

\classfeature{Undying Faith (Su):}{Fifteenth level templars are extremely difficult to kill. The templar may elect to gain the benefit of a raise dead spell at any time within 1 minute of being killed. If they do, their return is announced by a powerful flash of light (as a daylight spell) for 1 round. Instead of the normal level loss, they instead suffer 2 points of Charisma burn. Once used, they may not return from the dead in this way for 24 hours; a templar who dies twice in a day will need someone else to bring them back to continue their work. At eighteenth level, this ability improves to offer the benefit of a resurrection spell instead, though the templar only returns with half of their maximum hit points.}

\classfeature{All Things Are Possible (Sp):}{The prayers of a twentieth level templar are taken very seriously. Once per day they may cast miracle as a spell-like ability, though they must still spend experience points if the effect would require them from a spellcaster casting it.}

\subsection{Ex-Templars}

A templar who wishes to pursue other classes is welcome to do so. There are no multiclssing restrictions against the templar.
A templar who willingly leaves his faith or who is cast out loses all spells, spell-like, and supernatural abilities, as well as any ability stemming from one of their vows. They may return to the faith if a ranking member casts an atonement for them. They may also pursue a new faith entirely. They must still find a member of the faith to atone them, however. When joining a new faith in this way, the templar loses all of their old vows. They may swear a new one each day until they have reached the level allotted them based on their level.

\subsection{Vows}
\quot{``So many vows, they make you swear and swear. Defend the King, obey the King. Obey your father. Protect the innocent. Defend the weak. What if your father despises the King? What if the King massacres the innocent? It's too much. No matter what you do, you're forsaking one vow or another.''}

\vow{Charity
}{A bone to the dog is not charity. Charity is the bone shared with the dog, when you are just as hungry as the dog.
}{Once per round on your turn you may aid another as a free action.
}{Once per round when you are targeted by a spell with an effect beneficial to you, you may allow another creature within Close Range to also gain the benefits of that spell. The spell must also be beneficial to the creature you wish to share it with (interpreted at the DM's discretion), or the sharing fails.
}{An ally within Close range of you may use your spell slots to cast a spell of an equivalent or lower spell level, so long as you possess the minimum charisma score to use the slot yourself. Your ally may use this slot to cast any spell that they have prepared or that they know (in the case of spontaneous casters), using your slot instead of their own. Your ally may also cast spells from your spell list, even if they would not normally be capable of casting divine spells. Anyone casting a spell in this fashion uses their own attributes, feats, and character level to determine the effects and DC of the spell. They do not need to meet the minimum charisma score requirement for a particular spell level cast from your list, but they must be of a sufficient level that they would be able to use the spell slot were they a templar of the same level.
}{Perhaps your church decrees that its members must give aid to others, or maybe you give out of the goodness of your heart. You are the quintessential selfless knight, giving to others without necessarily thinking of your own gains. There are times when you may give up more important things than money; the truest sacrifice a templar can make is to offer their own life in the service of their cause.}

\subsection{Avenger of the Faith Styles}
As there are many different vows that a templar can swear, so to are there different combat styles that they may practice. A templar selects one of these styles as their primary style and another as a secondary. They are both then advanced as the templar gains levels.

\faith{Charger
}{A charger is a very straightforward templar. They see their foes, and they run or ride out to meet them. This generally leads to the defeat of their foes.
}{\ability{Knight Errant (Ex):}{A charger needs to work around the limitations of the bulky armor that is so often part of his attire. You no longer suffer penalties to your base speed from wearing medium or heavy armor. You also gain additional benefits while charging. You may make 1 turn up to 90 degrees as part of your charge action, though you must still travel at least 10 feet in a straight line immediately before you attack a target. Additionally, you are not required to move to the closest space to your opponent during a charge, and may make your charge attack when your opponent is in any of your threatened spaces. This would allow you to take a charge attack while running past an opponent, but this movement would provoke attacks of opportunity as normal.}
}{\ability{Cataphract (Ex):}{When charging you gain a +4 bonus to your attack roll instead of the normal +2 and you may make a full attack on a charge. You also may charge up to three times your normal base speed when you make a charge as a full-round action. If you would only be limited to a partial charge, you may move twice your base speed as part of that action. You may not make a full-attack when you perform a partial charge, however. This benefit also applies while you are mounted.}
}{\ability{Charge of Necessity (Su):}{While charging or running, you gain the benefit or air walk for the round, until the start of your next turn. If you do not continue running or charging at the start of the next round, you instead fall to the ground under the effect of feather fall. If you begin a fall from other circumstances you do not benefit from this effect. This benefit also applies while you are mounted.}
}{\ability{Charge of Glory (Ex):}{You can trample over those who fall before your charge, continuing to seek more blood. If you destroy an effect in your path, render a charged opponent unconscious or dead, or otherwise clear the way forward while charging you may continue the charge along the same path (following all normal restrictions as they apply) up to your full allowed distance. You may make additional attacks against those in your way along this additional distance as if they were your intended charge target. This benefit also applies while you are mounted.}
}{\ability{Charge of Destruction (Su):}{When a foe is struck with your charge attack and killed, they are destroyed utterly as if they had been immolated or disintegrated. Further, while charging or running you may leave behind a blade barrier as you leaves each space. The wall need not be continuous, and may have as many or as few breaks in it as you desire. This wall deals 15d6 points of damage, has a save DC of 16 + the templar's Charisma modifer, and dissipates at the start of your next turn. This benefit also applies while you are mounted.}
}

\chapter{Skills}
\section{How Skills Work}
foo
\section{Appraise}
foo
\section{Athletics}
foo
\section{Balance}
foo
\section{Bluff}
foo
\section{Concentration}
foo
\section{Craft}
foo
\section{Decipher Script}
foo
\section{Diplomacy}
foo
\section{Disable Device}
foo
\section{Disguise}
foo
\section{Escape Artist}
foo
\section{Forgery}
foo
\section{Gather Information}
foo
\section{Handle Animal}
foo
\section{Heal}
foo
\section{Intimidate}
foo
\section{Knowledge}
foo
\section{Perception}
foo
\section{Perform}
foo
\section{Profession}
foo
\section{Ride}
foo
\section{Search}
foo
\section{Sense Motive}
foo
\section{Sleight of Hand}
foo
\section{Speak Language}
foo
\section{Spellcraft}
foo
\section{Stealth}
foo
\section{Survival}
foo
\section{Tumble}
foo
\section{Use Magic Device}
foo
%%%%%%%%%%%%%%%%%%%%%%%%
%%Feats Chapter Formatting
%%%%%%%%%%%%%%%%%%%%%%%%
\newcommand{\combatfeat}[8]{
\belowpdfbookmark{#1}{feat:#1}\paragraph{\Large#1}\large\textbf{#2}

\normalsize\noindent\textit{#3}
%\indent\hypertarget{feat:#1}{}\textbf{#1 [Combat] #2} \\
\newline\indent\ability{+0 BAB}{#4}
\newline\indent\ability{+1 BAB}{#5}
\newline\indent\ability{+6 BAB}{#6}
\newline\indent\ability{+11 BAB}{#7}
\newline\indent\ability{+16 BAB}{#8}
%\begin{description}
%\item[Benefit:] #4
%\item[BAB +1:] #5
%\item[BAB +6:] #6
%\item[BAB +11:] #7
%\item[BAB +16:] #8
%\end{description}
%~\\*
}

\newcommand{\skillfeat}[8]{
\belowpdfbookmark{#1}{feat:#1}\paragraph{\Large#1}\large\textbf{#2}

\normalsize\noindent\textit{#3}
%\indent\hypertarget{feat:#1}{}\textbf{#1 [Skill:#2] #3} \\
\newline\indent\ability{0 Ranks}{#4}
\newline\indent\ability{4 Ranks}{#5}
\newline\indent\ability{9 Ranks}{#6}
\newline\indent\ability{14 Ranks}{#7}
\newline\indent\ability{19 Ranks}{#8}
%\begin{description}
%\item[Benefit:] #4
%\item[4 Ranks:] #5
%\item[9 Ranks:] #6
%\item[14 Ranks:] #7
%\item[19 Ranks:] #8
%\end{description}
%~\\*
}

\newcommand{\spellfeat}[8]{
\belowpdfbookmark{#1}{feat:#1}\paragraph{\Large#1}\large\textbf{#2}

\normalsize\noindent\textit{#3}
%\indent\hypertarget{feat:#1}{}\textbf{#1 [Spellcasting] #2} \\
\newline\indent\ability{0th Level}{#4}
\newline\indent\ability{1st Level}{#5}
\newline\indent\ability{3rd Level}{#6}
\newline\indent\ability{6th Level}{#7}
\newline\indent\ability{9th Level}{#8}
%\begin{description}
%\item[Benefit:] #4
%\item[Level 1:] #5
%\item[Level 3:] #6
%\item[Level 6:] #7
%\item[Level 9:] #8
%\end{description}
%~\\*
}

\newcommand{\genfeat}[3]{
\belowpdfbookmark{#1}{feat:#1}\paragraph{\Large#1}\large\textbf{#2}

\normalsize\noindent\textit{#3}
}

\newcommand{\featprereq}[1]{
\newline\indent\ability{Prerequisite}{#1}
}

\newcommand{\featbenefit}[1]{
\newline\indent\ability{Benefit}{#1}
}

\newcommand{\featspecial}[1]{
\newline\indent\ability{Special}{#1}
}


%%%%%%%%%%%%%%%%%%%%%%%%

\chapter{Feats}
\section{How Feats Work}
foo

\input{phb/feats/generalfeats}
\section{Combat Feats}

%\begin{multicols}{2}

\combatfeat{Blind Fighting}{[Combat]}
{You don't have to see to kill.}
{You may reroll your miss chances caused by concealment.}
{While in darkness, you may move your normal speed without difficulty.}
{You have Blindsense out to 60', this allows you to know the location of all creatures within 60'.}
{You have Tremorsense out to 120', this allows you to ``see" anything within 120' that is touching the earth.}
{You cannot be caught flat footed.}
\combatfeat{Blitz}{[Combat]}
{You go all out and try to achieve goals in a proactive manner.}
{While charging, you may opt to lose your Dexterity Bonus to AC for one round, but inflicting an extra d6 of damage if you hit.}
{You may go all out when attacking, adding your Base Attack Bonus to your damage, but provoking an Attack of Opportunity.}
{Bonus attacks made in a Full Attack for having a high BAB are made with a -2 penalty instead of a -5 penalty.}
{Every time you inflict damage upon an opponent with your melee attacks, you may immediately use an Intimidate attempt against that opponent as a bonus action.}
{You may make a Full Attack action as a Standard Action.}
\combatfeat{Combat Looting}{[Combat]}
{You can put things into your pants in the middle of combat.}
{You may sheathe or store an object as a free action.}
{You get a +3 bonus to disarm attempts. Picking up objects off the ground does not provoke an attack of opportunity.}
{As a Swift action, you may take a ring, amulet/necklace, headband, bracer, or belt from an opponent you have successfully grappled. You may pick up an item off the ground in the middle of a move action.}
{If you are grappling with an opponent, you may activate or deactivate their magic items with a successful Use Magic Device check. You may make Appraise checks as a free action.}
{You can take 10 on Use Magic Device and Sleight of Hand checks.}
\combatfeat{Combat School}{[Combat]}
{You are a member of a completely arbitrary fighting school that has a number of recognizable signature fighting moves.}
{First, name your fighting style (such as ``Hammer and Anvil Technique'' or ``Crescent Moon Style'', or ``Way of the Lightning Mace''). This fighting style only works with a small list of melee weapons that you have to describe the connectedness to the DM in a half-way believable way. Now, whenever you are using that technique in melee combat, you gain a +2 bonus on attack rolls.}
{Your immersion in your technique gives you great martial prowess, you gain a +2 to damage rolls in melee combat.}
{When you strike your opponent with the signature moves of your fighting school in melee, they must make a Fortitude Save (DC 10 + 1/2 your level + your Strength bonus) or become dazed for one round. If they succeed on this save they are immune to further dazing attempts for one round.}
{You may take 10 on attack rolls while using your special techniques. The DC to disarm you of a school-appropriate weapon is increased by 4.}
{You may add +5 to-hit on any one attack you make after the first each turn. If you hit an opponent twice in one round, all further attacks this round against that opponent are made with The Edge.}
\spellentry{Command}

Enchantment (Compulsion) [Language-Dependent, Mind-Affecting]

\textbf{Level:} Clr 1

\textbf{Components:} V

\textbf{Casting Time:} 1 standard action

\textbf{Range:} Close (25 ft. + 5 ft./2 levels)

\textbf{Target:} One living creature

\textbf{Duration:} 1 round

\textbf{Saving Throw:} Will negates

\textbf{Spell Resistance:} Yes

You give the subject a single command, which it obeys to the best of its ability 
at its earliest opportunity. You may select from the following options.

\textit{Approach:} On its turn, the subject moves toward you as quickly and directly 
as possible for 1 round. The creature may do nothing but move during its turn, 
and it provokes attacks of opportunity for this movement as normal.

\textit{Drop:} On its turn, the subject drops whatever it is holding. It can't 
pick up any dropped item until its next turn.

\textit{Fall:} On its turn, the subject falls to the ground and remains prone for 
1 round. It may act normally while prone but takes any appropriate penalties.

\textit{Flee:} On its turn, the subject moves away from you as quickly as possible 
for 1 round. It may do nothing but move during its turn, and it provokes attacks 
of opportunity for this movement as normal.

\textit{Halt:} The subject stands in place for 1 round. It may not take any actions 
but is not considered helpless.

If the subject can't carry out your command on its next turn, the spell automatically 
fails.


\combatfeat{Danger Sense}{[Combat]}
{Maybe Spiders tell you what's up. You certainly react to danger with uncanny effectiveness.}
{You get a +3 bonus on Initiative checks.}
{For the purpose of Search, Spot, and Listen, you are always considered to be ``actively searching". You also get Uncanny Dodge.}
{You may take 10 on Listen, Spot, and Search checks.}
{You may make a Sense Motive check (opposed by your opponent's Bluff check) immediately whenever any creature approaches within 60' of you with harmful intent. If you succeed, you know the location of the creature even if you cannot see it.}
{You are never surprised and always act on the first round of any combat.}
\combatfeat{Elusive Target}{[Combat]}
{You are very hard to hit when you want to be.}
{You gain a +2 Dodge bonus to AC.}
{Your opponents do not gain flanking or higher ground bonuses against you.}
{Your opponents do not inflict extra damage from the power attack option.}
{Diverting Defense -- As an immediate action, you may redirect an attack against you to any creature in your threatened range, friend or foe. You may not redirect an attack to the creature making the attack.}
{As an immediate action, you may make an attack that would normally hit you miss instead.}
\combatfeat{Expert Tactician}{[Combat]}
{You benefit your allies so good they remember you long time.}
{You gain a +4 bonus when flanking instead of the normal +2 bonus. Your allies who flank with you gain the same advantage.}
{You may feint as an Immediate action.}
{As a move action, you may make any 5' square adjacent to yourself into difficult ground.}
{For determining flanking with your allies, you may count your location as being 5' in any direction from your real location.}
{You ignore Cover bonuses less than full cover.}
\combatfeat{Ghost Hunter}{[Combat]}
{You smack around those folks in the spirit world.}
{Your attacks have a 50\% chance of striking incorporeal opponents even if they are not magical.}
{You can hear incorporeal and ethereal creatures as if they lacked those traits (note that shadows and the like rarely bother to actively move silently).}
{You can see invisible and ethereal creatures as if they lacked those traits.}
{Your attacks count as if you had the Ghost Touch property on your weapons.}
{Any Armor or shield you use benefits from the Ghost Touch quality.}
\combatfeat{Giant Slayer}{[Combat]}
{Everyone has a specialty. Yours is miraculously finding ways to stab creatures in the face when it seems improbable that you would be able to reach that high.}
{When you perform a grab on grapple maneuver, you do not provoke an attack of opportunity.}
{You gain a +4 Dodge bonus to your AC and Reflex Saves against attacks from any creature with a longer natural reach than your own.}
{You have The Edge against any creature you attack that is larger than you. Also, an opponent using the Improved Grab ability on you provokes an attack of opportunity from you. You may take this attack even if you do not threaten a square occupied by your opponent.}
{When you attempt to trip an opponent, you may choose whether your opponent resists with Strength or Dexterity.}
{When involved in an opposed bull rush, grapple, or trip check as the attacker or defender, you may negate the size modifier of both participants. You may not choose to negate the size modifier of only one character.}
\combatfeat{Great Fortitude}{[Combat]}
{You are so tough. Your belly is like a prism.}
{You gain a +3 bonus to your Fortitude Saves.}
{You die at -20 instead of -10.}
{You gain 1 hit point per level.}
{You gain DR of 5/-.}
{You are immune to the fatigued and exhausted conditions. If you are already immune to these conditions, you gain 1 hit point per level for each condition you were already immune to.}
\combatfeat{Horde Breaker}{[Combat]}
{You kill really large numbers of people.}
{You gain a number of extra attacks of opportunity each round equal to your Dexterity Bonus (if positive).}
{Whenever you drop an opponent with a melee attack, you are entitled to a bonus ``cleave" attack against another opponent you threaten. You may not take a 5' step or otherwise move before taking this bonus attack. This cleave attack is considered an attack of opportunity.}
{You may take a bonus 5' step every time you are entitled to a cleave attack, which you may take either before or after the attack.}
{You may generate an aura of fear on any opponents within 10' of yourself whenever you drop an opponent in melee. The save DC is 10 + the Hit Dice of the dropped creature.}
{Opponents you have the Edge against provoke an attack of opportunity from you by moving into your threatened area or attacking you.}
\combatfeat{Hunter}{[Combat]}
{You can move around and shoot things with surprising effectiveness.}
{The penalties for using a ranged weapon from an unstable platform (such as a ship or a moving horse) are halved.}
{Shot on the Run -- you may take a standard action to attack with a ranged weapon in the middle of a move action, taking some of your movement before and some of your movement after your attack. That still counts as your standard and move action for the round.}
{You suffer no penalties for firing from unstable ground, a running steed, or any of that.}
{You may take a full round action to take a double move and make a single ranged attack from any point during your movement.}
{You may take a full round action to run a full four times your speed and make a single ranged attack from any point during your movement. You retain your Dexterity modifier to AC while running.}
\combatfeat{Insightful Strike}{[Combat]}
{You Hack people down with inherent awesomeness.}
{You may use your Wisdom Modifier in place of your Strength Modifier for your melee attack rolls.}
{Your attacks have The Edge against an opponent who has a lower Wisdom and Dexterity than your own Wisdom, regardless of relative BAB.}
{Your melee attacks have a doubled critical threat range.}
{You make horribly telling blows. The extra critical multiplier of your melee attacks is doubled (x2 becomes x3, x3 becomes x5, and x4 becomes x7).}
{Any Melee attack you make is considered to be made with a magic weapon that has an enhancement bonus equal to your Wisdom Modifier (if positive).}
\combatfeat{Iron Will}{[Combat]}
{You are able to grit your teeth and shake off mental influences.}
{You gain a +3 bonus to your Willpower saves.}
{You gain the slippery mind ability of a Rogue.}
{If you are stunned, you are dazed instead.}
{You do not suffer penalties from pain and fear.}
{You are immune to compulsion effects.}
\combatfeat{Juggernaut}{[Combat]}
{You are an unstoppable Juggernaut.}
{You may be considered one size category larger for the purposes of any size dependant roll you make (such as a Bull Rush, Overrun, or Lift action).}
{You do not provoke an attack of opportunity for entering an opponent's square.}
{You gain a +4 bonus to attack and damage rolls to destroy objects. You may shatter a Force Effect by inflicting 30 damage on it.}
{When you successfully bullrush or overrun an opponent, you automatically Trample them, inflicting damage equal to a natural slam attack for a creature of your size.}
{You gain the Rock Throwing ability of any standard Giant with a strength equal to or less than yourself.}
\combatfeat{Lightning Reflexes}{[Combat]}
{You are fasty McFastFast. It helps keep you alive.}
{You gain a +3 bonus to your Reflex saves.}
{You gain Evasion, if you already have Evasion, that stacks to Improved Evasion.}
{You may make a Balance Check in place of your Reflex save.}
{You gain a +3 bonus to your Initiative.}
{When you take the Full Defense Action, add your level to your AC.}
\combatfeat{Mage Slayer}{[Combat]}
{You have trained long and hard to kill magic users. Maybe you hate them, maybe you just noticed that most of the really dangerous creatures in the world use magic.}
{You gain Spell Resistance of 5 + Character Level.}
{Damage you inflict is considered ``ongoing damage" for the purposes of concentration checks made before the beginning of your next round. All your attacks in a round are considered the same source of continuing damage.}
{Creatures cannot cast defensively within your threat range.}
{Your attacks ignore Deflection bonuses to AC.}
{When a creature uses a [Teleportation] effect within medium range of yourself, you may choose to be transported as well. This is not an action.}
\combatfeat{Murderous Intent}{[Combat]}
{You stab people in the face.}
{You may make a Coup de Grace as a standard action.}
{When you kill an opponent, you gain a +2 Morale Bonus to your attack and damage rolls for 1 minute.}
{Once per round, you may take an attack of opportunity against an opponent who is denied their Dexterity bonus to AC.}
{You may take a Coup de Grace action against opponents who are stunned.}
{You may take a Coup de Grace action against opponents who are dazed.}
\combatfeat{Phalanx Fighter}{[Combat]}
{You fight well in a group.}
{You may take attacks of opportunity even while flat footed.}
{Any Dodge bonus to AC you gain is also granted to any adjacent allies for as long as you benefit from the bonus and your ally remains adjacent.}
{Charging is an action that provokes an attack of opportunity from you. This attack is considered to be a ``readied attack" if it matters for purposes like setting against a charge.}
{You may attack with a reach weapon as if it was not a reach weapon. Thus, a medium creature would normally threaten at 5' and 10' with a reach weapon.}
{You may take an Aid Another action once per round as a free action. You provide double normal bonuses from this effect.}
\combatfeat{Point Blank Shot}{[Combat]}
{You are crazy good using a ranged weapon in close quarters.}
{When you are within 30' of your target, your attacks with a ranged weapon gain a +3 bonus to-hit.}
{You add your base attack bonus to damage with any ranged attack within the first range increment.}
{You do not provoke an attack of opportunity when you make a ranged attack.}
{When armed with a Ranged Weapon, you may make attacks of opportunity against opponents who provoke them within 30' of you. Movement within this area does not provoke an attack of opportunity.}
{With a Full Attack action, you may fire a ranged weapon once at every single opponent within the first range increment of your weapon. You gain no additional attacks for having a high BAB. Make a single attack roll for the entire round, and compare to the armor class of each opponent within range.}
\combatfeat{Sniper}{[Combat]}
{Your shooting is precise and dangerous.}
{Your range increments are 50\% longer than they would ordinarily be. Any benefit of being within 30' of an opponent is retained out to 60'.}
{You do not suffer a -4 penalty when firing a ranged weapon into melee and never hit an unintended target in close combats or grapples.}
{Your ranged attacks ignore Cover Bonuses (total cover still bones you).}
{Opponents struck by your ranged attacks do not automatically know what square your attack came from, and must attempt to find you normally.}
{Any time you hit an opponent with a ranged weapon, it is counted as a critical threat. If your weapon already had a 19-20 threat range, increase its critical multiplier by 1.}
\combatfeat{Subtle Cut}{[Combat]}
{You cut people so bad they have to ask you about it later.}
{Any time you damage an opponent, that damage is increased by 1.}
{As a standard action, you can make a weapon attack that also reduces a creature's movement rate. For every 5 points of damage this attack does, reduce the creature's movement by 5'. This penalties lasts until the damage is healed.}
{As a standard action, you may make a weapon attack that also does 2d4 points of Dexterity damage.}
{Any weapon attack that you make at this level acts as if the weapon had the wounding property.}
{As a standard action, you may make an attack that dazes your opponent. This effect lasts one round, and has a DC of 10 + half your level + your Intelligence bonus.}
\combatfeat{Two Weapon Fighting}{[Combat]}
{When armed with two weapons, you fight with two weapons rather than picking and choosing and fighting with only one. Kind of obvious in retrospect.}
{You suffer no penalty for doing things with your off-hand. When you make an attack or full-attack action, you may make a number of attacks with your off-hand weapon equal to the number of attacks you are afforded with your primary weapon.}
{While armed with two weapons, you gain an extra Attack of Opportunity each round for each attack you would be allowed for your BAB, these extra attacks of opportunity must be made with your off-hand.}
{You gain a +2 Shield Bonus to your armor class when fighting with two weapons and not flat footed.}
{You may feint as a Swift action.}
{While fighting with two weapons and not flat footed you may add the enhancement bonus of either your primary or your off-hand weapon to your Shield Bonus to AC.}
\combatfeat{Weapon Finesse}{[Combat]}
{You are incredibly deft with a sword.}
{You may use your Dexterity Modifier instead of your Strength modifier for calculating your melee attack bonus.}
{Your special attacks are considered to have the Edge when you attack an opponent with a Dexterity modifier smaller than yours, even if your Base Attack Bonus is not larger.}
{You may use your Dexterity modifier in place of your Strength modifier when attempting to trip an opponent.}
{You may use your Dexterity modifier in place of your Strength modifier for calculating your melee damage.}
{Once per turn, when an opponent is struck, you may take an attack of opportunity on that opponent.}
%\combatfeat{Weapon of Righteous Destruction}{[Combat]}
{Your hands make whatever is being held by them holy and on fire. For some reason this doesn't make them melt or burn up.}
{Whatever weapon you are wielding is considered Magical (+\sfrac{1}{3} bonus/level) in addition to any other properties that it has. Your unarmed attacks, even if not proficient, count for this effect.}
{The above, plus Flaming.}
{The above, Holy instead of Flaming.}
{The above, plus Sun weapon, Fort save. (BoG)}
{The above, plus Vorpal weapon (BoG).}
\combatfeat{Whirlwind}{[Combat]}
{You are just as dangerous to everyone around you as to anyone around you.}
{As a full round action, you may make a whirlwind attack - you may make a single attack against each opponent you can reach. Roll one attack roll and compare to each available opponent's AC individually.}
{You gain a +3 bonus to Balance checks.}
{When you make a whirlwind attack, you may also take a regular move action. You may make a single attack against each opponent you can reach at any point during your movement. Roll one attack roll and compare to each available opponent's AC individually, as normal.}
{Until your next round after making a whirlwind attack, you may take an attack of opportunity against any opponent that enters your threatened area.}
{When you make a whirlwind attack, you may also take a double move action as if you had charged. You overrun any creature in your path and may make a single attack against each opponent you can reach at any point during your movement. Roll one attack roll and compare to each available opponent's AC individually, as normal.}
\combatfeat{Zen Archery}{[Combat]}
{You are very calm about shooting people in the face. That's a good place to be.}
{You may use your Wisdom Modifier in place of your Dexterity Modifier on ranged attack rolls.}
{Any opponent you can hear is considered an opponent you can see for purposes of targeting them with ranged attacks.}
{If you cast a Touch Spell, you can deliver it with a ranged weapon (though you must hit with a normal attack to deliver the spell).}
{As a Full Round Action, you may make one ranged attack with a +20 Insight bonus to hit.}
{As a Full Round Action, you may make one ranged attack with a +20 Insight bonus to hit. If this attack hits, your attack is automatically upgraded to a critical threat. If the threat range of your weapon is 19-20, your critical multiplier is increased by one.}

%\end{multicols}
\subsection{Skill Feats} \label{comm:skillfeats}

\begin{multicols}{2}
\label{comm:feat:acquierereye}\skillfeat{Acquirer's Eye [Skill]}{
You know what you want, even if other people have it right now.}{
Appraise Ranks:}{
You gain +3 to your Appraise checks.}{
You automatically know if something is ordinary, masterwork, or magic when looking at it.}{
You can discover the properties of a magic item, including how to activate it (if appropriate) and how many charges are left (if it has them), with a successful Appraise check (DC item's caster level + 10) and 10 minutes of work.}{
Once per round as a free action, you can examine a magic item and attempt an Appraise check (DC item's caster level + 20) to determine its properties, including its functions, how to activate those functions (if necessary), and how many charges it has left (if it has charges).}{
You know what the most valuable piece of treasure is in any collection, such as the most valuable magic item an enemy is wearing or the most valuable object in a dragon's horde, just by looking at the collection. You automatically recognize an artifact when looking at it.}


\label{comm:feat:acrobatic}\skillfeat{Acrobatic [Skill]}{
You can totally flip out and kill someone with your gymnastic prowess.}{
Tumble Ranks:}{
You gain a +3 bonus to Tumble checks.}{
When using the Combat Expertise option, your dodge bonus to AC increases by +1. This further increases by +1 for every ten ranks of Tumble you have (+2 at 14, +3 at 24, and so on).}{
If an opponent attempts to bull-rush, overrun, or trample you, if you succeed on Tumble check of DC 25 + their base attack bonus, their movement continues in a straight line to the maximum allowed by their speed, you remain where you were, and you don't suffer from the effects of their bull-rush, overrun, or trample. If you fail, you provoke an attack of opportunity from that enemy.}{
If you succeed on a DC 40 Tumble check, you can move 10 feet when taking a 5-foot step.}{
If you succeed on a Tumble check against a DC of 30 + an opponent's base attack bonus, an action that would normally provoke an attack of opportunity doesn't.}


\label{comm:feat:alertness}\skillfeat{Alertness [Skill]}{
Your ears are so sharp you probably wouldn't miss your eyes.}{
Listen Ranks:}{
You gain a +3 bonus to Listen checks.}{
You can make a Listen check once a round as a free action. You don't take penalties for distractions on your Listen checks.}{
You gain blindsense to 60 feet. You don't take penalties for ambient noise, such as loud winds. Divide any distance penalties you take on Listen checks by two.}{
You gain blindsight to 120 feet.}{
You can hear through magical silence and similar effects, but you take a -20 penalty on your check. Divide any distance penalties you take on Listen checks by five.}


\label{comm:feat:animalaffinity}\skillfeat{Animal Affinity [Skill]}{
You're one of those people animals just won't leave alone for no apparent reason.}{
Handle Animal ranks:}{
You gain the wild empathy ability, with your check equal to your character level plus your Charisma modifier plus any other applicable bonuses. If you already have wild empathy, or later gain it from another source, you gain a +3 bonus on Handle Animal checks.}{
You can handle an animal as a free action, and push it as a move action.}{
You gain the benefits of \spell{speak with animals} permanently as an extraordinary ability. The DCs for you to rear and train creatures are halved.}{
With a DC 30 Handle Animal check, you can use a mass version of \spell{charm animal} as a spell-like ability, with save DC equal to 10 + \half your character level + your Cha modifier and effective caster level equal to your bonus on Handle Animal checks.}{
You can summon animals to your aid. Choose an animal with a CR equal to or less than your character level, and make a Handle Animal check at a DC of 25 + your character level. If you succeed, you summon a number of animals depending on how much the animal's CR is less than your character level for an hour. You can't use this ability again until any animals you've summoned with it have unsummoned or you've dismissed them.
\vspace{2pt}
\begin{tabular}{l|l}
	\textbf{CR} &\textbf{Number Appearing}\\
	Level - 1 &1\\
	Level - 2 &1d3\\
	Level - 3 &1d4\\
	Level - 4 &1d6\\
	Level - 5 &1d8\\
	Level - 6 &1d10\\
	Level - 7 &2d6\\
	Level - 8 &3d6\\
	Level - 9 &3d10\\
	Level - 10 &10+3d6\\
	Level - 11 &15+3d10\\
	Level - 12 &40\\
	Level - 13 &50\\
	Level - 14 &60\\
	Level - 15 &80\\
	Level - 16 &100\\
	Level - 17 &150\\
	Level - 18 &200\\
	Level - 19 &300
\end{tabular}}
%\listone
%	\bolditem{CR;}{Number Appearing}
%	\bolditem{Level - 1;}{1}
%	\bolditem{Level - 2;}{1d3}
%	\bolditem{Level - 3;}{1d4}
%	\bolditem{Level - 4;}{1d6}
%	\bolditem{Level - 5;}{1d8}
%	\bolditem{Level - 6;}{1d10}
%	\bolditem{Level - 7;}{2d6}
%	\bolditem{Level - 8;}{3d6}
%	\bolditem{Level - 9;}{3d10}
%	\bolditem{Level - 10;}{10+3d6}
%	\bolditem{Level - 11;}{15+3d10}
%	\bolditem{Level - 12;}{40}
%	\bolditem{Level - 13;}{50}
%	\bolditem{Level - 14;}{60}
%	\bolditem{Level - 15;}{80}
%	\bolditem{Level - 16;}{100}
%	\bolditem{Level - 17;}{150}
%	\bolditem{Level - 18;}{200}
%	\bolditem{Level - 19;}{300}
%\end{list}


\label{comm:feat:battlefieldsurgeon}\skillfeat{Battlefield Surgeon [Skill]}{
You like to cut people open with a saw. But it's good for them. Seriously.}{
Heal ranks:}{
You gain +3 to your Heal checks.}{
You can make first aid, treat poison, and treat wound checks as move actions.}{
For every 5 points your Heal check exceeds the DC for long term care, your patients recover another +100\% faster. For instance, if your Heal check result is 23, your patients would heal at thrice the normal rate.}{
If you operate on a patient for a minute, they regain hit points equal to your Heal check result. You also may, instead of healing hit point damage, cure any condition that heal could, reattach severed limbs, or repair ruined organs, if you succeed on a DC 30 check. Patients under your long-term care heal permanent ability drain as if it was ability damage.}{
With one hour of work, 25,000 gp worth of materials (which are consumed in the process), and a DC 40 Heal check, you can restore a creature that died within the last twenty-four hours to life. The subject's soul must be free and willing to return for the effect to work.}


\label{comm:feat:combatcasting}\skillfeat{Combat Casting [Skill]}{
Having a sword sticking out of your chest doesn't noticeably impede your ability to do\ldots well, just about anything.}{
Concentration ranks:}{
You gain +3 to your Concentration checks.}{
You can take 10 on Concentration checks and caster level checks.}{
You may maintain concentration on a spell as a move action (DC 25 + spell level). If you beat the DC by 10 or more, you can maintain concentration as a swift action. If you fail your check, you lose concentration.}{
If you would be nauseated, you're sickened instead.}{
All Concentration DCs are halved for you.}


\label{comm:feat:conartist}\skillfeat{Con Artist [Skill]}{
You can fool some of the people, all of the time.}{
Bluff ranks:}{
You gain a +3 bonus to Bluff checks.}{
Magic effects that would detect your lies or force you to speak the truth must succeed on a caster level check with DC equal to 10 plus your ranks in Bluff or fail.}{
Divination magic used on you detects a false alignment of your choice. You can present false surface thoughts to \spell{detect thoughts} and similar effects, changing your apparent Intelligence score (and thus your apparent mental strength) by as much as 10 points and can place any thought in your "surface thoughts" to be read by such spells or effects.}{
If you beat someone's Sense Motive check by 25, you can instill a \spell{suggestion} in them, as the spell. This suggestion lasts for one hour for each of your character levels.}{
You are protected from all spells and effects that detect or read emotions or thoughts, as by \spell{mind blank}.}


\label{comm:feat:cryptographer}\skillfeat{Cryptographer [Skill]}{
You're good at reading things no one intended you to.}{
Decipher Script ranks:}{
You gain +3 to your Decipher Script checks.}{
You can decipher a written spell (like a scroll) without using \spell{read magic}, if you succeed on a Decipher Script check of DC 20 + the spell's level. You can try once per day on any particular written spell.}{
You don't trigger written magic traps (like \spell{explosive runes} or \spell{symbols}) by reading them. You can disable them with Decipher Script as if you were using Disable Device. You can read the material hidden by a \spell{secret page} with a DC 25 Decipher Script check.}{
When you cast a spell from a scroll, the spell's save DC is equal to 10 + the spell's level + your Intelligence modifier + any other applicable bonuses, and its caster level is equal to your character level, plus other applicable bonuses.}{
Reading text using Decipher Script is a free action for you. You may disable written magical traps as a swift action, and you can cast 5th-level or lower spells from scrolls as a swift action.}


\label{comm:feat:deftfingers}\skillfeat{Deft Fingers [Skill]}{
Your amazing manual dexterity is the talk of princes and princesses.}{
Sleight of Hand ranks:}{
You gain a +3 bonus on your Sleight of Hand checks.}{
If you draw a hidden weapon and attack with it in the same round, your opponent loses their Dexterity bonus to AC against your first attack with that weapon that round. This ability can only be used once per round.}{
You can make an adjacent creature or object your size or smaller 'disappear' with your legerdemain. If you succeed on a DC 30 Sleight of Hand check as a standard action, your target can make a Hide check, or you can make the Hide check for them or it. As usual, you can hide larger creatures or objects by taking a -20 cumulative penalty for each size category larger they are than you.}{
With a DC 30 Sleight of Hand check, you can use \spell{shrink item} as a spell-like ability.}{
With a DC 40 Sleight of Hand check, you can use \spell{teleport object} as a spell-like ability. You can also retrieve items placed in the Ethereal Plane using \spell{teleport object}. With a DC 40 Sleight of Hand check, you can use \spell{instant summons} as a spell-like ability without requiring \spell{arcane mark}, but you may only designate one item at a time.}


\label{comm:feat:detective}\skillfeat{Detective [Skill]}{
You're good at finding things out just by conversing with townsfolk.}{
Gather Information ranks:}{
You gain a +3 bonus on your Gather Information checks.}{
Your ability to pick up on the social context aids you in establishing rapport. After succeeding on a Gather Information check, you gain a +2 bonus to Knowledge checks, Sense Motive checks, and checks for Cha-based skills in the same milieu.}{
With 2d6 hours of research, you can study a specific topic, such as a particular location or a well-known local monster, and substitute a Gather Information check for any Knowledge checks pertaining to the topic. You need access to local informants, a library, scholars, or other appropriate sources to use this ability.}{
You can gain the benefits of \spell{legend lore} with a DC 30 Gather Information check. If you have the person or thing at hand, or are in the place, this takes a day; otherwise, it consumes the time as normal for \spell{legend lore}. You need access to individuals or resources with relevant knowledge to use this ability.}{
With a DC 40 Gather Information check and 1d4+1 days of talking to people, you can either find an answer to any question you can pose in ten words or less, or find out where you need to go to get the answer. You need access to individuals or resources with relevant knowledge to use this ability.}


\label{comm:feat:dreadfuldemeanor}\skillfeat{Dreadful Demeanor [Skill]}{
People know you're a badass motherfvcker the instant you enter the room.}{
Intimidate ranks:}{
You gain +3 to your Intimidate checks.}{
You can demoralize an opponent as a move action.}{
Opponents you've demoralized remain \condition{shaken} until they lose sight of you.}{
Opponents who would be \condition{panicked} because of your fear effects are \condition{cowered} instead for the duration of the effect.}{
Any time you confirm a critical hit in melee, your target is \spell{cowered} until they lose sight of you. This is a fear effect.}


\label{comm:feat:expertcounterfeiter}\skillfeat{Expert Counterfeiter [Skill]}{
You aren't a common forger, you're an \textit{artiste}.}{
Forgery ranks:}{
You gain a +3 bonus to Forgery checks.}{
When creating a forgery, you roll twice and take the better result.}{
In situations where you can present a legal document of some sort, you can substitute a Forgery check for a Bluff, Diplomacy, or Intimidate check.}{
You can purchase items with counterfeit bills of exchange, falsified credit vouchers, and the like. You can acquire any item available through the gold economy in this method. Normally, your counterfeits are so good they don't provoke suspicion, but if someone examines them, they must still beat you in an opposed Forgery check to recognize they're not the real thing.}{
You can duplicate a scroll with eight hours of work and a Forgery check against DC 35 + the spell's level. The duplicate functions in all manners like the original scroll. You must have appropriate materials on hand for scribing the scroll, and if the spell requires XP or expensive material components, you must provide the requisite components or make up the XP cost in materials.}


\label{comm:feat:ghoststep}\skillfeat{Ghost Step [Skill]}{
You might as well be incorporeal for all the noise you make.}{
Move Silently ranks:}{
You gain +3 to your Move Silently checks.}{
Anyone attempting to use Survival to track you must beat you in an opposed check against Move Silently.}{
Creatures with blindsense, blindsight, tremorsense, or similar abilities do not automatically detect your presence, but must succeed on a Listen check, opposed by your Move Silently check, to notice you.}{
With success on a DC 30 Move Silently check as a standard action, you can control ambient sounds within 30 feet of yourself for a round. You can specifically duplicate any effect from \spell{control sound} (XPH), \spell{silence}, or \spell{ventriloquism}, and in general can make sound you've heard come from any part of the area, displace sounds in the area, or suppress any sounds or sounds. Also, if you take a -10 DC penalty on your Move Silently check, anyone within 30 feet of you can substitute your check result for their own.}{
You're so quiet that people don't even remember you when you're standing right next to them. Your opponents count as flat-footed whenever you attack them.}


\label{comm:feat:investigator}\skillfeat{Investigator [Skill]}{
You have an eye for detail and so much patience that going through a 100' by 100' room inch-by-inch doesn't even try it.}{
Search ranks:}{
You can use Search to find traps like a character with trapfinding. If you already have that ability, you gain +3 to your Search checks. Search is always a class skill for you.}{
You can Search a 10' by 10' area with a full-round action.}{
You automatically sense any active magic effects in an area you search. If you succeed on a DC 20 Search check, you can determine their number, strength, and school, as if using \spell{detect magic}.}{
You can Search objects or areas within 30 feet of yourself. You can make a Search check as a swift action.}{
You have an intuitive sense for hidden things. Anytime something that someone has hidden is within 60 feet of you, you know it; if there are multiple things, you know how many. However, you must still make Search checks as normal to locate them.}


\label{comm:feat:itemmaster}\skillfeat{Item Master [Skill]}{
You make magic items do things you want.}{
Use Magic Device ranks:}{
You gain a +3 bonus to Use Magic Device checks.}{
You don't suffer mishaps with magic items.}{
When rolling Use Magic Device checks or random effects from magic items, you may roll twice and take the better result.}{
With a swift action and a successful Use Magic Device check against a DC of 30 + the item's caster level, you can gain the benefits of a slotted magic item without needing to have a slot available (for instance, a third ring on your finger) for one round.}{
When you activate a wand or staff, you can substitute a spell slot instead of using a charge. The spell slot must be one you have not used for the day, though you may lose a prepared spell to emulate a wand charge (you may not lose prepared spells from your school of specialty, if any). The spell slot lost must be equal to or higher in level than the spell stored in the wand, including any level-increasing metamagic enhancements. When using spell trigger, spell completion, or other consumable magic items, if you succeed on a Use Magic Device check of 40 + the caster level of the item as a swift action, the item or charges thereof are not consumed.}


\label{comm:feat:leadership}\skillfeat{Leadership [Leadership] [Skill]}{
You convince people that obeying you is a good career move.}{
Diplomacy ranks:}{
You can awe even strangers and enemies into following your orders. With a DC 20 Diplomacy check, you can use \spell{command} as a spell-like ability, with save DC equal to 10 + \half your character level + your Cha modifier.}{
Your natural talent for leaderships attracts followers. Your leadership score is equal to your ranks in Diplomacy plus your Charisma modifier.}{
You persuade someone that you are so awesome that they should follow you around all the time, acquiring a cohort. A cohort is an intelligent and loyal creature with a CR at least 2 less than your character level. Cohorts gain levels when you do.}{
Your natural majesty stirs guilt in those who refuse your demands. With a DC 30 Diplomacy check, you can use \spell{geas} as a spell-like ability, but it offers a Will save at DC 10 + \half your character level + your Cha modifier.}{
You command the loyalty of armies\ldots even opposing ones. With a DC 40 Diplomacy check, you can use \spell{greater command} as a spell-like ability, with save DC equal to 10 + \half your character level + your Cha modifier and effective caster level equal to your bonus on Diplomacy checks.}


\label{comm:feat:legendarywrangler}\skillfeat{Legendary Wrangler [Skill]}{
No one can tell where you end and your ropes begin.}{
Use Rope ranks:}{
You gain a +3 bonus to Use Rope checks and proficiency with the bolas, net, and whip.}{
You can use a rope as if it was a bolas or whip, and you can substitute your ranks in Use Rope for your Base Attack Bonus for combat maneuvers made with it. You can also use it as a net, replacing the normal DC 20 Escape Artist check for someone entangled with it with your Use Rope check. You can throw a grappling hook, tie a knot, tie a special knot, or tie a rope around yourself one-handed as a move action. You don't provoke attacks of opportunity for using Use Rope.}{
You can use a rope, whip, grappling hook, or similar item to manipulate any item within 30 feet of yourself as easily as if it was in your hands; you can also make disarm, entangling (as if with a net), and trip attempts with it. You can move around on ropes and similar structures, like webs, as easily as you can on the ground.}{
With a DC 30 Use Rope check, you can use \spell{animate rope} as a spell-like ability; you can use any ability you can with an ordinary rope with an animated rope.}{
You can manipulate items out to 60 feet with ropes and similar items. You can use ropes for the grab on and hold down grapple maneuvers. When using combat maneuvers with ropes, you can replace the relevant check (disarm, grapple, trip, etc.) with a Use Rope check.}


\label{comm:feat:magicalaptitude}\skillfeat{Magical Aptitude [Skill]}{
You're crazy good at manipulating magic.}{
Spellcraft ranks:}{
You gain a +3 bonus on Spellcraft checks.}{
When counterspelling, you may use a spell of the same school that is one or more spell levels higher than the target spell.}{
You can dismiss a spell as a free action. You can redirect a spell as a move action, if it normally requires a standard action, or a swift action, if it normally takes a move action. You gain a +3 bonus on dispel checks.}{
You can counter a spell as an immediate action.}{
You automatically know which spells or magic effects are active on upon any individual object you see, as if you had \spell{greater arcane sight} active on yourself.}


\label{comm:feat:many-faced}\skillfeat{Many-Faced [Skill]}{
You change identities so often even you don't remember what you look like anymore.}{
Disguise ranks:}{
You gain +3 to your Disguise checks.}{
When creating a disguise, you roll twice and take the better result.}{
You can use \spell{Nystul's magic aura} as a spell-like ability at will, with a caster level equal to your character level and a save DC of 10 + \half your character level + your Cha modifier.}{
You can create a disguise as a full-round action, but you take a -10 penalty to your Disguise check. You can't be under direct observation while doing this, but you can use Bluff to create a diversion to allow you to change guises, as for the Hide skill.}{
You can choose an appearance that anyone viewing you with scrying or other divination magic sees instead of your "real" appearance. Even someone who benefits from \spell{true seeing} must succeed on a caster level check (DC 11 + your ranks in Disguise) to penetrate the illusion.}


\label{comm:feat:masterofterror}\skillfeat{Master of Terror [Leadership] [Skill]}{
You scare people so bad they follow you around hoping you won't hurt them.}{
Intimidate ranks:}{
Whenever you use Intimidate in combat, it affects everyone within 30 feet of you.}{
You gain followers. Your leadership score is equal to your ranks in Intimidate plus your Charisma modifier.}{
You gain a cohort who enjoys frightening your underlings almost as much as you do. A cohort is an intelligent and loyal creature with a CR at least 2 less than your character level. Cohorts gain levels when you do.}{
You gain the frightful presence ability. When you speak or attack, enemies within 30 feet of you must succeed on a Will save (DC 10 + \half your character level + your Cha modifier) or become shaken for 5d6 rounds. An opponent that succeeds on its saving throw is immune to your frightful presence for 24 hours.}{
Your opponents take a -2 morale penalty to saving throws if they can see you and you are within medium range (based on your character level).}


\label{comm:feat:naturalempath}\skillfeat{Natural Empath [Skill]}{
You read people like books.}{
Sense Motive ranks:}{
You gain a +3 bonus to Sense Motive checks.}{
You can quickly size up potential opponents. If you succeed on a Sense Motive check as a free action, opposed by their Bluff, you can tell if they're an even match (their CR equals your character level), an easy challenge (their CR is 1-3 less than your level), irrelevant (their CR is 4 or more less than your level), stronger (their CR is 1-3 higher than your level), or overwhelmingly powerful (their CR is 4 or more higher than your level). You can use this ability once on a particular creature every 24 hours.}{
If you succeed on a Sense Motive check, opposed by Bluff, you know your opponent's alignment. If you beat their Bluff by 20 or more, you can read their surface thoughts, as if during the third round of \spell{detect thoughts}.}{
You have an uncanny intuition for when people are interested in you. Any time someone uses a remote spell or effect, like \spell{scrying}, to examine you, you know you're under observation and if you make a Sense Motive check that beats their Bluff check, you know some details about them: if you've met them before, you recognize them, but if not, you get a basic idea of their reasons for their interest in you. Similarly, if you use Sense Motive on someone influenced by an enchantment effect, you can find out who created the effect with a Sense Motive check opposed by the controller's Bluff, getting the same information.}{
You know what people are going to do before they do. Any time someone you're aware of attacks you, make a Sense Motive check opposed by their Bluff: if you succeed, you get a free surprise round.}


\label{comm:feat:persuasive}\skillfeat{Persuasive [Skill]}{
When you tell you people something that contradicts the evidence of their own eyes, they believe you.}{
Diplomacy ranks:}{
You gain a +3 bonus to Diplomacy checks.}{
Your words can stop fights before they start. Any creature that can hear you speak must make a Will save (DC 10 + \half your character level + your Cha modifier) or it can't attack you directly; however, you aren't protected from its area or effect spells, or similar abilities. Any creature that succeeds on its save is immune to this ability for 24 hours. You may use nonattack spells or otherwise act, but if you attack the creature or its allies, it may attack you. This is a mind-affecting, language-dependent charm effect.}{
You can fascinate creatures with your silver tongue. You can affect as many HD of creatures as your bonus on Diplomacy checks; any creature that fails a Will save (DC 10 + \half your character level + your Cha modifier) becomes fascinated. If you use this ability in combat, each target gains a +2 bonus on its saving throw. If the spell affects only a single creature not in combat at the time, the saving throw has a penalty of -2. While a subject is fascinated by this spell, it reacts as though it were two steps more friendly in attitude, allowing you to make a single request of an affected creature. The request must be brief and reasonable. Even after the spell ends, the creature retains its new attitude toward you, but only with respect to that particular request. A creature that fails its saving throw does not remember that you enspelled it.}{
You can influence even hostile creatures into talking things over with you. With a DC 30 Diplomacy check, you can use a language-dependent version of \spell{charm monster} as a spell-like ability, with save DC equal to 10 + \half your character level + your Cha modifier; this is a mind-affecting charm effect.}{
You can convince an entire group of enemies to listen to you. If you succeed on a DC 40 Diplomacy check, your \spell{charm monster} ability improves to \spell{mass charm monster}, with a caster level equal to your bonus on Diplomacy checks.}


\label{comm:feat:professionalluddite}\skillfeat{Professional Luddite [Skill]}{
You've learned to break machines because you're an antitechnology fanatic -- or maybe you just work for the local protection racket.}{
Disable Device ranks:}{
You can use Disable Device on magic traps like a character with trapfinding. If you already have that ability, you gain +3 to your Disable Device checks. Disable Device is always a class skill for you.}{
You can use your Dexterity modifier instead of your Intelligence modifier for Disable Device checks. Darkness and blindness do not hinder your ability to disable devices.}{
You can reduce the amount of time required to disable a device. For each multiple of 10 you beat the required DC, you can decrease the time required from 2d4 rounds to 1d4 rounds to 1 round to a standard action to a move-equivalent action to a free action.}{
You can use Disable Device to end any persistent effect or area spell effect as if it was a magic trap, but the DC is 25 + twice the spell's level.}{
As an attack action, you can disable magic items. You must succeed on a melee touch attack roll for attended objects. Make a Disable Device check against a DC of 15 + the item's caster level: if your check succeeds, the item must make a Will save against a DC of 10 + \half your character level or be turned into a normal item, and even if it saves, its magical properties are suppressed for 1d4 rounds.}


\label{comm:feat:sharp-eyed}\skillfeat{Sharp-Eyed [Skill]}{
Nothing escapes you.}{
Spot ranks:}{
You gain a +3 bonus to Spot checks.}{
You can make a Spot check once a round as a free action. You don't take penalties for distractions on your Spot checks.}{
As a move action, you can make a Spot check against a DC of an opponent's Armor Class: if you succeed, you can ignore their Armor and Natural Armor bonus to AC for the next attack you make against them. If you accept a -20 penalty to your check, you can attempt this check as a swift action. Divide any distance penalties you take on Spot checks by two.}{
If you beat an opponent's Hide check with a Spot check at a -10 penalty, you can ignore concealment. If you beat their Hide check at a -30 penalty, you can ignore total concealment.}{
You can see through solid objects, but you take a -20 penalty on your Spot check for each 5'. Divide any distance penalties you take on Spot checks by five.}


\label{comm:feat:slipperycontortionist}\skillfeat{Slippery Contortionist [Skill]}{
Your childhood nickname was "Greasy the Pig," but now people call you "The Great Hamster."}{
Escape Artist ranks:}{
You gain +3 to your Escape Artist checks.}{
While squeezing into a space at least half as wide as your normal space, you may move your normal speed and you take no penalty to your attack rolls or AC for squeezing.}{
You can squeeze through a tight space or an extremely tight space as a full-round action, but you take a -10 penalty to your Escape Artist check. Opponents grappling you don't get positive size modifiers added to their grapple bonus when you use Escape Artist to try to break their hold.}{
If you succeed on a DC 30 Escape Artist check, you can ignore magical effects that impede movement as if you were under the effects of \spell{freedom of movement} for one round; this is not an action. You can also slip through a \spell{wall of force} or similar barrier with a DC 40 check.}{
You can make an Escape Artist check instead of a saving throw for any effect that would keep you from taking actions. (This does not help against effects that don't allow a saving throw.)}


\label{comm:feat:steadystance}\skillfeat{Steady Stance [Skill]}{
You can fight just about anywhere.}{
Balance ranks:}{
You gain a +3 bonus to your Balance checks.}{
If an effect would knock you prone, if you succeed on a DC 20 Balance check, you remain standing.}{
If your opponent is balancing, you gain a +3 dodge bonus to AC against their attacks unless they succeed at beating you in an opposed Balance check.}{
All Balance DCs are halved for you.}{
You never suffer any impairment or damage from anything you're standing on, whether it's molten lava, a cloud, or even another creature. Ambient conditions, such as lighting or weather, can still impair you.}


\label{comm:feat:stealthy}\skillfeat{Stealthy [Skill]}{
If someone sees you, you have to kill them.}{
Hide ranks:}{
You gain a +3 bonus to your Hide checks.}{
You can Hide as a free action after attacking, and snipe with melee attacks (or ranged attacks from closer than 10').}{
A constant \spell{nondetection effect} protects you and your equipment, with an effective caster level equal to your ranks in Hide.}{
You can attempt to Hide even when under direct observation, but you take the usual -20 penalty to your check.}{
Even opponents who can see you have trouble locating you. If they succeed at beating your Hide check with Spot (and thus can see you), they have a 50\% concealment miss chance when attacking you, which decreases by 5\% for each point they beat your Hide DC.}


\label{comm:feat:swimlikeafish}\skillfeat{Swim Like a Fish [Skill]}{
You're at least as home in the water as you are on land.}{
Swim ranks:}{
You gain +3 to your Swim checks.}{
You gain a swim speed equal to your base land speed, with the attendant benefits. You don't take armor check penalties to your Swim checks.}{
You can breathe water, and you can attack through water as if under the effects of \spell{freedom of movement}.}{
While under water, you can substitute Swim checks for Reflex saves, and you gain a +4 bonus to attack and damage rolls.}{
As a swift action, you can add your ranks in Swim as a dodge bonus to your Armor Class while under water.}


\label{comm:feat:track}\skillfeat{Track [Skill]}{
You feel at home no matter where you are.}{
Survival ranks:}{
You can follow tracks using Survival, as the Track and Legendary Tracker feats.}{
You can identify the race/kind of creatures from their tracks.}{
You can move through or over difficult natural terrain without being slowed, taking nonlethal damage, or suffering other impairment. You take no penalties for moving your speed when tracking, and only -10 when moving double your speed. You can track subjects protected by \spell{pass without trace} or similar spells at a -20 penalty.}{
You can track through the Astral Plane with a DC 35 Survival check. You can determine the destination of a teleportation spell when standing at the point of departure with a DC 40 Survival check; if you have \spell{teleport} or a similar spell, you can follow as if you had seen the destination once.}{
You're immune to natural planar effects as if you had \spell{planar tolerance} always active.}
\end{multicols}

\input{phb/feats/spellcastingfeats}

%%%%%%%%%%%%%%%%%%%%%%%%
%%Goods and Services chapter formatting
%%%%%%%%%%%%%%%%%%%%%%%%

\newcommand{\armorentry}[4]{\ability{#1}{#3\newline\indent\textbullet{ #4}}}

%%%%%%%%%%%%%%%%%%%%%%%%

\chapter{Goods and Services}
\section{The Three Economies}
foo
\section{Armor}
\afterpage{\newcommand{\armorcat}[1]{\cellcolor{headercolor}\textbf{#1}& \cellcolor{headercolor}AC& \cellcolor{headercolor}Max  Dex& \cellcolor{headercolor}ACP& \cellcolor{headercolor}Cost & \cellcolor{headercolor}Weight}

\newcommand{\armortablerow}[2]{#1 & #2\\}

\begin{table}
\caption{Armor Types}
{\tabulinesep=1mm
\rowcolors{1}{colorone}{colortwo}
\begin{tabu} to \linewidth {X[3l] c X[c] c l c   |   X[3l] c X[c] c l c}
\armortablerow{\armorcat{Non-\linebreak Armors}}{\armorcat{Medium\linebreak Armor}} \hline
\armortablerow{Camouflaged \linebreak Clothing & 0 & - & 0 & 1gp & 5 lb}{Gith Armor & +5 & +4 & -2 & 900gp & 25 lb}
\armortablerow{Functional \linebreak Clothing & 0 & - & 0 & 1gp & 5 lb}{Animal Spirit Armor & +4 & +3 & -3 & 1,000gp & 30 lb}
\armortablerow{Thieves \linebreak Clothing & 0 & - & 0 & 1gp & 5 lb}{Dragonscale Shirt\textsuperscript{1} & +7 & +5 & -4 & 3,400gp & 30 lb}
\armortablerow{Travelers \linebreak Clothing & 0 & - & 0 & 1gp & 5 lb}{Mithril Suit\textsuperscript{1} & +7 & +5 & -2 & 5,000gp & 25 lb}
\armortablerow{Fancy \linebreak Clothing & 0 & +6 & -1 & 30gp & 5 lb}{Adamantine Breastplate\textsuperscript{1} & +8 & +3 & -6 & 5,000gp & 60 lb}
\armortablerow{\armorcat{Light \linebreak Armor}}{\armorcat{Heavy \linebreak Armor}} \hline
\armortablerow{Cord \linebreak Armor & +2 & - & 0 & 20gp & 10 lb}{Hoplite\linebreak Armor & +6 & +1 & -9 & 250gp & 60 lb}
\armortablerow{Winter Clothes & +2 & +4 & -4 & 30gp & 25 lb}{Half Plate & +7 & +2 & -5 & 800gp & 45 lb}
\armortablerow{Leather\linebreak Armor & +2  & +7  & 0 & 45gp & 10 lb}{Coral Armor & +6 & +2 & -3 & 850gp & 40 lb}
\armortablerow{Still Suit & +2 & +5 & -3 & 50gp & 25 lb}{Great Armor & +7 & +2 & -7 & 1,000gp & 50 lb}
\armortablerow{Wicker \linebreak Armor & +3 & +7  & -1 & 55gp & 10 lb}{Full Plate & +8 & +1 & -6 & 1,100gp & 50 lb}
\armortablerow{Studded Leather & +3 & +6 & -1 & 65gp & 16 lb}{Silk Steel\linebreak Armor\textsuperscript{1} & +7 & +3 & -4 & 3,400gp & 50 lb}
\armortablerow{Chain Shirt & +4 & +5 & -2 & 100gp & 25 lb}{Dragonscale Suit\textsuperscript{1} & +9 & +4 & -5 & 5,500gp & 40 lb}
\armortablerow{Brigadine & +4 & +6 & -2 & 110gp & 25 lb}{Adamantine Carapace\textsuperscript{1} & +11 & +2 & -9 & 9,000gp & 75 lb}
\armortablerow{Gray \linebreak Armor & +4 & - & 0 & 1,000gp & 15 lb}{\armorcat{Shields}}\tabucline{7-12}
\armortablerow{Ironskin Leather\textsuperscript{1}& +5 & +5 & -2 & 3,300 gp & 15 lb}{Buckler & +1 & - & -1 & 15gp & 5 lb}
\armortablerow{Mithril Shirt\textsuperscript{1} & +5 & +6 & -1 & 4,000 gp & 15 lb}{Wooden Shield & +1 & - & 0 & 15gp & 5 lb}
\armortablerow{\armorcat{Medium \linebreak Armor}}{Steel Shield & +2 & - & -1 & 50gp & 10 lb}\tabucline{1-6}
\armortablerow{Hide & +3 & +4 & -4 & 15gp & 25 lb}{Dragonscale Shield\textsuperscript{1} & +3 & - & -1 & 950gp & 15 lb}
\armortablerow{Scale Mail & +4	 & +3 & -5 & 50gp & 35 lb}{Mithril Shield\textsuperscript{1} & +2 & - & 0 & 1,020gp & 5 lb}
\armortablerow{Ringmail & +4 & +6 & -2 & 75gp & 30 lb}{Adamantine & +3 & - & -1 & 2,000gp & 25 lb}
\armortablerow{Chainmail & +5 & +3 & -3 & 150gp & 40 lb}{\armorcat{Great \linebreak Shields}}\tabucline{7-12}
\armortablerow{Lamellar & +5 & +4 & -4 & 190gp & 40 lb}{Tower Shield & +4 & - & -10 & 100gp & 45 lb}
\armortablerow{Elaborate Gown & +1 & +3 & -5 & 300gp & 15 lb}{Kite Shield & +4 & - & -5 & 120gp & 35 lb}
\armortablerow{Lobster Mail & +5 & +2 & -5 & 350gp & 40 lb}{Bone Wall & +3 & - & -10 & 150gp & 30 lb}
\armortablerow{Breastplate & +6 & +2 & -4 & 450gp & 40 lb}{Kappa Shell & +3 & - & -12 & 500gp & 30 lb}
\armortablerow{Chitin\linebreak Carapace & +6 & +4 & -3 & 500gp & 30 lb}{& & & & & }
\hline\multicolumn{12}{l}{\textsuperscript{1} This armor is already masterwork, and cannot be improved further.}\\ \hline
\end{tabu}}
\end{table}}

\subsection{Armor Qualities}
To wear heavier armor effectively, a character can select the Armor Proficiency feats, but most classes are automatically proficient with the armors that work best for them. Characters who are not proficient with an armor or shield incur additional penalties when wearing them. Armor and shields can take damage from some types of attacks. Here is the format for armor entries (given as column headings on the table below).

\subsection{Category}
Armors and shields come in different categories. Nonarmor, light armor, medium armor, and heavy armor for armors, and shields and great shields for shields. Characters who are wearing armor that they do not have the appropriate proficiency with take additional penalties for wearing them (see Armor Class Penalties below). All characters are proficient with nonarmors.

\subsection{Cost}
The cost of the armor for Small or Medium humanoid creatures. See Armor for Unusual Creatures, below, for armor prices for other creatures.

\subsection{Armor Class (AC) Bonus}
Each armor grants an armor bonus to AC, while shields grant a shield bonus to AC. The armor bonus from a suit of armor doesn't stack with other effects or items that grant an armor bonus. Similarly, the shield bonus from a shield doesn't stack with other effects that grant a shield bonus.

\subsection{Maximum Dex Bonus (Max Dex)}
This number is the maximum Dexterity bonus to AC that this type of armor allows. Heavier armors limit mobility, reducing the wearer's ability to dodge blows. This restriction doesn't affect any other Dexterity-related abilities.

Even if a character's Dexterity bonus to AC drops to 0 because of armor, this situation does not count as losing a Dexterity bonus to AC.

Your character's encumbrance (the amount of gear he or she carries) may also restrict the maximum Dexterity bonus that can be applied to his or her Armor Class.

Shields do not affect a character's maximum Dexterity bonus.

\subsection{Armor Check Penalty (ACP)}
Any armor heavier than leather hurts a character's ability to use some skills. An armor check penalty number is the penalty that applies to Balance, Climb, Escape Artist, Hide, Jump, Move Silently, Sleight of Hand, or Tumble checks by a character wearing a certain kind of armor. Double the normal armor check penalty is applied to Swim checks. If a character casts arcane spells in armor which has an Armor Check Penalty, their spells with somatic components have a 5\% chance of failing for every point of armor check penalty. A character's encumbrance (the amount of gear carried, including armor) may also apply an armor check penalty.

If a character is wearing armor and using a shield, both armor check penalties apply.

If a character has a current armor check penalty, from his armor and shield combined, that is greater than their Base Attack Bonus, they move at reduced speed (\sfrac{2}{3}~Speed). If the Armor Check Penalty is 4 points greater than their Base Attack Bonus, they cannot take the Run action. If the Armor Check Penalty is ten points greater than their Base Attack Bonus, they can only stagger around, and only gain a standard action each round instead of a full-round action.

\subsection{Nonproficient with Armor Worn}
A character who wears armor and/or uses a shield with which he or she is not proficient treats that armor as if it had an Armor Check Penalty 4 greater than normal.

\subsection{Sleeping in Armor}
A character who sleeps in medium or heavy armor is automatically fatigued the next day. He or she takes a -2 penalty on Strength and Dexterity and can't charge or run. Sleeping in light armor does not cause fatigue.

\subsection{Weight}
This column gives the weight of the armor sized for a Medium wearer. Armor fitted for Small characters weighs half as much, and armor for Large characters weighs twice as much.

\subsection{Individual Armor Entries}
In addition to the qualities given on the table, every armor and shield provides a special benefit unique to it, given below.

\armorentry{Adamantine Breastplate}{Medium Armor}{Made of one of the hardest and most durable known metals, this breastplate provides excellent protection.}{You gain DR equal to your BaB/Adamantine.}

\armorentry{Adamantine Carapace}{Heavy Armor}{Made of one of the hardest and most durable known metals, this armor completely encases you.}{You gain DR equal to your BaB/Adamantine}

\armorentry{Adamantine Shield}{Shield}{A target shield constructed of pure Adamantine, it is nearly indestructible and can be placed between your important bits and deadly weapons.}{As an Immediate action you may force an opponent to reroll a successful attack against you.}

\armorentry{Animal Spirit Armor}{Medium Armor}{Fashioned of the skin of an angry beast, this armor still carries its spirit and will lend you its strength.}{Whilst Charging all your attacks deal +1d6 damage. If you have additional attacks from a high BaB this damage increases by +1d6 per extra attack you have gained.}

\armorentry{Bone Wall}{Great Shield}{A seemingly random assortment of bones collected into a large shield.}{A Bone Wall provides a bonus to saves against Death Effects and Necromancy spells equal to its Shield bonus}

\armorentry{Breastplate}{Medium Armor}{A solid steel armor underlayed with chainmail, it protects all of your vital bits.}{Whenever you suffer lethal physical damage you can convert an amount of that damage equal to your BAB into nonlethal damage.}

\armorentry{Brigadine}{Light Armor}{A sleeveless leather shirt with small steel plates riveted to the fabric. Like a medieval flak jacket, these plates protect your organs from stabbing weapons.}{Brigandine prevents an amount of Piercing damage per attack equal to your BAB.}

\armorentry{Buckler}{Shield}{A small shield strapped to the wrist or forearm used for parrying.}{A buckler provides no bonuses while you are denied your Dexterity bonus to AC. You may use a weapon with the hand using the Buckler, but doing so causes you to suffer a -1 penalty to attack rolls using this hand (including two handed weapons).}

\armorentry{Chain Shirt}{Light Armor}{Interwoven links of the finest steel cover your torso.}{A Chain shirt prevents an amount of Slashing damage per attack equal to your BAB.}

\armorentry{Chainmail}{Medium Armor}{Interwoven links of the finest steel cover your entire body.}{Chainmail prevents an amount of Slashing damage per attack equal to your BAB.}

\armorentry{Chitin Carapace}{Medium Armor}{Made out of an Ankheg or something, it's amazingly light and makes you look like a crazy mantis man when you wear it.}{You gain a climb speed equal to your land speed.}

\armorentry{Coral Armor}{Heavy Armor}{Made of living Coral, this armor is as dangerous to your opponents as it is protective.}{All Coral Armor counts as having been made with Armor Spikes. Enemies you damage with the Coral spikes are poisoned (DC 10 + \half your Level + Con Bonus), initial and secondary damage of 1d3 Dex.}

\armorentry{Cord Armor}{Light Armor}{A series of knots wrapped about your person protect you from incoming attacks.}{As an Immediate Action, you can replace the AC for one attack made against you with the result of a tumble check. You may use this ability after an attack has hit you, but not after damage has been rolled.}

\armorentry{Dragonscale Shield}{Shield}{Fashioned from a single dragon scale, this shield deflects dragon breath and other similar attacks.}{Whilst carrying this shield you may use an Immediate Action to gain Evasion for 1 round.}

\armorentry{Dragonscale Shirt}{Medium Armor}{Forged from the hide of a dragon, this armor provides a warding effect against its still living brethren.}{You gain Energy Resistance to one Energy type [chosen when the armor is made] equal to the Armor Bonus the shirt provides.}

\armorentry{Dragonscale Suit}{Heavy Armor}{This armor forged from the hide of a dragon provides excellent protection from swords and flames alike.}{You gain Energy Resistance to one Energy type equal to the Armor Bonus the suit provides (inc. any Enhancement bonus).}

\armorentry{Elaborate Gown}{Medium Armor}{It's a big frilly dress, or a bulky robe, or something else that's expensive and hard to move in.}{You gain a +2 bonus to Intimidate and Perform checks. You can hide weapons of a size up to your own inside your outfit with a normal Sleight of hand check.}

\armorentry{Full Plate}{Heavy Armor}{Plates of steel, flexible chain armor, and leather straps encase your entire body.}{Whenever you suffer lethal physical damage you may convert an amount of that damage equal to your BAB into nonlethal damage.}

\armorentry{Gith Armor}{Medium Armor}{The Gith have mastered the techniques of manipulating Astral Driftmetal and the chaotic stuff of Limbo to make a reasonably lightweight, yet oddly protective garment.}{You do not suffer any Arcane Spell Failure in this armor if you are proficient with it.}

\armorentry{Gray Armor}{Light Armor}{Made from slaadskin, gray armor shifts to anticipate attacks with the power of Limbo.}{You always act in the surprise round of any combat.}

\armorentry{Great Armor}{Heavy Armor}{Large, bulky armor typically worn by Samurai, Great Armor consists of many bands of lacquered iron with great shoulder pads known as O-sode.}{Great Armor includes arm panels which provide a +2 Shield bonus to AC}

\armorentry{Half Plate}{Heavy Armor}{A mixture of rigid plates and flexible chain, Half Plate combines protection with flexibility.}{Whenever you suffer lethal physical damage you may convert an amount of that damage equal to your BAB into nonlethal damage.}

\armorentry{Hide Armor}{Medium Armor}{You wear some other creature as a hat.}{Animals will not attack you unless you attack or otherwise threaten them first, and you cannot be detected by Scent.}

\armorentry{Hoplite Armor}{Heavy Armor}{Oldschool armor made of bronze and layered on in thick sheets all over the vitals.}{Hoplite Armor provides Medium Fortification as an Ex ability whilst worn.}

\armorentry{Ironskin Armor}{Light Armor}{The skin of creatures resistant to nonmagical weapons must be cut using enchanted tools, however the resulting armor is both light and hard as steel.}{You gain DR equal to your BaB/Magic and Piercing.}

\armorentry{Kappa Shell}{Great Shield}{Fitting on the back of a character like the Kappa's Shell it is named after, the Kappa Shell provides decent protection for units moving across a dangerous battle field.}{You may use both your hands while using this shield, but your attacks suffer a -2 penalty.}

\armorentry{Kite Shield}{Great Shield}{Shaped metal shields fit easily over a rider and his steed, allowing good protection for mounted troops.}{Your mount gains any benefit you gain from this shield (AC, feat bonuses etc.).}

\armorentry{Lamellar Armor}{Medium Armor}{Small metal or leather plates that are linked together to protect you, it provides a smooth surface that causes attacks to slide away rather than pierce the wearer.}{Prevents an amount of Piercing damage per attack equal to your BaB.}

\armorentry{Leather Armor}{Light Armor}{Made from cured cowhide, it's cheap, but does its job.}{You gain fire resistance 5. This fire resistance does not protect against environmental effects.}

\armorentry{Lobster Mail}{Medium Armor}{A carapace of a deep aquatic design. The engineering style is reminiscent of that of Kwalish, but much sleeker and individualized.}{Whilst underwater this armor allows you to breathe water, and its ACP is reduced to 0 if you are proficient with it.}

\armorentry{Mithril Shield}{Shield}{It's as strong as titanium and as light as titanium. Who are we fooling? This is a great shield.}{Attacks against you by enemies that are not flanking you have a 20\% miss chance.}

\armorentry{Mithril Shirt}{Light Armor}{The links that make up this chain shirt are woven much tighter and made of mithril.}{You gain DR equal to your BaB/Bludgeoning}

\armorentry{Mithril Suit}{Medium Armor}{A full body covering of light metal. Very shiny, and nearly skin tight, the mithril suit is surprisingly protective.}{You gain DR equal to your BaB/Bludgeoning.}

\armorentry{Ringmail}{Medium Armor}{Steel rings are woven onto a leather armor backing to provide extra reinforcement.}{Ringmail prevents the first 10 points of damage from a critical hit or sneak attack damage.}

\armorentry{Scale Mail}{Medium Armor}{Many small armor scales attached to each other in overlapping rows over a backing of leather, its rigid surface provides better protection from blunt attacks than mail.}{Prevents an amount of Bludgeoning damage per attack equal to your BAB.}

\armorentry{Silk Steel Armor}{Heavy Armor}{Made using an ancient bugbear technique, overlapping plates of steel are held apart by layers of silk and the entire carapace slides virtually without sound. While bulky, this black armor is remarkably stealthy.}{Whenever you suffer lethal physical damage you may convert an amount of that damage equal to your BAB into nonlethal damage. The ACP of this armor is ignored when using Move Silently.}

\armorentry{Steel Shield}{Shield}{It can be round or square or shaped like something in particular. It's not important, the key is that it's between you and sharp objects and it's made out of steel.}{Whilst there are no enemies within your natural reach you count as having cover from ranged attacks.}

\armorentry{Still Suit}{Light Armor}{A watertight suit envelops your whole body, recycling all of your excretions and protecting you from the heat.}{Whilst wearing a Still Suit you do not have to make Fortitude checks for hot environments and use half the daily allowance of water.}

\armorentry{Studded Leather Armor}{Light Armor}{Leather armor that has been adorned with metal studs to help protect your vitals.}{You may ignore the first 10 points of damage from each critical hit or sneak attack damage.}

\armorentry{Tower Shield}{Great Shield}{Giant pieces of wood or metal, tower shields offer tremendous protection, but cannot be effectively used while mounted.}{You may claim \half Cover. This must be declared at the start of your action and lasts until the start of your next action, but your own attacks suffer a -2 to hit penalty during this time.}

\armorentry{Wicker Armor}{Light Armor}{This armor is made of woven willow branches, and makes you look a bit like furniture with wings.}{Whilst wearing Wicker Armor you are immune to falling damage and gain a +3 circumstance bonus to Jump checks.}

\armorentry{Winter Clothes}{Light Armor}{Thick clothing that keeps you warm.}{You do not have to make Fortitude checks for cold environments whilst wearing Winter Clothing and you gain Cold resistance 5.}

\armorentry{Wooden Shield}{Shield}{Made of wood and held together with bands of steel or strips of leather, a wooden shield makes up in shock absorbance what it loses in resilience.}{Whilst there are no enemies within your natural reach you count as having cover from ranged attacks.}

\subsection{Masterwork Armor}

You can purchase or craft masterwork versions of armor or 
shields. Such a well-made item functions like the normal version, except that its 
armor check penalty is lessened by 1. 

A masterwork suit of armor or shield costs an extra 150 gp over and above the normal 
cost for that type of armor or shield.

The masterwork quality of a suit of armor or shield never provides a bonus on attack 
or damage rolls, even if the armor or shield is used as a weapon.

All magic armors and shields are automatically considered to be of masterwork quality.

You can't add the masterwork quality to armor or a shield after it is created; 
it must be crafted as a masterwork item.

\subsection{Armor For Unusual Creatures}
\begin{wraptable}{r}{.5\linewidth}
\caption{Nonstandard Armors}
\rowcolors{1}{colorone}{colortwo}
\centering
{\tabulinesep=1mm
\begin{tabu}to \linewidth {X[l] X[c] X[c] X[c] X[c]}
\header\textbf{Size} & \multicolumn{2}{c}{\textbf{Humanoid}} & \multicolumn{2}{c}{\textbf{Non-Humanoid}}\\ 
\header & \textit{Cost} & \textit{Weight} & \textit{Cost} & \textit{Weight} \\ \hline
Tiny or smaller\textsuperscript{1} & x\sfrac{1}{2} & x\sfrac{1}{10} & x1  & x\sfrac{1}{10}\\
Small & x1 & x\sfrac{1}{2} & x1 & x\sfrac{1}{2}\\
Medium & x1 & x1 & x2 & x1\\
Large & x2 & x2 & x4 & x2\\
Huge & x4 & x5 & x8 & x5\\
Gargantuan& x8 & x8 & x16 & x8\\
Colossal & x16 & x12 & x32 & x12\\ \hline
\multicolumn{5}{l}{\textsuperscript{1}Divide armor bonus by 2.}\\ \hline
\end{tabu}
}
\end{wraptable}

Armor and shields for unusually big creatures, unusually little creatures, and nonhumanoid creatures have different costs and weights from those given on the armors table. Refer to the appropriate line on Table: Nonstandard Armors and apply the multipliers to cost and weight for the armor type in question.

\subsection{Getting Into And Out Of Armor}

The time required to don armor depends on its type, see the Table: Donning and Removing Armor.

\textbf{Don:} This column tells how long it takes a character to put the armor 
on. (One minute is 10 rounds.) Readying (strapping on) a shield is only a move 
action.

\textbf{Don Hastily:} This column tells how long it takes to put the armor on in 
a hurry. The armor check penalty and armor bonus for hastily donned armor are each 
1 point worse than normal. 

\textbf{Remove:} This column tells how long it takes to get the armor off. Loosing 
a shield (removing it from the arm and dropping it) is only a move action.

~\\*
\begin{table}[b]
\caption{Donning and Removing Armor}
{\tabulinesep=1mm\centering
\rowcolors{1}{colorone}{colortwo}
\begin{tabu}to \textwidth {X l l l}
\header\textbf{Armor Type} & \textbf{Don} & \textbf{Don Hastily} & \textbf{Remove}\\ \hline
Shield or Great Shield & 1 move action & \sfrac{n}{a} & 1 move action\\
Nonarmor & 5 rounds & 1 full-round action & 1 full-round action\\
Light Armor & 1 minute & 5 rounds & 1 minute\textsuperscript{1}\\
Medium Armor & 4 minutes\textsuperscript{1} & 1 minute & 1 minute\textsuperscript{1}\\
Heavy Armor & 4 minutes\textsuperscript{2} & 4 minutes\textsuperscript{1} & 1d4+1 minutes\textsuperscript{1}\\ \hline
\multicolumn{4}{p{\textwidth}}{\textsuperscript{1} If the character has some help, cut this time in half. A single character doing nothing else can help one or two adjacent characters. Two characters can't help each other don armor at the same time.}\\
\multicolumn{4}{p{\textwidth}}{\textsuperscript{2} The wearer must have help to don this armor. Without help, it can be donned only hastily.}\\ \hline
\end{tabu}}
\end{table}
\section{Weapons}
At their core, a weapon is just an object of a particular size and complexity that you wield against a foe in an attempt to disable them. The size, and complexity of that object are not negligible parts of it though, and are in fact the basis of an effective weapon.

\subsection{Weapon Descriptors}

\subsubsection{Melee Weapons}
Melee weapons are used for making attacks against nearby foes, and generally threaten every space within a creature's natural reach.

\textbf{Reach Weapons:} A reach weapon is a melee weapon with a much longer haft than normal, allowing its wielder to strike at targets that aren't adjacent to him or her. Most reach weapons double the wielder's natural reach, meaning that a typical Small or Medium wielder of such a weapon can attack a creature 10 feet away, but not a creature in an adjacent square. A typical Large character wielding a reach weapon of the appropriate size can attack a creature 15 or 20 feet away, but not adjacent creatures or creatures up to 10 feet away. There may be limits on how you can use a reach weapon, consult each weapons individual entry. 

\textbf{Hurled Weapons:} Some melee weapons also list a range. These may be hurled at a target, and function as a ranged weapon when they are used in this way. See "Thrown Weapons" for more information about using a hurled melee weapon in this way.

\textbf{Double Weapons:} A double weapon has a damaging head on both the ends of the weapon. A character can fight with both ends of a double weapon as if fighting with two weapons, but they incur all the normal attack penalties associated with two-weapon combat as if they were wielding a one-handed weapon and a light weapon. The character can also choose to use a double weapon two handed, attacking with only one end of it. A creature wielding a double weapon in one hand can't use it as a double weapon-only one end of the weapon can be used in any given round. 

\subsubsection{Ranged Weapons}
Ranged weapons are suited for striking distant foes, and threaten no spaces. Using a ranged weapon within a threat range generally provokes attacks of opportunity, making them well suited to firing from out of the fray.

\textbf{Projectile Weapons:} Crossbows, repeating crossbows, bows, compound bows, and slings are projectile weapons. Most projectile weapons require two hands to use (see specific weapon descriptions). A character does not add their Strength bonus on damage rolls with a projectile weapon unless it's a composite bow or sling. If the character has a penalty for low Strength, it is added to damage rolls when they use a projectile weapon other than a crossbow.

\textbf{Ammunition:} Projectile weapons use ammunition: arrows (for bows), bolts (for crossbows), or sling bullets (for slings). When using a bow, a character can draw ammunition as a free action; crossbows and slings require an action for reloading. Generally speaking, ammunition that hits its target is destroyed or rendered useless, while normal ammunition that misses has a 50\% chance of being destroyed or lost. 

Attempting to use an arrow or bolt as a melee weapon incurs a -4 non-proficiency penalty, and deals damage equal to the bow or crossbow it was designed for. Sling bullets may not be used as a melee weapon.

\textbf{Ranged weapons and Mounts:} Thrown weapons and crossbows can be used from mounts without complication (aside from the normal penalties for using ranged weapons from mounts). Bows must be at least one size category smaller than the wielder to be used on a mount.

\textbf{Thrown Weapons:} In order to use a thrown weapon properly, it must be small enough for the wielder to use one handed. Ranged weapons the same size as the wielder can be thrown with two hands, but doing so incurs a -4 penalty on the attack roll. The wielder applies his or her Strength modifier to damage dealt by thrown weapons (except for splash weapons). It is possible to throw a weapon that isn't designed to be thrown (that is, a melee weapon that doesn't have a numeric entry in the Range Increment column on Table: Weapons), but a character who does so takes a -4 penalty on the attack roll. Throwing a light or one-handed weapon is a standard action, while throwing a two-handed weapon is a full-round action. Regardless of the type of weapon, such an attack scores a threat only on a natural roll of 20 and deals double damage on a critical hit. Such a weapon has a range increment of 10 feet. Any weapon three sizes smaller than the wielder can be thrown with a 10 foot range increment without penalty. 

\subsubsection{Improvised Weapons}
Sometimes objects not crafted to be weapons nonetheless see use in combat. Because such objects are not designed for this use, any creature that uses one in combat is considered to be nonproficient with it and takes a -4 penalty on attack rolls made with that object. To determine the size category and appropriate damage for an improvised weapon, compare its relative size and damage potential to the weapon list to find a reasonable match. An improvised weapon scores a threat on a natural roll of 20 and deals double damage on a critical hit. An improvised thrown weapon has a range increment of 10 feet. Objects heaver than a character's light load cannot be used as weapons.

\subsubsection{Weapon Size}
Every weapon, like every object and creature, has a size category that indicates how different sized creatures can interact with it.

\textbf{Two-Handed:} A two-handed weapon is one that is the same size category as the wielder. Two-handed weapons must be wielded with both the primary and off hand to be effective. Attacks with a two-handed melee weapon add 1-1/2 times the character's Strength bonus to damage rolls.

\textbf{One-Handed:} A one-handed weapon is one that is one size category smaller than the wielder. One-handed weapons can be used in either the primary hand or the off hand. Attacks with a one-handed melee weapon add the wielder's Strength bonus to damage rolls if it's used in the primary hand, or 1/2 their Strength bonus if it's used in the off hand. If a one-handed melee weapon is wielded with two hands during combat, 1-1/2 times the character's Strength bonus is added to damage rolls. 

\textbf{Light:} A light weapon is one that is two or more size categories smaller than the wielder. Light weapons can be used in either the primary hand or the off hand. Attacks with a light weapon add the wielder's Strength bonus (if any) to damage rolls for melee attacks with a light weapon if it's used in the primary hand, or one-half the wielder's Strength bonus if it's used in the off hand. Using two hands to wield a light weapon gives no advantage on damage; the Strength bonus applies as though the weapon were held in the wielder's primary hand only. It is even easier to use in one's off hand than a one-handed weapon is, however, and they are well suited for two-weapon fighting styles. Light melee weapons can also be used while grappling. An unarmed strike is always considered a light weapon.

\textbf{Inappropriately Sized Weapons:} A creature can't use weapons larger than itself.

\subsubsection{Weapon Complexity}
In addition to their size, every weapon is grouped according to how difficult the weapon is to master. Every weapon falls into one of three broad categories: simple, martial, and exotic. 

\textbf{Simple Weapons:} Simple weapons are those that require the least training and practice to use effectively. Many are simple 'swing and pray' or 'point and thrust' weapons that are hard to mess up, or simply weapons that are so small that they only work in very straightforward ways. Most classes are proficient in all simple weapons.

\textbf{Martial Weapons:} Martial weapons are those that require dedication and training to use effectively. 

Because of this martial weapons that are three sizes smaller than the wielder are always treated as simple weapons for the wielder.

\textbf{Exotic Weapons:} 

\subsection{Weapon Description}
Weapon entries follow the following format.

\noindent\textbf{Cost:} This value is the weapon's cost in gold pieces (gp) or silver pieces (sp). The cost includes miscellaneous gear that goes with the weapon. The cost is the same for a Small or Medium version of the weapon, while a Large version costs twice the listed price. Versions smaller than Small cost half as much for each size category reduction.

\textbf{Damage:} Each type of weapon deals damage based on its size.

\textbf{Critical:} The entry in this column notes how the weapon is used with the rules for critical hits. The first number indicates the minimum d20 roll that will generate a critical threat. When your character confirms a threat and scores a critical hit, roll the weapon damage two, three, or four times as indicated by its critical multiplier (using all applicable modifiers to weapon damage on each roll), and add all the results together. Extra damage over and above a weapon's normal damage, such as sneak attack damage, is not multiplied when you score a critical hit.

\begin{itemize}
  \item\noindent\textbf{20x2:}{The weapon deals double damage on a critical hit.}
  \item\noindent\textbf{20x3:}{The weapon deals triple damage on a critical hit.}
  \item\noindent\textbf{x3/x4:}{One head of this double weapon deals triple damage on a critical hit. The other head deals quadruple damage on a critical hit.}
  \item\noindent\textbf{20x4:}{The weapon deals quadruple damage on a critical hit.}
  \item\noindent\textbf{19-20/x2:}{The weapon scores a threat on a natural roll of 19 or 20 (instead of just 20) and deals double damage on a critical hit. (The weapon has a threat range of 19-20.)}
  \item\noindent\textbf{18-20/x2:}{The weapon scores a threat on a natural roll of 18, 19, or 20 (instead of just 20) and deals double damage on a critical hit. (The weapon has a threat range of 18-20.)}
  \item\noindent\textbf{19-20/x3:}{The weapon scores a threat on a natural roll of 19 or 20 (instead of just 20) and deals triple damage on a critical hit. (The weapon has a threat range of 19-20.)}
\end{itemize}

\textbf{Range:} Any attack at less than this distance is not penalized for range. However, each full range increment imposes a cumulative -2 penalty on the attack roll. A thrown weapon has a maximum range of five range increments. A projectile weapon can shoot out to ten range increments.

\textbf{Type:} Weapons are classified according to the type of damage they deal: bludgeoning, piercing, or slashing. Some monsters may be resistant or immune to attacks from certain types of weapons. Some weapons deal damage of multiple types. If a weapon is of two types, the damage it deals is not half one type and half another; all of it is both types. Therefore, a creature would have to be immune to both types of damage to ignore any of the damage from such a weapon.  In other cases, a weapon can deal either of two types of damage. In a situation when the damage type is significant, the wielder can choose which type of damage to deal with such a weapon.

\textbf{Special:} Some weapons have special features. See the weapon descriptions for details.

\afterpage{\newcommand{\wepcell}[3]{\begin{tabular}{l}#1\\#2\\#3\end{tabular}}

\begin{small}
{\tabulinesep=1mm
\rowcolors{1}{colorone}{colortwo}
\begin{longtabu} to \textwidth {X[3.75, l] X[3, l] X[4, l] X[3, l] X[2, l] X[2, l] X[2, l] X[2, l] X[2, l] X[2, l]}
\header\textbf{Simple Weapons}&&&&&&&&\\
\hline
\rowcolor{colortwo}\textbf{Weapon} &\textbf{Critical} &\textbf{Type} &\textbf{Range} &\textbf{Dimin.} &\textbf{Tiny} &\textbf{Small} &\textbf{Medium} &\textbf{Large} \endhead
 Club &20x2 &Bludgeoning &Melee &1d3 &1d4  &1d6   &1d8 &2d6 \\[1ex]
 Crowssbow &19-20x2 &Piercing &120 ft. &1d4 &1d6 &1d8 &1d10 &2d8 \\[1ex]
 Gauntlet\footnotemark[1] &20x2 &Bludgeoning &Melee &- &1 &1d2 &1d3 &1d4 \\[1ex]
 Hammer &20x2 &Bludgeoning &Melee &1d3 &1d4 &1d6 &1d8 &2d6 \\[1ex]
 Longspear &20x3 &Piercing &Reach &1d3 &1d4 &1d6 &1d8 &2d6 \\[1ex]
 Morning Star &20x2 &Bludgeoning \& Piercing &Melee &1d4 &1d6 &1d8 &2d6 &3d6 \\
 Sling &20x2 &Bludgeoning &50 ft. &1 &1d2 &1d3 &1d4 &1d6 \\[1ex]
 Spear &20x3 &Piercing &Melee or 20 ft. &1d3 &1d4 &1d6 &1d8 &2d6 \\
 Staff &20x2 &Bludgeoning &Melee &\sfrac{1d2}{1d2} &\sfrac{1d3}{1d3} &\sfrac{1d4}{1d4} &\sfrac{1d6}{1d6} &\sfrac{1d8}{1d8} \\
 Spiked Gauntlet\footnotemark[1] &20x2 &Piercing and Bludgeoning &Melee &1 &1d2 &1d3 &1d4 &1d6 \\
 Unarmed\footnotemark[1] &20x2 &Bludgeoning &Melee &- &1 &1d2 &1d3 &1d4 \\[1ex] \hline
\end{longtabu} 
\vspace{-1pt}
\vspace{-1\baselineskip}
\noindent\begin{tabu}to \textwidth{X}
\cellcolor{colortwo}\footnotemark[1] The size and damage for this weapon indicates the size of the creature using it, instead of the actual size of the weapon. These weapons are always considered light weapons.\\ \hline
 \end{tabu}
%
\rowcolors{1}{colorone}{colortwo}
\begin{longtabu} to \textwidth {X[6, l] X[3, l] X[4, l] X[3, l] X[2, l] X[2, l] X[2, l] X[2, l] X[2, l] X[2, l]}
\header\textbf{Martial Weapons}&&&&&&&&&\\
\hline
\rowcolor{colortwo}\textbf{Weapon} &\textbf{Critical} &\textbf{Type} &\textbf{Range} &\textbf{Fine} &\textbf{Dimin.} &\textbf{Tiny} &\textbf{Small} &\textbf{Medium} &\textbf{Large} \endhead
 Axe &20x3 &Bludgeoning \& Slashing &Melee &1d3 &1d4 &1d6 &1d8 &1d12 &3d6 \\
 Bastard Sword &19-20x2 &Slashing or Piercing &Melee &1d3 &1d4 &1d6 &1d8 &1d10 &2d8 \\
 Bow &20x3 &Piercing &100 ft. &- &1d3 &1d4 &1d6 &1d8 &2d6 \\[1ex]
 Composite Bow &20x3 &Piercing &110 ft.&- &1d3 &1d4 &1d6 &1d8 &2d6 \\[1ex]
 Curved Sword &18-20x2 &Slashing &Melee &1d2 &1d3 &1d4 &1d6 &2d4 &2d6 \\[1ex]
 Dwarven Axe &20x3 &Bludgeoning \& Slashing &Melee &1d3 &1d4 &1d6 &1d8 &1d10 &1d12 \\
 Flail &19-20x2 &Bludgeoning &Melee &1d3 &1d4 &1d6 &1d8 &1d10 &2d6 \\[1ex]
 Glaive &20x3 &Slashing &Reach &1d3 &1d4 &1d6 &1d8 &1d10 &2d6 \\[1ex]
 Greatclub &20x2 &Bludgeoning &Melee &1d3 &1d4 &1d6 &1d8 &1d10 &2d6 \\[1ex]
 Guisarme &20x3 &Slashing &Reach &1d2 &1d3 &1d4 &1d6 &2d4 &2d6 \\[1ex]
 Halberd &20x3 &Percing or Slashing &Melee &1d3 &1d4 &1d6 &1d8 &1d10 &2d6 \\
 Pick &20x4 &Piercing &Melee &1 &1d2 &1d4 &1d6 &1d8 &1d10 \\[1ex]
 Ranseur &20x3 &Piercing &Reach &1d2 &1d3 &1d4 &1d6 &2d4 &2d6 \\[1ex]
 Sap &20x2 &Bludgeoning &Melee &1d3 &1d4 &1d6 &1d8 &1d10 &2d6 \\[1ex]
 Scythe &20x4 &Piercing or Slashing &Melee &1d2 &1d3 &1d4 &1d6 &2d4 &2d6 \\
 Shield &20x2 &Bludgeoning &Melee &1 &1d2 &1d3 &1d4 &1d6 &1d8 \\[1ex]
 Spiked Armor\footnotemark[1] &20x2 &Piercing &Melee &1 &1d2 &1d3 &1d4 &1d6 &1d8 \\[1ex]
 Spiked Shield &20x2 &Bludgeoning \& Piercing &Melee &1d3 &1d4 &1d6 &1d8 &1d10 &2d6 \\[1ex]
 Sword &19-20x2 &Slashing or Piercing &Melee &1d3 &1d4 &1d6 &1d8 &2d6 &3d6 \\
 Thinblade &19-20x3 &Piercing &Melee &1d2 &1d3 &1d4 &1d6 &2d4 &2d6 \\[1ex]
 Throwing Axe &20x2 &Bludgeoning \& Slashing &10 ft. &1d2 &1d3 &1d4 &1d6 &1d8 &1d10 \\
 Throwing Hammer &20x2 &Bludgeoning &20 ft. &1d2 &1d3 &1d4 &1d6 &1d8 &1d10  \\[1ex]
 Trident &20x2 &Piercing &Melee or 10 ft. &1d3 &1d4 &1d6 &1d8 &1d10 &2d6 \\
 Warhammer &20x3 or 20x4 &Bludgeoning or Piercing &Melee &1d3 or 1d2 &1d4 or 1d3 &1d6 or 1d4 &1d8 or 1d6 &2d6 or 1d8 &3d6 or 2d6 \\
 \hline
\end{longtabu}
%
\rowcolors{1}{colorone}{colortwo}
\begin{longtabu} to \textwidth {X[6, l] X[3, l] X[4, l] X[3, l] X[2, l] X[2, l] X[2, l] X[2, l] X[2, l] X[2, l]}
\header\textbf{Exotic Weapons}&&&&&&&&&\\
\hline
\rowcolor{colortwo}\textbf{Weapon} &\textbf{Critical} &\textbf{Type} &\textbf{Range} &\textbf{Fine} &\textbf{Dimin.} &\textbf{Tiny} &\textbf{Small} &\textbf{Medium} &\textbf{Large} \endhead
 Bolas &20x2 &Bludgeoning &10 ft. &- &1 &1d2 &1d3 &1d4 &1d6 \\[1ex]
 Dire Flail &19-20x2/ 19-20x2 &Bludgeoning &Melee &1d2/ 1d2 &1d3/ 1d3 &1d4/ 1d4 &1d6/ 1d6 &1d8/ 1d8 &1d10/ 1d10 \\
 Double Axe &20x3/ 20x3 &Bludgeoning \& Slashing &Melee &1d3/ 1d3 &1d3/ 1d3 &1d4/ 1d4 &1d6/ 1d6 &1d8/ 1d8 &1d10/ 1d10 \\
 Double Sword &19-20x2/ 19-20x2 &Piercing or Slashing &Melee &1d2/ 1d2 &1d3/ 1d3 &1d4/ 1d4 &1d6/ 1d6 &1d8/ 1d8 &1d10/ 1d10 \\
 Hook-Hammer &20x3/ 20x4 &Bludgeoning/ Piercing &Melee &- &1d2/\newline{}1 &1d3/ 1d2 &1d4/ 1d3 &1d6/ 1d4 &1d8/ 1d6 \\
 Kama &20x2 &Slashing &Melee &1d3 &1d4 &1d6 &1d8 &1d10 &2d6 \\[1ex]
 Kasurigama &20x2 &Slashing &Melee or Reach &1 &1d2 &1d3 &1d4 &1d6 &2d4 \\
 Net &N/A &N/A &Reach &- &- &- &- &- &- \\[1ex]
 Nunchaku &20x2 &Bludgeoning &Melee &1d3 &1d4 &1d6 &1d8 &1d10 &2d6 \\[1ex]
 Repeating Crossbow &19-20x2 &Piercing &120 ft. &1d3 &1d4 &1d6 &1d8 &1d10 &2d6\\[1ex]
 Sai &20x2 &Bludgeoning &Melee or 10 ft. &1d2 &1d3 &1d4 &1d6 &1d8 &1d10 \\
 Shuriken &20x2 &Piercing &10 ft &1 &1d2 &1d3 &1d4 &1d6 &1d8 \\[1ex]
 Siangham &20x2 &Piercing &Melee &1d3 &1d4 &1d6 &1d8 &1d10 &2d6 \\[1ex]
 Spiked Chain &20x2 &Piercing &Special &1d2 &1d3 &1d4 &1d6 &2d4 & \\[1ex]
 Urgrosh &20x3/\newline{}20x3 &Slashing/ Piercing &Melee &1d2/\newline{}1 &1d3/ 1d2 &1d4/ 1d3 &1d6/ 1d4 &1d8/ 1d6 &1d10/ 1d8 \\
 Whip &20x2 &Slashing &Special &- &1 &1d2 &1d3 &1d4 &1d6 \\[1ex]
 \hline
\end{longtabu}}
\end{small}}

\subsection{Individual Weapon Rules}

In addition to the qualities given on the table, some weapons have additional rules, given below.

\textbf{Bastard Sword:} A character with exotic weapon proficiency can wield a bastard sword as if they were one size larger than they are.

\textbf{Bolas:} You can use this weapon to make a ranged trip attack against an opponent. You can't be tripped during your own trip attempt when using a set of bolas. As a thrown weapon, bolas must be one size smaller than you to be used effectively.

\textbf{Bow:} Bows are projectile weapons, the range given is for a medium sized bow. For every size category larger or smaller than medium, add or subtract 30 feet from the bows range. You need at least two hands to use a bow, regardless of its size. A bow the same size as you is too unwieldy to use while you are mounted. If you have a penalty for low Strength, apply it to damage rolls when you use a Bow. If you have a bonus for high Strength, you can apply it to damage rolls when you use a composite bow (see below) but not a regular bow.

\textbf{Composite Bow:} You need at least two hands to use a composite bow, regardless of its size. You can use a composite bow up to your size while mounted. All composite bows are made with a particular minimum strength rating (that is, each requires a minimum Strength score to use with proficiency). If your Strength score is less than the strength rating of the composite bow, you can't use it. The default composite longbow requires a Strength score of 10 or higher to use. A composite longbow can be made with a high strength rating to take advantage of an above-average Strength score; this feature allows you to add your Strength bonus to damage, as long as you meet the strength rating for the bow you can add either your Strength bonus, or the strength bonus that would be derived from the bows strength rating +4, to your damage rolls, whichever is lower.

\textbf{Crossbow:} Crossbows are ranged weapons that use bolts. The range listed for the crossbow is for one of medium size, for every size category larger or smaller than medium increase or decrease the range by 40 ft. Reloading a crossbow provokes an attack of opportunity, Reloading a light and one-handed crossbows is a move action, two-handed crossbows require a full round action to reload. Reloading a crossbow requires two hands.

\textbf{Dire Flail:} A dire flail is a double weapon. You can fight with it as if fighting with two weapons, but if you do, you incur all the normal attack penalties associated with fighting with two weapons, just as if you were using a one-handed weapon and a light weapon. A creature wielding a dire flail in one hand can't use it as a double weapon— only one end of the weapon can be used in any given round. When using a dire flail, you get a +2 bonus on attack rolls made to disarm an enemy. You can also use this weapon to make trip attacks. If you are tripped during your own trip attempt, you can drop the dire flail to avoid being tripped.

\textbf{Double Axe:} A double axe is a double weapon. You can fight with it as if fighting with two weapons, but if you do, you incur all the normal attack penalties associated with fighting with two weapons, just as if you were using a one-handed weapon and a light weapon. A creature wielding an orc double axe in one hand can't use it as a double weapon-only one end of the weapon can be used in any given round.

\textbf{Double Sword:} A double sword is a double weapon. You can fight with it as if fighting with two weapons, but if you do, you incur all the normal attack penalties associated with fighting with two weapons, just as if you were using a one-handed weapon and a light weapon. A creature wielding a two-bladed sword in one hand can't use it as a double weapon-only one end of the weapon can be used in any given round.

\textbf{Dwarven Axe:} A character with exotic proficiency with a Dwarven Axe can wield one as if they were one size category larger than they are. Dwarves only need martial proficiency with them to do this.

\textbf{Flail:} With a flail, you get a +2 bonus on attack rolls made to disarm an enemy. You can also use this weapon to make trip attacks. If you are tripped during your own trip attempt, you can drop the flail to avoid being tripped.

\textbf{Gauntlet:} This metal glove lets you deal lethal damage rather than nonlethal damage with unarmed strikes. A strike with a gauntlet is otherwise considered an unarmed attack. Medium and heavy armors (except breastplate) come with gauntlets. The damage listings given are for a gauntlet made for a creature of the indicated size, instead of fo a gauntlet of the indicated size. You may not wear gauntlets made for a creature of a different size than you.

\textbf{Glaive:} A glaive has reach. The glaives reach property can only be used when it is a two-handed weapon. You can strike opponents 10 feet away with it, but you can't use it against an adjacent foe.

\textbf{Guisarme:} A guisarme has reach. The guisarmes reach property can only be used when it is a two-handed weapon. You can strike opponents 10 feet away with it, but you can't use it against an adjacent foe. You can also use it to make trip attacks. If you are tripped during your own trip attempt, you can drop the guisarme to avoid being tripped.

\textbf{Halberd:} If you use a ready action to set a halberd against a charge, you deal double damage on a successful hit against a charging character. You can use a halberd to make trip attacks. If you are tripped during your own trip attempt, you can drop the halberd to avoid being tripped.

\textbf{Hook-Hammer:} A hook-hammer is a double weapon. You can fight with it as if fighting with two weapons, but if you do, you incur all the normal attack penalties associated with fighting with two weapons, just as if you were using a one-handed weapon and a light weapon. On a medium sized hook-hammer the hammer's blunt head is a bludgeoning weapon that deals 1d6 points of damage (crit ×3) and its hook is a piercing weapon that deals 1d4 points of damage (crit ×4). You can use either head as the primary weapon. The other head is the offhand weapon. A creature wielding a gnome hook-hammer in one hand can't use it as a double weapon-only one end of the weapon can be used in any given round. You can use a hook-hammer to make trip attacks. If you are tripped during your own trip attempt, you can drop the gnome hooked hammer to avoid being tripped. Gnomes treat hook-hammers as martial weapons.

\textbf{Kusarigama:} A kusarigama has reach, so you can strike opponents 10 feet away with it. The kusarigamas reach property can only be used when it is wielded in two hands (though not necessarily a two-handed weapon). In addition, unlike most other weapons with reach, it can be used against an adjacent foe. You can make trip attacks with the chain. If you are tripped during your own trip attempt, you can drop the chain to avoid being tripped. When using a spiked chain, you get a +2 bonus on opposed attack rolls made to disarm an opponent (including the roll to avoid being disarmed if such an attempt fails).

\textbf{Longspear:} A longspear has reach. The longspears reach property can only be used when it is a two-handed weapon. You can strike opponents 10 feet away with it, but you can't use it against an adjacent foe. If you use a ready action to set a longspear against a charge, you deal double damage on a successful hit against a charging character. While mounted, you can wield a lance with one hand. A longspear couched in a military saddle deals double damage on a charge.

\textbf{Net:} A net is a reach weapon used to entangle enemies. Unlike other reach weapons, a net the same size as you can be used with one hand. When you use a net, you make a ranged touch attack against your target. If you hit, the target is entangled. An entangled creature takes a -2 penalty on attack rolls and a -4 penalty on Dexterity, can move at only half speed, and cannot charge or run. If you control the trailing rope by succeeding on an opposed Strength check while holding it, the entangled creature can move only within the limits that the rope allows. If the entangled creature attempts to cast a spell, it must make a DC 15 Concentration check or be unable to cast the spell. An entangled creature can escape with a DC 20 Escape Artist check (a full-round action). The net has 5 hit points and can be burst with a DC 25 Strength check (also a full-round action). A net is useful only against creatures within one size category of you. A net must be folded to be thrown effectively. The first time you throw your net in a fight, you make a normal ranged touch attack roll. After the net is unfolded, you take a -4 penalty on attack rolls with it. It takes 2 rounds for a proficient user to fold a net and twice that long for a nonproficient one to do so.

\textbf{Nunchaku:} The nunchaku is a special monk weapon. This designation gives a monk wielding a nunchaku special options. With a nunchaku, you get a +2 bonus on attack rolls made to disarm an enemy. Nunchakus only count as monk weapons if they are light.

\textbf{Ranseur:} A ranseur has reach. The ranseurs reach property can only be used when it is a two-handed weapon. You can strike opponents 10 feet away with it, but you can't use it against an adjacent foe. With a ranseur, you get a +2 bonus on attack rolls made to disarm an opponent.

\textbf{Repeating Crossbow:} The repeating crossbow holds 5 crossbow bolts. As long as it holds bolts, you can reload it by pulling the reloading lever (a free action). Loading a new case of 5 bolts is a full-round action that provokes attacks of opportunity. A repeating crossbow functions identically to a crossbow in all other ways.

\textbf{Sai:} With a sai, you get a +4 bonus on opposed attack rolls made to disarm an enemy. The sai is a special monk weapon. This designation gives a monk wielding a sai special options. Sais only count as monk weapons if they are light.

\textbf{Shield:} You can bash with a shield instead of using it for defense. Doing so incurs all the normal penalties for two weapon fighting. Great Shields are one size smaller than the size of creature it was designed for, normal shields are two sizes smaller.

\textbf{Shuriken:} A shuriken is a special monk weapon. This designation gives a monk wielding shuriken special options. A shuriken can't be used as a melee weapon. Although they are thrown weapons, shuriken are treated as ammunition for the purposes of drawing them as long as they are three size categories smaller than you.

\textbf{Siangham:} The siangham is a special monk weapon. This designation gives a monk wielding a siangham special options. Siangham must be light to be used as a monk weapon.

\textbf{Sickle:} A sickle can be used to make trip attacks. If you are tripped during your own trip attempt, you can drop the sickle to avoid being tripped.

\textbf{Sling:} Your Strength modifier applies to damage rolls when you use a sling, just as it does for thrown weapons. You can fire, but not load, a sling the same size as you with one hand. Loading a sling is a move action that requires two hands and provokes attacks of opportunity. You can hurl ordinary stones with a sling, but stones are not as dense or as round as bullets. Thus, such an attack deals damage as if the weapon were designed for a creature one size category smaller than you and you take a -1 penalty on attack rolls. The range given is for a sling of medium size, for every size larger or smaller than medium increase or decrease the range by 15 feet.

\textbf{Spear:} If you use a ready action to set a spear against a charge, you deal double damage on a successful hit against a charging character. A spear one size smaller than you can be used as a thrown weapon with a 20 foot range incriment.

\textbf{Spiked Armor:} You can outfit your armor with spikes, which can deal damage in a grapple or as a separate attack. The damage listed is for armor made for a creature of the given size. Spiked armor is a light weapon.

\textbf{Spiked Chain:} A spiked chain has reach, so you can strike opponents 10 feet away with it. The spiked chains reach property can only be used when it is wielded in two hands (though not necessarily a two-handed weapon). In addition, unlike most other weapons with reach, it can be used against an adjacent foe. You can make trip attacks with the chain. If you are tripped during your own trip attempt, you can drop the chain to avoid being tripped. When using a spiked chain, you get a +2 bonus on opposed attack rolls made to disarm an opponent (including the roll to avoid being disarmed if such an attempt fails).

\textbf{Spiked Gauntlet:} Your opponent cannot use a disarm action to disarm you of spiked gauntlets. An attack with a spiked gauntlet is considered an armed attack. The damage listings given are for a spiked gauntlet made for a creature of the indicated size, instead of fo a spiked gauntlet of the indicated size. You may not wear gauntlets made for a creature of a different size than you.

\textbf{Spiked Shield:} You can bash with a spiked shield instead of using it for defense. If you use a ready action to set a spear against a charge, you deal double damage on a successful hit against a charging character.

\textbf{Staff:} A staff is a double weapon. You can fight with it as if fighting with two weapons, but if you do, you incur all the normal attack penalties associated with fighting with two weapons, just as if you were using a one-handed weapon and a light weapon. A creature wielding a quarterstaff in one hand can't use it as a double weapon-only one end of the weapon can be used in any given round. The quarterstaff is a special monk weapon. This designation gives a monk wielding a staff special options.

\textbf{Trident:} This weapon can be thrown as long as it is one size category smaller than you. If you use a ready action to set a trident against a charge, you deal double damage on a successful hit against a charging character.

\textbf{Unarmed Strike:} The damage listed for each size of unarmed strike is the size of the creature using unarmed strike. You can deal leathal or non-leathal damage at your option with an unarmed strike. The damage from an unarmed strike is considered weapon damage for the purposes of effects that give you a bonus on weapon damage rolls. An unarmed strike is always considered a light weapon.

\textbf{Urgrosh:} An urgrosh is a double weapon. You can fight with it as if fighting with two weapons, but if you do, you incur all the normal attack penalties associated with fighting with two weapons, just as if you were using a one-handed weapon and a light weapon. The urgrosh's axe head is a slashing weapon that deals 1d8 points of damage. Its spear head is a piercing weapon that deals 1d6 points of damage. You can use either head as the primary weapon. The other is the off-hand weapon. A creature wielding an urgrosh in one hand can't use it as a double weapon-only one end of the weapon can be used in any given round. If you use a ready action to set an urgrosh against a charge, you deal double damage if you score a hit against a charging character. If you use an urgrosh against a charging character, the spear head is the part of the weapon that deals damage. Dwarves treat urgroshes as martial weapons.

\textbf{Warhammer:} A warhammer has two sides that can be used interchangably. One side deals bludgeoning and has a critical range of 20x3, the other deals piercing damage and has a critical range of 20x4. As a medium weapon the hammer side deals 2d6 damage and the pick side deals 1d8 damage. You can choose which side you make an attack with at the beginning of each attack. It is not a double weapon, and cannot be weilded as one. Enhancements to the weapon effect both sides.

\textbf{Whip:} A whip has a 15 foot reach and can be used to attack any creature within range, including adjacent foes. The whips reach property can only be used when it is a one-handed or light weapon. A whip deals nonlethal damage. It also deals no damage to any creature with an armor bonus of +1 or higher, or a natural armor bonus of +3 or higher. Using a whip provokes an attack of opportunity as if you had used a ranged weapon. You cannot use a whip as a two-handed weapon. You can make trip attacks with a whip. If you are tripped during your own trip attempt, you can drop the whip to avoid being tripped. When using a whip, you get a +2 bonus on opposed attack rolls made to disarm an opponent.

\subsection{Masterwork Weapons}

A masterwork weapon is a finely crafted version of a normal weapon. Wielding it 
provides a +1 enhancement bonus on attack rolls.

You can't add the masterwork quality to a weapon after it is created; it must be 
crafted as a masterwork weapon (see the Craft skill). The masterwork quality adds 
300 gp to the cost of a normal weapon (or 6 gp to the cost of a single unit of 
ammunition). Adding the masterwork quality to a double weapon costs twice the normal 
increase (+600 gp).

Masterwork ammunition is damaged (effectively destroyed) when used. The enhancement 
bonus of masterwork ammunition does not stack with any enhancement bonus of the 
projectile weapon firing it.

All magic weapons are automatically considered to be of masterwork quality. The 
enhancement bonus granted by the masterwork quality doesn't stack with the enhancement 
bonus provided by the weapon's magic.

Even though some types of armor and shields can be used as weapons, you can't create 
a masterwork version of such an item that confers an enhancement bonus on attack 
rolls. Instead, masterwork armor and shields have lessened armor check penalties.
\section{Adventuring Gear}
\begin{table}
\rowcolors{1}{colorone}{colortwo}
\caption{Adventuring Gear}
{\tabulinesep=1mm
\begin{tabu}to \linewidth{X c c | X c c}
\header\textbf{Item} & \textbf{Cost} & \textbf{Weight} & \textbf{Item} & \textbf{Cost} & \textbf{Weight}\\ \hline
10ft Ladder & 5 cp & 20 lb. & Lantern (bullseye) & 12 gp & 3 lb. \\
10ft Pole & 2 sp & 8 lb. & Lantern (hooded) & 7 gp & 2 lb. \\
Backpack (empty) & 2 gp & 2 lb.\textsuperscript{1} & Lock & -- & 1 lb. \\
Barrel (empty) & 2 gp & 30 lb. & \hspace{.25cm}Very simple & 20 gp & 1 lb. \\
Basket (empty) & 4 sp & 1 lb. & \hspace{.25cm}Average & 40 gp & 1 lb. \\
Bedroll & 1 sp & 5 lb.\textsuperscript{1} & \hspace{.25cm}Good & 80 gp & 1 lb. \\
Bell & 1 gp & -- & \hspace{.25cm}Amazing & 150 gp & 1 lb. \\
Belt Pouch (empty) & 1 gp & \sfrac{1}{2} lb.\textsuperscript{1} & Manacles (common) & 15 gp & 2 lb. \\
Block and tackle & 5 gp & 5 lb. & Manacles (masterwork) & 50 gp & 2 lb. \\
Bucket (empty) & 5 sp & 2 lb. & Miner's Pick & 3 gp & 10 lb. \\
Caltrops & 1 gp & 2 lb. & Oil (1-pint flask) & 1 sp & 1 lb. \\
Candle & 1 cp & -- & Paper (sheet) & 4 sp & -- \\
Canvas (sq. yd.) & 1 sp & 1 lb. & Parchment (sheet) & 2 sp & -- \\
Case, map or scroll & 1 gp & \sfrac{1}{2} lb. & Piton & 1 sp & \sfrac{1}{2} lb. \\
Chain (10 ft.) & 30 gp & 2 lb. & Portable Ram & 10 gp & 20 lb. \\
Chalk, 1 piece & 1 cp & -- & Rope (hempen, 50 ft.) & 1 gp & 10 lb. \\
Chest (empty) & 2 gp & 25 lb. & Rope (silk, 50 ft.) & 10 gp & 5 lb. \\
Clay Jug & 3 cp & 9 lb. & Sack (empty) & 1 sp & \sfrac{1}{2} lb.\textsuperscript{1} \\
Clay Mug/Tankard & 2 cp & 1 lb. & Sealing wax & 1 gp & 1 lb. \\
Clay Pitcher & 2 cp & 5 lb. & Sewing needle & 5 sp & -- \\
Common Lamp & 1 sp & 1 lb. & Signal whistle & 8 sp & -- \\
Crowbar & 2 gp & 5 lb. & Signet ring & 5 gp & -- \\
Firewood (per day) & 1 cp & 20 lb. & Sledge & 1 gp & 10 lb. \\
Fishhook & 1 sp & -- & Small Steel Mirror & 10 gp & \sfrac{1}{2} lb. \\
Fishing net, 25 sq. ft. & 4 gp & 5 lb. & Soap (per lb.) & 5 sp & 1 lb. \\
Flask (empty) & 3 cp & 1.5 lb. & Spade or shovel & 2 gp & 8 lb. \\
Flint and steel & 1 gp & -- & Spyglass & 1,000 gp & 1 lb. \\
Glass Wine Bottle & 2 gp & -- & Tent & 10 gp & 20 lb.\textsuperscript{1} \\
Grappling hook & 1 gp & 4 lb. & Torch & 1 cp & 1 lb. \\
Hammer & 5 sp & 2 lb. & Trail Rations (per day) & 5 sp & 1 lb.\textsuperscript{1} \\
Ink (1 oz. vial) & 8 gp & -- & Vial, ink or potion & 1 gp & \sfrac{1}{10} lb. \\
Inkpen & 1 sp & -- & Waterskin & 1 gp & 4 lb.\textsuperscript{1} \\
Iron Pot & 5 sp & 10 lb. & Whetstone & 2 cp & 1 lb. \\
&&&Winter Blanket & 5 sp & 3 lb.\textsuperscript{1} \\ \hline
\multicolumn{6}{p{\linewidth}}{\textsuperscript{1} These items weigh one-quarter this amount when made for Small characters. Containers for Small characters also carry one-quarter the normal amount.}\\ \hline
\end{tabu}}
\end{table}

A few of the pieces of adventuring gear found on Table: Adventuring Gear are described 
below, along with any special benefits they confer on the user ("you").

\textbf{Caltrops:} A caltrop is a four-pronged iron spike crafted so that one prong 
faces up no matter how the caltrop comes to rest. You scatter caltrops on the ground 
in the hope that your enemies step on them or are at least forced to slow down 
to avoid them. One 2- pound bag of caltrops covers an area 5 feet square.

Each time a creature moves into an area covered by caltrops (or spends a round 
fighting while standing in such an area), it might step on one. The caltrops make 
an attack roll (base attack bonus +0) against the creature. For this attack, the 
creature's shield, armor, and deflection bonuses do not count. If the creature 
is wearing shoes or other footwear, it gets a +2 armor bonus to AC. If the caltrops 
succeed on the attack, the creature has stepped on one. The caltrop deals 1 point 
of damage, and the creature's speed is reduced by one-half because its foot is 
wounded. This movement penalty lasts for 24 hours, or until the creature is successfully 
treated with a DC 15 Heal check, or until it receives at least 1 point of magical 
curing. A charging or running creature must immediately stop if it steps on a caltrop. 
Any creature moving at half speed or slower can pick its way through a bed of caltrops 
with no trouble.

Caltrops may not be effective against unusual opponents.

\textbf{Candle:} A candle dimly illuminates a 5-foot radius and burns for 1 hour.

\textbf{Chain:} Chain has hardness 10 and 5 hit points. It can be burst with a 
DC 26 Strength check.

\textbf{Clay Jug:} This basic ceramic jug is fitted with a stopper and holds 1 gallon of liquid.

\textbf{Common Lamp:} A lamp clearly illuminates a 15-foot radius, provides shadowy 
illumination out to a 30-foot radius, and burns for 6 hours on a pint of oil. You 
can carry a lamp in one hand. 

\textbf{Crowbar:} A crowbar it grants a +2 circumstance bonus on Strength checks 
made for such purposes. If used in combat, treat a crowbar as a one-handed improvised 
weapon that deals bludgeoning damage equal to that of a club of its size.

\textbf{Flint and Steel:} Lighting a torch with flint and steel is a full-round 
action, and lighting any other fire with them takes at least that long.

\textbf{Grappling Hook:} Throwing a grappling hook successfully requires a Use 
Rope check (DC 10, +2 per 10 feet of distance thrown).

\textbf{Hammer:} If a hammer is used in combat, treat it as a one-handed improvised 
weapon that deals bludgeoning damage equal to that of a spiked gauntlet of its 
size.

\textbf{Ink:} This is black ink. You can buy ink in other colors, but it costs 
twice as much.

\textbf{Lantern, Bullseye:} A bullseye lantern provides clear illumination in a 
60-foot cone and shadowy illumination in a 120-foot cone. It burns for 6 hours 
on a pint of oil. You can carry a bullseye lantern in one hand.

\textbf{Lantern, Hooded:} A hooded lantern clearly illuminates a 30-foot radius 
and provides shadowy illumination in a 60-foot radius. It burns for 6 hours on 
a pint of oil. You can carry a hooded lantern in one hand.

\textbf{Lock:} The DC to open a lock with the Open Lock skill depends on the lock's 
quality: simple (DC 20), average (DC 25), good (DC 30), or superior (DC 40).

\textbf{Manacles and Manacles, Masterwork:} Manacles can bind a Medium creature. 
A manacled creature can use the Escape Artist skill to slip free (DC 30, or DC 
35 for masterwork manacles). Breaking the manacles requires a Strength check (DC 
26, or DC 28 for masterwork manacles). Manacles have hardness 10 and 10 hit points.

Most manacles have locks; add the cost of the lock you want to the cost of the 
manacles.

For the same cost, you can buy manacles for a Small creature.

For a Large creature, manacles cost ten times the indicated amount, and for a Huge 
creature, one hundred times this amount. Gargantuan, Colossal, Tiny, Diminutive, 
and Fine creatures can be held only by specially made manacles.

\textbf{Oil:} A pint of oil burns for 6 hours in a lantern. You can use a flask 
of oil as a splash weapon. Use the rules for alchemist's fire, except that it takes 
a full round action to prepare a flask with a fuse. Once it is thrown, there is 
a 50\% chance of the flask igniting successfully.

You can pour a pint of oil on the ground to cover an area 5 feet square, provided 
that the surface is smooth. If lit, the oil burns for 2 rounds and deals 1d3 points 
of fire damage to each creature in the area.

\textbf{Portable Ram:} This iron-shod wooden beam gives you a +2 circumstance 
bonus on Strength checks made to break open a door and it allows a second person 
to help you without having to roll, increasing your bonus by 2.

\textbf{Rope, Hempen:} This rope has 2 hit points and can be burst with a DC 23 
Strength check.

\textbf{Rope, Silk:} This rope has 4 hit points and can be burst with a DC 24 Strength 
check. It is so supple that it provides a +2 circumstance bonus on Use Rope checks.

\textbf{Spyglass:} Objects viewed through a spyglass are magnified to twice their 
size.

\textbf{Torch:} A torch burns for 1 hour, clearly illuminating a 20-foot radius 
and providing shadowy illumination out to a 40- foot radius. If a torch is used 
in combat, treat it as a one-handed improvised weapon that deals bludgeoning damage 
equal to that of a gauntlet of its size, plus 1 point of fire damage.

\textbf{Vial:} A vial holds 1 ounce of liquid. The stoppered container usually 
is no more than 1 inch wide and 3 inches high.

%%%%%%%%%%%%%%%%%%%%%%%%%
\subsection{Special Substances and Items}
%%%%%%%%%%%%%%%%%%%%%%%%%

Any of these substances except for the everburning torch and holy water can be 
made by a character with the Craft (alchemy) skill.

\textbf{Acid:} You can throw a flask of acid as a splash weapon. Treat this attack 
as a ranged touch attack with a range increment of 10 feet. A direct hit deals 
1d6 points of acid damage. Every creature within 5 feet of the point where the 
acid hits takes 1 point of acid damage from the splash.

\textbf{Alchemist's Fire:} You can throw a flask of alchemist's fire as a splash 
weapon. Treat this attack as a ranged touch attack with a range increment of 10 
feet.

A direct hit deals 1d6 points of fire damage. Every creature within 5 feet of the 
point where the flask hits takes 1 point of fire damage from the splash. On the 
round following a direct hit, the target takes an additional 1d6 points of damage. 
If desired, the target can use a full-round action to attempt to extinguish the 
flames before taking this additional damage. Extinguishing the flames requires 
a DC 15 Reflex save. Rolling on the ground provides the target a +2 bonus on the 
save. Leaping into a lake or magically extinguishing the flames automatically smothers 
the fire.

\textbf{Antitoxin:} If you drink antitoxin, you get a +5 alchemical bonus on Fortitude 
saving throws against poison for 1 hour.

\textbf{Everburning Torch:} This otherwise normal torch has a \textit{continual 
flame }spell cast upon it. An everburning torch clearly illuminates a 20-foot radius 
and provides shadowy illumination out to a 40-foot radius.

\textbf{Holy Water:} Holy water damages undead creatures and evil outsiders almost 
as if it were acid. A flask of holy water can be thrown as a splash weapon.

Treat this attack as a ranged touch attack with a range increment of 10 feet. A 
flask breaks if thrown against the body of a corporeal creature, but to use it 
against an incorporeal creature, you must open the flask and pour the holy water 
out onto the target. Thus, you can douse an incorporeal creature with holy water 
only if you are adjacent to it. Doing so is a ranged touch attack that does not 
provoke attacks of opportunity.

\begin{wraptable}{r}{.5\linewidth}
\rowcolors{1}{colorone}{colortwo}
\caption{Special Substances and Items}
{\tabulinesep=1mm
\begin{tabu}to \linewidth{X c c}
\header\textbf{Item} & \textbf{Cost} & \textbf{Weight}\\ \hline
Acid (flask) & 10 gp & 1 lb.\\
Alchemist's fire (flask) & 20 gp & 1 lb.\\
Antitoxin (vial) & 50 gp & --\\
Everburning torch & 110 gp & 1 lb.\\
Holy water (flask) & 25 gp & 1 lb.\\
Smokestick & 20 gp & \sfrac{1}{2} lb.\\
Sunrod & 2 gp & 1 lb.\\
Tanglefoot bag & 50 gp & 4 lb.\\
Thunderstone & 30 gp & 1 lb.\\
Tindertwig & 1 gp & --\\ \hline
\end{tabu}}
\end{wraptable}

A direct hit by a flask of holy water deals 2d4 points of damage to an undead creature 
or an evil outsider. Each such creature within 5 feet of the point where the flask 
hits takes 1 point of damage from the splash.

Temples to good deities sell holy water at cost (making no profit).

\textbf{Smokestick:} This alchemically treated wooden stick instantly creates thick, 
opaque smoke when ignited. The smoke fills a 10- foot cube (treat the effect as 
a \textit{fog cloud }spell, except that a moderate or stronger wind dissipates 
the smoke in 1 round). The stick is consumed after 1 round, and the smoke dissipates 
naturally.

\textbf{Sunrod:} This 1-foot-long, gold-tipped, iron rod glows brightly when struck. 
It clearly illuminates a 30-foot radius and provides shadowy illumination in a 
60-foot radius. It glows for 6 hours, after which the gold tip is burned out and 
worthless.

\textbf{Tanglefoot Bag:} When you throw a tanglefoot bag at a creature (as a ranged 
touch attack with a range increment of 10 feet), the bag comes apart and the goo 
bursts out, entangling the target and then becoming tough and resilient upon exposure 
to air. An entangled creature takes a -2 penalty on attack rolls and a -4 penalty 
to Dexterity and must make a DC 15 Reflex save or be glued to the floor, unable 
to move. Even on a successful save, it can move only at half speed. Huge or larger 
creatures are unaffected by a tanglefoot bag. A flying creature is not stuck to 
the floor, but it must make a DC 15 Reflex save or be unable to fly (assuming it 
uses its wings to fly) and fall to the ground. A tanglefoot bag does not function 
underwater.

A creature that is glued to the floor (or unable to fly) can break free by making 
a DC 17 Strength check or by dealing 15 points of damage to the goo with a slashing 
weapon. A creature trying to scrape goo off itself, or another creature assisting, 
does not need to make an attack roll; hitting the goo is automatic, after which 
the creature that hit makes a damage roll to see how much of the goo was scraped 
off. Once free, the creature can move (including flying) at half speed. A character 
capable of spellcasting who is bound by the goo must make a DC 15 Concentration 
check to cast a spell. The goo becomes brittle and fragile after 2d4 rounds, cracking 
apart and losing its effectiveness. An application of \textit{universal solvent} 
to a stuck creature dissolves the alchemical goo immediately.

\textbf{Thunderstone:} You can throw this stone as a ranged attack with a range 
increment of 20 feet. When it strikes a hard surface (or is struck hard), it creates 
a deafening bang that is treated as a sonic attack. Each creature within a 10-foot-radius 
spread must make a DC 15 Fortitude save or be deafened for 1 hour. A deafened creature, 
in addition to the obvious effects, takes a -4 penalty on initiative and has a 
20\% chance to miscast and lose any spell with a verbal component that it tries 
to cast.

Since you don't need to hit a specific target, you can simply aim at a particular 
5-foot square. Treat the target square as AC 5.

\textbf{Tindertwig:} The alchemical substance on the end of this small, wooden 
stick ignites when struck against a rough surface. Creating a flame with a tindertwig 
is much faster than creating a flame with flint and steel (or a magnifying glass) 
and tinder. Lighting a torch with a tindertwig is a standard action (rather than 
a full-round action), and lighting any other fire with one is at least a standard 
action.

%%%%%%%%%%%%%%%%%%%%%%%%%
\subsection{Tools and Skill Kits}
%%%%%%%%%%%%%%%%%%%%%%%%%

\begin{wraptable}{r}{.5\linewidth}
\rowcolors{1}{colorone}{colortwo}
\caption{Tools and Skill Kits}
{\tabulinesep=1mm
\begin{tabu}to \linewidth{X c c}
\header\textbf{Item} & \textbf{Cost} & \textbf{Weight}\\ \hline
Alchemist's lab & 500 gp & 40 lb.\\
Artisan's tools (common) & 5 gp & 5 lb.\\
Artisan's tools (masterwork) & 55 gp & 5 lb.\\
Climber's kit & 80 gp & 5 lb.\textsuperscript{1}\\
Disguise kit & 50 gp & 8 lb.\textsuperscript{1}\\
Healer's kit & 50 gp & 1 lb.\\
Holly and mistletoe & -- & --\\
Holy symbol (silver) & 25 gp & 1 lb.\\
Holy symbol (wooden) & 1 gp & --\\
Hourglass & 25 gp & 1 lb.\\
Magnifying glass & 100 gp & --\\
Masterwork  Tool & 50 gp & 1 lb.\\
Merchant's Scale & 2 gp & 1 lb.\\
Musical instrument (common) & 5 gp & 3 lb.\textsuperscript{1}\\
Musical instrument (masterwork) & 100 gp & 3 lb.\textsuperscript{1}\\
Spell component pouch & 5 gp & 2 lb.\\
Thieves' tools (common) & 30 gp & 1 lb.\\
Thieves' tools (masterwork) & 100 gp & 2 lb.\\
Water clock & 1,000 gp & 200 lb.\\
Wizard's Spellbook (blank) & 15 gp & 3 lb.\\ \hline
\multicolumn{3}{p{\linewidth}}{\textsuperscript{1} These items weigh one-quarter this amount when made for Small characters.}\\
\hline
\end{tabu}}
\end{wraptable}

\indent\textbf{Alchemist's Lab:} An alchemist's lab always has the perfect tool for making 
alchemical items, so it provides a +2 circumstance bonus on Craft (alchemy) checks. 
It has no bearing on the costs related to the Craft (alchemy) skill. Without this 
lab, a character with the Craft (alchemy) skill is assumed to have enough tools 
to use the skill but not enough to get the +2 bonus that the lab provides.

\textbf{Artisan's Tools (common):} These special tools include the items needed to pursue 
any craft. Without them, you have to use improvised tools (-2 penalty on Craft 
checks), if you can do the job at all.

\textbf{Artisan's Tools (masterwork):} These tools serve the same purpose as artisan's 
tools (above), but masterwork artisan's tools are the perfect tools for the job, 
so you get a +2 circumstance bonus on Craft checks made with them.

\textbf{Climber's Kit:} This is the perfect tool for climbing and gives you a +2 
circumstance bonus on Climb checks.

\textbf{Disguise Kit:} The kit is the perfect tool for disguise and provides a 
+2 circumstance bonus on Disguise checks. A disguise kit is exhausted after ten 
uses.

\textbf{Healer's Kit:} It is the perfect tool for healing and provides a +2 circumstance 
bonus on Heal checks. A healer's kit is exhausted after ten uses.

\textbf{Holy Symbol, Silver or Wooden:} A holy symbol focuses positive energy. 
A cleric or paladin uses it as the focus for his spells and as a tool for turning 
undead. Each religion has its own holy symbol.

\textbf{Magnifying Glass:} This simple lens allows a closer look at small objects. 
It is also useful as a substitute for flint and steel when starting fires. Lighting 
a fire with a magnifying glass requires light as bright as sunlight to focus, tinder 
to ignite, and at least a full-round action. A magnifying glass grants a +2 circumstance 
bonus on Appraise checks involving any item that is small or highly detailed.

\textbf{Masterwork Tool:} This well-made item is the perfect tool for the job. 
It grants a +2 circumstance bonus on a related skill check (if any). Bonuses provided 
by multiple masterwork items used toward the same skill check do not stack.

\textbf{Merchant's Scale:} A scale grants a +2 circumstance bonus on Appraise 
checks involving items that are valued by weight, including anything made of precious 
metals.

\textbf{Musical Instrument, Common or Masterwork:} A masterwork instrument grants 
a +2 circumstance bonus on Perform checks involving its use.

\textbf{Spell Component Pouch:} A spellcaster with a spell component pouch is assumed 
to have all the material components and focuses needed for spellcasting, except 
for those components that have a specific cost, divine focuses, and focuses that 
wouldn't fit in a pouch.

\textbf{Thieves' Tools (common):} This kit contains the tools you need to use the Disable 
Device and Open Lock skills. Without these tools, you must improvise tools, and 
you take a -2 circumstance penalty on Disable Device and Open Locks checks.

\textbf{Thieves' Tools (masterwork):} This kit contains extra tools and tools of 
better make, which grant a +2 circumstance bonus on Disable Device and Open Lock 
checks.

\textit{Unholy Symbols:} An unholy symbol is like a holy symbol except that it 
focuses negative energy and is used by evil clerics (or by neutral clerics who 
want to cast evil spells or command undead).

\textbf{Water Clock:} This large, bulky contrivance gives the time accurate to 
within half an hour per day since it was last set. It requires a source of water, 
and it must be kept still because it marks time by the regulated flow of droplets 
of water.

\textbf{Wizard's Spellbook (blank):} A spellbook has 100 pages of parchment, and 
each spell takes up one page per spell level (one page each for 0-level spells).

%%%%%%%%%%%%%%%%%%%%%%%%%
\subsection{Clothing}
%%%%%%%%%%%%%%%%%%%%%%%%%

\textbf{Artisan's Outfit:} This outfit includes a shirt with buttons, a skirt or 
pants with a drawstring, shoes, and perhaps a cap or hat. It may also include a 
belt or a leather or cloth apron for carrying tools.

\textbf{Cleric's Vestments:} These ecclesiastical clothes are for performing priestly 
functions, not for adventuring.

\textbf{Cold Weather Outfit:} A cold weather outfit includes a wool coat, linen 
shirt, wool cap, heavy cloak, thick pants or skirt, and

boots. This outfit grants a +5 circumstance bonus on Fortitude saving throws against 
exposure to cold weather.

\begin{wraptable}{r}{.5\linewidth}
\rowcolors{1}{colorone}{colortwo}
\caption{Clothing}
{\tabulinesep=1mm
\begin{tabu}to \linewidth{X c c}
\header\textbf{Item} & \textbf{Cost} & \textbf{Weight}\\ \hline
Artisan's outfit & 1 gp & 4 lb\textsuperscript{1}\\
Cleric's vestments & 5 gp & 6 lb\textsuperscript{1}\\
Cold weather outfit & 8 gp & 7 lb\textsuperscript{1}\\
Courtier's outfit & 30 gp & 6 lb\textsuperscript{1}\\
Entertainer's outfit & 3 gp & 4 lb\textsuperscript{1}\\
Explorer's outfit & 10 gp & 8 lb\textsuperscript{1}\\
Monk's outfit & 5 gp & 2 lb\textsuperscript{1}\\
Noble's outfit & 75 gp & 10 lb\textsuperscript{1}\\
Peasant's outfit & 1 sp & 2 lb\textsuperscript{1}\\
Royal outfit & 200 gp & 15 lb\textsuperscript{1}\\
Scholar's outfit & 5 gp & 6 lb\textsuperscript{1}\\
Traveler's outfit & 1 gp & 5 lb\textsuperscript{1}\\ \hline
\multicolumn{3}{p{\linewidth}}{\textsuperscript{1} These items weigh one-quarter this amount when made for Small characters.}\\
\hline
\end{tabu}}
\end{wraptable}

\textbf{Courtier's Outfit:} This outfit includes fancy, tailored clothes in whatever 
fashion happens to be the current style in the courts of the nobles. Anyone trying 
to influence nobles or courtiers while wearing street dress will have a hard time 
of it (-2 penalty on Charisma-based skill checks to influence such individuals). 
If you wear this outfit without jewelry (costing an additional 50 gp), you look 
like an out-of-place commoner.

\textbf{Entertainer's Outfit:} This set of flashy, perhaps even gaudy, clothes 
is for entertaining. While the outfit looks whimsical, its practical design lets 
you tumble, dance, walk a tightrope, or just run (if the audience turns ugly).

\textbf{Explorer's Outfit:} This is a full set of clothes for someone who never 
knows what to expect. It includes sturdy boots, leather breeches or a skirt, a 
belt, a shirt (perhaps with a vest or jacket), gloves, and a cloak. Rather than 
a leather skirt, a leather overtunic may be worn over a cloth skirt. The clothes 
have plenty of pockets (especially the cloak). The outfit also includes any extra 
items you might need, such as a scarf or a wide-brimmed hat.

\textbf{Monk's Outfit:} This simple outfit includes sandals, loose breeches, and 
a loose shirt, and is all bound together with sashes. The outfit is designed to 
give you maximum mobility, and it's made of high-quality fabric. You can hide small 
weapons in pockets hidden in the folds, and the sashes are strong enough to serve 
as short ropes.

\textbf{Noble's Outfit:} This set of clothes is designed specifically to be expensive 
and to show it. Precious metals and gems are worked into the clothing. To fit into 
the noble crowd, every would-be noble also needs a signet ring (see Adventuring 
Gear, above) and jewelry (worth at least 100 gp).

\textbf{Peasant's Outfit:} This set of clothes consists of a loose shirt and baggy 
breeches, or a loose shirt and skirt or overdress. Cloth wrappings are used for 
shoes.

\textbf{Royal Outfit:} This is just the clothing, not the royal scepter, crown, 
ring, and other accoutrements. Royal clothes are ostentatious, with gems, gold, 
silk, and fur in abundance.

\textbf{Scholar's Outfit:} Perfect for a scholar, this outfit includes a robe, 
a belt, a cap, soft shoes, and possibly a cloak.

\textbf{Traveler's Outfit:} This set of clothes consists of boots, a wool skirt 
or breeches, a sturdy belt, a shirt (perhaps with a vest or jacket), and an ample 
cloak with a hood.
\section{Animals and Related Gear}

\begin{wraptable}{r}{.45\textwidth}
\rowcolors{1}{colorone}{colortwo}
\caption{Mounts and Related Gear}
{\tabulinesep=1mm
\begin{tabu}to \linewidth{X r r}
\header\textbf{Item} & \textbf{Cost} & \textbf{Weight}\\ \hline
Barding&&\\
\hspace{.5cm}Medium creature & x2 & x1\\
\hspace{.5cm}Large creature & x4 & x2\\
Bit and bridle & 2 gp & 1 lb.\\
Dog&&\\
\hspace{.5cm}Guard Dog & 25 gp & --\\
\hspace{.5cm}Riding Dog & 150 gp & --\\
Donkey or mule & 8 gp & --\\
Feed (per day) & 5 cp & 10 lb.\\
Horse&&\\
\hspace{.5cm}Heavy Horse & 200 gp & --\\
\hspace{.5cm}Heavy Warhorse & 400 gp & --\\
\hspace{.5cm}Light Horse & 75 gp & --\\
\hspace{.5cm}Light Warhorse & 150 gp & --\\
\hspace{.5cm}Pony & 30 gp & --\\
\hspace{.5cm}Warpony & 100 gp & --\\
Saddle (common)&&\\
\hspace{.5cm}Military & 20 gp & 30 lb.\\
\hspace{.5cm}Pack & 5 gp & 15 lb.\\
\hspace{.5cm}Riding & 10 gp & 25 lb.\\
Saddle (exotic)&&\\
\hspace{.5cm}Military & 60 gp & 40 lb.\\
\hspace{.5cm}Pack & 15 gp & 20 lb.\\
\hspace{.5cm}Riding & 30 gp & 30 lb.\\
Saddlebags & 4 gp & 8 lb.\\
Stabling (per day) & 5 sp & --\\
\hline
\end{tabu}}

\rowcolors{1}{colorone}{colortwo}
\caption{Mount Speed In Armor}
{\tabulinesep=1mm
\begin{tabu}to \linewidth{X c c c}
\header\textbf{Barding} & \textbf{40ft} & \textbf{50ft} & \textbf{60ft}\\ \hline
Medium & 30ft & 35ft & 40ft\\
Heavy & 30ft\textsuperscript{1} & 35ft\textsuperscript{1} & 40ft\textsuperscript{1}\\ \hline
\multicolumn{4}{p{\linewidth}}{\textsuperscript{1} A mount wearing heavy armor moves at only triple its normal speed when running, instead of quadruple.}\\ \hline
\end{tabu}}
\end{wraptable}

\textbf{Barding, Medium Creature and Large Creature:} Barding is a type of armor 
that covers the head, neck, chest, body, and possibly legs of a horse or other 
mount. Barding made of medium or heavy armor provides better protection than light 
barding, but at the expense of speed. Barding can be made of any of the armor types 
found on Table: Armor and Shields.

Armor for a horse (a Large nonhumanoid creature) costs four times as much as armor 
for a human (a Medium humanoid creature) and also weighs twice as much as the armor 
found on Table: Armor and Shields (see Armor for Unusual Creatures). If the barding 
is for a pony or other Medium mount, the cost is only double, and the weight is 
the same as for Medium armor worn by a humanoid. Medium or heavy barding slows 
a mount that wears it, as shown on the table below.

Flying mounts can't fly in medium or heavy barding.

Removing and fitting barding takes five times as long as the figures given on Table: 
Donning Armor. A barded animal cannot be used to carry any load other than the 
rider and normal saddlebags.

\textbf{Dog, Riding:} This Medium dog is specially trained to carry a Small humanoid 
rider. It is brave in combat like a warhorse. You take no damage when you fall 
from a riding dog.

\textbf{Donkey or Mule:} Donkeys and mules are stolid in the face of danger, hardy, 
surefooted, and capable of carrying heavy loads over vast distances. Unlike a horse, 
a donkey or a mule is willing (though not eager) to enter dungeons and other strange 
or threatening places.

\textbf{Feed:} Horses, donkeys, mules, and ponies can graze to sustain themselves, 
but providing feed for them is much better. If you have a riding dog, you have 
to feed it at least some meat.

\textbf{Horse:} A horse (other than a pony) is suitable as a mount for a human, 
dwarf, elf, half-elf, or half-orc. A pony is smaller than a horse and is a suitable 
mount for a gnome or halfling.

Warhorses and warponies can be ridden easily into combat. Light horses, ponies, 
and heavy horses are hard to control in combat.

\textbf{Saddle, Exotic:} An exotic saddle is like a normal saddle of the same sort 
except that it is designed for an unusual mount. Exotic saddles come in military, 
pack, and riding styles.

\textbf{Saddle, Military:} A military saddle braces the rider, providing a +2 circumstance 
bonus on Ride checks related to staying in the saddle. If you're knocked unconscious 
while in a military saddle, you have a 75\% chance to stay in the saddle (compared 
to 50\% for a riding saddle).

\textbf{Saddle, Pack:} A pack saddle holds gear and supplies, but not a rider. 
It holds as much gear as the mount can carry.

\textbf{Saddle, Riding:} The standard riding saddle supports a rider
\section{Other Goods and Services}
%%%%%%%%%%%%%%%%%%%%%%%%%
\subsection{Food, Drink, and Lodging}
%%%%%%%%%%%%%%%%%%%%%%%%%

\textbf{Inn:} Poor accommodations at an inn amount to a place on the floor near 
the hearth. Common accommodations consist of a place on a raised, heated floor, 
the use of a blanket and a pillow. Good accommodations consist of a small, private 
room with one bed, some amenities, and a covered chamber pot in the corner.

\textbf{Meals:} Poor meals might be composed of bread, baked turnips, onions, and 
water. Common meals might consist of bread, chicken stew, carrots, and watered-down 
ale or wine. Good meals might be composed of bread and pastries, beef, peas, and 
ale or wine.

\begin{table}[h]
\rowcolors{1}{colorone}{colortwo}
\caption{Food, Drink, and Lodging}
\centering
{\tabulinesep=1mm
\begin{tabu}to \linewidth{X[2] X X | X[2] X X}
\header \textbf{Item} & \textbf{Cost} & \textbf{Weight} & \textbf{Item} & \textbf{Cost} & \textbf{Weight}\\ \hline
Banquet (per person) & 10 gp & --&Meals (per day)&&\\
Chunk of Meat & 3 sp & \sfrac{1}{2} lb.&\hspace{.5cm}Good & 5 sp & --\\
Hunk of Cheese & 1 sp & \sfrac{1}{2} lb.&\hspace{.5cm}Common & 3 sp & --\\
Loaf of Bread & 2 cp & \sfrac{1}{2} lb.&\hspace{.5cm}Poor & 1 sp & --\\
Ale&&&Wine&&\\
\hspace{.5cm}Gallon & 2 sp & 8 lb.&\hspace{.5cm}Common (pitcher) & 2 sp & 6 lb.\\
\hspace{.5cm}Mug & 4 cp & 1 lb. &\hspace{.5cm}Fine (bottle) & 10 gp & 1.5 lb.\\
Inn stay (per day)&&&&&\\
\hspace{.5cm}Good & 2 gp & --&&&\\
\hspace{.5cm}Common & 5 sp & --&&&\\
\hspace{.5cm}Poor & 2 sp & --&&&\\ \hline
\end{tabu}}
\end{table}

%%%%%%%%%%%%%%%%%%%%%%%%%
\subsection{Transport}
%%%%%%%%%%%%%%%%%%%%%%%%%

\textbf{Carriage}: This four-wheeled vehicle can transport as many as four people 
within an enclosed cab, plus two drivers. In general, two horses (or other beasts 
of burden) draw it. A carriage comes with the harness needed to pull it.

\textbf{Cart:} This two-wheeled vehicle can be drawn by a single horse (or other 
beast of burden). It comes with a harness.

\textbf{Galley:} This three-masted ship has seventy oars on either side and requires 
a total crew of 200. A galley is 130 feet long and 20 feet wide, and it can carry 
150 tons of cargo or 250 soldiers. For 8,000 gp more, it can be fitted with a ram 
and castles with firing platforms fore, aft, and amidships. This ship cannot make 
sea voyages and sticks to the coast. It moves about 4 miles per hour when being 
rowed or under sail.

\textbf{Keelboat:} This 50- to 75-foot-long ship is 15 to 20 feet wide and has 
a few oars to supplement its single mast with a square sail. It has a crew of eight 
to fifteen and can carry 40 to 50 tons of cargo or 100 soldiers. It can make sea 
voyages, as well as sail down rivers (thanks to its flat bottom). It moves about 
1 mile per hour.

\begin{wraptable}{r}{.4\linewidth}
\rowcolors{1}{colorone}{colortwo}
\caption{Transport}
\centering
{\tabulinesep=1mm
\begin{tabu}to \linewidth{X r r}
\header\textbf{Item} & \textbf{Cost} & \textbf{Weight}\\\hline
Carriage & 100 gp & 600 lb.\\
Cart & 15 gp & 200 lb.\\
Galley & 30,000 gp & --\\
Keelboat & 3,000 gp & --\\
Longship & 10,000 gp & --\\
Rowboat & 50 gp & 100 lb.\\
Oar & 2 gp & 10 lb.\\
Sailing ship & 10,000 gp & --\\
Sled & 20 gp & 300 lb.\\
Wagon & 35 gp & 400 lb.\\
Warship & 25,000 gp & --\\ \hline
\end{tabu}}
\end{wraptable}

\textbf{Longship:} This 75-foot-long ship with forty oars requires a total crew 
of 50. It has a single mast and a square sail, and it can carry 50 tons of cargo 
or 120 soldiers. A longship can make sea voyages. It moves about 3 miles per hour 
when being rowed or under sail.

\textbf{Rowboat:} This 8- to 12-foot-long boat holds two or three Medium passengers. 
It moves about 1.5 miles per hour.

\textbf{Sailing Ship:} This larger, seaworthy ship is 75 to 90 feet long and 20 
feet wide and has a crew of 20. It can carry 150 tons of cargo. It has square sails 
on its two masts and can make sea voyages. It moves about 2 miles per hour.

\textbf{Sled:} This is a wagon on runners for moving through snow and over ice. 
In general, two horses (or other beasts of burden) draw it. A sled comes with the 
harness needed to pull it.

\textbf{Wagon:} This is a four-wheeled, open vehicle for transporting heavy loads. 
In general, two horses (or other beasts of burden) draw it. A wagon comes with 
the harness needed to pull it.

\textbf{Warship:} This 100-foot-long ship has a single mast, although oars can 
also propel it. It has a crew of 60 to 80 rowers. This ship can carry 160 soldiers, 
but not for long distances, since there isn't room for supplies to support that 
many people. The warship cannot make sea voyages and sticks to the coast. It is 
not used for cargo. It moves about 2.5 miles per hour when being rowed or under 
sail.

%%%%%%%%%%%%%%%%%%%%%%%%%
\subsection{Spellcasting and Services}
%%%%%%%%%%%%%%%%%%%%%%%%%

\begin{table}[b]
\rowcolors{1}{colorone}{colortwo}
\caption{Spellcasting and Services}
{\tabulinesep=1mm
\begin{tabu}to \textwidth{X X}
\header\textbf{Service} & \textbf{Cost}\\ \hline
Coach cab & 3 cp per mile\\
Messenger & 2 cp per mile\\
Road or gate toll & 1 cp\\
Ship's passage & 1 sp per mile\\
Spell, 0th-level & Caster level x 5 gp\textsuperscript{1}\\
Spell, 1st-level & Caster level x 10 gp\textsuperscript{1}\\
Spell, 2nd-level & Caster level x 20 gp\textsuperscript{1}\\
Spell, 3rd-level & Caster level x 30 gp\textsuperscript{1}\\
Spell, 4th-level & Caster level x 40 gp\textsuperscript{1}\\
Spell, 5th-level & Caster level x 50 gp\textsuperscript{1}\\
Spell, 6th-level & Caster level x 60 gp\textsuperscript{1}\\
Spell, 7th-level & Caster level x 70 gp\textsuperscript{1}\\
Spell, 8th-level & Caster level x 80 gp\textsuperscript{1}\\
Spell, 9th-level & Caster level x 90 gp\textsuperscript{1}\\
Trained Hireling & 3 sp per day\\
Untrained Hireling & 1 sp per day\\\hline
\end{tabu}
\begin{tabu}to \linewidth{X}
\rowcolor{colortwo}\textsuperscript{1} See spell description for additional costs. If the additional costs put the spell's total cost above 3,000 gp, that spell is not generally available.\\ \hline
\end{tabu}}
\end{table}

Sometimes the best solution for a problem is to hire someone else to take care 
of it.

\textbf{Coach Cab:} The price given is for a ride in a coach that transports people 
(and light cargo) between towns. For a ride in a cab that transports passengers 
within a city, 1 copper piece usually takes you anywhere you need to go.

\textbf{Hireling, Trained:} The amount given is the typical daily wage for mercenary 
warriors, masons, craftsmen, scribes, teamsters, and other trained hirelings. This 
value represents a minimum wage; many such hirelings require significantly higher 
pay.

\textbf{Hireling, Untrained:} The amount shown is the typical daily wage for laborers, 
porters, cooks, maids, and other menial workers.

\textbf{Messenger:} This entry includes horse-riding messengers and runners. Those 
willing to carry a message to a place they were going anyway may ask for only half 
the indicated amount.

\textbf{Road or Gate Toll:} A toll is sometimes charged to cross a well-trodden, 
well-kept, and well-guarded road to pay for patrols on it and for its upkeep. Occasionally, 
a large walled city charges a toll to enter or exit (or sometimes just to enter).

\textbf{Ship's Passage:} Most ships do not specialize in passengers, but many have 
the capability to take a few along when transporting cargo. Double the given cost 
for creatures larger than Medium or creatures that are otherwise difficult to bring 
aboard a ship.

\textbf{Spell:} The indicated amount is how much it costs to get a spellcaster 
to cast a spell for you. This cost assumes that you can go to the spellcaster and 
have the spell cast at his or her convenience (generally at least 24 hours later, 
so that the spellcaster has time to prepare the spell in question). If you want 
to bring the spellcaster somewhere to cast a spell you need to negotiate with him 
or her, and the default answer is no.

The cost given is for a spell with no cost for a material component or focus component 
and no XP cost. If the spell includes a material component, add the cost of that 
component to the cost of the spell.

If the spell has a focus component (other than a divine focus), add 1/10 the cost 
of that focus to the cost of the spell. If the spell has an XP cost, add 5 gp per 
XP lost. 

Furthermore, if a spell has dangerous consequences, the spellcaster will certainly 
require proof that you can and will pay for dealing with any such consequences 
(that is, assuming that the spellcaster even agrees to cast such a spell, which 
isn't certain). In the case of spells that transport the caster and characters 
over a distance, you will likely have to pay for two castings of the spell, even 
if you aren't returning with the caster.

In addition, not every town or village has a spellcaster of sufficient level to 
cast any spell. In general, you must travel to a small town (or larger settlement) 
to be reasonably assured of finding a spellcaster capable of casting 1st-level 
spells, a large town for 2nd-level spells, a small city for 3rd- or 4th-level spells, 
a large city for 5th- or 6th-level spells, and a metropolis for 7th- or 8th-level 
spells. Even a metropolis isn't guaranteed to have a local spellcaster able to 
cast 9th-level spells.
\chapter{Description}
\section{Physical Appearance}
foo
\section{Personality}
foo
\section{Alignment}
foo
\section{Religion}
foo
%%%%%%%%%%%%%%%%%%%%%%%%%%%%%%%%%%%%%%%%%%%%%%%%%%
%%%%%%%%%%%%%%%%%%%%%%%%%%%%%%%%%%%%%%%%%%%%%%%%%%
\chapter{Adventuring}\label{chapter:Adventuring}
%%%%%%%%%%%%%%%%%%%%%%%%%%%%%%%%%%%%%%%%%%%%%%%%%%
%%%%%%%%%%%%%%%%%%%%%%%%%%%%%%%%%%%%%%%%%%%%%%%%%%

%%%%%%%%%%%%%%%%%%%%%%%%%%%%%%%%%%%%%%%%%%%%%%%%%%
\section{Carrying Capacity}\index{Carrying Capacity}
%%%%%%%%%%%%%%%%%%%%%%%%%%%%%%%%%%%%%%%%%%%%%%%%%%

Encumbrance rules determine how much a character's armor and equipment slow him 
or her down. Encumbrance comes in two parts: encumbrance by armor and encumbrance 
by total weight.

\textbf{Encumbrance by Armor:} A character's armor defines his or her maximum Dexterity 
bonus to AC, armor check penalty, speed, and running speed. Unless your character 
is weak or carrying a lot of gear, that's all you need to know. The extra gear 
your character carries won't slow him or her down any more than the armor already 
does.

If your character is weak or carrying a lot of gear, however, then you'll need 
to calculate encumbrance by weight. Doing so is most important when your character 
is trying to carry some heavy object.

\textbf{Weight:} If you want to determine whether your character's gear is heavy 
enough to slow him or her down more than the armor already does, total the weight 
of all the character's items, including armor, weapons, and gear. Compare this 
total to the character's Strength on Table: Carrying Capacity. Depending on how 
the weight compares to the character's carrying capacity, he or she may be carrying 
a light, medium, or heavy load. Like armor, a character's load affects his or her 
maximum Dexterity bonus to AC, carries a check penalty (which works like an armor 
check penalty), reduces the character's speed, and affects how fast the character 
can run, as shown on Table: Carrying Loads. A medium or heavy load counts as medium 
or heavy armor for the purpose of abilities or skills that are restricted by armor. 
Carrying a light load does not encumber a character.

If your character is wearing armor, use the worse figure (from armor or from load) 
for each category. Do not stack the penalties.

\textbf{Lifting and Dragging:} A character can lift as much as his or her maximum 
load over his or her head.

A character can lift as much as double his or her maximum load off the ground, 
but he or she can only stagger around with it. While overloaded in this way, the 
character loses any Dexterity bonus to AC and can move only 5 feet per round (as 
a full-round action).

A character can generally push or drag along the ground as much as five times his 
or her maximum load. Favorable conditions can double these numbers, and bad circumstances 
can reduce them to one-half or less.

\textbf{Bigger and Smaller Creatures:} The figures on Table: Carrying Capacity 
are for Medium bipedal creatures. A larger bipedal creature can carry more weight 
depending on its size category, as follows: Large x2, Huge x4, Gargantuan x8, Colossal 
x16. A smaller creature can carry less weight depending on its size category, as 
follows: Small x3/4, Tiny x1/2, Diminutive x1/4, Fine x1/8.

Quadrupeds can carry heavier loads than characters can. Instead of the multipliers 
given above, multiply the value corresponding to the creature's Strength score 
from Table: Carrying Capacity by the appropriate modifier, as follows: Fine x1/4, 
Diminutive x1/2, Tiny x3/4, Small x1, Medium x1.5, Large x3, Huge x6, Gargantuan 
x12, Colossal x24.

\textbf{Tremendous Strength:} For Strength scores not shown on Table: Carrying 
Capacity, find the Strength score between 20 and 29 that has the same number in 
the "ones" digit as the creature's Strength score does and multiply the numbers 
in that for by 4 for every ten points the creature's strength is above the score 
for that row.

\begin{table}[htb]
\rowcolors{1}{white}{offyellow}
\caption{Carrying Capacity}
\centering
\begin{tabular}{c c c c}
\textbf{Strength} & \textbf{Light Load} & \textbf{Medium Load} & \textbf{Heavy Load}\\
1 & 3 lb. or less & 4-6 lb. & 7-10 lb.\\
2 & 6 lb. or less & 7-13 lb. & 14-20 lb.\\
3 & 10 lb. or less & 11-20 lb. & 21-30 lb.\\
4 & 13 lb. or less & 14-26 lb. & 27-40 lb.\\
5 & 16 lb. or less & 17-33 lb. & 34-50 lb.\\
6 & 20 lb. or less & 21-40 lb. & 41-60 lb.\\
7 & 23 lb. or less & 24-46 lb. & 47-70 lb.\\
8 & 26 lb. or less & 27-53 lb. & 54-80 lb.\\
9 & 30 lb. or less & 31-60 lb. & 61-90 lb.\\
10 & 33 lb. or less & 34-66 lb. & 67-100 lb.\\
11 & 38 lb. or less & 39-76 lb. & 77-115 lb.\\
12 & 43 lb. or less & 44-86 lb. & 87-130 lb.\\
13 & 50 lb. or less & 51-100 lb. & 101-150 lb.\\
14 & 58 lb. or less & 59-116 lb. & 117-175 lb.\\
15 & 66 lb. or less & 67-133 lb. & 134-200 lb.\\
16 & 76 lb. or less & 77-153 lb. & 154-230 lb.\\
17 & 86 lb. or less & 87-173 lb. & 174-260 lb.\\
18 & 100 lb. or less & 101-200 lb. & 201-300 lb.\\
19 & 116 lb. or less & 117-233 lb. & 234-350 lb.\\
20 & 133 lb. or less & 134-266 lb. & 267-400 lb.\\
21 & 153 lb. or less & 154-306 lb. & 307-460 lb.\\
22 & 173 lb. or less & 174-346 lb. & 347-520 lb.\\
23 & 200 lb. or less & 201-400 lb. & 401-600 lb.\\
24 & 233 lb. or less & 234-466 lb. & 467-700 lb.\\
25 & 266 lb. or less & 267-533 lb. & 534-800 lb.\\
26 & 306 lb. or less & 307-613 lb. & 614-920 lb.\\
27 & 346 lb. or less & 347-693 lb. & 694-1,040 lb.\\
28 & 400 lb. or less & 401-800 lb. & 801-1,200 lb.\\
29 & 466 lb. or less & 467-933 lb. & 934-1,400 lb.\\
+10 & x4 & x4 & x4\\
\end{tabular}
\end{table}

\begin{table}[htb]
\rowcolors{1}{white}{offyellow}
\caption{Carrying Loads}
\centering
\begin{tabular}{l c c c c c}
\textbf{Load} & \textbf{Max Dex} & \textbf{Check Penalty} & \textbf{30ft} & \textbf{20ft} & \textbf{Run}\\
Medium & +3 & -3 & 20ft & 15ft & x4\\
Heavy & +1 & -6 & 20ft & 15ft & x3\\
\end{tabular}
\end{table}

%%%
\subsubsection{Armor and Encumbrance for Other Base Speeds}
%%%

The table below provides reduced speed figures for all base speeds from 20 feet 
to 100 feet (in 10-foot increments).

\begin{table}[htb]
\rowcolors{1}{white}{offyellow}
\caption{Reduced Rates For Other Speeds}
\centering
\begin{tabular}{c c c c}
\textbf{Base Speed} & \textbf{Reduced Speed} & \textbf{Base Speed} & \textbf{Reduced Speed}\\
20ft & 15ft & 70ft & 50ft\\
30ft & 20ft & 80ft & 55ft\\
40ft & 30ft & 90ft & 60ft\\
50ft & 35ft & 100ft & 70ft\\
60ft & 40ft & & \\
\end{tabular}
\end{table}

%%%%%%%%%%%%%%%%%%%%%%%%%%%%%%%%%%%%%%%%%%%%%%%%%%
\section{Movement}\index{Movement}
%%%%%%%%%%%%%%%%%%%%%%%%%%%%%%%%%%%%%%%%%%%%%%%%%%

There are three movement scales, as follows.

\begin{itemize}
\item Tactical, for combat, measured in feet (or squares) per round.
\item Local, for exploring an area, measured in feet per minute.
\item Overland, for getting from place to place, measured in miles per hour or miles per day.
\end{itemize}

\textbf{Modes of Movement:} While moving at the different movement scales, creatures 
generally walk, hustle, or run.

\textit{Walk:} A walk represents unhurried but purposeful movement at 3 miles per 
hour for an unencumbered human.

\textit{Hustle:} A hustle is a jog at about 6 miles per hour for an unencumbered 
human. A character moving his or her speed twice in a single round, or moving that 
speed in the same round that he or she performs a standard action or another move 
action is hustling when he or she moves.

\textit{Run (x3):} Moving three times speed is a running pace for a character in 
heavy armor. It represents about 7 miles per hour for a human in full plate.

\textit{Run (x4):} Moving four times speed is a running pace for a character 
in light, medium, or no armor. It represents about 13 miles per hour for an unencumbered 
human, or 9 miles per hour for a human in chainmail.

%%%%%%%%%%%%%%%%%%%%%%%%%
\subsection{Tactical Movement}
%%%%%%%%%%%%%%%%%%%%%%%%%

Use tactical movement for combat. Characters generally don't walk during combat -- they 
hustle or run. A character who moves his or her speed and takes some action is 
hustling for about half the round and doing something else the other half.

\textbf{Hampered Movement:}\index{Hampered Movement} Difficult terrain, obstacles, or poor visibility can 
hamper movement. When movement is hampered, each square moved into usually counts 
as two squares, effectively reducing the distance that a character can cover in 
a move. 

If more than one condition applies, multiply together all additional costs that 
apply. (This is a specific exception to the normal rule for doubling) 

In some situations, your movement may be so hampered that you don't have sufficient 
speed even to move 5 feet (1 square). In such a case, you may use a full-round 
action to move 5 feet (1 square) in any direction, even diagonally. Even though 
this looks like a 5-foot step, it's not, and thus it provokes attacks of opportunity 
normally. (You can't take advantage of this rule to move through impassable terrain 
or to move when all movement is prohibited to you.)

You can't run or charge through any square that would hamper your movement.

%%%%%%%%%%%%%%%%%%%%%%%%%
\subsection{Local Movement}
%%%%%%%%%%%%%%%%%%%%%%%%%

Characters exploring an area use local movement, measured in feet per minute.

\textbf{Walk:} A character can walk without a problem on the local scale.

\textbf{Hustle:} A character can hustle without a problem on the local scale. See 
Overland Movement, below, for movement measured in miles per hour.

\textbf{Run:} A character with a Constitution score of 9 or higher can run for 
a minute without a problem. Generally, a character can run for a minute or two 
before having to rest for a minute

%%%%%%%%%%%%%%%%%%%%%%%%%
\subsection{Overland Movement}
%%%%%%%%%%%%%%%%%%%%%%%%%

Characters covering long distances cross-country use overland movement. Overland 
movement is measured in miles per hour or miles per day. A day represents 8 hours 
of actual travel time. For rowed watercraft, a day represents 10 hours of rowing. 
For a sailing ship, it represents 24 hours.

\textbf{Walk:} A character can walk 8 hours in a day of travel without a problem. 
Walking for longer than that can wear him or her out (see Forced March, below).

\textbf{Hustle:} A character can hustle for 1 hour without a problem. Hustling 
for a second hour in between sleep cycles deals 1 point of nonlethal damage, and 
each additional hour deals twice the damage taken during the previous hour of hustling. 
A character who takes any nonlethal damage from hustling becomes fatigued.

A fatigued character can't run or charge and takes a penalty of -2 to Strength 
and Dexterity. Eliminating the nonlethal damage also eliminates the fatigue.

\textbf{Run:} A character can't run for an extended period of time.

Attempts to run and rest in cycles effectively work out to a hustle.

\textbf{Terrain:} The terrain through which a character travels affects how much 
distance he or she can cover in an hour or a day (see Table: Terrain and Overland 
Movement). A highway is a straight, major, paved road. A road is typically a dirt 
track. A trail is like a road, except that it allows only single-file travel and 
does not benefit a party traveling with vehicles. Trackless terrain is a wild area 
with no paths.

\textbf{Forced March:} In a day of normal walking, a character walks for 8 hours. 
The rest of the daylight time is spent making and breaking camp, resting, and eating.

A character can walk for more than 8 hours in a day by making a forced march. For 
each hour of marching beyond 8 hours, a Constitution check (DC 10, +2 per extra 
hour) is required. If the check fails, the character takes 1d6 points of nonlethal 
damage. A character who takes any nonlethal damage from a forced march becomes 
fatigued. Eliminating the nonlethal damage also eliminates the fatigue. It's possible 
for a character to march into unconsciousness by pushing himself too hard.

\textbf{Mounted Movement:} A mount bearing a rider can move at a hustle. The damage 
it takes when doing so, however, is lethal damage, not nonlethal damage. The creature 
can also be ridden in a forced march, but its Constitution checks automatically 
fail, and, again, the damage it takes is lethal damage. Mounts also become fatigued 
when they take any damage from hustling or forced marches.

See Table: Mounts and Vehicles for mounted speeds and speeds for vehicles pulled 
by draft animals.

\textbf{Waterborne Movement:} See Table: Mounts and Vehicles for speeds for water 
vehicles.

\begin{table}[htb]
\rowcolors{1}{white}{offyellow}\mcinherit
\caption{Movement and Distance}
\centering
\begin{tabular}{l c c c c}
\textbf{Travel Type} & \textbf{15ft} & \textbf{20ft} & \textbf{30ft} & \textbf{40ft}\\
\multicolumn{5}{l}{\textbf{One Round (Tactical)\textsuperscript{1}}}\\
Walk & 15ft & 20ft & 30ft & 40ft\\
Hustle & 30ft & 40ft & 60ft & 80ft\\
Run (x3) & 45ft & 60ft & 90ft & 120ft\\
Run (x4) & 60ft & 80ft & 120ft & 160ft\\
\multicolumn{5}{l}{\textbf{One Minute (Local)}}\\
Walk & 150ft & 200ft & 300ft & 400ft\\
Hustle & 300ft & 400ft & 600ft & 800ft\\
Run (x3) & 450ft & 600ft & 900ft & 1,200ft\\
Run (x4) & 600ft & 800ft & 1,200ft & 1,600ft\\
\multicolumn{5}{l}{\textbf{One Hour (Overland)}}\\
Walk & 1.5 miles & 2 miles & 3 miles & 4 miles\\
Hustle & 3 miles & 4 miles & 6 miles & 8 miles\\
Run & -- & -- & -- & --\\
\multicolumn{5}{l}{\textbf{One Day (Overland)}}\\
Walk & 12 miles & 16 miles & 24 miles & 32 miles\\
Hustle & -- & -- & -- & --\\
Run & -- & -- & -- & --\\
\multicolumn{5}{p{8cm}}{\textsuperscript{1} Tactical movement is often measured in squares on the battle grid (1sq = 5ft) rather than feet.}\\
\end{tabular}
\end{table}

\begin{table}[htb]
\rowcolors{1}{white}{offyellow}
\caption{Hampered Movement}
\centering
\begin{tabular}{l c}
\textbf{Condition} & \textbf{Additional Movement Cost}\\
Difficult Terrain & x2\\
Obstacle\textsuperscript{1} & x2\\
Poor Visibility & x2\\
Impassable & --\\
\multicolumn{2}{l}{\textsuperscript{1} May require a skill check.}\\
\end{tabular}
\end{table}

\begin{table}[htb]
\rowcolors{1}{white}{offyellow}
\caption{Terrain and Overland Movement}
\centering
\begin{tabular}{l c c c}
\textbf{Terrain} & \textbf{Highway} & \textbf{Road or Trail} & \textbf{Trackless}\\
Forest & x1 & x1 & x1/2\\
Frozen Tundra & x1 & x3/4 & x3/4\\
Hills & x1 & x3/4 & x1/2\\
Jungle & x1 & x3/4 & x1/4\\
Moor & x1 & x1 & x3/4\\
Mountains & x3/4 & x3/4 & x1/2\\
Plains & x1 & x1 & x3/4\\
Sandy Desert & x1 & x1/2 & x1/2\\
Swamp & x1 & x3/4 & x1/2\\
\end{tabular}
\end{table}

\begin{table}[htb]
\rowcolors{1}{white}{offyellow}
\caption{Mounts and Vehicles}
\centering
\begin{tabular}{l l l}
\textbf{Mount/Vehicle} & \textbf{Per Hour} & \textbf{Per Day}\\
Mount (carrying load) & & \\
\hspace{1cm}Donkey (51-150 lb.)\textsuperscript{1} & 2 miles & 16 miles\\
\hspace{1cm}Donkey or mule & 3 miles & 24 miles\\
\hspace{1cm}Heavy horse (201-600 lb.)\textsuperscript{1} & 3.5 miles & 28 miles\\
\hspace{1cm}Heavy horse or heavy warhorse & 5 miles & 40 miles\\
\hspace{1cm}Heavy warhorse (301-900 lb.)\textsuperscript{1} & 3.5 miles & 28 miles\\
\hspace{1cm}Light horse (151-450 lb.)\textsuperscript{1} & 4 miles & 32 miles\\
\hspace{1cm}Light horse or light warhorse & 6 miles & 48 miles\\
\hspace{1cm}Light warhorse (231-690 lb.)\textsuperscript{1} & 4 miles & 32 miles\\
\hspace{1cm}Mule (231-690 lb.)\textsuperscript{1} & 2 miles & 16 miles\\
\hspace{1cm}Pony (76-225 lb.)\textsuperscript{1} & 3 miles & 24 miles\\
\hspace{1cm}Pony or warpony & 4 miles & 32 miles\\
\hspace{1cm}Riding Dog (101-300 lb.)\textsuperscript{1} & 3 miles & 24 miles\\
\hspace{1cm}Riding Dog & 4 miles & 32 miles\\
\hspace{1cm}Warpony (101-300 lb.)\textsuperscript{1} & 3 miles & 24 miles\\
Cart or wagon & 2 miles & 16 miles\\
Ship & & \\
\hspace{1cm}Galley (rowed and sailed) & 4 miles & 96 miles\\
\hspace{1cm}Keelboat (rowed)\textsuperscript{2} & 1 mile & 10 miles\\
\hspace{1cm}Longship (sailed and rowed) & 3 miles & 72 miles\\
\hspace{1cm}Raft or barge (poled or towed)\textsuperscript{2} & 1/2 mile & 5 miles\\
\hspace{1cm}Rowboat (rowed)\textsuperscript{2} & 1.5 miles & 15 miles\\
\hspace{1cm}Sailing ship (sailed) & 2 miles & 48 miles\\
\hspace{1cm}Warship (sailed and rowed) & 2.5 miles & 60 miles\\
\multicolumn{3}{p{9.5cm}}{\textsuperscript{1} Quadrupeds, such as horses, can carry heavier loads than characters can. See Carrying Capacity, above, for more information.}\\
\multicolumn{3}{p{9.5cm}}{\textsuperscript{2} Rafts, barges, keelboats, and rowboats are used on lakes and rivers.
If going downstream, add the speed of the current (typically 3 miles per hour) to the speed of the vehicle. In addition to 10 hours of being rowed, the vehicle can also float an additional 14 hours, if someone can guide it, so add an additional 42 miles to the daily distance traveled. These vehicles can’t be rowed against any significant current, but they can be pulled upstream by draft animals on the shores.}\\
\end{tabular}
\end{table}

%%%%%%%%%%%%%%%%%%%%%%%%%
\subsection{Moving In Three Dimensions}\index{Movement!Flying}\index{Maneuverability}
%%%%%%%%%%%%%%%%%%%%%%%%%

Once movement becomes three-dimensional and involves turning in midair and maintaining 
a minimum velocity to stay aloft, it gets more complicated. Most flying creatures 
have to slow down at least a little to make a turn, and many are limited to fairly 
wide turns and must maintain a minimum forward speed. Each flying creature has 
a maneuverability, as shown on Table: Maneuverability. The entries on the table 
are defined below.

\textit{Minimum Forward Speed:} If a flying creature fails to maintain its minimum 
forward speed, it must land at the end of its movement. If it is too high above 
the ground to land, it falls straight down, descending 150 feet in the first round 
of falling. If this distance brings it to the ground, it takes falling damage. 
If the fall doesn't bring the creature to the ground, it must spend its next turn 
recovering from the stall. It must succeed on a DC 20 Reflex save to recover. Otherwise 
it falls another 300 feet. If it hits the ground, it takes falling damage. Otherwise, 
it has another chance to recover on its next turn.

\textit{Hover:} The ability to stay in one place while airborne. 

\textit{Move Backward:} The ability to move backward without turning around.

\textit{Reverse:} A creature with good maneuverability uses up 5 feet of its speed 
to start flying backward.

\textit{Turn:} How much the creature can turn after covering the stated distance.

\textit{Turn in Place:} A creature with good or average maneuverability can use 
some of its speed to turn in place.

\textit{Maximum Turn:} How much the creature can turn in any one space. 

\textit{Up Angle:} The angle at which the creature can climb.

\textit{Up Speed:} How fast the creature can climb.

\textit{Down Angle:} The angle at which the creature can descend.

\textit{Down Speed:} A flying creature can fly down at twice its normal flying 
speed.

\textit{Between Down and Up:} An average, poor, or clumsy flier must fly level 
for a minimum distance after descending and before climbing. Any flier can begin 
descending after a climb without an intervening distance of level flight.

\begin{table}[htb]
\rowcolors{1}{white}{offyellow}
\caption{Maneuverability}
\centering
\begin{tabular}{*{6}{l}}
 & \textbf{Perfect} & \textbf{Good} & \textbf{Average} & \textbf{Poor} & \textbf{Clumsy}\\
Minimum forward speed & None & None & Half & Half & Half\\
Hover & Yes & Yes & No & No & No\\
Move backward & Yes & Yes & No & No & No\\
Reverse & Free & -5 ft. & No & No & No\\
Turn & Any & 90\textdegree{}/5 ft. & 45\textdegree{}/5 ft. & 45\textdegree{}/5 ft. & 45\textdegree{}/10 ft.\\
Turn in place & Any & +90\textdegree{}/-5 ft. & +45\textdegree{}/-5 ft. & No & No\\
Maximum turn & Any & Any & 90\textdegree{} & 45\textdegree{} & 45\textdegree{}\\
Up angle & Any & Any & 60\textdegree{} & 45\textdegree{} & 45\textdegree{}\\
Up speed & Full & Half & Half & Half & Half\\
Down angle & Any & Any & Any & 45\textdegree{} & 45\textdegree{}\\
Down speed & Double & Double & Double & Double & Double\\
Between down and up & 0 ft. & 0 ft. & 5 ft. & 10 ft. & 20 ft.\\
\end{tabular}
\end{table}

%%%%%%%%%%%%%%%%%%%%%%%%%
\subsection{Evasion and Pursuit}\index{Chase}
%%%%%%%%%%%%%%%%%%%%%%%%%

In round-by-round movement, simply counting off squares, it's impossible for a 
slow character to get away from a determined fast character without mitigating 
circumstances. Likewise, it's no problem for a fast character to get away from 
a slower one. 

When the speeds of the two concerned characters are equal, there's a simple way 
to resolve a chase: If one creature is pursuing another, both are moving at the 
same speed, and the chase continues for at least a few rounds, have them make opposed 
Dexterity checks to see who is the faster over those rounds. If the creature being 
chased wins, it escapes. If the pursuer wins, it catches the fleeing creature. 

Sometimes a chase occurs overland and could last all day, with the two sides only 
occasionally getting glimpses of each other at a distance. In the case of a long 
chase, an opposed Constitution check made by all parties determines which can keep 
pace the longest. If the creature being chased rolls the highest, it gets away. 
If not, the chaser runs down its prey, outlasting it with stamina.

%%%%%%%%%%%%%%%%%%%%%%%%%
\subsection{Moving Around In Squares}
%%%%%%%%%%%%%%%%%%%%%%%%%

In general, when the characters aren't engaged in round-by-round combat, they should 
be able to move anywhere and in any manner that you can imagine real people could. 
A 5-foot square, for instance, can hold several characters; they just can't all 
fight effectively in that small space. The rules for movement are important for 
combat, but outside combat they can impose unnecessary hindrances on character 
activities.

%%%%%%%%%%%%%%%%%%%%%%%%%%%%%%%%%%%%%%%%%%%%%%%%%%
\section{Exploration}
%%%%%%%%%%%%%%%%%%%%%%%%%%%%%%%%%%%%%%%%%%%%%%%%%%

%%%%%%%%%%%%%%%%%%%%%%%%%
\subsection{Vision and Light}
%%%%%%%%%%%%%%%%%%%%%%%%%

Dwarves and half-orcs have darkvision, but everyone else needs light to see by. 
See Table: Light Sources and Illumination for the radius that a light source illuminates 
and how long it lasts.

In an area of bright light, all characters can see clearly. A creature can't hide 
in an area of bright light unless it is invisible or has cover.

In an area of shadowy illumination, a character can see dimly. Creatures within 
this area have concealment relative to that character. A creature in an area of 
shadowy illumination can make a Hide check to conceal itself.

In areas of darkness, creatures without darkvision are effectively blinded. In 
addition to the obvious effects, a blinded creature has a 50\% miss chance in combat 
(all opponents have total concealment), loses any Dexterity bonus to AC, takes 
a -2 penalty to AC, moves at half speed, and takes a -4 penalty on Search checks 
and most Strength and Dexterity-based skill checks.

Characters with \gameterm{Low-light Vision} (elves, gnomes, and half-elves) can see objects 
twice as far away as the given radius. Double the effective radius of bright light 
and of shadowy illumination for such characters.

Characters with \gameterm{Darkvision} (dwarves and half-orcs) can see lit areas normally as 
well as dark areas within 60 feet. A creature can't hide within 60 feet of a character 
with darkvision unless it is invisible or has cover.

\begin{table}[htb]
\rowcolors{1}{white}{offyellow}
\caption{Light Sources and Illumination}
\centering
\begin{tabular}{l c c c}
\textbf{Object} & \textbf{Bright} & \textbf{Shadowy} & \textbf{Duration}\\
Candle & n/a\textsuperscript{1} & 5 ft. & 1 hr.\\
Common Lamp & 15 ft. & 30 ft. & 6 hr./pint\\
Everburning torch & 20 ft. & 40 ft. & Permanent\\
Lantern (bullseye)\textsuperscript{2} & 60-ft. cone & 120-ft. cone & 6 hr./pint\\
Lantern (hooded) & 30 ft. & 60 ft. & 6 hr./pint\\
Sunrod & 30 ft. & 60 ft. & 6 hr.\\
Torch & 20 ft. & 40 ft. & 1 hr.\\
\textbf{Spell} & \textbf{Bright} & \textbf{Shadowy} & \textbf{Duration}\\
Continual flame & 20 ft. & 40 ft. & Permanent\\
Dancing lights (torches) & 20 ft. (each) & 40 ft. (each) & 1 min.\\
Daylight & 60 ft. & 120 ft. & 30 min.\\
Light & 20 ft. & 40 ft. & 10 min.\\
\multicolumn{4}{l}{\textsuperscript{1} A candle does not provide bright illumination, only shadowy illumination.}\\
\multicolumn{4}{l}{\textsuperscript{2} A bullseye lantern illuminates a cone, not a radius.}\\
\end{tabular}
\end{table}

%%%%%%%%%%%%%%%%%%%%%%%%%
\subsection{Breaking and Entering}
%%%%%%%%%%%%%%%%%%%%%%%%%

When attempting to break an object, you have two choices: smash it with a weapon 
or break it with sheer strength.

%%%
\subsubsection{Smashing an Object}
%%%

Smashing a weapon or shield with a slashing or bludgeoning weapon is accomplished 
by the sunder special attack. Smashing an object is a lot like sundering a weapon 
or shield, except that your attack roll is opposed by the object's AC. Generally, 
you can smash an object only with a bludgeoning or slashing weapon.

\textbf{Armor Class:} Objects are easier to hit than creatures because they usually 
don't move, but many are tough enough to shrug off some damage from each blow. 
An object's Armor Class is equal to 10 + its size modifier + its Dexterity modifier. 
An inanimate object has not only a Dexterity of 0 (-5 penalty to AC), but also 
an additional -2 penalty to its AC. Furthermore, if you take a full-round action 
to line up a shot, you get an automatic hit with a melee weapon and a +5 bonus 
on attack rolls with a ranged weapon.

\textbf{\gameterm{Hardness}:} Each object has hardness -- a number that represents how well 
it resists damage. Whenever an object takes damage, subtract its hardness from 
the damage. Only damage in excess of its hardness is deducted from the object's 
hit points (see Table: Common Armor, Weapon, and Shield Hardness and Hit Points; 
Table: Substance Hardness and Hit Points; and Table: Object Hardness and Hit Points).

\textbf{Hit Points:} An object's hit point total depends on what it is made of 
and how big it is (see Table: Common Armor, Weapon, and Shield Hardness and Hit 
Points; Table: Substance Hardness and Hit Points; and Table: Object Hardness and 
Hit Points). When an object's hit points reach 0, it's ruined.

Very large objects have separate hit point totals for different sections.

\textit{Energy Attacks:} Acid and sonic attacks deal damage to most objects just 
as they do to creatures; roll damage and apply it normally after a successful hit. 
Electricity and fire attacks deal half damage to most objects; divide the damage 
dealt by 2 before applying the hardness. Cold attacks deal one-quarter damage to 
most objects; divide the damage dealt by 4 before applying the hardness.

\textit{Ranged Weapon Damage:} Objects take half damage from ranged weapons (unless 
the weapon is a siege engine or something similar). Divide the damage dealt by 
2 before applying the object's hardness.

\textit{Ineffective Weapons:} Certain weapons just can't effectively deal damage 
to certain objects.

\textit{Immunities:} Objects are immune to nonlethal damage and to critical hits.
Even animated objects, which are otherwise considered creatures, have these immunities 
because they are constructs.

\textit{Magic Armor, Shields, and Weapons:} Each +1 of enhancement bonus adds 2 
to the hardness of armor, a weapon, or a shield and +10 to the item's hit points.

\textit{Vulnerability to Certain Attacks:} Certain attacks are especially successful 
against some objects. In such cases, attacks deal double their normal damage and 
may ignore the object's hardness.

\textit{Damaged Objects:} A damaged object remains fully functional until the item's 
hit points are reduced to 0, at which point it is destroyed.
Damaged (but not destroyed) objects can be repaired with the \linkskill{Craft} skill.

\textbf{Saving Throws:}\index{Saving Throws!of Items} Nonmagical, unattended items never make saving throws. 
They are considered to have failed their saving throws, so they always are affected 
by spells. An item attended by a character (being grasped, touched, or worn) makes 
saving throws as the character (that is, using the character's saving throw bonus).

Magic items always get saving throws. A magic item's Fortitude, Reflex, and Will 
save bonuses are equal to 2 + one-half its caster level. An attended magic item 
either makes saving throws as its owner or uses its own saving throw bonus, whichever 
is better.

\textit{Animated Objects:} Animated objects count as creatures for purposes of 
determining their Armor Class (do not treat them as inanimate objects).

%%%
\subsubsection{Breaking Items}\index{Breaking Items}
%%%

When a character tries to break something with sudden force rather than by dealing 
damage, use a Strength check (rather than an attack roll and damage roll, as with 
the sunder special attack) to see whether he or she succeeds. The DC depends more 
on the construction of the item than on the material.

If an item has lost half or more of its hit points, the DC to break it drops by 2.

Larger and smaller creatures get size bonuses and size penalties on Strength checks 
to break open doors as follows: Fine -16, Diminutive -12, Tiny -8, Small -4, Large 
+4, Huge +8, Gargantuan +12, Colossal +16.

A crowbar or portable ram improves a character's chance of breaking open a door.

\begin{table}[htb]
\rowcolors{1}{white}{offyellow}
\caption{Common Armor, Weapon, and Shield Hardness and Hit Points}
\centering
\begin{tabular}{l c c}
\textbf{Weapon or Shield} & \textbf{Hardness} & \textbf{HP\textsuperscript{1}}\\
Light blade & 10 & 2\\
One-handed blade & 10 & 5\\
Two-handed blade & 10 & 10\\
Light metal-hafted weapon & 10 & 10\\
One-handed metal-hafted weapon & 10 & 20\\
Light hafted weapon & 5 & 2\\
One-handed hafted weapon & 5 & 5\\
Two-handed hafted weapon & 5 & 10\\
Projectile weapon & 5 & 5\\
Armor & special\textsuperscript{2} & armor bonus x 5\\
Buckler & 10 & 5\\
Light wooden shield & 5 & 7\\
Heavy wooden shield & 5 & 15\\
Light steel shield & 10 & 10\\
Heavy steel shield & 10 & 20\\
Tower shield & 5 & 20\\
\multicolumn{3}{p{10cm}}{\textsuperscript{1} The hp value given is for Medium armor, weapons, and shields. Divide by 2 for each size category of the item smaller than Medium, or multiply it by 2 for each size category larger than Medium.}\\
\multicolumn{3}{p{10cm}}{\textsuperscript{2} Varies by material; see Table: Substance Hardness and Hit Points.}\\
\end{tabular}
\end{table}

\begin{table}[htb]
\rowcolors{1}{white}{offyellow}
\caption{Substance Hardness and Hit Points}
\centering
\begin{tabular}{l c c}
\textbf{Substance} & \textbf{Hardness} & \textbf{Hit Points}\\
Paper or cloth & 0 & 2/inch of thickness\\
Rope & 0 & 2/inch of thickness\\
Ice & 0 & 3/inch of thickness\\
Glass & 1 & 1/inch of thickness\\
Leather or hide & 2 & 5/inch of thickness\\
Wood/Darkwood & 5 & 10/inch of thickness\\
Stone & 8 & 15/inch of thickness\\
Dragonhide & 10 & 10/inch of thickness\\
Iron/Cold Iron/Steel & 10 & 30/inch of thickness\\
Mithral & 15 & 30/inch of thickness\\
Adamantine & 20 & 40/inch of thickness\\
\end{tabular}
\end{table}

\begin{table}[htb]
\rowcolors{1}{white}{offyellow}
\caption{Size and Armor Class of Objects}
\centering
\begin{tabular}{l c}
\textbf{Size} & \textbf{AC Modifier}\\
Colossal & -8\\
Gargantuan & -4\\
Huge & -2\\
Large & -1\\
Medium & +0\\
Small & +1\\
Tiny & +2\\
Diminutive & +4\\
Fine & +8\\
\end{tabular}
\end{table}

\begin{table}[htb]
\rowcolors{1}{white}{offyellow}
\caption{Object Hardness and Hit Points}
\centering
\begin{tabular}{l c c c}
\textbf{Object} & \textbf{Hardness} & \textbf{Hit Points} & \textbf{Break DC}\\
Rope (1 inch diam.) & 0 & 2 & 23\\
Simple wooden door & 5 & 10 & 13\\
Small chest & 5 & 1 & 17\\
Good wooden door & 5 & 15 & 18\\
Treasure chest & 5 & 15 & 23\\
Strong wooden door & 5 & 20 & 23\\
Masonry wall (1 ft. thick) & 8 & 90 & 35\\
Hewn stone (3 ft. thick) & 8 & 540 & 50\\
Chain & 10 & 5 & 26\\
Manacles & 10 & 10 & 26\\
Masterwork manacles & 10 & 10 & 28\\
Iron door (2 in. thick) & 10 & 60 & 28\\
\end{tabular}
\end{table}

\begin{table}[htb]
\rowcolors{1}{white}{offyellow}
\caption{DCs to Break or Burst Items}
\centering
\begin{tabular}{l c}
\textbf{Strength Check to:} & \textbf{DC}\\
Break down simple door & 13\\
Break down good door & 18\\
Break down strong door & 23\\
Burst rope bonds & 23\\
Bend iron bars & 24\\
Break down barred door & 25\\
Burst chain bonds & 26\\
Break down iron door & 28\\
\textbf{Condition} & \textbf{DC Mod\textsuperscript{1}}\\
Hold portal & +5\\
Arcane lock & +10\\
\multicolumn{2}{l}{\textsuperscript{1} If both apply, use the larger number.}\\
\end{tabular}
\end{table}

%%%%%%%%%%%%%%%%%%%%%%%%%
\subsection{Special Materials}
%%%%%%%%%%%%%%%%%%%%%%%%%

In addition to magic items created with spells, some substances have innate special 
properties.

If you make a suit of armor or weapon out of more than one special material, you 
get the benefit of only the most prevalent material. However, you can build a double 
weapon with each head made of a different special material. 

Each of the special materials described below has a definite game effect. Some 
creatures have damage reduction based on their creature type or core concept. Some 
are resistant to all but a special type of damage, such as that dealt by evil-aligned 
weapons or bludgeoning weapons. Others are vulnerable to weapons of a particular 
material. Characters may choose to carry several different types of weapons, depending 
upon the campaign and types of creatures they most commonly encounter. 

%%%
\subsubsection{Alchemical Silver}\index{Alchemical Silver}
%%%

A complex process involving metallurgy and alchemy 
can bond silver to a weapon made of steel so that it bypasses the damage reduction 
of creatures such as lycanthropes.

On a successful attack with a silvered weapon, the wielder takes a -1 penalty on 
the damage roll (with the usual minimum of 1 point of damage). The alchemical silvering 
process can't be applied to nonmetal items, and it doesn't work on rare metals 
such as adamantine, cold iron, and mithral.

Alchemical silver has 10 hit points per inch of thickness and hardness 8.

\begin{table}[htb]
\rowcolors{1}{white}{offyellow}
\caption{Alchemical Silver Prices}
\centering
\begin{tabular}{l r}
\textbf{Type of Alchemical Silver Item} & \textbf{Cost Modifier}\\
Ammunition & +2gp\\
Light weapon & +20gp\\
One-handed weapon, or one head of a double weapon & +90gp\\
Two-handed weapon, or both heads of a double weapon & +180gp\\
\end{tabular}
\end{table}

%%%
\subsubsection{Adamantine}\index{Adamantine}
%%%

This ultra-hard metal adds to the quality of a weapon or suit 
of armor. Weapons fashioned from adamantine have a natural ability to bypass hardness 
when sundering weapons or attacking objects, ignoring hardness less than 20. Armor 
made from adamantine grants its wearer damage reduction of 1/- if it's light armor, 
2/- if it's medium armor, and 3/- if it's heavy armor. Adamantine is so costly 
that weapons and armor made from it are always of masterwork quality; the masterwork 
cost is included in the prices given below. Thus, adamantine weapons and ammunition 
have a +1 enhancement bonus on attack rolls, and the armor check penalty of adamantine 
armor is lessened by 1 compared to ordinary armor of its type. Items without metal 
parts cannot be made from adamantine. An arrow could be made of adamantine, but 
a quarterstaff could not.

Only weapons, armor, and shields normally made of metal can be fashioned from adamantine. 
Weapons, armor and shields normally made of steel that are made of adamantine have 
one-third more hit points than normal. Adamantine has 40 hit points per inch of 
thickness and hardness 20.

\begin{table}[htb]
\rowcolors{1}{white}{offyellow}
\caption{Adamantine Prices}
\centering
\begin{tabular}{l r}
\textbf{Type of Adamantine Item} & \textbf{Cost Modifier}\\
Ammunition & +60gp\\
Light Armor & +5,000gp\\
Medium Armor & +10,000gp\\
Heavy Armor & +15,000gp\\
Weapon & +3,000gp\\
\end{tabular}
\end{table}

%%%
\subsubsection{Cold Iron}\index{Cold Iron}
%%%

This iron, mined deep underground, known for its effectiveness 
against fey creatures, is forged at a lower temperature to preserve its delicate 
properties. Weapons made of cold iron cost twice as much to make as their normal 
counterparts. Also, any magical enhancements cost an additional 2,000 gp. 

Items without metal parts cannot be made from cold iron. An arrow could be made 
of cold iron, but a quarterstaff could not.

A double weapon that has only half of it made of cold iron increases its cost by 
50\%.

Cold iron has 30 hit points per inch of thickness and hardness 10.

%%%
\subsubsection{Darkwood}\index{Darkwood}
%%%

This rare magic wood is as hard as normal wood but very light. 
Any wooden or mostly wooden item (such as a bow, an arrow, or a spear) made from 
darkwood is considered a masterwork item and weighs only half as much as a normal 
wooden item of that type. Items not normally made of wood or only partially of 
wood (such as a battleaxe or a mace) either cannot be made from darkwood or do 
not gain any special benefit from being made of darkwood. The armor check penalty 
of a darkwood shield is lessened by 2 compared to an ordinary shield of its type. 
To determine the price of a darkwood item, use the original weight but add 10 gp 
per pound to the price of a masterwork version of that item.

Darkwood has 10 hit points per inch of thickness and hardness 5.

%%%
\subsubsection{Dragonhide}\index{Dragonhide}
%%%

Armorsmiths can work with the hides of dragons to produce 
armor or shields of masterwork quality. One dragon produces enough hide for a single 
suit of masterwork hide armor for a creature one size category smaller than the 
dragon. By selecting only choice scales and bits of hide, an armorsmith can produce 
one suit of masterwork banded mail for a creature two sizes smaller, one suit of 
masterwork half-plate for a creature three sizes smaller, or one masterwork breastplate 
or suit of full plate for a creature four sizes smaller. In each case, enough hide 
is available to produce a small or large masterwork shield in addition to the armor, 
provided that the dragon is Large or larger.

Because dragonhide armor isn't made of metal, druids can wear it without penalty.

Dragonhide armor costs double what masterwork armor of that type ordinarily costs, 
but it takes no longer to make than ordinary armor of that type.

Dragonhide has 10 hit points per inch of thickness and hardness 10.

%%%
\subsubsection{Mithral}\index{Mithral}
%%%

Mithral is a very rare silvery, glistening metal that is lighter 
than iron but just as hard. When worked like steel, it becomes a wonderful material 
from which to create armor and is occasionally used for other items as well. Most 
mithral armors are one category lighter than normal for purposes of movement and 
other limitations. Heavy armors are treated as medium, and medium armors are treated 
as light, but light armors are still treated as light. Spell failure chances for 
armors and shields made from mithral are decreased by 10\%, maximum Dexterity bonus 
is increased by 2, and armor check penalties are lessened by 3 (to a minimum of 
0).

An item made from mithral weighs half as much as the same item made from other 
metals. In the case of weapons, this lighter weight does not change a weapon's 
size category or the ease with which it can be wielded (whether it is light, one-handed, 
or two-handed). Items not primarily of metal are not meaningfully affected by being 
partially made of mithral. (A longsword can be a mithral weapon, while a scythe 
cannot be.)

Weapons or armors fashioned from mithral are always masterwork items as well; the 
masterwork cost is included in the prices given below.

Mithral has 30 hit points per inch of thickness and hardness 15.

\begin{table}[htb]
\rowcolors{1}{white}{offyellow}
\caption{Mithral Prices}
\centering
\begin{tabular}{l r}
\textbf{Type of Mithral Item} & \textbf{Cost Modifier}\\
Light Armor & +1,000gp\\
Medium Armor & +4,000gp\\
Heavy Armor & +9,000gp\\
Shield & +1,000gp\\
Other Item & +500gp/lb\\
\end{tabular}
\end{table}
\section{The Constructanomicon}
\vspace*{-10pt}
\quot{``How does that even stay up?"}

Perhaps the most important question surrounding Dungeons and Dragons is the question why there are Dungeons and Dragons. When you think about it, that's pretty weird.

\subsection{Dungeons: By the Gods, Why?}

Alright, we know that you love dungeons. We love them too, despite the fact that we're pretty sure there is no good reason for the silly things. The average D\&D game world is frankly incapable of the technology or manpower needed to build vast underground complexes. I mean, look at our own world history: aside from a single underground city in Turkey and a couple of pyramids and tombs, the ancient world took a pass on underground life. Even the old excuse of ``Wizards can magic it up and they do it because its defensible'' is a bit lame considering that we are talking about a world with teleport and burrowing and ethereal travel; being underground is actually a liability since its harder to escape and people can drop the roof onto you, not to mention the incredible costs involved in doing it even if magic is available.

So here is what we suggest: dungeons have an actual magical purpose. By putting anything behind at least 40' of solid, continuous material (like solid walls of dirt, stone, ice, or whatever, but not a forest of trees or rooms of furniture) the area is immune to unlimited-range or ``longer than Long Range'' spells like Scrying and transportation magic like teleport, greater teleport, the travel version of gate, and other effects. You can use these magics inside a dungeon, but you also stopped by a 40' solid, continuous material in a Line of Effect; this means you can use these effects inside a dungeon to bypass doors and walls, but entering and leaving the dungeon is a problem, and parts of the dungeon that have more than 30' of material in the way between your position and the target of your effect will be effectively isolated from your position.

In summary, in a best-case scenario you can transport yourself to a dungeon, then bust in the entrance and enter the dungeon, then transport yourself to the place you want to be inside the dungeon. In a worse-case scenario, the dungeon designer will have built the dungeon in such a way that only someone aware of the layout can take full advantage of unlimited range or transportation spells like teleports and Scry, or even that most or all areas if the dungeon are inaccessible to these effects.

Of course, there are exceptions. The idea of permanent portals, gates, or teleport circles are just too common in D\&D and too fun to just abandon. Permanent effects will continue to regardless of materials in the way, and will be the premier way to enter and leave dungeons, as well as the best way to move inside a dungeon.

By incorporating these changes in your D\&D world, you are ensuring that players actually explore rooms in your dungeons that you have painstakingly built, you avoid all the problems with Scry-and-Die tactics, and you'll find that players actually care about dungeon geography. It also adds a bit to suspension of disbelief in your setting, which is only good for a cooperative storytelling game.

\subsubsection{Dungeons: building dungeons for fun, profit, and defense}
As an old hand at D\&D, I've seen more dungeons than I can count. Most have followed a ``random generator and a new pad of graph paper'' philosophy to dungeon construction, and frankly that's got to go. Here are a few tips to constructing a dungeon that makes real sense:

\subsubsection{Chokepoints Are Your Friend}
Most dungeons are built like a modern building: ease of use and easy access are emphasized. Don't do that. Remember that a dungeon is built with the idea that it will be invaded at some point by a hostile and possibly supernatural attacker. At the very least, this means that rooms will not have doors to every adjacent room, and single hallways to single rooms will also be avoided.

Chokepoints are your single most important consideration. You want to make sure that attackers get bottled up in them and your forces don't get caught up in them. That's trickier than it sounds. Generally, place your chokepoints at the entrances and exits of your dungeon, and possibly at ``fall back'' positions where troops can make another stand if their position is overrun. Key locations should have their own chokepoints like prisons, treasuries, and quarters for potentially hostile quests. Locations that should not be blocked off by choke points include barracks, armories, and key storage rooms, since you never know when your troops might need some arms or materials to react to a threat.

\subsubsection{The Three ``M''s: Mobility, Manpower, and Morale}

A dungeon is built to house a fighting force, and several considerations come into play in its design. If your dungeon is an abandoned ruin, then the current residents might not exploit these features, but be sure that the original designers had them in mind..

\textbf{Mobility:} Choke points are the first stage in the idea of mobility, as they assume that your enemies will be stuck gathering their forces at once point and behind that chokepoint you are gathering your forces as well; however, that does not need to be true. The designers of a dungeon can easily place one-way secret doors that allow them to get behind an enemy position and outflank an enemy, sending forces from two sides to crush an enemy.

Also, the common feature of long hallways with rooms off to the sides must be avoided. While this is a simple arrangement (and easy to draw on graph paper), it allows attackers to make straight shots toward key areas. It is better to mix-up the layout of non-essential rooms like storerooms so that enemy forces become split as they search rooms and take different routes. A common mistake like a long hallway or a central room with doors allows the enemy to send scouting forces to check rooms, then they can quickly surge forward if one of those forces finds a threat. It is better to split an enemy's forces between several collections of rooms, leaving groups isolated in the event of a counterattack.

\textbf{Manpower:} A well-designed dungeon needs guardians, and there are no solid rules about who you need in your dungeon. Generally, you want troops that are loyal, intelligent, and powerful, but often other considerations come into play. Dumb beasts can be chained at a choke point, and they are perfectly suitable as guardians, and large numbers of weak but smart defenders can set off traps, block passages, or slow the advance of the enemy with caltrops or even their own lifeblood. Depending on the type of guardians the dungeon was intended for, it can have wildly different layouts. For example, a dungeon may have a room that is merely a pit with ladders leading to an entrance and exit, and this room simply houses a dangerous beast like a Dire Bear. Any enemy who wants to take this chokepoint would need to fight the bear. Another example could be a dungeon designed to have kobolds as defenders; this kind of dungeon may have small-sized corridors so that they can move quickly from rooms to room (so that any medium-sized creature must squeeze in) and covered shooting galleries where the kobolds can use crossbows to fire on attackers from relative safety.

\textbf{Morale:} An often overlooked aspect of dungeon construction is morale, which is the simple question of ``are my troops happy enough to stay and confident enough to fight.'' Kitchens and ample food stores are a good first place to start, as are comforts like good barracks or personal rooms, timely payment of salary, and amusements. While a Half-fiend Chimera can be locked in box without food or air, its loyalty and willingness to fight is definitely in question. Some dungeon creators use mindless beasts or unintelligent monsters like oozes, while other creators use controlled monsters like undead, but these troops are generally less effective than dedicated and intelligent troops.

If the dungeon has luxuries like escape routes, common rooms to socialize in, entertainments like gaming rooms, and places to worship gods, troops will be more willing to fight when attackers threaten. Without these things, troops might surrender or flee from a hostile threat, or even turn on the dungeon creator.

\subsubsection{Form Follows Function}

Sometimes, dungeons can be designed in a crazy fashion that is fun to play in, but makes no tactical sense. That's fine, since it can mean that the dungeon was built as part of a magical effect or for some mystical reason. A certain arrangement of rooms may create a dungeon-wide effect that blocks ethereal travel or teleportation, or maybe the fact that the dungeon is arranged like a demon's face means that the dungeon is a giant mystical trap for a bound demon.

The sky is the limit for this kind of thing, and we encourage you to ``go nuts" as it creates flavorful dungeons that you will remember years later. I'm certain people are more likely to remember a dungeon built as a giant hive with hexagonal rooms, honeycombed passages, and undead bees than they are going to remember a standard temple of Orcus.

\subsubsection{Castles and Manors: Taking the Dungeon out of the Dungeon}

Traps, choke-points, humanoid defenders, and monstrous occupants can all be found guarding treasures and lifestyles above ground as well as below. Unfortunately, a building that extends above the surface is inherently more vulnerable than a true Dungeon to the most feared of D\&D tactics: Scry \& Die.

\subsubsection{Unimportance}

While a castle is by definition subject to scry \& die tactics, the number of creatures actually capable of pulling that off is fairly limited and if they don't care enough about your buildings, you're pretty safe. A building doesn't have to be bereft of valuable loot and major players in the game of thrones to avoid teleport assaults -- it just has to look that way. In many ways a run-down shack is safer than a gleaming adamantine fortress. And that means that illusions like hallucinatory terrain and mirage arcana are very valuable to any fortress whose purpose is to keep its occupants and their treasure safe. If no one cares, your swag and your family are safe.

\subsubsection{Magically Foiling Diviners}

When you don't have 40' of solid stone between you and the hostile world outside, scry \& die is a real problem for you. Especially if you're trying to keep order and rule a region, and therefore hiding your fortress really isn't an option, magically protecting yourself from attack magic is going to look pretty tempting. For those of you who are old school, attention has to be drawn to the fact that \spell{nondetection} actually doesn't work at all. It costs you money every day, and the would-be teleport assassins have a chance of spell failure every time they attempt to scry on your location. But nothing happens to them if it doesn't work, so at best \spell{nondetection} makes them try again later. Eventually they're still going to come for you, and you're out a small pile of diamond dust.

The big winners here are \spell{mirage arcana} and \spell{mindblank}. \spell{Mindblank} always wins, even against gods, but it only stops people from pulling a scry \& die on you. Your enemies can still teleport ambush your house, or your butler, and just sort of assume that if your servants are preparing your favorite food in your house that you're probably in there somewhere. This means that if you are living a high profile gangster lifestyle, \spell{mindblank} is of limited utility, but if you are willing to be a shadowy sage who lives on a demiplane somewhere that no one has heard of, it's totally the win. Mirage arcana simply makes a room appear as a different room. This means that when someone attempts a scry \& die, they end up shunted to some completely different room that presumably has deadly magical traps all over it. Unfortunately, there are ways for a clever diviner to bypass that sort of thing, and there's not a whole lot you can do about it. Ultimately, only stupid Wizards lose when they pull Scry \& Die, so based on the Intelligence requirements of Wizarding\ldots\  you pretty much know how this is going to go down. Still, a clever Illusion trap can nab an impatient Wizard, and that's often good enough.

A special shout-out needs to go to \spell{dimensional lock}, because the effects on would-be teleport assassins is hilarious. It doesn't cause the spell to fail, it merely stops dimensional movement into the warded area. So the assassin moves to the Astral Plane, is shifted at high speed over to the segment that corresponds to next to your bed, and then the shift back into the material world fails. This leaves them all buffed up and stranded on the Astral Plane. You can even amuse yourself by putting lethal traps on that portion of the Astral Plane to nail these guys on the way in. The downside of course is that a lock is only 40 feet across, so covering enough of a castle to make teleport ambushes impractical is difficult. Still, if you have enough 8th level spell slots lying around (or less, remember that it's a lower level spell for the Summoner), it provides the basis of some very nice protection. Also good is the fact that since dimensional locks can be tiled, it can also leave spaces that you can use as a means of entrance/egress and which can be potentially defended if they are used as attack points by hostiles.

The \spell{anticipate teleport} line of spells is a cantrip on the Summoner list for a reason. Those spells don't actually stop a scry \& die, and the areas are very small and duration unexceptional. Even if you are a Summoner, defending your house with \spell{anticipate teleport} is probably implausible. The final consideration is the elephant in the room: \spell{Screen}. It's an enormously powerful spell where it fools scrying ``automatically", but unfortunately it is defined so vaguely as to be essentially unusable without creating an argument. Which is really a shame, because it's otherwise the best hope for defending yourself. Your best bet is to make certain key rooms appear like other rooms so that teleport ambushes end up in the wrong areas -- which means that it's basically just \spell{mirage arcane} that's several levels higher.

\subsubsection{The Public Square: When Divination Doesn't Matter}

Sometimes your building is Courthouse, or a Market, or a Factory, and the entire point is that the general public goes in and out of the building all the time. In such a circumstance, all the divination magic in the world doesn't mean anything because your enemies can actually just walk into your building to scout the place for a teleport ambush or even buff themselves up on the outside and then run in while 1/round a level spells are counting down their awesome. In these circumstances, you're going to want a fall-back position to be readily available on little or no notice. Contingent magic and magical traps may well want to pull key personnel out rather than send summoned monsters or impediments in. After all, if you put off the final confrontation for 20 minutes, the teleport ambush has essentially failed.

\subsection{Traps}
\vspace*{-8pt}
\quot{``How did that boulder not crush those displacer beasts?"}

Dungeons are classically filled with monsters and traps. That can be cool, but I'll be the first to admit that it's pretty weird. Traps and monsters are profoundly counter synergistic.

\subsubsection{Designing Traps}

There are numerous collections of devious traps that can easily kill a single character or an entire party. But let's face it: most of them are dumb. Making a trap that will kill or humiliate characters doesn't make you a genius, making traps that kill player characters is easy. Just have the roof cave in to inflict more damage than the PCs have in hit points, it's not even hard. The difficulty is making traps that make sense, as well as traps that will add to the enjoyment of the game rather than paralyze it with a continuous ``I check the banister, Mother May I?" fest.

\subsubsection{Placing Traps}

For a trap to be effective, it has to have essentially no chance of backfiring against its creators. Remember that the dungeon occupants are going to spend a lot more time in the vicinity of any traps than any invading force is, so there has to be a pretty good reason why the trap wouldn't backfire. Traps can cordon off areas that are too big or too small for the normal residents to set them off (Kobolds might put in a collapsing floor that triggered off a weight of over 100 pounds, and Stone Giants might put nasty traps all over any 5' hallways that ran through a dungeon they occupied), or areas that are for whatever other reason off-limits (Dwarves might trap tapped-out shafts in their mines to nail burrowing monsters trying to sneak in the back way). Some traps sound like they'd be plenty selective enough to put everywhere -- like magical symbols that only blast the forces of Good or heretics who don't follow your god. Be careful with those, as just because they won't explode on any of the normal residents doesn't mean that they won't be a liability. After all, what's the point of being a Cleric of Loviatar if you can't have captured Paladins brought to your chambers for interrogation?

Traps can also be left in an ``inactive" state much of the time, and then triggered into activity only when the dungeon's occupants believe that they are under attack. A switch that activates traps in many non-essential areas (like the rec room or the loading dock) is a very real possibility. These can also be activated in layers, a prearranged fallback point might have the mechanisms to activate traps in the outer area that has presumably been compromised by intruders.

Remember that a trap, once active, makes an area more difficult to use. Sometimes that's OK, as is the case when the area in question is being invaded by Bugbears or is itself a tomb prison meant to hold a powerful demon god. But sometimes that's really inconvenient. Active traps just don't make any sense in the mess hall or the barracks. Your own soldiers are going to fall into that pit full of spikes about a thousand times more often than invading adventurers are if you put it right next to the beer kegs.

\subsubsection{Organizational Traps}

The least obvious, but in many ways most useful trap is one which simply allows defenders to respond appropriately to an oncoming attack. An alarm spell is, in the right hands, the most powerful trap in the core rules. You can put it anywhere, and all it does is make a sound when someone enters the area. Like the bell that sounds when you enter a 7-11, the effects of this trap do not meaningfully interfere with the normal operations of the facility they are ensconced in. These traps have as their core utility that they alert the defenders or delay an attacker. Really swank traps will do both.

Obviously, these traps are only worth anything if you have defenders. But remember that a dungeon filled with giant centipedes, or some other mindless monster really isn't going to take full advantage of an alarm system (a ringing bell may wake a sleeping mindless defender up, but it's not going to be able to figure out whether the bell indicates a customer or an invader). Traps designed to misdirect, delay, or otherwise hamper invading forces are only going to appear in unoccupied regions of a dungeon if they are capable of diverting unauthorized entrants into lethal traps. The name for that kind of set-up is a ``Rube Goldberg Mechanism" and it generally has no place in D\&D. Looney Tunes or Mousetrap perhaps, but generally not Dungeons and Dragons.

\subsubsection{Lethal Traps}

Lethal traps are in no way less dangerous to their creators than they are to invaders. Remember always that the creatures in a dungeon intend to live there for perhaps years or even centuries, and the statistics on mine fields just aren't good. The residents of a dungeon have to be completely convinced that a potential trap can't cut off their jangly bits when they are making their way to the privy in the middle of their sleep cycle. That doesn't even mean that lethal traps can be in places that unauthorized residents aren't allowed (like the master's bedchamber) -- that's going to end up beheading servants and guests.

Lethal traps appear in only a couple of kinds of places:

\listone
    \item \textbf{Battlefields:} If an area is contested, right now, having a lethal trap in there is an antisocial but plausible technique.
    \item \textbf{Deserted Regions:} If you leave the dungeon to go on a pilgrimage to a Planar Touchstone that you dig, it's quite thinkable to activate some nasty traps while you're gone.
    \item \textbf{Inaccessible Areas:} If you take over a Brownie hole, there's going to be a lot of crazy hallways that you can't even get into. Filling the mouse holes with mousetraps is fine.
    \item \textbf{Vaults:} If you have something, like a repository of important treasure perhaps, that is really hard to open and is supposed to be used infrequently and possibly only in some sort of crazy ``two guys whip out their keys at the same time" scheme -- trapping that is totally expected.
    \item \textbf{Discerning Traps:} Some magical traps are able to detect certain kinds of creatures and only detonate on specific ones. Unless you're a crazy loner wizard who has no friends and conducts no commerce, those are pretty much a liability. But hey, if you are a Lich-Master Hermit, then those sorts of traps are fine.
\end{list}

\vspace*{8pt}

What this means is that if a dungeon isn't on a war footing right now, any lethal traps in it are probably going to be inactive. If the hobgoblins don't believe that they are under attack right now, the pressure plates all over the dungeon are going to be in their locked position and opening doors is not going to cause poison blades to shoot out. Once they fall back and pull the ``totally being attacked" lever -- then you can go back to worrying if Gygaxian traps lurk behind every door or neck-level tripwires might release torrents of green slime.

\subsubsection{Living Traps}

Some creatures are essentially traps, distinct only in that they have a Wisdom and Charisma score. The monstrous spider, the dire bear in a pit, and the golem are all classics, but the sky is really the limit here. Creatures can act like guard dogs if they are intelligent or magically controlled enough to tell friend from foe. Or they can act like punji sticks at the bottom of a pit if they are uncontrolled.

To be useful to a dungeon's occupants, a living trap has to be unable to turn on its masters. The occupants live in this place so any ``wandering monsters" had best be capable of discerning intruders from VIPs. Any monsters that can't make discernments like that need to be kept in cages or other inaccessible regions of the dungeon until someone specifically unleashes them in the event of a dungeon invasion. What this means for a dungeoneer is that successfully disguising yourself as a Dungeon Resident will keep the trained displacer beasts from attacking you. Furthermore, if you sneak into a dungeon, the untrainable creatures (monstrous vermin, ooze monsters, whatever) are all going to be locked up until an alarm gets sounded. A little discretion can make the dungeon environs a lot safer for the would-be raiders.

\subsubsection{Beneficial Traps}

Game mechanically, any localized triggered magical event is a ``trap". So if you whip out a room that heals everyone in it every round or an immobile pool that you can scry right out of, that's going to be a trap as far as the game is concerned. That means that the residents of a dungeon can shill out surprisingly small amounts of nuyen to get their pads to do all kinds of crazy stuff. Unlimited healing, permanent scrying pools, and more will be a fact of many rooms in virtually any dungeon. Moving these things is impractical, so ownership of a dungeon can be a very lucrative proposition

%%%%%%%%%%%%%%%%%%%%%%%%%%%%%%%%%%%%%%%%%%%%%%%%%%
\section{Wilderness}
%%%%%%%%%%%%%%%%%%%%%%%%%%%%%%%%%%%%%%%%%%%%%%%%%%

%%%%%%%%%%%%%%%%%%%%%%%%%
\subsection{Getting Lost}\index{Getting Lost}
%%%%%%%%%%%%%%%%%%%%%%%%%

There are many ways to get lost in the wilderness. Following an obvious road, trail, 
or feature such as a stream or shoreline prevents any possibility of becoming lost, 
but travelers striking off cross-country may become disoriented -- especially in 
conditions of poor visibility or in difficult terrain. 

\textbf{Poor Visibility:} Any time characters cannot see at least 60 feet in the 
prevailing conditions of visibility, they may become lost. Characters traveling 
through fog, snow, or a downpour might easily lose the ability to see any landmarks 
not in their immediate vicinity. Similarly, characters traveling at night may be 
at risk, too, depending on the quality of their light sources, the amount of moonlight, 
and whether they have darkvision or lowlight vision.

\textbf{Difficult Terrain:} Any character in forest, moor, hill, or mountain terrain 
may become lost if he or she moves away from a trail, road, stream, or other obvious 
path or track. Forests are especially dangerous because they obscure far-off landmarks 
and make it hard to see the sun or stars.

\textbf{Chance to Get Lost:} If conditions exist that make getting lost a possibility, 
the character leading the way must succeed on a \linkskill{Survival} check or become lost. 
The difficulty of this check varies based on the terrain, the visibility conditions, 
and whether or not the character has a map of the area being traveled through. 
Refer to the table below and use the highest DC that applies.

\begin{table}[htb]
\rowcolors{1}{white}{offyellow}
\caption{Survival DCs to avoid getting Lost}
\centering
\begin{tabular}{l c l c}
 & \textbf{Survival DC} & & \textbf{Survival DC}\\
Moor or hill (map) & 6 & Moor or hill (no map) & 10\\
Mountain (map) & 8 & Mountain (no map) & 12\\
Poor visibility & 12 & Forest & 15\\
\end{tabular}
\end{table}

A character with at least 5 ranks in \linkskill{Knowledge} (geography) or Knowledge (local) 
pertaining to the area being traveled through gains a +2 bonus on this check.

Check once per hour (or portion of an hour) spent in local or overland movement 
to see if travelers have become lost. In the case of a party moving together, only 
the character leading the way makes the check.

\textbf{Effects of Being Lost:} If a party becomes lost, it is no longer certain 
of moving in the direction it intended to travel. Randomly determine the direction 
in which the party actually travels during each hour of local or overland movement. 
The characters' movement continues to be random until they blunder into a landmark 
they can't miss, or until they recognize that they are lost and make an effort 
to regain their bearings.

\textit{Recognizing that You're Lost:} Once per hour of random travel, each character 
in the party may attempt a Survival check (DC 20, -1 per hour of random travel) 
to recognize that they are no longer certain of their direction of travel. Some 
circumstances may make it obvious that the characters are lost.

\textit{Setting a New Course:} A lost party is also uncertain of determining in 
which direction it should travel in order to reach a desired objective. Determining 
the correct direction of travel once a party has become lost requires a Survival 
check (DC 15, +2 per hour of random travel). If a character fails this check, he 
chooses a random direction as the "correct" direction for resuming travel.

Once the characters are traveling along their new course, correct or incorrect, 
they may get lost again. If the conditions still make it possible for travelers 
to become lost, check once per hour of travel as described in Chance to Get Lost, 
above, to see if the party maintains its new course or begins to move at random 
again.

\textit{Conflicting Directions:} It's possible that several characters may attempt 
to determine the right direction to proceed after becoming lost. Make a Survival 
check for each character in secret, then tell the players whose characters succeeded 
the correct direction in which to travel, and tell the players whose characters 
failed a random direction they think is right. 

\textbf{Regaining Your Bearings:} There are several ways to become un-lost. First, 
if the characters successfully set a new course and follow it to the destination 
they're trying to reach, they're not lost anymore. Second, the characters through 
random movement might run into an unmistakable landmark. Third, if conditions suddenly 
improve -- the fog lifts or the sun comes up -- lost characters may attempt to set 
a new course, as described above, with a +4 bonus on the Survival check. Finally, 
magic may make their course clear.

%%%%%%%%%%%%%%%%%%%%%%%%%
\subsection{Forest Terrain}
%%%%%%%%%%%%%%%%%%%%%%%%%

Forest terrain can be divided into three categories: sparse, medium, and dense. 
An immense forest could have all three categories within its borders, with more 
sparse terrain at the outer edge of the forest and dense forest at its heart. 

The table below describes in general terms how likely it is that a given square 
has a terrain element in it.

%%%
\subsubsection{Forest Terrain Features}
%%%

\begin{table}[htb]
\rowcolors{1}{white}{offyellow}
\caption{Random Forest Features}
\centering
\begin{tabular}{l c c c}
 & \multicolumn{3}{c}{\textbf{Forest Category}}\\
\textbf{Feature} & \textbf{Sparse} & \textbf{Medium} & \textbf{Dense}\\
Typical Trees & 50\% & 70\% & 80\%\\
Massive Trees & -- & 10\% & 20\%\\
Light Undergrowth & 50\% & 70\% & 50\%\\
Heavy Undergrowth& -- & 20\% & 50\%\\
\end{tabular}
\end{table}

\textbf{Trees:} The most important terrain element in a forest is the trees, obviously. 
A creature standing in the same square as a tree gains a +2 bonus to Armor Class 
and a +1 bonus on Reflex saves (these bonuses don't stack with cover bonuses from 
other sources). The presence of a tree doesn't otherwise affect a creature's fighting 
space, because it's assumed that the creature is using the tree to its advantage 
when it can. The trunk of a typical tree has AC 4, hardness 5, and 150 hp. A DC 
15 \linkskill{Climb} check is sufficient to climb a tree. Medium and dense forests have massive 
trees as well. These trees take up an entire square and provide cover to anyone 
behind them. They have AC 3, hardness 5, and 600 hp. Like their smaller counterparts, 
it takes a DC 15 Climb check to climb them.

\textbf{Undergrowth:} Vines, roots, and short bushes cover much of the ground in 
a forest. A space covered with light undergrowth costs 2 squares of movement to 
move into, and it provides concealment. Undergrowth increases the DC of \linkskill{Tumble} 
and \linkskill{Move Silently} checks by 2 because the leaves and branches get in the way. Heavy 
undergrowth costs 4 squares of movement to move into, and it provides concealment 
with a 30\% miss chance (instead of the usual 20\%). It increases the DC of Tumble 
and Move Silently checks by 5. Heavy undergrowth is easy to hide in, granting a 
+5 circumstance bonus on Hide checks. Running and charging are impossible. Squares 
with undergrowth are often clustered together. Undergrowth and trees aren't mutually 
exclusive; it's common for a 5-foot square to have both a tree and undergrowth.

\textbf{Forest Canopy:} It's common for elves and other forest dwellers to live 
on raised platforms far above the surface floor. These wooden platforms generally 
have rope bridges between them. To get to the treehouses, characters generally 
ascend the trees' branches (Climb DC 15), use rope ladders (Climb DC 0), or take 
pulley elevators (which can be made to rise a number of feet equal to a Strength 
check, made each round as a full-round action). Creatures on platforms or branches 
in a forest canopy are considered to have cover when fighting creatures on the 
ground, and in medium or dense forests they have concealment as well.

\textbf{Other Forest Terrain Elements:} Fallen logs generally stand about 3 feet 
high and provide cover just as low walls do. They cost 5 feet of movement to cross. 
Forest streams are generally 5 to 10 feet wide and no more than 5 feet deep. Pathways 
wind through most forests, allowing normal movement and providing neither cover 
nor concealment. These paths are less common in dense forests, but even unexplored 
forests will have occasional game trails.

\textbf{Stealth and Detection in a Forest:} In a sparse forest, the maximum distance 
at which a \linkskill{Spot} check for detecting the nearby presence of others can succeed is 
3d6x10 feet. In a medium forest, this distance is 2d8x10 
feet, and in a dense forest it is 2d6x10 feet.

Because any square with undergrowth provides concealment, it's usually easy for 
a creature to use the Hide skill in the forest. Logs and massive trees provide 
cover, which also makes hiding possible.

The background noise in the forest makes \linkskill{Listen} checks more difficult, increasing 
the DC of the check by 2 per 10 feet, not 1 (but note that Move Silently is also 
more difficult in undergrowth). 

%%%
\subsubsection{Forest Fires (CR 6)}
%%%

Most campfire sparks ignite nothing, but if conditions are dry, winds are strong, 
or the forest floor is dried out and flammable, a forest fire can result. Lightning 
strikes often set trees afire and start forest fires in this way. Whatever the 
cause of the fire, travelers can get caught in the conflagration.

A forest fire can be spotted from as far away as 2d6x100 feet 
by a character who makes a \linkskill{Spot} check, treating the fire as a Colossal creature 
(reducing the DC by 16). If all characters fail their Spot checks, the fire moves 
closer to them. They automatically see it when it closes to half the original distance.

Characters who are blinded or otherwise unable to make Spot checks can feel the 
heat of the fire (and thus automatically "spot" it) when it is 100 feet away.

The leading edge of a fire (the downwind side) can advance faster than a human 
can run (assume 120 feet per round for winds of moderate strength). Once a particular 
portion of the forest is ablaze, it remains so for 2d4x10 minutes 
before dying to a smoking smolder. Characters overtaken by a forest fire may find 
the leading edge of the fire advancing away from them faster than they can keep 
up, trapping them deeper and deeper in its grasp.

Within the bounds of a forest fire, a character faces three dangers: heat damage, 
catching on fire, and smoke inhalation. 

\textbf{Heat Damage:} Getting caught within a forest fire is even worse than being 
exposed to extreme heat (see Heat Dangers). Breathing the air causes a character 
to take 1d6 points of damage per round (no save). In addition, a character must 
make a Fortitude save every 5 rounds (DC 15, +1 per previous check) or take 1d4 
points of nonlethal damage. A character who holds his breath can avoid the lethal 
damage, but not the nonlethal damage. Those wearing heavy clothing or any sort 
of armor take a -4 penalty on their saving throws. In addition, those wearing metal 
armor or coming into contact with very hot metal are affected as if by a \linkspell{Heat Metal} spell.

\textbf{Catching on Fire:} Characters engulfed in a forest fire are at risk of 
catching on fire when the leading edge of the fire overtakes them, and are then 
at risk once per minute thereafter (see \linksec{Catching on Fire}).

\textbf{Smoke Inhalation:} Forest fires naturally produce a great deal of smoke. 
A character who breathes heavy smoke must make a Fortitude save each round (DC 
15, +1 per previous check) or spend that round choking and coughing. A character 
who chokes for 2 consecutive rounds takes 1d6 points of nonlethal damage. Also, 
smoke obscures vision, providing concealment to characters within it.

%%%%%%%%%%%%%%%%%%%%%%%%%
\subsection{Marsh Terrain}
%%%%%%%%%%%%%%%%%%%%%%%%%

Two categories of marsh exist: relatively dry moors and watery swamps. Both are 
often bordered by lakes (described in Aquatic Terrain, below), which effectively 
are a third category of terrain found in marshes.

The table below describes terrain features found in marshes.

%%%
\subsubsection{Marsh Terrain Features}
%%%

\begin{table}[htb]
\rowcolors{1}{white}{offyellow}
\caption{Random Marsh Features}
\centering
\begin{tabular}{l c c c}
 & \multicolumn{2}{c}{\textbf{Marsh Category}}\\
\textbf{Feature} & \textbf{Moor} & \textbf{Swamp}\\
Shallow bog & 20\% & 40\%\\
Deep bog & 5\% & 20\%\\
Light undergrowth & 30\% & 20\%\\
Heavy Undergrowth & 10\% & 20\%\\
\end{tabular}
\end{table}

\textbf{Bogs:} If a square is part of a shallow bog, it has deep mud or standing 
water of about 1 foot in depth. It costs 2 squares of movement to move into a square 
with a shallow bog, and the DC of \linkskill{Tumble} checks in such a square increases by 2. 

A square that is part of a deep bog has roughly 4 feet of standing water. It costs 
Medium or larger creatures 4 squares of movement to move into a square with a deep 
bog, or characters can swim if they wish. Small or smaller creatures must swim 
to move through a deep bog. Tumbling is impossible in a deep bog.

The water in a deep bog provides cover for Medium or larger creatures. Smaller 
creatures gain improved cover (+8 bonus to AC, +4 bonus on Reflex saves). Medium 
or larger creatures can crouch as a move action to gain this improved cover. Creatures 
with this improved cover take a -10 penalty on attacks against creatures that aren't 
underwater.

Deep bog squares are usually clustered together and surrounded by an irregular 
ring of shallow bog squares.

Both shallow and deep bogs increase the DC of \linkskill{Move Silently} checks by 2.

\textbf{Undergrowth:} The bushes, rushes, and other tall grasses in marshes function 
as undergrowth does in a forest (see above). A square that is part of a bog does 
not also have undergrowth. 

\textbf{Quicksand:} Patches of quicksand present a deceptively solid appearance 
(appearing as undergrowth or open land) that may trap careless characters. A character 
approaching a patch of quicksand at a normal pace is entitled to a DC 8 \linkskill{Survival} 
check to spot the danger before stepping in, but charging or running characters 
don't have a chance to detect a hidden bog before blundering in. A typical patch 
of quicksand is 20 feet in diameter; the momentum of a charging or running character 
carries him or her 1d2x5 feet into the quicksand.

\textit{Effects of Quicksand:} Characters in quicksand must make a DC 10 \linkskill{Swim} check 
every round to simply tread water in place, or a DC 15 Swim check to move 5 feet 
in whatever direction is desired. If a trapped character fails this check by 5 
or more, he sinks below the surface and begins to drown whenever he can no longer 
hold his breath (see the Swim skill description).

Characters below the surface of a bog may swim back to the surface with a successful 
Swim check (DC 15, +1 per consecutive round of being under the surface).

\textit{Rescue:} Pulling out a character trapped in quicksand can be difficult. 
A rescuer needs a branch, spear haft, rope, or similar tool that enables him to 
reach the victim with one end of it. Then he must make a DC 15 Strength check to 
successfully pull the victim, and the victim must make a DC 10 Strength check to 
hold onto the branch, pole, or rope. If the victim fails to hold on, he must make 
a DC 15 Swim check immediately to stay above the surface. If both checks succeed, 
the victim is pulled 5 feet closer to safety.

\textbf{Hedgerows:} Common in moors, hedgerows are tangles of stones, soil, and 
thorny bushes. Narrow hedgerows function as low walls, and it takes 15 feet of 
movement to cross them. Wide hedgerows are more than 5 feet tall and take up entire 
squares. They provide total cover, just as a wall does. It takes 4 squares of movement 
to move through a square with a wide hedgerow; creatures that succeed on a DC 10 
\linkskill{Climb} check need only 2 squares of movement to move through the square.

\textbf{Other Marsh Terrain Elements:} Some marshes, particularly swamps, have 
trees just as forests do, usually clustered in small stands. Paths lead across 
many marshes, winding to avoid bog areas. As in forests, paths allow normal movement 
and don't provide the concealment that undergrowth does.

\textbf{Stealth and Detection in a Marsh:} In a moor, the maximum distance at which 
a \linkskill{Spot} check for detecting the nearby presence of others can succeed is 6d6x10 
feet. In a swamp, this distance is 2d8x10 feet.

Undergrowth and deep bogs provide plentiful concealment, so it's easy to hide in 
a marsh.

A marsh imposes no penalties on \linkskill{Listen} checks, and using the \linkskill{Move Silently} skill 
is more difficult in both undergrowth and bogs.

%%%%%%%%%%%%%%%%%%%%%%%%%
\subsection{Hills Terrain}
%%%%%%%%%%%%%%%%%%%%%%%%%

A hill can exist in most other types of terrain, but hills can also dominate the 
landscape. Hills terrain is divided into two categories: gentle hills and rugged 
hills. Hills terrain often serves as a transition zone between rugged terrain such 
as mountains and flat terrain such as plains.

%%%
\subsubsection{Hills Terrain Features}
%%%

\begin{table}[htb]
\rowcolors{1}{white}{offyellow}
\caption{Random Hills Features}
\centering
\begin{tabular}{l c c c}
 & \multicolumn{2}{c}{\textbf{Hills Category}}\\
\textbf{Feature} & \textbf{Gentle Hill} & \textbf{Rugged Hill}\\
Gradual slope & 75\% & 40\%\\
Steep slope & 20\% & 50\%\\
Cliff & 5\% & 10\%\\
Light undergrowth & 15\% & 15\%\\
\end{tabular}
\end{table}

\textbf{Gradual Slope:} This incline isn't steep enough to affect movement, but 
characters gain a +1 bonus on melee attacks against foes downhill from them.

\textbf{Steep Slope:} Characters moving uphill (to an adjacent square of higher 
elevation) must spend 2 squares of movement to enter each square of steep slope. 
Characters running or charging downhill (moving to an adjacent square of lower 
elevation) must succeed on a DC 10 \linkskill{Balance} check upon entering the first steep 
slope square. Mounted characters make a DC 10 \linkskill{Ride} check instead. Characters who 
fail this check stumble and must end their movement 1d2x5 feet 
later. Characters who fail by 5 or more fall prone in the square where they end 
their movement. A steep slope increases the DC of Tumble checks by 2.

\textbf{Cliff:} A cliff typically requires a DC 15 \linkskill{Climb} check to scale and is 
1d4x10 feet tall, although the needs of your map may mandate 
a taller cliff. A cliff isn't perfectly vertical, taking up 5-foot squares if it's 
less than 30 feet tall and 10-foot squares if it's 30 feet or taller. 

\textbf{Light Undergrowth:} Sagebrush and other scrubby bushes grow on hills, athough 
they rarely cover the landscape as they do in forests and marshes. Light undergrowth 
provides concealment and increases the DC of \linkskill{Tumble} and \linkskill{Move Silently} checks by 
2. 

\textbf{Other Hills Terrain Elements:} Trees aren't out of place in hills terrain, 
and valleys often have active streams (5 to 10 feet wide and no more than 5 feet 
deep) or dry streambeds (treat as a trench 5 to 10 feet across) in them. If you 
add a stream or streambed, remember that water always flows downhill.

\textbf{Stealth and Detection in Hills:} In gentle hills, the maximum distance 
at which a \linkskill{Spot} check for detecting the nearby presence of others can succeed is 
2d10x10 feet. In rugged hills, this distance is 2d6x10 
feet.

Hiding in hills terrain can be difficult if there isn't undergrowth around. A hilltop 
or ridge provides enough cover to hide from anyone below the hilltop or ridge.

Hills don't affect \linkskill{Listen} or Move Silently checks. 

%%%%%%%%%%%%%%%%%%%%%%%%%
\subsection{Mountain Terrain}
%%%%%%%%%%%%%%%%%%%%%%%%%

The three mountain terrain categories are alpine meadows, rugged mountains, and 
forbidding mountains. As characters ascend into a mountainous area, they're likely 
to face each terrain category in turn, beginning with alpine meadows, extending 
through rugged mountains, and reaching forbidding mountains near the summit.

Mountains have an important terrain element, the rock wall, that is marked on the 
border between squares rather than taking up squares itself. 

%%%
\subsubsection{Mountain Terrain Features}
%%%

\begin{table}[htb]
\rowcolors{1}{white}{offyellow}
\caption{Random Mountain Features}
\centering
\begin{tabular}{l c c c}
 & \multicolumn{3}{c}{\textbf{Mountain Category}}\\
\textbf{Feature} & \textbf{Alpine Meadow} & \textbf{Rugged} & \textbf{Forbidding}\\
Gradual slope & 50\% & 25\% & 15\%\\
Steep slope & 40\% & 55\% & 55\%\\
Cliff & 10\% & 15\% & 20\%\\
Chasm & -- & 5\% & 10\%\\
Light undergrowth & 20\% & 10\% & --\\
Scree & -- & 20\% & 30\%\\
Dense rubble & -- & 20\% & 30\%\\
\end{tabular}
\end{table}

\textbf{Gradual and Steep Slopes:} These function as described in Hills Terrain, 
above.

\textbf{Cliff:} These terrain elements also function like their hills terrain counterparts, 
but they're typically 2d6x10 feet tall. Cliffs taller than 80 
feet take up 20 feet of horizontal space.

\textbf{Chasm:} Usually formed by natural geological processes, chasms function 
like pits in a dungeon setting. Chasms aren't hidden, so characters won't fall 
into them by accident (although bull rushes are another story). A typical chasm 
is 2d4x10 feet deep, at least 20 feet long, and anywhere from 
5 feet to 20 feet wide. It takes a DC 15 \linkskill{Climb} check to climb out of a chasm. In 
forbidding mountain terrain, chasms are typically 2d8x10 feet 
deep.

\textbf{Light Undergrowth:} This functions as described in Forest Terrain, above.

\textbf{Scree:} A field of shifting gravel, scree doesn't affect speed, but it 
can be treacherous on a slope. The DC of \linkskill{Balance} and \linkskill{Tumble} checks increases by 
2 if there's scree on a gradual slope and by 5 if there's scree on a steep slope. 
The DC of \linkskill{Move Silently} checks increases by 2 if the scree is on a slope of any 
kind.

\textbf{Dense Rubble:} The ground is covered with rocks of all sizes. It costs 
2 squares of movement to enter a square with dense rubble. The DC of Balance and 
Tumble checks on dense rubble increases by 5, and the DC of Move Silently checks 
increases by +2. 

\textbf{Rock Wall:} A vertical plane of stone, rock walls require DC 25 Climb checks 
to ascend. A typical rock wall is 2d4x10 feet tall in rugged 
mountains and 2d8x10 feet tall in forbidding mountains. Rock 
walls are drawn on the edges of squares, not in the squares themselves.

\textbf{Cave Entrance:} Found in cliff and steep slope squares and next to rock 
walls, cave entrances are typically between 5 and 20 feet wide and 5 feet deep. 
Beyond the entrance, a cave could be anything from a simple chamber to the entrance 
to an elaborate dungeon. Caves used as monster lairs typically have 1d3 rooms that 
are 1d4x10 feet across. 

\textbf{Other Mountain Terrain Features:} Most alpine meadows begin above the tree 
line, so trees and other forest elements are rare in the mountains. Mountain terrain 
can include active streams (5 to 10 feet wide and no more than 5 feet deep) and 
dry streambeds (treat as a trench 5 to 10 feet across). Particularly high-altitude 
areas tend to be colder than the lowland areas that surround them, so they may 
be covered in ice sheets (described below).

\textbf{Stealth and Detection in Mountains:} As a guideline, the maximum distance 
in mountain terrain at which a \linkskill{Spot} check for detecting the nearby presence of 
others can succeed is 4d10x10 feet. Certain peaks and ridgelines 
afford much better vantage points, of course, and twisting valleys and canyons 
have much shorter spotting distances. Because there's little vegetation to obstruct 
line of sight, the specifics on your map are your best guide for the range at which 
an encounter could begin. As in hills terrain, a ridge or peak provides enough 
cover to hide from anyone below the high point.

It's easier to hear faraway sounds in the mountains. The DC of \linkskill{Listen} checks increases 
by 1 per 20 feet between listener and source, not per 10 feet.

%%%
\subsubsection{Avalanches (CR 7)}
%%%

The combination of high peaks and heavy snowfalls means that avalanches are a deadly 
peril in many mountainous areas. While avalanches of snow and ice are common, it's 
also possible to have an avalanche of rock and soil.

An avalanche can be spotted from as far away as 1d10x500 feet 
downslope by a character who makes a DC 20 Spot check, treating the avalanche as 
a Colossal creature. If all characters fail their Spot checks to determine the 
encounter distance, the avalanche moves closer to them, and they automatically 
become aware of it when it closes to half the original distance. It's possible 
to hear an avalanche coming even if you can't see it. Under optimum conditions 
(no other loud noises occurring), a character who makes a DC 15 Listen check can 
hear the avalanche or landslide when it is 1d6x500 feet away. 
This check might have a DC of 20, 25, or higher in conditions where hearing is 
difficult (such as in the middle of a thunderstorm). 

A landslide or avalanche consists of two distinct areas: the bury zone (in the 
direct path of the falling debris) and the slide zone (the area the debris spreads 
out to encompass). Characters in the bury zone always take damage from the avalanche; 
characters in the slide zone may be able to get out of the way. Characters in the 
bury zone take 8d6 points of damage, or half that amount if they make a DC 15 Reflex 
save. They are subsequently buried (see below). Characters in the slide zone take 
3d6 points of damage, or no damage if they make a DC 15 Reflex save. Those who 
fail their saves are buried. 

Buried characters take 1d6 points of nonlethal damage per minute. If a buried character 
falls unconscious, he or she must make a DC 15 Constitution check or take 1d6 points 
of lethal damage each minute thereafter until freed or dead.

The typical avalanche has a width of 1d6x100 feet, from one edge 
of the slide zone to the opposite edge. The bury zone in the center of the avalanche 
is half as wide as the avalanche's full width.

To determine the precise location of characters in the path of an avalanche, roll 
1d6x20; the result is the number of feet from the center of the 
path taken by the bury zone to the center of the party's location. Avalanches of 
snow and ice advance at a speed of 500 feet per round, and rock avalanches travel 
at a speed of 250 feet per round.

%%%
\subsubsection{Mountain Travel}
%%%

High altitude can be extremely fatiguing -- or sometimes deadly -- to creatures that 
aren't used to it. Cold becomes extreme, and the lack of oxygen in the air can 
wear down even the most hardy of warriors.

\textbf{Acclimated Characters:} Creatures accustomed to high altitude generally 
fare better than lowlanders. Any creature with an Environment entry that includes 
mountains is considered native to the area, and acclimated to the high altitude. 
Characters can also acclimate themselves by living at high altitude for a month. 
Characters who spend more than two months away from the mountains must reacclimate 
themselves when they return. Undead, constructs, and other creatures that do not 
breathe are immune to altitude effects.

\textbf{Altitude Zones:} In general, mountains present three possible altitude 
bands: low pass, low peak/high pass, and high peak. 

\textit{Low Pass (lower than 5,000 feet):} Most travel in low mountains takes place 
in low passes, a zone consisting largely of alpine meadows and forests. Travelers 
may find the going difficult (which is reflected in the movement modifiers for 
traveling through mountains), but the altitude itself has no game effect.

\textit{Low Peak or High Pass (5,000 to 15,000 feet):} Ascending to the highest 
slopes of low mountains, or most normal travel through high mountains, falls into 
this category. All nonacclimated creatures labor to breathe in the thin air at 
this altitude. Characters must succeed on a Fortitude save each hour (DC 15, +1 
per previous check) or be fatigued. The fatigue ends when the character descends 
to an altitude with more air. Acclimated characters do not have to attempt the 
Fortitude save. 

\textit{High Peak (more than 15,000 feet):} The highest mountains exceed 20,000 
feet in height. At these elevations, creatures are subject to both high altitude 
fatigue (as described above) and altitude sickness, whether or not they're acclimated 
to high altitudes. Altitude sickness represents long-term oxygen deprivation, 
and it affects mental and physical ability scores. After each 6-hour period a character 
spends at an altitude of over 15,000 feet, he must succeed on a Fortitude save 
(DC 15, +1 per previous check) or take 1 point of damage to all ability scores. 
Creatures acclimated to high altitude receive a +4 competence bonus on their saving 
throws to resist high altitude effects and altitude sickness, but eventually even 
seasoned mountaineers must abandon these dangerous elevations. 

%%%%%%%%%%%%%%%%%%%%%%%%%
\subsection{Desert Terrain}
%%%%%%%%%%%%%%%%%%%%%%%%%

Desert terrain exists in warm, temperate, and cold climates, but all deserts share 
one common trait: little rain. The three categories of desert terrain are tundra 
(cold deserts), rocky desert (often temperate), and sandy desert (often warm).

Tundra differs from the other desert categories in two important ways. Because 
snow and ice cover much of the landscape, it's easy to find water. And during the 
height of summer, the permafrost thaws to a depth of a foot or so, turning the 
landscape into a vast field of mud. The muddy tundra affects movement and skill 
use as the shallow bogs described in marsh terrain, although there's little standing 
water.

The table above describes terrain elements found in each of the three desert categories. 
The terrain elements on this table are mutually exclusive; for instance, a square 
of tundra may contain either light undergrowth or an ice sheet, but not both.

%%%
\subsubsection{Desert Terrain Features}
%%%

\begin{table}[htb]
\rowcolors{1}{white}{offyellow}
\caption{Random Desert Features}
\centering
\begin{tabular}{l c c c}
 & \multicolumn{3}{c}{\textbf{Desert Category}}\\
\textbf{Feature} & \textbf{Tundra} & \textbf{Rocky} & \textbf{Sandy}\\
Light undergrowth & 15\% & 5\% & 5\%\\
Ice sheet & 25\% & -- & --\\
Light rubble & 5\% & 30\% & 10\%\\
Dense rubble & -- & 30\% & 5\%\\
Sand dunes & -- & -- & 50\%\\
\end{tabular}
\end{table}

\textbf{Light Undergrowth:} Consisting of scrubby, hardy bushes and cacti, light 
undergrowth functions as described for other terrain types.

\textbf{Ice Sheet:} The ground is covered with slippery ice. It costs 2 squares 
of movement to enter a square covered by an ice sheet, and the DC of \linkskill{Balance} and 
Tumble checks there increases by 5. A DC 10 Balance check is required to run or 
charge across an ice sheet. 

\textbf{Light Rubble:} Small rocks are strewn across the ground, making nimble 
movement more difficult more difficult. The DC of Balance and \linkskill{Tumble} checks increases 
by 2. 

\textbf{Dense Rubble:} This terrain feature consists of more and larger stones. 
It costs 2 squares of movement to enter a square with dense rubble. The DC of Balance 
and Tumble checks increases by 5, and the DC of \linkskill{Move Silently} checks increases 
by 2.

\textbf{Sand Dunes:} Created by the action of wind on sand, sand dunes function 
as hills that move. If the wind is strong and consistent, a sand dune can move 
several hundred feet in a week's time. Sand dunes can cover hundreds of squares. 
They always have a gentle slope pointing in the direction of the prevailing wind 
and a steep slope on the leeward side.

\textbf{Other Desert Terrain Features:} Tundra is sometimes bordered by forests, 
and the occasional tree isn't out of place in the cold wastes. Rocky deserts have 
towers and mesas consisting of flat ground surrounded on all sides by cliffs and 
steep slopes (described in Mountain Terrain, above). Sandy deserts sometimes have 
quicksand; this functions as described in Marsh Terrain, above, although desert 
quicksand is a waterless mixture of fine sand and dust. All desert terrain is crisscrossed 
with dry streambeds (treat as trenches 5 to 15 feet wide) that fill with water 
on the rare occasions when rain falls.

\textbf{Stealth and Detection in the Desert:} In general, the maximum distance 
in desert terrain at which a \linkskill{Spot} check for detecting the nearby presence of others 
can succeed is 6d6x20 feet; beyond this distance, elevation changes 
and heat distortion in warm deserts makes spotting impossible. The presence of 
dunes in sandy deserts limits spotting distance to 6d6x10 feet. 

The desert imposes neither bonuses nor penalties on \linkskill{Listen} or Spot checks. The 
scarcity of undergrowth or other elements that offer concealment or cover makes 
hiding more difficult.

%%%
\subsubsection{Sandstorms}
%%%

A sandstorm reduces visibility to 1d10x5 feet and provides a 
-4 penalty on Listen, Search, and Spot checks. A sandstorm deals 1d3 points of 
nonlethal damage per hour to any creatures caught in the open, and leaves a thin 
coating of sand in its wake. Driving sand creeps in through all but the most secure 
seals and seams, to chafe skin and contaminate carried gear. 

%%%%%%%%%%%%%%%%%%%%%%%%%
\subsection{Plains Terrain}
%%%%%%%%%%%%%%%%%%%%%%%%%

Plains come in three categories: farms, grasslands, and battlefields. Farms are 
common in settled areas, of course, while grasslands represent untamed plains. 
The battlefields where large armies clash are temporary places, usually reclaimed 
by natural vegetation or the farmer's plow. Battlefields represent a third terrain 
category because adventurers tend to spend a lot of time there, not because they're 
particularly prevalent.

The table below shows the proportions of terrain elements in the different categories 
of plains. On a farm, light undergrowth represents most mature grain crops, so 
farms growing vegetable crops will have less light undergrowth, as will all farms 
during the time between harvest and a few months after planting.

The terrain elements in the table below are mutually exclusive.

%%%
\subsubsection{Plains Terrain Features}
%%%

\begin{table}[htb]
\rowcolors{1}{white}{offyellow}
\caption{Random Plains Features}
\centering
\begin{tabular}{l c c c}
 & \multicolumn{3}{c}{\textbf{Plains Category}}\\
\textbf{Feature} & \textbf{Farm} & \textbf{Grassland} & \textbf{Battlefield}\\
Light undergrowth & 40\% & 20\% & 10\%\\
Heavy undergrowth & -- & 10\% & --\\
Light rubble & -- & -- & 10\%\\
Trench & 5\% & -- & 5\%\\
Berm & -- & -- & 5\%\\
\end{tabular}
\end{table}

\textbf{Undergrowth:} Whether they're crops or natural vegetation, the tall grasses 
of the plains function like light undergrowth in a forest. Particularly thick bushes 
form patches of heavy undergrowth that dot the landscape in grasslands.

\textbf{Light Rubble:} On the battlefield, light rubble usually represents something 
that was destroyed: the ruins of a building or the scattered remnants of a stone 
wall, for example. It functions as described in the desert terrain section above.

\textbf{Trench:} Often dug before a battle to protect soldiers, a trench functions 
as a low wall, except that it provides no cover against adjacent foes. It costs 
2 squares of movement to leave a trench, but it costs nothing extra to enter one. 
Creatures outside a trench who make a melee attack against a creature inside the 
trench gain a +1 bonus on melee attacks because they have higher ground. In farm 
terrain, trenches are generally irrigation ditches.

\textbf{Berm:} A common defensive structure, a berm is a low, earthen wall that 
slows movement and provides a measure of cover. Put a berm on the map by drawing 
two adjacent rows of steep slope (described in Hills Terrain, above), with the 
edges of the berm on the downhill side. Thus, a character crossing a two-square 
berm will travel uphill for 1 square, then downhill for 1 square. Two square berms 
provide cover as low walls for anyone standing behind them. Larger berms provide 
the low wall benefit for anyone standing 1 square downhill from the top of the 
berm. 

\textbf{Fences:} Wooden fences are generally used to contain livestock or impede 
oncoming soldiers. It costs an extra square of movement to cross a wooden fence. 
A stone fence provides a measure of cover as well, functioning as low walls. Mounted 
characters can cross a fence without slowing their movement if they succeed on 
a DC 15 Ride check. If the check fails, the steed crosses the fence, but the rider 
falls out of the saddle.

\textbf{Other Plains Terrain Features:} Occasional trees dot the landscape in many 
plains, although on battlefields they're often felled to provide raw material for 
siege engines (described in Urban Features). Hedgerows (described in Marsh Terrain) 
are found in plains as well. Streams, generally 5 to 20 feet wide and 5 to 10 feet 
deep, are commonplace.

\textbf{Stealth and Detection in Plains:} In plains terrain, the maximum distance 
at which a \linkskill{Spot} check for detecting the nearby presence of others can succeed is 
6d6x40 feet, although the specifics of your map may restrict 
line of sight. Plains terrain provides no bonuses or penalties on \linkskill{Listen} and Spot 
checks. Cover and concealment are not uncommon, so a good place of refuge is often 
nearby, if not right at hand.

%%%%%%%%%%%%%%%%%%%%%%%%%
\subsection{Aquatic Terrain}
%%%%%%%%%%%%%%%%%%%%%%%%%

Aquatic terrain is the least hospitable to most PCs, because they can't breathe 
there. Aquatic terrain doesn't offer the variety that land terrain does. The ocean 
floor holds many marvels, including undersea analogues of any of the terrain elements 
described earlier in this section. But if characters find themselves in the water 
because they were bull rushed off the deck of a pirate ship, the tall kelp beds 
hundreds of feet below them don't matter. Accordingly, these rules simply divide 
aquatic terrain into two categories: flowing water (such as streams and rivers) 
and nonflowing water (such as lakes and oceans).

\textbf{Flowing Water:} Large, placid rivers move at only a few miles per hour, 
so they function as still water for most purposes. But some rivers and streams 
are swifter; anything floating in them moves downstream at a speed of 10 to 40 
feet per round. The fastest rapids send swimmers bobbing downstream at 60 to 90 
feet per round. Fast rivers are always at least rough water (\linkskill{Swim} DC 15), and whitewater 
rapids are stormy water (Swim DC 20). If a character is in moving water, move her 
downstream the indicated distance at the end of her turn. A character trying to 
maintain her position relative to the riverbank can spend some or all of her turn 
swimming upstream.

\textit{Swept Away:} Characters swept away by a river moving 60 feet per round 
or faster must make DC 20 Swim checks every round to avoid going under. If a character 
gets a check result of 5 or more over the minimum necessary, he arrests his motion 
by catching a rock, tree limb, or bottom snag -- he is no longer being carried along 
by the flow of the water. Escaping the rapids by reaching the bank requires three 
DC 20 Swim checks in a row. Characters arrested by a rock, limb, or snag can't 
escape under their own power unless they strike out into the water and attempt 
to swim their way clear. Other characters can rescue them as if they were trapped 
in quicksand (described in Marsh Terrain, above). 

\textbf{Non-Flowing Water:} Lakes and oceans simply require a swim speed or successful 
Swim checks to move through (DC 10 in calm water, DC 15 in rough water, DC 20 in 
stormy water). Characters need a way to breathe if they're underwater; failing 
that, they risk drowning. When underwater, characters can move in any direction 
as if they were flying with perfect maneuverability.

\textbf{Stealth and Detection Underwater:} How far you can see underwater depends 
on the water's clarity. As a guideline, creatures can see 4d8x10 
feet if the water is clear, and 1d8x10 feet if it's murky. Moving 
water is always murky, unless it's in a particularly large, slow-moving river.

It's hard to find cover or concealment to hide underwater (except along the seafloor). 
Listen and Move Silently checks function normally underwater.

\textit{Invisibility:} An invisible creature displaces water and leaves a visible, 
body-shaped "bubble" where the water was displaced. The creature still has concealment 
(20\% miss chance), but not total concealment (50\% miss chance).

%%%
\subsubsection{Underwater Combat}
%%%

Land-based creatures can have considerable difficulty when fighting in water. Water 
affects a creature's Armor Class, attack rolls, damage, and movement. In some cases 
a creature's opponents may get a bonus on attacks. The effects are summarized in 
the accompanying table. They apply whenever a character is swimming, walking in 
chest-deep water, or walking along the bottom. 

\textbf{Ranged Attacks Underwater:} Thrown weapons are ineffective underwater, 
even when launched from land. Attacks with other ranged weapons take a -2 penalty 
on attack rolls for every 5 feet of water they pass through, in addition to the 
normal penalties for range. 

\textbf{Attacks from Land:} Characters swimming, floating, or treading water on 
the surface, or wading in water at least chest deep, have improved cover (+8 bonus 
to AC, +4 bonus on Reflex saves) from opponents on land. Landbound opponents who 
have \linkspell{Freedom of Movement} effects ignore this cover when making melee attacks 
against targets in the water. A completely submerged creature has total cover against 
opponents on land unless those opponents have \textit{Freedom of Movement} effects. 
Magical effects are unaffected except for those that require attack rolls (which 
are treated like any other effects) and fire effects.

\textbf{Fire:} Nonmagical fire (including alchemist's fire) does not burn underwater. 
Spells or spell-like effects with the fire descriptor are ineffective underwater 
unless the caster makes a \linkskill{Spellcraft} check (DC 20 + spell level). If the check 
succeeds, the spell creates a bubble of steam instead of its usual fiery effect, 
but otherwise the spell works as described. A supernatural fire effect is ineffective 
underwater unless its description states otherwise. The surface of a body of water 
blocks line of effect for any fire spell. If the caster has made a Spellcraft check 
to make the fire spell usable underwater, the surface still blocks the spell's 
line of effect.

\begin{table}[htb]
\rowcolors{1}{white}{offyellow}
\caption{Combat Adjustments Underwater}
\centering
\begin{tabular}{l c c c c}
& \multicolumn{2}{c}{\textbf{Attack / Damage}} & &\\
\textbf{Condition} & \textbf{Slashing or Bludgeoning} & \textbf{Tail} & \textbf{Movement} & \textbf{Off Balance?\textsuperscript{4}}\\
\textit{Freedom of Movement} & normal/normal & normal/normal & normal & No\\
Has a Swim Speed & -2 / half & normal & normal & No\\
Successful Swim Check & -2 / half\textsuperscript{1} & -2 / half & quarter or half\textsuperscript{2} & No\\
Firm footing\textsuperscript{3} & -2 / half & -2 / half & half & No\\
None of the above & -2 / half & -2 / half & normal & Yes\\
\multicolumn{5}{p{16cm}}{\textsuperscript{1} A creature without a \textit{Freedom of Movement} effect or a swim speed makes grapple checks underwater at a -2 penalty, but deals damage normally when grappling.}\\
\multicolumn{5}{p{16cm}}{\textsuperscript{2} A successful Swim check lets a creature move one-quarter its speed as a move action, or one-half its speed as a full-round action.}\\
\multicolumn{5}{p{16cm}}{\textsuperscript{3} Creatures have firm footing when walking along the bottom, braced againat a ship's hull, or the like. A Creature can only walk along the bottom of it wears or carried enough gear to weigh itself down -- at least 16 pounds for Medium creatures, twice that for each Size larger, half that for each Size smaller.}\\
\multicolumn{5}{p{16cm}}{\textsuperscript{4} Creatures flailing about in the water (usually because they failed their Swim checks) have a hard time fighting effectively. An off-balance creature loses its Dexterity bonus to AC, and opponents gain a +2 bonus on atacks against it.}\\
\end{tabular}
\end{table}

%%%
\subsubsection{Floods}
%%%

In many wilderness areas, river floods are a common occurrence.

In spring, an enormous snowmelt can engorge the streams and rivers it feeds. Other 
catastrophic events such as massive rainstorms or the destruction of a dam can 
create floods as well.

During a flood, rivers become wider, deeper, and swifter. Assume that a river rises 
by 1d10+10 feet during the spring flood, and its width increases by a factor of 
1d4x50\%. Fords may disappear for days, bridges may be swept 
away, and even ferries might not be able to manage the crossing of a flooded river. 
A river in flood makes Swim checks one category harder (calm water becomes rough, 
and rough water becomes stormy). Rivers also become 50\% swifter.

%%%%%%%%%%%%%%%%%%%%%%%%%%%%%%%%%%%%%%%%%%%%%%%%%%
\section{Urban Adventures}
%%%%%%%%%%%%%%%%%%%%%%%%%%%%%%%%%%%%%%%%%%%%%%%%%%

At first glance, a city is much like a dungeon, made up of walls, doors, rooms, 
and corridors. Adventures that take place in cities have two salient differences 
from their dungeon counterparts, however. Characters have greater access to resources, 
and they must contend with law enforcement.

\textbf{Access to Resources:} Unlike in dungeons and the wilderness, characters 
can buy and sell gear quickly in a city. A large city or metropolis probably has 
high-level NPCs and experts in obscure fields of knowledge who can provide assistance 
and decipher clues. And when the PCs are battered and bruised, they can retreat 
to the comfort of a room at the inn.

The freedom to retreat and ready access to the marketplace means that the players 
have a greater degree of control over the pacing of an urban adventure.

\textbf{Law Enforcement:}\index{Law Enforcement} The other key distinctions between adventuring in a city 
and delving into a dungeon is that a dungeon is, almost by definition, a lawless 
place where the only law is that of the jungle: Kill or be killed. A city, on the 
other hand, is held together by a code of laws, many of which are explicitly designed 
to prevent the sort of behavior that adventurers engage in all the time: killing 
and looting. Even so, most cities' laws recognize monsters as a threat to the stability 
the city relies on, and prohibitions about murder rarely apply to monsters such 
as aberrations or evil outsiders. Most evil humanoids, however, are typically protected 
by the same laws that protect all the citizens of the city. Having an evil alignment 
is not a crime (except in some severely theocratic cities, perhaps, with the magical 
power to back up the law); only evil deeds are against the law. Even when adventurers 
encounter an evildoer in the act of perpetrating some heinous evil upon the populace 
of the city, the law tends to frown on the sort of vigilante justice that leaves 
the evildoer dead or otherwise unable to testify at a trial.

\textbf{Weapon And Spell Restrictions:} Different cities have different laws about such issues as carrying weapons in public 
and restricting spellcasters. The city's laws may not affect all characters equally. A monk isn't hampered at 
all by a law about peace-bonding weapons, but a cleric is reduced to a fraction 
of his power if all holy symbols are confiscated at the city's gates.

%%%%%%%%%%%%%%%%%%%%%%%%%
\subsection{Urban Features}
%%%%%%%%%%%%%%%%%%%%%%%%%

Walls, doors, poor lighting, and uneven footing: In many ways a city is much like 
a dungeon. Some new considerations for an urban setting are covered below.

%%%
\subsubsection{Walls and Gates}
%%%

Many cities are surrounded by walls. A typical small city wall is a fortified stone 
wall 5 feet thick and 20 feet high. Such a wall is fairly smooth, requiring a DC 
30 \linkskill{Climb} check to scale. The walls are crenellated on one side to provide a low 
wall for the guards atop it, and there is just barely room for guards to walk along 
the top of the wall. A typical small city wall has AC 3, hardness 8, and 450 hp 
per 10-foot section.

A typical large city wall is 10 feet thick and 30 feet high, with crenellations 
on both sides for the guards on top of the wall. It is likewise smooth, requiring 
a DC 30 Climb check to scale. Such a wall has AC 3, hardness 8, and 720 hp per 
10-foot section.

A typical metropolis wall is 15 feet thick and 40 feet tall. It has crenellations 
on both sides and often has a tunnel and small rooms running through its interior. 
Metropolis walls have AC 3, hardness 8, and 1,170 hp per 10-foot section.

Unlike smaller cities, metropolises often have interior walls as well as surrounding 
walls -- either old walls that the city has outgrown, or walls dividing individual 
districts from each other. Sometimes these walls are as large and thick as the 
outer walls, but more often they have the characteristics of a large city's or 
small city's walls.

\textbf{Watch Towers:} Some city walls are adorned with watch towers set at irregular 
intervals. Few cities have enough guards to keep someone constantly stationed at 
every tower, unless the city is expecting attack from outside. The towers provide 
a superior view of the surrounding countryside as well as a point of defense against 
invaders.

Watch towers are typically 10 feet higher than the wall they adjoin, and their 
diameter is 5 times the thickness of the wall. Arrow slits line the outer sides 
of the upper stories of a tower, and the top is crenellated like the surrounding 
walls are. In a small tower (25 feet in diameter adjoining a 5-foot-thick wall), 
a simple ladder typically connect the tower's stories and the roof. In a larger 
tower, stairs serve that purpose. 

Heavy wooden doors, reinforced with iron and bearing good locks (\linkskill{Open Lock} DC 30), 
block entry to a tower, unless the tower is in regular use. As a rule, the captain 
of the guard keeps the key to the tower secured on her person, and a second copy 
is in the city's inner fortress or barracks.

\textbf{Gates:} A typical city gate is a gatehouse with two portcullises and murder 
holes above the space between them. In towns and some small cities, the primary 
entry is through iron double doors set into the city wall.

Gates are usually open during the day and locked or barred at night. Usually, one 
gate lets in travelers after sunset and is staffed by guards who will open it for 
someone who seems honest, presents proper papers, or offers a large enough bribe 
(depending on the city and the guards).

%%%
\subsubsection{Guards and Soldiers}
%%%

A city typically has full-time military personnel equal to 1\% of its adult population, 
in addition to militia or conscript soldiers equal to 5\% of the population. The 
full-time soldiers are city guards responsible for maintaining order within the 
city, similar to the role of modern police, and (to a lesser extent) for defending 
the city from outside assault. Conscript soldiers are called up to serve in case 
of an attack on the city.

A typical city guard force works on three eight-hour shifts, with 30\% of the force 
on a day shift (8am to 4pm), 
35\% on an evening shift (4pm to 12am), 
and 35\% on a night shift (12am to 8am). 
At any given time, 80\% of the guards on duty are on the streets patrolling, while 
the remaining 20\% are stationed at various posts throughout the city, where they 
can respond to nearby alarms. At least one such guard post is present within each 
neighborhood of a city (each neighborhood consisting of several districts).

The majority of a city guard force is made up of warriors, mostly 1st level. Officers 
include higher-level warriors, fighters, a fair number of clerics, and wizards 
or sorcerers, as well as multiclass fighter/spellcasters.

%%%
\subsubsection{Siege Engines}
%%%

Siege engines are large weapons, temporary structures, or pieces of equipment traditionally 
used in besieging a castle or fortress.

\begin{table}[htb]
\rowcolors{1}{white}{offyellow}
\caption{Seige Engines}
\centering
\begin{tabular}{l r c c c c}
\textbf{Item} & \textbf{Cost} & \textbf{Damage} & \textbf{Critical} & \textbf{Range Increment} & \textbf{Typical Crew}\\
Ballista & 500gp & 3d8 & 19-20 & 120ft & 1\\
Heavy Catapult & 800gp & 6d6 & -- & 200ft (100ft minimum) & 4\\
Light Catapult & 550gp & d6 & -- & 150ft (100ft minimum) & 2\\
Ram & 1,000gp & 3d6\textsuperscript{*} & -- & -- & 10\\
Seige Tower & 2,000gp & -- & -- & -- & 20\\
\multicolumn{6}{l}{\textsuperscript{*} See description for special rules.}\\
\end{tabular}
\end{table}

\begin{table}[htb]
\rowcolors{1}{white}{offyellow}
\caption{Catapult Attack Modifiers}
\centering
\begin{tabular}{p{10cm} l}
\textbf{Condition} & \textbf{Modifier}\\
No line of sight to target square & -6\\
Successive Shots (crew can see where most recent misses landed) & Cumulative +2 per miss (Max +10)\\
Successive Shots (crew can't see where most recent miss landed, but observer is providing feedback) & Cumulative +1 per previous miss (Max +5)\\
\end{tabular}
\end{table}

\textbf{Ballista:} A ballista is essentially a Huge heavy crossbow fixed in place. 
Its size makes it hard for most creatures to aim it. Thus, a Medium creature 
takes a -4 penalty on attack rolls when using a ballista, and a Small creature 
takes a -6 penalty. It takes a creature smaller than Large two full-round actions 
to reload the ballista after firing.

A ballista takes up a space 5 feet across.

\textbf{Heavy Catapult:} A heavy catapult is a massive engine capable of throwing 
rocks or heavy objects with great force. Because the catapult throws its payload 
in a high arc, it can hit squares out of its line of sight. To fire a heavy catapult, 
the crew chief makes a special check against DC 15 using only his base attack bonus, 
Intelligence modifier, range increment penalty, and the appropriate modifiers from 
the lower section of Table 3-26. If the check succeeds, the catapult stone hits 
the square the catapult was aimed at, dealing the indicated damage to any object 
or character in the square. Characters who succeed on a DC 15 Reflex save take 
half damage. Once a catapult stone hits a square, subsequent shots hit the same 
square unless the catapult is re-aimed or the wind changes direction or speed.

If a catapult stone misses, roll 1d8 to determine where it lands. This determines 
the misdirection of the throw, with 1 being back toward the catapult and 2 through 
8 counting clockwise around the target square. Then, count 3 squares away from 
the target square for every range increment of the attack.

Loading a catapult requires a series of full-round actions. It takes a DC 15 Strength 
check to winch the throwing arm down; most catapults have wheels to allow up to 
two crew members to use the aid another action, assisting the main winch operator. 
A DC 15 Profession (siege engineer) check latches the arm into place, and then 
another DC 15 Profession (siege engineer) check loads the catapult ammunition. 
It takes four full-round actions to re-aim a heavy catapult (multiple crew members 
can perform these full-round actions in the same round, so it would take a crew 
of four only 1 round to re-aim the catapult).

A heavy catapult takes up a space 15 feet across.

\textbf{Light Catapult:} This is a smaller, lighter version of the heavy catapult. 
It functions as the heavy catapult, except that it takes a DC 10 Strength check 
to winch the arm into place, and only two full-round actions are required to re-aim 
the catapult.

A light catapult takes up a space 10 feet across.

\textbf{Ram:} This heavy pole is sometimes suspended from a movable scaffold that 
allows the crew to swing it back and forth against objects. As a full-round action, 
the character closest to the front of the ram makes an attack roll against the 
AC of the construction, applying the -4 penalty for lack of proficiency. (It's 
not possible to be proficient with this device.) In addition to the damage given 
on Table: Siege Engines, up to nine other characters holding the ram can add their 
Strength modifier to the ram's damage, if they devote an attack action to doing 
so. It takes at least one Huge or larger creature, two Large creatures, four Medium-size 
creatures, or eight Small creatures to swing a ram. (Tiny or smaller creatures 
can't use a ram.)

A ram is typically 30 feet long. In a battle, the creatures wielding the ram stand 
in two adjacent columns of equal length, with the ram between them.

\textbf{Siege Tower:} This device is a massive wooden tower on wheels or rollers 
that can be rolled up against a wall to allow attackers to scale the tower and 
thus to get to the top of the wall with cover. The wooden walls are usually 1 foot 
thick.

A typical siege tower takes up a space 15 feet across. The creatures inside push 
it at a speed of 10 feet (and a siege tower can't run). The eight creatures pushing 
on the ground floor have total cover, and those on higher floors get improved cover 
and can fire through arrow slits.

%%%
\subsubsection{City Streets}
%%%

Typical city streets are narrow and twisting. Most streets average 15 to 20 feet 
wide [(1d4+1)x5 feet)], while alleys range from 10 feet wide 
to only 5 feet. Cobblestones in good condition allow normal movement, but ones 
in poor repair and heavily rutted dirt streets are considered light rubble, increasing 
the DC of \linkskill{Balance} and \linkskill{Tumble} checks by 2.

Some cities have no larger thoroughfares, particularly cities that gradually grew 
from small settlements to larger cities. Cities that are planned, or perhaps have 
suffered a major fire that allowed authorities to construct new roads through formerly 
inhabited areas, might have a few larger streets through town. These main roads 
are 25 feet wide -- offering room for wagons to pass each other -- with 5-foot-wide 
sidewalks on either side.

\textbf{Crowds:} Urban streets are often full of people going about their daily 
lives. In most cases, it isn't necessary to put every 1st-level commoner on the 
map when a fight breaks out on the city's main thoroughfare. Instead just indicate 
which squares on the map contain crowds. If crowds see something obviously dangerous, 
they'll move away at 30 feet per round at initiative count 0. It takes 2 squares 
of movement to enter a square with crowds. The crowds provide cover for anyone 
who does so, enabling a Hide check and providing a bonus to Armor Class and on 
Reflex saves.

\textit{Directing Crowds:} It takes a DC 15 \linkskill{Diplomacy} check or DC 20 \linkskill{Intimidate} 
check to convince a crowd to move in a particular direction, and the crowd must 
be able to hear or see the character making the attempt. It takes a full-round 
action to make the Diplomacy check, but only a free action to make the Intimidate 
check.

If two or more characters are trying to direct a crowd in different directions, 
they make opposed Diplomacy or Intimidate checks to determine whom the crowd listens 
to. The crowd ignores everyone if none of the characters' check results beat the 
DCs given above.

%%%
\subsubsection{Above and beneath the Streets}
%%%

\textbf{Rooftops:} Getting to a roof usually requires climbing a wall (see the 
Walls section), unless the character can reach a roof by jumping down from a higher 
window, balcony, or bridge. Flat roofs, common only in warm climates (accumulated 
snow can cause a flat roof to collapse), are easy to run across. Moving along the 
peak of a roof requires a DC 20 Balance check. Moving on an angled roof surface 
without changing altitude (moving parallel to the peak, in other words) requires 
a DC 15 Balance check. Moving up and down across the peak of a roof requires a 
DC 10 Balance check.

Eventually a character runs out of roof, requiring a long jump across to the next 
roof or down to the ground. The distance to the next closest roof is usually 1d3x5 
feet horizontally, but the roof across the gap is equally likely to be 5 feet higher, 
5 feet lower, or the same height. Use the guidelines in the \linkskill{Jump} skill 
(a horizontal jump's peak height is one-fourth of the horizontal distance) to 
determine whether a character can make a jump.

\textbf{Sewers:} To get into the sewers, most characters open a grate (a full-round 
action) and jump down 10 feet. Sewers are built exactly like dungeons, except that 
they're much more likely to have floors that are slippery or covered with water. 
Sewers are also similar to dungeons in terms of creatures liable to be encountered 
therein. Some cities were built atop the ruins of older civilizations, so their 
sewers sometimes lead to treasures and dangers from a bygone age.

%%%
\subsubsection{City Buildings}
%%%

Most city buildings fall into three categories. The majority of buildings in the 
city are two to five stories high, built side by side to form long rows separated 
by secondary or main streets. These row houses usually have businesses on the ground 
floor, with offices or apartments above.

Inns, successful businesses, and large warehouses -- as well as millers, tanners, 
and other businesses that require extra space -- are generally large, free-standing 
buildings with up to five stories. 

Finally, small residences, shops, warehouses, or storage sheds are simple, one-story 
wooden buildings, especially if they're in poorer neighborhoods.

Most city buildings are made of a combination of stone or clay brick (on the lower 
one or two stories) and timbers (for the upper stories, interior walls, and floors). 
Roofs are a mixture of boards, thatch, and slates, sealed with pitch. A typical 
lower-story wall is 1 foot thick, with AC 3, hardness 8, 90 hp, and a Climb DC 
of 25. Upper-story walls are 6 inches thick, with AC 3, hardness 5, 60 hp, and 
a \linkskill{Climb} DC of 21. Exterior doors on most buildings are good wooden doors that are 
usually kept locked, except on public buildings such as shops and taverns.

%%%%%%%%%%%%%%%%%%%%%%%%%
\subsection{Buying Buildings}
%%%%%%%%%%%%%%%%%%%%%%%%%

Characters might want to buy their own buildings or even construct

their own castle. Use the prices in Table: Buildings directly, or as a guide when 
for extrapolating costs for more exotic structures.

\begin{table}[htb]
\rowcolors{1}{white}{offyellow}
\caption{Building Prices}
\centering
\begin{tabular}{l r}
\textbf{Building} & \textbf{Cost}\\
Simple House & 1,000gp\\
Grand House & 5,000gp\\
Mansion & 100,000gp\\
Tower & 50,000gp\\
Keep & 150,000gp\\
Castle & 500,000gp\\
Huge Castle & 1,000,000gp\\
Moat with Bridge & 50,000gp\\
\end{tabular}
\end{table}

\textit{Simple House:} This one- to three-room house is made of wood and has a 
thatched roof.

\textit{Grand House:} This four- to ten-room house is made of wood and has a thatched 
roof.

\textit{Mansion:} This ten- to twenty-room residence has two or three stories and 
is made of wood and brick. It has a slate roof.

\textit{Tower:} This round or square, three-level tower is made of stone.

\textit{Keep:} This fortified stone building has fifteen to twenty-five rooms.

\textit{Castle:} A castle is a keep surrounded by a 15-foot stone wall with four 
towers. The wall is 10 feet thick.

\textit{Huge Castle:} A huge castle is a particularly large keep with numerous 
associated buildings (stables, forge, granaries, and so on) and an elaborate 20-foot-high 
wall that creates bailey and courtyard areas. The wall has six towers and is 10 
feet thick.

\textit{Moat with Bridge:} The moat is 15 feet deep and 30 feet wide. The bridge 
may be a wooden drawbridge or a permanent stone structure.

%%%
\subsubsection{City Lights}
%%%

If a city has main thoroughfares, they are lined with lanterns hanging at a height 
of 7 feet from building awnings. These lanterns are spaced 60 feet apart, so their 
illumination is all but continuous. Secondary streets and alleys are not lit; it 
is common for citizens to hire lantern-bearers when going out after dark.

Alleys can be dark places even in daylight, thanks to the shadows of the tall buildings 
that surround them. A dark alley in daylight is rarely dark enough to afford true 
concealment, but it can lend a +2 circumstance bonus on \linkskill{Hide} checks.

%%%%%%%%%%%%%%%%%%%%%%%%%%%%%%%%%%%%%%%%%%%%%%%%%%
\section{Weather}
%%%%%%%%%%%%%%%%%%%%%%%%%%%%%%%%%%%%%%%%%%%%%%%%%%

Sometimes weather can play an important role in an adventure.

Table: Random Weather is an appropriate weather table for general use, and can 
be used as a basis for a local weather tables. Terms on that table are defined 
as follows.

\textbf{Calm:} Wind speeds are light (0 to 10 mph).

\textbf{Cold:} Between 0\textdegree{} and 40\textdegree{} Fahrenheit during the day, 10 to 20 degrees 
colder at night.

\textbf{Cold Snap:} Lowers temperature by -10\textdegree{} F.

\textbf{Downpour:} Treat as rain (see Precipitation, below), but conceals as fog. 
Can create floods (see above). A downpour lasts for 2d4 hours.

\textbf{Heat Wave:} Raises temperature by +10\textdegree{} F.

\textbf{Hot:} Between 85\textdegree{} and 110\textdegree{} Fahrenheit during the day, 10 to 20 degrees 
colder at night.

\textbf{Moderate:} Between 40\textdegree{} and 60\textdegree{} Fahrenheit during the day, 10 to 20 degrees 
colder at night.

\textbf{Powerful Storm (Windstorm/Blizzard/Hurricane/Tornado):}
Wind speeds are over 50 mph (see Table: Wind Effects). In addition, blizzards 
are accompanied by heavy snow (1d3 feet), and hurricanes are accompanied by downpours 
(see above). Windstorms last for 1d6 hours. Blizzards last for 1d3 days. Hurricanes 
can last for up to a week, but their major impact on characters will come in a 
24-to-48-hour period when the center of the storm moves through their area. Tornadoes 
are very short-lived (1d6x10 minutes), typically forming as part 
of a thunderstorm system. 

\textbf{Precipitation:} Roll d\% to determine whether the precipitation is fog 
(01-30), rain/snow (31-90), or sleet/hail (91-00). Snow and sleet occur only when 
the temperature is 30\textdegree{} Fahrenheit or below. Most precipitation lasts for 2d4 hours. 
By contrast, hail lasts for only 1d20 minutes but usually accompanies 1d4 hours 
of rain.

\textbf{Storm (Duststorm/Snowstorm/Thunderstorm):} Wind 
speeds are severe (30 to 50 mph) and visibility is cut by three-quarters. Storms 
last for 2d4-1 hours. See Storms, below, for more details. 

\textbf{Warm:} Between 60\textdegree{} and 85\textdegree{} Fahrenheit during the day, 10 to 20 degrees 
colder at night.

\textbf{Windy:} Wind speeds are moderate to strong (10 to 30 mph); see Table: Wind 
Effects on the following page.

\begin{table}[htb]
\rowcolors{1}{white}{offyellow}
\caption{Random Weather}
\centering
\begin{tabular}{c l l l l}
\textbf{d\%} & \textbf{Weather} & \textbf{Cold Climate} & \textbf{Temperate Climate\textsuperscript{1}} & \textbf{Desert}\\
01-70 & Normal weather & Cold, calm & Normal for season\textsuperscript{2} & Hot, calm\\
71-80 & Abnormal weather & \shortstack{Heat wave (01-30) or\\ cold snap (31-100)} & \shortstack{Heat wave (01-50) or\\ cold snap (51-100)} & Hot, windy\\
81-90 & Inclement weather & Precipitation (snow) & Precipitation (normal for season) & Hot, windy\\
91-99 & Storm & Snowstorm & Thunderstorm, snowstorm\textsuperscript{3} & Duststorm\\
100 & Powerful storm & Blizzard & Windstorm, blizzard\textsuperscript{4}, hurricane, tornado & Downpour\\
\multicolumn{5}{l}{\textsuperscript{1} Temperate includes forest, hills, marsh, mountains, plains, and warm aquatic.}\\
\multicolumn{5}{l}{\textsuperscript{2} Winter is cold, summer is warm, spring and autumn are temperate. Marsh regions are slightly warmer in winter.}\\
%superscript 3 and superscript 4 aren't provided in the actual SRD, there's nothing to complete this table with.
\end{tabular}
\end{table}

%%%%%%%%%%%%%%%%%%%%%%%%%
\subsection{Rain, Snow, Sleet, and Hail}
%%%%%%%%%%%%%%%%%%%%%%%%%

Bad weather frequently slows or halts travel and makes it virtually impossible 
to navigate from one spot to another. Torrential downpours and blizzards obscure 
vision as effectively as a dense fog.

Most precipitation is rain, but in cold conditions it can manifest as snow, sleet, 
or hail. Precipitation of any kind followed by a cold snap in which the temperature 
dips from above freezing to 30\textdegree{} F or below may produce ice.

\textit{Rain:} Rain 
reduces visibility ranges by half, resulting in a -4 penalty on \linkskill{Spot} and \linkskill{Search} 
checks. It has the same effect on flames, ranged weapon attacks, and Listen checks 
as severe wind.

\textit{Snow:} Falling snow has the same effects on visibility, ranged weapon attacks, 
and skill checks as rain, and it costs 2 squares of movement to enter a snow-covered 
square. A day of snowfall leaves 1d6 inches of snow on the ground.

\textit{Heavy Snow:} Heavy snow has the same effects as normal snowfall, but also 
restricts visibility as fog does (see Fog, below). A day of heavy snow leaves 1d4 
feet of snow on the ground, and it costs 4 squares of movement to enter a square 
covered with heavy snow. Heavy snow accompanied by strong or severe winds may result 
in snowdrifts 1d4x5 feet deep, especially in and around objects 
big enough to deflect the wind -- a cabin or a large tent, for instance. There is 
a 10\% chance that a heavy snowfall is accompanied by lightning (see Thunderstorm, 
below). Snow has the same effect on flames as moderate wind.

\textit{Sleet:} Essentially frozen rain, sleet has the same effect as rain while 
falling (except that its chance to extinguish protected flames is 75\%) and the 
same effect as snow once on the ground. 

\textit{Hail:} Hail does not reduce visibility, but the sound of falling hail makes 
Listen checks more difficult (-4 penalty). Sometimes (5\% chance) hail can become 
large enough to deal 1 point of lethal damage (per storm) to anything in the open. 
Once on the ground, hail has the same effect on movement as snow.

%%%%%%%%%%%%%%%%%%%%%%%%%
\subsection{Storms}\index{Storms}
%%%%%%%%%%%%%%%%%%%%%%%%%

The combined effects of precipitation (or dust) and wind that accompany all storms 
reduce visibility ranges by three quarters, imposing a -8 penalty on \linkskill{Spot}, \linkskill{Search}, 
and \linkskill{Listen} checks. Storms make ranged weapon attacks impossible, except for those 
using siege weapons, which have a -4 penalty on attack rolls. They automatically 
extinguish candles, torches, and similar unprotected flames. They cause protected 
flames, such as those of lanterns, to dance wildly and have a 50\% chance to extinguish 
these lights. See Table: Wind Effects for possible consequences to creatures caught 
outside without shelter during such a storm. Storms are divided into the following 
three types. 

\textit{Duststorm (CR 3):} These desert storms differ from other storms in that 
they have no precipitation. Instead, a duststorm blows fine grains of sand that 
obscure vision, smother unprotected flames, and can even choke protected flames 
(50\% chance). Most duststorms are accompanied by severe winds and leave behind 
a deposit of 1d6 inches of sand. However, there is a 10\% chance for a greater 
duststorm to be accompanied by windstorm-magnitude winds (see Table: Wind Effects). 
These greater duststorms deal 1d3 points of nonlethal damage each round to anyone 
caught out in the open without shelter and also pose a choking hazard (see Drowning -- except 
that a character with a scarf or similar protection across her mouth and nose does 
not begin to choke until after a number of rounds equal to 10 x 
her Constitution score). Greater duststorms leave 2d3-1 feet of fine sand in their 
wake.

\textit{Snowstorm:} In addition to the wind and precipitation common to other storms, 
snowstorms leave 1d6 inches of snow on the ground afterward. 

\textit{Thunderstorm:} In addition to wind and precipitation (usually rain, but 
sometimes also hail), thunderstorms are accompanied by lightning that can pose 
a hazard to characters without proper shelter (especially those in metal armor). 
As a rule of thumb, assume one bolt per minute for a 1-hour period at the center 
of the storm. Each bolt causes electricity damage equal to 1d10 eight-sided dice. 
One in ten thunderstorms is accompanied by a tornado (see below). 

\textbf{Powerful Storms:} Very high winds and torrential precipitation reduce visibility 
to zero, making Spot, Search, and Listen checks and all ranged weapon attacks impossible. 
Unprotected flames are automatically extinguished, and protected flames have a 
75\% chance of being doused. Creatures caught in the area must make a DC 20 Fortitude 
save or face the effects based on the size of the creature (see Table: Wind Effects). 
Powerful storms are divided into the following four types.

\textit{Windstorm:} While accompanied by little or no precipitation, windstorms 
can cause considerable damage simply through the force of their wind.

\textit{Blizzard:} The combination of high winds, heavy snow (typically 1d3 feet), 
and bitter cold make blizzards deadly for all who are unprepared for them.

\textit{Hurricane:} In addition to very high winds and heavy rain, hurricanes are 
accompanied by floods. Most adventuring activity is impossible under such conditions.

\textit{Tornado:} One in ten thunderstorms is accompanied by a tornado.

%%%%%%%%%%%%%%%%%%%%%%%%%
\subsection{Fog}\index{Fog}
%%%%%%%%%%%%%%%%%%%%%%%%%

Whether in the form of a low-lying cloud or a mist rising from the ground, fog 
obscures all sight, including darkvision, beyond 5 feet. Creatures 5 feet away 
have concealment (attacks by or against them have a 20\% miss chance).

%%%%%%%%%%%%%%%%%%%%%%%%%
\subsection{Winds}\index{Wind}
%%%%%%%%%%%%%%%%%%%%%%%%%

The wind can create a stinging spray of sand or dust, fan a large fire, heel over 
a small boat, and blow gases or vapors away. If powerful enough, it can even knock 
characters down (see Table: Wind Effects), interfere with ranged attacks, or impose 
penalties on some skill checks.

\textit{Light Wind:} A gentle breeze, having little or no game effect.

\textit{Moderate Wind:} A steady wind with a 50\% chance of extinguishing small, 
unprotected flames, such as candles.

\textit{Strong Wind:} Gusts that automatically extinguish unprotected flames (candles, 
torches, and the like). Such gusts impose a -2 penalty on ranged attack rolls and 
on \linkskill{Listen} checks.

\textit{Severe Wind:} In addition to automatically extinguishing any unprotected 
flames, winds of this magnitude cause protected flames (such as those of lanterns) 
to dance wildly and have a 50\% chance of extinguishing these lights. Ranged weapon 
attacks and Listen checks are at a -4 penalty. This is the velocity of wind produced 
by a \textit{gust of wind }spell.

\textit{Windstorm:} Powerful enough to bring down branches if not whole trees, 
windstorms automatically extinguish unprotected flames and have a 75\% chance of 
blowing out protected flames, such as those of lanterns. Ranged weapon attacks 
are impossible, and even siege weapons have a -4 penalty on attack rolls. Listen 
checks are at a -8 penalty due to the howling of the wind. 

\textit{Hurricane-Force Wind:} All flames are extinguished. Ranged attacks are 
impossible (except with siege weapons, which have a -8 penalty on attack rolls). 
Listen checks are impossible: All characters can hear is the roaring of the wind. 
Hurricane-force winds often fell trees.

\textit{Tornado (CR 10):} All flames are extinguished. All ranged attacks are impossible 
(even with siege weapons), as are Listen checks. Instead of being blown away (see 
Table: Wind Effects), characters in close proximity to a tornado who fail their 
Fortitude saves are sucked toward the tornado. Those who come in contact with the 
actual funnel cloud are picked up and whirled around for 1d10 rounds, taking 6d6 
points of damage per round, before being violently expelled (falling damage may 
apply). While a tornado's rotational speed can be as great as 300 mph, the funnel 
itself moves forward at an average of 30 mph (roughly 250 feet per round). A tornado 
uproots trees, destroys buildings, and causes other similar forms of major destruction.

\begin{table}[htb]
\rowcolors{1}{white}{offyellow}
\caption{Wind Effects}
\centering
\begin{tabular}{l l l l l c}
\textbf{Wind Force} & \textbf{Wind Speed} & \multicolumn{1}{p{2.5cm}}{\textbf{Ranged Attacks Normal/Siege Weapons\textsuperscript{1}}} & \textbf{Creature Size\textsuperscript{2}} & \multicolumn{1}{p{2cm}}{\textbf{Wind Effect on Creatures}} & \textbf{Fort Save DC}\\
Light & 0-10 mph & -/- & Any & None & -\\
Moderate & 11-20 mph & -/- & Any & None & -\\
Strong & 21-30 mph & -2/- & Tiny or smaller & Knocked down & 10\\
& & & Small or larger & None & \\
Severe & 31-50 mph & -4/- & Tiny & Blown away & 15\\
& & & Small & Knocked down& \\
& & & Medium & Checked& \\
& & & Large or larger & None& \\
Windstorm & 51-74 mph & Impossible/-4 & Small or smaller & Blown away & 18\\
& & & Medium & Knocked down& \\
& & & Large or Huge & Checked& \\
& & & Gargantuan or Colossal & None& \\
Hurricane & 75-174 mph & Impossible/-8 & Medium or smaller & Blown away & 20\\
& & & Large & Knocked down& \\
& & & Huge & Checked& \\
& & & Gargantuan or Colossal & None& \\
Tornado & 175-300 mph & Impossible/impossible & Large or smaller & Blown away & 30\\
& & & Huge & Knocked down& \\
& & & Gargantuan or Colossal & Checked& \\
\multicolumn{6}{p{16cm}}{\textsuperscript{1} The siege weapon category includes ballista and catapult attacks as well as boulders tossed by giants.}\\
\multicolumn{6}{p{16cm}}{\textsuperscript{2} Flying or airborne creatures are treated as one size category smaller than their actual size, so an airborne Gargantuan dragon is treated as Huge for purposes of wind effects.}\\
\multicolumn{6}{p{16cm}}{\textit{Checked:} Creatures are unable to move forward against the force of the wind. Flying creatures are blown back 1d6x5 feet.}\\
\multicolumn{6}{p{16cm}}{\textit{Knocked Down:}\index{Knocked Down} Creatures are knocked prone by the force of the wind. Flying creatures are instead blown back 1d6x10 feet.}\\
\multicolumn{6}{p{16cm}}{\textit{Blown Away:} Creatures on the ground are knocked prone and rolled 1d4x10 feet, taking 1d4 points of nonlethal damage per 10 feet. Flying creatures are blown back 2d6×10 feet and take 2d6 points of nonlethal damage due to battering and buffeting.}\\
\end{tabular}
\end{table}

%%%%%%%%%%%%%%%%%%%%%%%%%%%%%%%%%%%%%%%%%%%%%%%%%%
\section{The Environment}
%%%%%%%%%%%%%%%%%%%%%%%%%%%%%%%%%%%%%%%%%%%%%%%%%%

Environmental hazards specific to one kind of terrain (such as an avalanche, which 
occurs in the mountains) are described in Wilderness, above. Environmental hazards 
common to more than one setting are detailed below.

%%%%%%%%%%%%%%%%%%%%%%%%%
\subsection{Acid Effects}
%%%%%%%%%%%%%%%%%%%%%%%%%

Corrosive acids deals 1d6 points of damage per round of exposure except in the 
case of total immersion (such as into a vat of acid), which deals 10d6 points of 
damage per round. An attack with acid, such as from a hurled vial or a monster's 
spittle, counts as a round of exposure.

The fumes from most acids are inhaled poisons. Those who come close enough to a 
body of acid large enough to dunk a creature in it must make a DC 13 Fortitude save or 
take 1 point of Constitution damage. All such characters must make a second save 
1 minute later or take another 1d4 points of Constitution damage.

Creatures immune to acid's caustic properties might still drown in it if they are 
totally immersed (see Drowning).

%%%%%%%%%%%%%%%%%%%%%%%%%
\subsection{Cold Dangers}\index{Cold Weather Effects}
%%%%%%%%%%%%%%%%%%%%%%%%%

Cold and exposure deal nonlethal damage to the victim. This nonlethal damage cannot 
be recovered until the character gets out of the cold and warms up again. Once 
a character is rendered unconscious through the accumulation of nonlethal damage, 
the cold and exposure begins to deal lethal damage at the same rate.

An unprotected character in cold weather (below 40\textdegree{} F) must make a Fortitude save 
each hour (DC 15, + 1 per previous check) or take 1d6 points of nonlethal damage. 
A character who has the \linkskill{Survival} skill may receive a bonus on this saving throw 
and may be able to apply this bonus to other characters as well (see the skill 
Description).

In conditions of severe cold or exposure (below 0\textdegree{} F), an unprotected character 
must make a Fortitude save once every 10 minutes (DC 15, +1 per previous check), 
taking 1d6 points of nonlethal damage on each failed save. A character who has 
the Survival skill may receive a bonus on this saving throw and may be able to 
apply this bonus to other characters as well (see the skill description). Characters 
wearing winter clothing only need check once per hour for cold and exposure damage.

A character who takes any nonlethal damage from cold or exposure is beset by frostbite 
or hypothermia (treat her as fatigued). These penalties end when the character 
recovers the nonlethal damage she took from the cold and exposure.

Extreme cold (below -20\textdegree{} F) deals 1d6 points of lethal damage per minute (no save). 
In addition, a character must make a Fortitude save (DC 15, +1 per previous check) 
or take 1d4 points of nonlethal damage. Those wearing metal armor or coming into 
contact with very cold metal are affected as if by a \textit{chill metal }spell.

%%%
\subsubsection{Ice Effects}
%%%

Characters walking on ice must spend 2 squares of movement to enter a square covered 
by ice, and the DC for \linkskill{Balance} and \linkskill{Tumble} checks increases by +5. Characters in 
prolonged contact with ice may run the risk of taking damage from severe cold (see 
above).

%%%%%%%%%%%%%%%%%%%%%%%%%
\subsection{Darkness}\index{Darkness}
%%%%%%%%%%%%%%%%%%%%%%%%%

Darkvision allows many characters and monsters to see perfectly well without any 
light at all, but characters with normal vision (or low-light vision, for that 
matter) can be rendered completely blind by putting out the lights. Torches or 
lanterns can be blown out by sudden gusts of subterranean wind, magical light sources 
can be dispelled or countered, or magical traps might create fields of impenetrable 
darkness.

In many cases, some characters or monsters might be able to see, while others are 
blinded. For purposes of the following points, a blinded creature is one who simply 
can't see through the surrounding darkness.

\begin{itemize}
\item Creatures blinded by darkness lose the ability to deal extra damage due to precision (for example, a sneak attack).
\item Blinded creatures are hampered in their movement, and pay 2 squares of movement 
per square moved into (double normal cost). Blinded creatures can't run or charge.
\item All opponents have total concealment from a blinded creature, so the blinded creature 
has a 50\% miss chance in combat. A blinded creature must first pinpoint the location 
of an opponent in order to attack the right square; if the blinded creature launches 
an attack without pinpointing its foe, it attacks a random square within its reach. 
For ranged attacks or spells against a foe whose location is not pinpointed, roll 
to determine which adjacent square the blinded creature is facing; its attack is 
directed at the closest target that lies in that direction.
\item A blinded creature loses its Dexterity adjustment to AC and takes a -2 penalty to AC.
\item A blinded creature takes a -4 penalty on \linkskill{Search} checks and most Strength- and Dexterity-based 
skill checks, including any with an armor check penalty. A creature blinded by 
darkness automatically fails any skill check relying on vision.
\item Creatures blinded by darkness cannot use gaze attacks and are immune to gaze attacks.
\item A creature blinded by darkness can make a \linkskill{Listen} check as a free action each round 
in order to locate foes (DC equal to opponents' Move Silently checks). A successful 
check lets a blinded character hear an unseen creature "over there somewhere".
It's almost impossible to pinpoint the location of an unseen creature. A Listen 
check that beats the DC by 20 reveals the unseen creature's square (but the unseen 
creature still has total concealment from the blinded creature).
\item A blinded creature can grope about to find unseen creatures. A character can make 
a touch attack with his hands or a weapon into two adjacent squares using a standard 
action. If an unseen target is in the designated square, there is a 50\% miss chance 
on the touch attack. If successful, the groping character deals no damage but has 
pinpointed the unseen creature's current location. (If the unseen creature moves, 
its location is once again unknown.)
\item If a blinded creature is struck by an unseen foe, the blinded character pinpoints 
the location of the creature that struck him (until the unseen creature moves, 
of course). The only exception is if the unseen creature has a reach greater than 
5 feet (in which case the blinded character knows the location of the unseen opponent, 
but has not pinpointed him) or uses a ranged attack (in which case, the blinded 
character knows the general direction of the foe, but not his location).
\item A creature with the scent ability automatically pinpoints unseen creatures within 5 feet of its location.
\end{itemize}

%%%%%%%%%%%%%%%%%%%%%%%%%
\subsection{Falling}\index{Falling}
%%%%%%%%%%%%%%%%%%%%%%%%%

\textbf{Falling Damage:} The basic rule is simple: 1d6 points of damage per 10 
feet fallen, to a maximum of 20d6.

If a character deliberately jumps instead of merely slipping or falling, the damage 
is the same but the first 1d6 is nonlethal damage. A DC 15 \linkskill{Jump} check or DC 15 
\linkskill{Tumble} check allows the character to avoid any damage from the first 10 feet fallen 
and converts any damage from the second 10 feet to nonlethal damage. Thus, a character 
who slips from a ledge 30 feet up takes 3d6 damage. If the same character deliberately 
jumped, he takes 1d6 points of nonlethal damage and 2d6 points of lethal damage. 
And if the character leaps down with a successful Jump or Tumble check, he takes 
only 1d6 points of nonlethal damage and 1d6 points of lethal damage from the plunge.

Falls onto yielding surfaces (soft ground, mud) also convert the first 1d6 of damage 
to nonlethal damage. This reduction is cumulative with reduced damage due to deliberate 
jumps and the Jump skill.

\textbf{Falling into Water:} Falls into water are handled somewhat differently. 
If the water is at least 10 feet deep, the first 20 feet of falling do no damage. 
The next 20 feet do nonlethal damage (1d3 per 10-foot increment). Beyond that, 
falling damage is lethal damage (1d6 per additional 10-foot increment).

Characters who deliberately dive into water take no damage on a successful DC 15 
Swim check or DC 15 Tumble check, so long as the water is at least 10 feet deep 
for every 30 feet fallen. However, the DC of the check increases by 5 for every 
50 feet of the dive. 

%%%%%%%%%%%%%%%%%%%%%%%%%
\subsection{Falling Objects}
%%%%%%%%%%%%%%%%%%%%%%%%%

Just as characters take damage when they fall more than 10 feet, so too do they 
take damage when they are hit by falling objects.

Objects that fall upon characters deal damage based on their weight and the distance 
they have fallen.

For each 200 pounds of an object's weight, the object deals 1d6 points of damage, 
provided it falls at least 10 feet. Distance also comes into play, adding an additional 
1d6 points of damage for every 10-foot increment it falls beyond the first (to 
a maximum of 20d6 points of damage).

Objects smaller than 200 pounds also deal damage when dropped, but they must fall 
farther to deal the same damage. Use Table: Damage from Falling Objects to see 
how far an object of a given weight must drop to deal 1d6 points of damage.

\begin{table}[htb]
\rowcolors{1}{white}{offyellow}
\caption{Damage from Falling Objects}
\centering
\begin{tabular}{cc}
\textbf{Object Weight} & \textbf{Falling Distance}\\
200-101lb & 20ft\\
100-51lb & 30ft\\
50-31lb & 40ft\\
30-11lb & 50ft\\
10-6lb & 60ft\\
5-1lb & 70ft\\
\end{tabular}
\end{table}

For each additional increment an object falls, it deals an additional 1d6 points 
of damage.

Objects weighing less than 1 pound do not deal damage to those they land upon, 
no matter how far they have fallen.

%%%%%%%%%%%%%%%%%%%%%%%%%
\subsection{Heat Dangers}\index{Heat Dangers}
%%%%%%%%%%%%%%%%%%%%%%%%%

Heat deals nonlethal damage that cannot be recovered until the character gets cooled 
off (reaches shade, survives until nightfall, gets doused in water, is targeted 
by \textit{endure elements}, and so forth). Once rendered unconscious through the 
accumulation of nonlethal damage, the character begins to take lethal damage at 
the same rate.

A character in very hot conditions (above 90\textdegree{} F) must make a Fortitude saving 
throw each hour (DC 15, +1 for each previous check) or take 1d4 points of nonlethal 
damage. Characters wearing heavy clothing or armor of any sort take a -4 penalty 
on their saves. A character with the \linkskill{Survival} skill may receive a bonus on this 
saving throw and may be able to apply this bonus to other characters as well (see 
the skill description). Characters reduced to unconsciousness begin taking lethal 
damage (1d4 points per hour).

In severe heat (above 110\textdegree{} F), a character must make a Fortitude save once every 
10 minutes (DC 15, +1 for each previous check) or take 1d4 points of nonlethal 
damage. Characters wearing heavy clothing or armor of any sort take a -4 penalty 
on their saves. A character with the Survival skill may receive a bonus on this 
saving throw and may be able to apply this bonus to other characters as well. Characters 
reduced to unconsciousness begin taking lethal damage (1d4 points per each 10-minute 
period).

A character who takes any nonlethal damage from heat exposure now suffers from 
heatstroke and is fatigued.

These penalties end when the character recovers the nonlethal damage she took from 
the heat.

Extreme heat (air temperature over 140\textdegree{} F, fire, boiling water, lava) deals lethal 
damage. Breathing air in these temperatures deals 1d6 points of damage per minute 
(no save). In addition, a character must make a Fortitude save every 5 minutes 
(DC 15, +1 per previous check) or take 1d4 points of nonlethal damage. Those wearing 
heavy clothing or any sort of armor take a -4 penalty on their saves. In addition, 
those wearing metal armor or coming into contact with very hot metal are affected 
as if by a \textit{heat metal }spell.

Boiling water deals 1d6 points of scalding damage, unless the character is fully 
immersed, in which case it deals 10d6 points of damage per round of exposure.

%%%
\subsubsection{Catching on Fire}
%%%

Characters exposed to burning oil, bonfires, and noninstantaneous magic fires 
might find their clothes, hair, or equipment on fire. Spells with an instantaneous 
duration don't normally set a character on fire, since the heat and flame 
from these come and go in a flash.

Characters at risk of catching fire are allowed a DC 15 Reflex save to avoid this 
fate. If a character's clothes or hair catch fire, he takes 1d6 points of damage 
immediately. In each subsequent round, the burning character must make another 
Reflex saving throw. Failure means he takes another 1d6 points of damage that round. 
Success means that the fire has gone out. (That is, once he succeeds on his saving 
throw, he's no longer on fire.)

A character on fire may automatically extinguish the flames by jumping into enough 
water to douse himself. If no body of water is at hand, rolling on the ground or 
smothering the fire with cloaks or the like permits the character another save 
with a +4 bonus.

Those unlucky enough to have their clothes or equipment catch fire must make DC 
15 Reflex saves for each item. Flammable items that fail take the same amount of 
damage as the character.

%%%
\subsubsection{Lava Effects}
%%%

Lava or magma deals 2d6 points of damage per round of exposure, except in the case 
of total immersion (such as when a character falls into the crater of an active 
volcano), which deals 20d6 points of damage per round.

Damage from magma continues for 1d3 rounds after exposure ceases, but this additional 
damage is only half of that dealt during actual contact (that is, 1d6 or 10d6 points 
per round).

An immunity or resistance to fire serves as an immunity to lava or magma. However, 
a creature immune to fire might still drown if completely immersed in lava (see 
Drowning, below).

%%%%%%%%%%%%%%%%%%%%%%%%%
\subsection{Smoke Effects}\index{Smoke}
%%%%%%%%%%%%%%%%%%%%%%%%%

A character who breathes heavy smoke must make a Fortitude save each round (DC 
15, +1 per previous check) or spend that round choking and coughing. A character 
who chokes for 2 consecutive rounds takes 1d6 points of nonlethal damage.

Smoke obscures vision, giving concealment (20\% miss chance) to characters within 
it.

%%%%%%%%%%%%%%%%%%%%%%%%%
\subsection{Starvation and Thirst}\index{Starvation}\index{Thirst}
%%%%%%%%%%%%%%%%%%%%%%%%%

Characters might find themselves without food or water and with no means to obtain 
them. In normal climates, Medium characters need at least a gallon of fluids and 
about a pound of decent food per day to avoid starvation. (Small characters need 
half as much.) In very hot climates, characters need two or three times as much 
water to avoid dehydration.

A character can go without water for 1 day plus a number of hours equal to his 
Constitution score. After this time, the character must make a Constitution check 
each hour (DC 10, +1 for each previous check) or take 1d6 points of nonlethal damage.

A character can go without food for 3 days, in growing discomfort. After this time, 
the character must make a Constitution check each day (DC 10, +1 for each previous 
check) or take 1d6 points of nonlethal damage.

Characters who have taken nonlethal damage from lack of food or water are fatigued. 
Nonlethal damage from thirst or starvation cannot be recovered until the character 
gets food or water, as needed -- not even magic that restores hit points heals this 
damage.

%%%%%%%%%%%%%%%%%%%%%%%%%
\subsection{Suffocation}\index{Suffocation}
%%%%%%%%%%%%%%%%%%%%%%%%%

A character who has no air to breathe can hold her breath for 2 rounds per point 
of Constitution. After this period of time, the character must make a DC 10 Constitution 
check in order to continue holding her breath. The save must be repeated each round, 
with the DC increasing by +1 for each previous success.

When the character fails one of these Constitution checks, she begins to suffocate. 
In the first round, she falls unconscious (0 hit points). In the following round, 
she drops to -1 hit points and is dying. In the third round, she suffocates.

\textbf{Slow Suffocation:} A Medium character can breathe easily for 6 hours in 
a sealed chamber measuring 10 feet on a side. After that time, the character takes 
1d6 points of nonlethal damage every 15 minutes. Each additional Medium character 
or significant fire source (a torch, for example) proportionally reduces the time 
the air will last.

Small characters consume half as much air as Medium characters. A larger volume 
of air, of course, lasts for a longer time. 

%%%%%%%%%%%%%%%%%%%%%%%%%
\subsection{Water Dangers}\index{Water Dangers}
%%%%%%%%%%%%%%%%%%%%%%%%%

Any character can wade in relatively calm water that isn't over his head, no check 
required. Similarly, swimming in calm water only requires skill checks with a DC 
of 10. Trained swimmers can just take 10. (Remember, however, that armor or heavy 
gear makes any attempt at swimming much more difficult. See the \linkskill{Swim} skill description.)

By contrast, fast-moving water is much more dangerous. On a successful DC 15 Swim 
check or a DC 15 Strength check, it deals 1d3 points of nonlethal damage per round 
(1d6 points of lethal damage if flowing over rocks and cascades). On a failed check, 
the character must make another check that round to avoid going under.

Very deep water is not only generally pitch black, posing a navigational hazard, 
but worse, it deals water pressure damage of 1d6 points per minute for every 100 
feet the character is below the surface. A successful Fortitude save (DC 15, +1 
for each previous check) means the diver takes no damage in that minute. Very cold 
water deals 1d6 points of nonlethal damage from hypothermia per minute of exposure.

%%%
\subsubsection{Drowning}\index{Drowning}
%%%

Any character can hold her breath for a number of rounds equal to twice her Constitution 
score. After this period of time, the character must make a DC 10 Constitution 
check every round in order to continue holding her breath. Each round, the DC increases 
by 1. 

When the character finally fails her Constitution check, she begins to drown. In 
the first round, she falls unconscious (0 hp). In the following round, she drops 
to -1 hit points and is dying. In the third round, she drowns.

It is possible to drown in substances other than water, such as sand, quicksand, 
fine dust, and silos full of grain.

%%%%%%%%%%%%%%%%%%%%%%%%%%%%%%%%%%%%%%%%%%%%%%%%%%
\section{Traps}\index{Traps}
%%%%%%%%%%%%%%%%%%%%%%%%%%%%%%%%%%%%%%%%%%%%%%%%%%

\textbf{Types of Traps:} A trap can be either mechanical or magic in nature. Mechanical 
traps include pits, arrow traps, falling blocks, water-filled rooms, whirling blades, 
and anything else that depends on a mechanism to operate. A mechanical trap can 
be constructed by a PC through successful use of the \linkskill{Craft} (trapmaking) skill (see 
Designing a Trap, below, and the skill description).

Magic traps are further divided into spell traps and magic device traps. Magic 
device traps initiate spell effects when activated, just as wands, rods, rings, 
and other magic items do. Creating a magic device trap requires the \linkfeat{Craft Wondrous Item}
feat (see Designing a Trap and the feat description).

Spell traps are simply spells that themselves function as traps. Creating 
a spell trap requires the services of a character who can cast the needed spell 
or spells, who is usually either the character creating the trap or an NPC spellcaster 
hired for the purpose.

%%%%%%%%%%%%%%%%%%%%%%%%%
\subsection{Mechanical Traps}
%%%%%%%%%%%%%%%%%%%%%%%%%

Dungeons are frequently equipped with deadly mechanical (nonmagical) traps. A trap 
typically is defined by its location and triggering conditions, how hard it is 
to spot before it goes off, how much damage it deals, and whether or not the heroes 
receive a saving throw to mitigate its effects. Traps that attack with arrows, 
sweeping blades, and other types of weaponry make normal attack rolls, with a specific 
attack bonus dictated by the trap's design.

Creatures who succeed on a DC 20 \linkskill{Search} check detect a simple mechanical trap before 
it is triggered. (A simple trap is a snare, a trap triggered by a tripwire, or 
a large trap such as a pit.)

A character with the trap sense class feature who succeeds on a DC 21 (or higher) 
Search check detects a well-hidden or complex mechanical trap before it is triggered. 
Complex traps are denoted by their triggering mechanisms and involve pressure plates, 
mechanisms linked to doors, changes in weight, disturbances in the air, vibrations, 
and other sorts of unusual triggers.

%%%%%%%%%%%%%%%%%%%%%%%%%
\subsection{Magic Traps}
%%%%%%%%%%%%%%%%%%%%%%%%%

Many spells can be used to create dangerous traps. Unless the spell or item description 
states otherwise, assume the following to be true.

\begin{itemize}
\item A successful Search check (DC 25 + spell level) made by a rogue (and only a rogue) 
detects a magic trap before it goes off. Other characters have no chance to find 
a magic trap with a Search check.
\item Magic traps permit a saving throw in order to avoid the effect (DC 10 + spell level x 1.5).
\item Magic traps may be disarmed by a rogue (and only a rogue) with a successful Disable 
Device check (DC 25 + spell level).
\end{itemize}

%%%%%%%%%%%%%%%%%%%%%%%%%
\subsection{Elements of a Trap}
%%%%%%%%%%%%%%%%%%%%%%%%%

All traps -- mechanical or magic -- have the following elements: trigger, reset, 
Search DC, Disable Device DC, attack bonus (or saving throw or onset delay), damage/effect, 
and Challenge Rating. Some traps may also include optional elements, such as poison 
or a bypass. These characteristics are described below.

%%%
\subsubsection{Trigger}
%%%

A trap's trigger determines how it is sprung.

\textbf{Location:} A location trigger springs a trap when someone stands in a particular 
square.

\textbf{Proximity:} This trigger activates the trap when a creature approaches 
within a certain distance of it. A proximity trigger differs from a location trigger 
in that the creature need not be standing in a particular square. Creatures that 
are flying can spring a trap with a proximity trigger but not one with a location 
trigger. Mechanical proximity triggers are extremely sensitive to the slightest 
change in the air. This makes them useful only in places such as crypts, where 
the air is unusually still.

The proximity trigger used most often for magic device traps is the \linkspell{Alarm} 
spell. Unlike when the spell is cast, an \textit{Alarm} spell used as a trigger 
can have an area that's no larger than the area the trap is meant to protect.

Some magic device traps have special proximity triggers that activate only when 
certain kinds of creatures approach. For example, a \linkspell{Detect Good} spell 
can serve as a proximity trigger on an evil altar, springing the attached trap 
only when someone of good alignment gets close enough to it.

\textbf{Sound:} This trigger springs a magic trap when it detects any sound. A 
sound trigger functions like an ear and has a +15 bonus on Listen checks. A successful 
Move Silently check, magical \linkspell{Silence}, and other effects that would negate 
hearing defeat it. A trap with a sound trigger requires the casting of \linkspell{Clairaudience} during its construction.

\textbf{Visual:} This trigger for magic traps works like an actual eye, springing 
the trap whenever it "sees" something. A trap with a visual trigger requires 
the casting of \linkspell{Arcane Eye}, \linkspell{Clairvoyance}, or \linkspell{True Seeing} during 
its construction. Sight range and the Spot bonus conferred on the trap depend on 
the spell chosen, as shown.

\begin{table}[htb]
\rowcolors{1}{white}{offyellow}
\caption{Spell Trap Spot Bonuses}
\centering
\begin{tabular}{lll}
\textbf{Spell} & \textbf{Sight Range} & \textbf{Spot Bonus}\\
Arcane Eye & Line of Sight (unlimited range) & +20\\
Clairvoyance & One preselected location & +15\\
True Seeing & Line of Sight (up to 120ft) & +30\\
\end{tabular}
\end{table}

If you want the trap to "see" in the dark, you must either choose the \textit{True Seeing}
option or add \textit{Darkvision} to the trap as well. (\textit{Darkvision}
limits the trap's sight range in the dark to 60 feet.) If invisibility, disguises, 
or illusions can fool the spell being used, they can fool the visual trigger as 
well. 

\textbf{Touch:} A touch trigger, which springs the trap when touched, is one of 
the simplest kinds of trigger to construct. This trigger may be physically attached 
to the part of the mechanism that deals the damage or it may not. You can make 
a magic touch trigger by adding \textit{Alarm} to the trap and reducing the area 
of the effect to cover only the trigger spot.

\textbf{Timed:} This trigger periodically springs the trap after a certain duration 
has passed.

\textbf{Spell:} All spell traps have this kind of trigger. The appropriate spell 
descriptions explain the trigger conditions for traps that contain spell 
triggers.

%%%
\subsubsection{Reset}
%%%

A reset element is the set of conditions under which a trap becomes ready to trigger 
again.

\textbf{No Reset:} Short of completely rebuilding the trap, there's no way to trigger 
it more than once. Spell traps have no reset element. 

\textbf{Repair:} To get the trap functioning again, you must repair it. 

\textbf{Manual:} Resetting the trap requires someone to move the parts back into 
place. This is the kind of reset element most mechanical traps have.

\textbf{Automatic:} The trap resets itself, either immediately or after a timed 
interval.

%%%
\subsubsection{Repairing and Resetting Mechanical Traps}
%%%

Repairing a mechanical trap requires a \linkskill{Craft} (trapmaking) check against a DC equal 
to the one for building it. The cost for raw materials is one-fifth of the trap's 
original market price. To calculate how long it takes to fix a trap, use the same 
calculations you would for building it, but use the cost of the raw materials required 
for repair in place of the market price.

Resetting a trap usually takes only a minute or so. For a trap with a more difficult 
reset method, you should set the time and labor required.

\vspace{12pt}
\textbf{Bypass (Optional Element)}

If the builder of a trap wants to be able to move past the trap after it is created 
or placed, it's a good idea to build in a bypass mechanism -- something that temporarily 
disarms the trap. Bypass elements are typically used only with mechanical traps; 
spell traps usually have built-in allowances for the caster to bypass them.

\textbf{Lock:} A lock bypass requires a DC 30 \linkskill{Open Lock} check to open. 

\textbf{Hidden Switch:} A hidden switch requires a DC 25 \linkskill{Search} check to locate.

\textbf{Hidden Lock:} A hidden lock combines the features above, requiring a DC 
25 Search check to locate and a DC 30 Open Lock check to open.

\vspace{12pt}
\textbf{Search and Disable Device DCs}

The builder sets the Search and Disable Device DCs for a mechanical trap. For a 
magic trap, the values depend on the highest-level spell used.

\textbf{Mechanical Trap:} The base DC for both Search and Disable Device checks 
is 20. Raising or lowering either of these DCs affects the base cost (Table: Cost 
Modifiers for Mechanical Traps) and possibly the CR (Table: CR Modifiers for Mechanical 
Traps). 

\textbf{Magic Trap:} The DC for both Search and Disable Device checks is equal 
to 25 + the spell level of the highest-level spell used. Only characters with the 
trap sense class feature can attempt a Search check or a Disable Device check involving 
a magic trap. These DCs do not affect the trap's cost or CR.

%%%
\subsubsection{Attack Bonus/Saving Throw DC}
%%%

A trap usually either makes an attack roll or forces a saving throw to avoid it. 
Occasionally a trap uses both of these options, or neither (see Never Miss).

\textbf{Pits:} These are holes (covered or not) that characters can fall into and 
take damage. A pit needs no attack roll, but a successful Reflex save (DC set by 
the builder) avoids it. Other save-dependent mechanical traps also fall into this 
category.

Pits in dungeons come in three basic varieties: uncovered, covered, and chasms. 
Pits and chasms can be defeated by judicious application of the Climb skill, the 
Jump skill, or various magical means.

Uncovered pits serve mainly to discourage intruders from going a certain way, although 
they cause much grief to characters who stumble into them in the dark, and they 
can greatly complicate a melee taking place nearby.

Covered pits are much more dangerous. They can be detected with a DC 20 Search 
check, but only if the character is taking the time to carefully examine the area 
before walking across it. A character who fails to detect a covered pit is still 
entitled to a DC 20 Reflex save to avoid falling into it. However, if she was running 
or moving recklessly at the time, she gets no saving throw and falls automatically.

Trap coverings can be as simple as piled refuse (straw, leaves, sticks, garbage), 
a large rug, or an actual trapdoor concealed to appear as a normal part of the 
floor. Such a trapdoor usually swings open when enough weight (usually about 50 
to 80 pounds) is placed upon it. Devious trap builders sometimes design trapdoors 
so that they spring back shut after they open. The trapdoor might lock once it's 
back in place, leaving the stranded character well and truly trapped. Opening such 
a trapdoor is just as difficult as opening a regular door (assuming the trapped 
character can reach it), and a DC 13 Strength check is needed to keep a spring-loaded 
door open.

Pit traps often have something nastier than just a hard floor at the bottom. A 
trap designer may put spikes, monsters, or a pool of acid, lava, or even water 
at the bottom. Spikes at the bottom of a pit deal damage as daggers with a +10 
attack bonus and a +1 bonus on damage for every 10 feet of the fall (to a maximum 
bonus on damage of +5). If the pit has multiple spikes, a falling victim is attacked 
by 1d4 of them. This damage is in addition to any damage from the fall itself. 

Monsters sometimes live in pits. Any monster that can fit into the pit might have 
been placed there by the dungeon's designer, or might simply have fallen in and 
not been able to climb back out. 

A secondary trap, mechanical or magical, at the bottom of a pit can be particularly 
deadly. Activated by a falling victim, the secondary trap attacks the already injured 
character when she's least ready for it.

\textbf{Ranged Attack Traps:} These traps fling darts, arrows, spears, or the like 
at whoever activated the trap. The builder sets the attack bonus. A ranged attack 
trap can be configured to simulate the effect of a composite bow with a high strength 
rating which provides the trap with a bonus on damage equal to its strength rating.

\textbf{Melee Attack Traps:} These traps feature such obstacles as sharp blades 
that emerge from walls and stone blocks that fall from ceilings. Once again, the 
builder sets the attack bonus.

%%%
\subsubsection{Damage/Effect}
%%%

The effect of a trap is what happens to those who spring it. Usually this takes 
the form of either damage or a spell effect, but some traps have special effects.

\textbf{Pits:} Falling into a pit deals 1d6 points of damage per 10 feet of depth. 

\textbf{Ranged Attack Traps:} These traps deal whatever damage their ammunition 
normally would. If a trap is constructed with a high strength rating, it has a 
corresponding bonus on damage.

\textbf{Melee Attack Traps:} These traps deal the same damage as the melee weapons 
they "wield." In the case of a falling stone block, you can assign any amount 
of bludgeoning damage you like, but remember that whoever resets the trap has to 
lift that stone back into place. 

A melee attack trap can be constructed with a built-in bonus on damage rolls, just 
as if the trap itself had a high Strength score. 

\textbf{Spell Traps:} Spell traps produce the spell's effect. Like all 
spells, a spell trap that allows a saving throw has a save DC of 10 + spell level 
+ caster's relevant ability modifier.

\textbf{Magic Device Traps:} These traps produce the effects of any spells included 
in their construction, as described in the appropriate entries. If the 
spell in a magic device trap allows a saving throw, its save DC is 10 + spell level 
x1.5. Some spells make attack rolls instead.

\textbf{Special:} Some traps have miscellaneous features that produce special effects, 
such as drowning for a water trap or ability damage for poison. Saving throws and 
damage depend on the poison or are set by the builder, as appropriate.

%%%
\subsubsection{Miscellaneous Trap Features}
%%%

Some traps include optional features that can make them considerably more deadly. 
The most common such features are discussed below.

\textbf{Alchemical Item:} Mechanical traps may incorporate alchemical devices or 
other special substances or items, such as tanglefoot bags, alchemist's fire, thunderstones, 
and the like. Some such items mimic spell effects. If the item mimics a spell effect, 
it increases the CR as shown on Table: CR Modifiers for Mechanical Traps.

\textbf{Gas:} With a gas trap, the danger is in the inhaled poison it delivers. 
Traps employing gas usually have the never miss and onset delay features (see below).

\textbf{Liquid:} Any trap that involves a danger of drowning is in this category. 
Traps employing liquid usually have the never miss and onset delay features (see 
below). 

\textbf{Multiple Target:} Traps with this feature can affect more than one character.

\textbf{Never Miss:} When the entire dungeon wall moves to crush you, your quick 
reflexes won't help, since the wall can't possibly miss. A trap with this feature 
has neither an attack bonus nor a saving throw to avoid, but it does have an onset 
delay (see below). Most traps involving liquid or gas are of the never miss variety. 

\textbf{Onset Delay:} An onset delay is the amount of time between when the trap 
is sprung and when it deals damage. A never miss trap always has an onset delay.

\textbf{Poison:} Traps that employ poison are deadlier than their nonpoisonous 
counterparts, so they have correspondingly higher CRs. To determine the CR modifier 
for a given poison, consult Table: CR Modifiers for Mechanical Traps. Only injury, 
contact, and inhaled poisons are suitable for traps; ingested types are not. Some 
traps simply deal the poison's damage. Others deal damage with ranged or melee 
attacks as well. 

\textbf{Pit Spikes:} Treat spikes at the bottom of a pit as daggers, each with 
a +10 attack bonus. The damage bonus for each spike is +1 per 10 feet of pit depth 
(to a maximum of +5). Each character who falls into the pit is attacked by 1d4 
spikes. Pit spikes do not add to the average damage of the trap (see \linksec{Challenge Rating of a Trap}{Average Damage}, 
below).

\textbf{Pit Bottom:} If something other than spikes waits at the bottom of a pit, 
it's best to treat that as a separate trap (see \linksec{Challenge Rating of a Trap}{Multiple Traps}, below) with a location 
trigger that activates on any significant impact, such as a falling character. 

\textbf{Touch Attack:} This feature applies to any trap that needs only a successful 
touch attack (melee or ranged) to hit.

%%%%%%%%%%%%%%%%%%%%%%%%%
\subsection{Sample Traps}
%%%%%%%%%%%%%%%%%%%%%%%%%

The costs listed for mechanical traps are market prices; those for magic traps 
are raw material costs. Caster level and class for the spells used to produce the 
trap effects are provided in the entries for magic device traps and spell traps. 
For all other spells used (in triggers, for example), the caster level is assumed 
to be the minimum required.

%%%
\subsubsection{CR 1 Traps}
%%%

\textbf{Basic Arrow Trap:} CR 1; mechanical; proximity trigger; manual reset; Atk 
+10 ranged (1d6/x3, arrow); Search DC 20; Disable Device DC 20. \textit{Market 
Price:} 2,000 gp.

\textbf{Camouflaged Pit Trap:} CR 1; mechanical; location trigger; manual reset; 
DC 20 Reflex save avoids; 10 ft. deep (1d6, fall); Search DC 24; Disable Device 
DC 20. \textit{Market Price:} 1,800 gp.

\textbf{Deeper Pit Trap:} CR 1; mechanical; location trigger; manual reset; hidden 
switch bypass (Search DC 25); DC 15 Reflex save avoids; 20 ft. deep (2d6, fall); 
multiple targets (first target in each of two adjacent 5-ft. squares); Search DC 
20; Disable Device DC 23. \textit{Market Price:} 1,300 gp.

\textbf{Fusillade of Darts:} CR 1; mechanical; location trigger; manual reset; 
Atk +10 ranged (1d4+1, dart); multiple targets (fires 1d4 darts at each target 
in two adjacent 5-ft. squares); Search DC 14; Disable Device DC 20. \textit{Market 
Price:} 500 gp.

\textbf{Poison Dart Trap:} CR 1; mechanical; location trigger; manual reset; Atk 
+8 ranged (1d4 plus poison, dart); poison (bloodroot, DC 12 Fortitude save resists, 
0/1d4 Con plus 1d3 Wis); Search DC 20; Disable Device DC 18. \textit{Market Price: 
}700 gp.

\textbf{Poison Needle Trap:} CR 1; mechanical; touch trigger; manual reset; Atk 
+8 ranged (1 plus greenblood oil poison); Search DC 22; Disable Device DC 20. \textit{Market 
Price:} 1,300 gp.

\textbf{Portcullis Trap:} CR 1; mechanical; location trigger; manual reset; Atk 
+10 melee (3d6); Search DC 20; Disable Device DC 20. \textit{Note:} Damage applies 
only to those underneath the portcullis. Portcullis blocks passageway. \textit{Market 
Price:} 1,400 gp.

\textbf{Razor-Wire across Hallway:} CR 1; mechanical; location trigger; no reset; 
Atk +10 melee (2d6, wire); multiple targets (first target in each of two adjacent 
5-ft. squares); Search DC 22; Disable Device DC 15. \textit{Market Price:} 400 
gp.

\textbf{Rolling Rock Trap:} CR 1; mechanical; location trigger; manual reset; Atk 
+10 melee (2d6, rock); Search DC 20; Disable Device DC 22. \textit{Market Price: 
}1,400 gp.

\textbf{Scything Blade Trap:} CR 1; mechanical; location trigger; automatic reset; 
Atk +8 melee (1d8/x3); Search DC 21; Disable Device DC 20. \textit{Market Price: 
}1,700 gp.

\textbf{Spear Trap:} CR 1; mechanical; location trigger; manual reset; Atk +12 
ranged (1d8/x3, spear); Search DC 20; Disable Device DC 20. \textit{Note:} 200-ft. 
max range, target determined randomly from those in its path. Market Price: 1,200 
gp.

\textbf{Swinging Block Trap:} CR 1; mechanical; touch trigger; manual reset; Atk 
+5 melee (4d6, stone block); Search DC 20; Disable Device DC 20. \textit{Market 
Price:} 500 gp.

\textbf{Wall Blade Trap:} CR 1; mechanical; touch trigger; automatic reset; hidden 
switch bypass (Search DC 25); Atk +10 melee (2d4/x4, scythe); Search DC 22; Disable 
Device DC 22. \textit{Market Price:} 2,500 gp.

%%%
\subsubsection{CR 2 Traps}
%%%

\textbf{Box of Brown Mold:} CR 2; mechanical; touch trigger (opening the box); 
automatic reset; 5-ft. cold aura (3d6, cold nonlethal); Search DC 22; Disable Device 
DC 16. \textit{Market Price:} 3,000 gp.

\textbf{Bricks from Ceiling:} CR 2; mechanical; touch trigger; repair reset; Atk 
+12 melee (2d6, bricks); multiple targets (all targets in two adjacent 5-ft. squares); 
Search DC 20; Disable Device DC 20. \textit{Market Price:} 2,400 gp.

\textbf{Burning Hands Trap:} CR 2; magic device; proximity trigger 
\textit{(alarm); }automatic reset; spell effect (\linkspell{Burning Hands}, 1st-level 
wizard, 1d4 fire, DC 11 Reflex save half damage); Search DC 26; Disable Device 
DC 26. \textit{Cost:} 500 gp, 40 XP.

\textbf{Camouflaged Pit Trap:} CR 2; mechanical; location trigger; manual reset; 
DC 20 Reflex save avoids; 20 ft. deep (2d6, fall); multiple targets (first target 
in each of two adjacent 5-ft. squares); Search DC 24; Disable Device DC 19. \textit{Market 
Price:} 3,400 gp.

\textbf{Inflict Light Wounds Trap:} CR 2; magic device; touch 
trigger; automatic reset; spell effect (\linkspell{Inflict Light Wounds}, 1st-level 
cleric, 1d8+1, DC 11 Will save half damage); Search DC 26; Disable Device DC 26. 
\textit{Cost:} 500 gp, 40 XP.

\textbf{Javelin Trap:} CR 2; mechanical; location trigger; manual reset; Atk +16 
ranged (1d6+4, javelin); Search DC 20; Disable Device DC 18. \textit{Market Price: 
}4,800 gp.

\textbf{Large Net Trap:} CR 2; mechanical; location trigger; manual reset; Atk 
+5 melee (see note); Search DC 20; Disable Device DC 25. \textit{Note:} Characters 
in 10-ft. square are grappled by net (Str 18) if they fail a DC 14 Reflex save. 
\textit{Market Price:} 3,000 gp.

\textbf{Pit Trap:} CR 2; mechanical, location trigger; manual reset; DC 20 Reflex 
save avoids; 40 ft. deep (4d6, fall); Search DC 20; Disable Device DC 20. \textit{Market 
Price:} 2,000 gp.

\textbf{Poison Needle Trap:} CR 2; mechanical; touch trigger; repair reset; lock 
bypass (Open Lock DC 30); Atk +17 melee (1 plus poison, needle); poison (blue whinnis, 
DC 14 Fortitude save resists (poison only), 1 Con/unconsciousness); Search DC 22; 
Disable Device DC 17. \textit{Market Price:} 4,720 gp.

\textbf{Spiked Pit Trap:} CR 2; mechanical; location trigger; automatic reset; 
DC 20 Reflex save avoids; 20 ft. deep (2d6, fall); multiple targets (first target 
in each of two adjacent 5-ft. squares); pit spikes (Atk +10 melee, 1d4 spikes per 
target for 1d4+2 each); Search DC 18; Disable Device DC 15. \textit{Market Price: 
}1,600 gp.

\textbf{Tripping Chain:} CR 2; mechanical; location trigger; automatic reset; multiple 
traps (tripping and melee attack); Atk +15 melee touch (trip), Atk +15 melee (2d4+2, 
spiked chain); Search DC 15; Disable Device DC 18. \textit{Market Price:} 3,800 
gp. \textit{Note:} This trap is really one CR 1 trap that trips and a second CR 
1 trap that attacks with a spiked chain. If the tripping attack succeeds, a +4 
bonus applies to the spiked chain attack because the opponent is prone.

\textbf{Well-Camouflaged Pit Trap:} CR 2; mechanical; location trigger; repair 
reset; DC 20 Reflex save avoids; 10 ft. deep (1d6, fall); Search DC 27; Disable 
Device DC 20. \textit{Market Price:} 4,400 gp.

%%%
\subsubsection{CR 3 Traps}
%%%

\textbf{Burning Hands Trap:} CR 3; magic device; proximity trigger 
\textit{(alarm); }automatic reset; spell effect (\linkspell{Burning Hands}, 5th-level 
wizard, 5d4 fire, DC 11 Reflex save half damage); Search DC 26; Disable Device 
DC 26. \textit{Cost:} 2,500 gp, 200 XP.

\textbf{Camouflaged Pit Trap:} CR 3; mechanical; location trigger; manual reset; 
DC 20 Reflex save avoids; 30 ft. deep (3d6, fall); multiple targets (first target 
in each of two adjacent squares); Search DC 24; Disable Device DC 18. \textit{Market 
Price:} 4,800 gp.

\textbf{Ceiling Pendulum:} CR 3; mechanical; timed trigger; automatic reset; Atk 
+15 melee (1d12+8/x3, greataxe); Search DC 15; Disable Device DC 27. \textit{Market 
Price:} 14,100 gp.

\textbf{Fire Trap:} CR 3; spell; spell trigger; no reset; spell 
effect (\linkspell{Fire Trap}, 3rd-level druid, 1d4+3 fire, DC 13 Reflex save half 
damage); Search DC 27; Disable Device DC 27. \textit{Cost:} 85 gp to hire NPC spellcaster.

\textbf{Extended Bane Trap:} CR 3; magic device; proximity 
trigger \textit{(detect good); }automatic reset; spell effect (extended \linkspell{Bane}, 
3rd-level cleric, DC 13 Will save negates); Search DC 27; Disable Device DC 27. 
\textit{Cost:} 3,500 gp, 280 XP.

\textbf{Ghoul Touch Trap:} CR 3; magic device; touch trigger; 
automatic reset; spell effect (\linkspell{Ghoul Touch}, 3rd-level wizard, DC 13 Fortitude 
save negates); Search DC 27; Disable Device DC 27. \textit{Cost:} 3,000 gp, 240 
XP. 

\textbf{Hail of Needles:} CR 3; mechanical; location trigger; manual reset; Atk 
+20 ranged (2d4); Search DC 22; Disable Device DC 22. \textit{Market Price:} 5,400 
gp.

\textbf{Acid Arrow Trap:} CR 3; magic device; proximity trigger 
\textit{(alarm); }automatic reset; Atk +2 ranged touch; spell effect (\linkspell{Acid Arrow},
3rd-level wizard, 2d4 acid/round for 2 rounds); Search DC 27; Disable Device 
DC 27. \textit{Cost:} 3,000 gp, 240 XP.

\textbf{Pit Trap:} CR 3; mechanical, location trigger; manual reset; DC 20 Reflex 
save avoids; 60 ft. deep (6d6, fall); Search DC 20; Disable Device DC 20. \textit{Market 
Price:} 3,000 gp.

\textbf{Poisoned Arrow Trap:} CR 3; mechanical; touch trigger; manual reset; lock 
bypass (Open Lock DC 30); Atk +12 ranged (1d8 plus poison, arrow); poison (Large 
monstrous scorpion venom, DC 14 Fortitude save resists, 1d4 Con/1d4 Con); Search 
DC 19; Disable Device DC 15. \textit{Market Price:} 2,900 gp.

\textbf{Spiked Pit Trap:} CR 3; mechanical; location trigger; manual reset; DC 
20 Reflex save avoids; 20 ft. deep (2d6, fall); multiple targets (first target 
in each of two adjacent 5-ft. squares); pit spikes (Atk +10 melee, 1d4 spikes per 
target for 1d4+2 each); Search DC 21; Disable Device DC 20. \textit{Market Price: 
}3,600 gp.

\textbf{Stone Blocks from Ceiling:} CR 3; mechanical; location trigger; repair 
reset; Atk +10 melee (4d6, stone blocks); Search DC 25; Disable Device DC 20. \textit{Market 
Price:} 5,400 gp.

%%%
\subsubsection{CR 4 Traps}
%%%

\textbf{Bestow Curse Trap:} CR 4; magic device; touch trigger 
\textit{(detect chaos); }automatic reset; spell effect (\linkspell{Bestow Curse}, 5th-level 
cleric, DC 14 Will save negates); Search DC 28; Disable Device DC 28. \textit{Cost: 
}8,000 gp, 640 XP.

\textbf{Camouflaged Pit Trap:} CR 4; mechanical; location trigger; manual reset; 
DC 20 Reflex save avoids; 40 ft. deep (4d6, fall); multiple targets (first target 
in each of two adjacent 5-ft. squares); Search DC 25; Disable Device DC 17. \textit{Market 
Price:} 6,800 gp. 

\textbf{Collapsing Column:} CR 4; mechanical; touch trigger (attached); no reset; 
Atk +15 melee (6d6, stone blocks); Search DC 20; Disable Device DC 24. \textit{Market 
Price:} 8,800 gp.

\textbf{Glyph of Warding (Blast):} CR 4; spell; spell trigger; 
no reset; spell effect (\linkspell{Glyph of Warding} [blast], 5th-level cleric, 2d8 
acid, DC 14 Reflex save half damage); multiple targets (all targets within 5 ft.); 
Search DC 28; Disable Device DC 28. \textit{Cost:} 350 gp to hire NPC spellcaster.

\textbf{Lightning Bolt Trap:} CR 4; magic device; proximity trigger 
\textit{(alarm); }automatic reset; spell effect (\linkspell{Lightning Bolt}, 5th-level 
wizard, 5d6 electricity, DC 14 Reflex save half damage); Search DC 28; Disable 
Device DC 28. \textit{Cost:} 7,500 gp, 600 XP.

\textbf{Pit Trap:} CR 4; mechanical, location trigger; manual reset; DC 20 Reflex 
save avoids; 80 ft. deep (8d6, fall); Search DC 20; Disable Device DC 20. \textit{Market 
Price:} 4,000 gp.

\textbf{Poisoned Dart Trap:} CR 4; mechanical; location trigger; manual reset; 
Atk +15 ranged (1d4+4 plus poison, dart); multiple targets (1 dart per target in 
a 10-ft.-by-10-ft. area); poison (Small monstrous centipede poison, DC 10 Fortitude 
save resists, 1d2 Dex/1d2 Dex); Search DC 21; Disable Device DC 22. \textit{Market 
Price:} 12,090 gp.

\textbf{Sepia Snake Sigil Trap:} CR 4; spell; spell trigger; 
no reset; spell effect (\linkspell{Sepia Snake Sigil}, 5th-level wizard, DC 14 Reflex 
save negates); Search DC 28; Disable Device DC 28. \textit{Cost:} 650 gp to hire 
NPC spellcaster.

\textbf{Spiked Pit Trap:} CR 4; mechanical; location trigger; automatic reset; 
DC 20 Reflex save avoids; 60 ft. deep (6d6, fall); pit spikes (Atk +10 melee, 1d4 
spikes per target for 1d4+5 each); Search DC 20; Disable Device DC 20. \textit{Market 
Price:} 4,000 gp. 

\textbf{Wall Scythe Trap:} CR 4; mechanical; location trigger; automatic reset; 
Atk +20 melee (2d4+8/x4, scythe); Search DC 21; Disable Device DC 18. \textit{Market 
Price:} 17,200 gp.

\textbf{Water-Filled Room Trap:} CR 4; mechanical; location trigger; automatic 
reset; multiple targets (all targets in a 10-ft.-by-10-ft. room); never miss; onset 
delay (5 rounds); liquid; Search DC 17; Disable Device DC 23. \textit{Market Price: 
}11,200 gp.

\textbf{Wide-Mouth Spiked Pit Trap:} CR 4; mechanical; location trigger; manual 
reset; DC 20 Reflex save avoids; 20 ft. deep (2d6, fall); multiple targets (first 
target in each of two adjacent 5-ft. squares); pit spikes (Atk +10 melee, 1d4 spikes 
per target for 1d4+2 each); Search DC 18; Disable Device DC 25. \textit{Market 
Price:} 7,200 gp.

%%%
\subsubsection{CR 5 Traps}
%%%

\textbf{Camouflaged Pit Trap:} CR 5; mechanical; location trigger; manual reset; 
DC 20 Reflex save avoids; 50 ft. deep (5d6, fall); multiple targets (first target 
in each of two adjacent 5-ft. squares); Search DC 25; Disable Device DC 17. \textit{Market 
Price:} 8,500 gp.

\textbf{Doorknob Smeared with Contact Poison:} CR 5; mechanical; touch trigger 
(attached); manual reset; poison (nitharit, DC 13 Fortitude save resists, 0/3d6 
Con); Search DC 25; Disable Device DC 19. \textit{Market Price:} 9,650 gp.

\textbf{Falling Block Trap:} CR 5; mechanical; location trigger; manual reset; 
Atk +15 melee (6d6); multiple targets (can strike all characters in two adjacent 
specified squares); Search DC 20; Disable Device DC 25. \textit{Market Price:} 15,000 
gp.

\textbf{Fire Trap:}CR 5; spell; spell trigger; no reset; spell effect 
(\linkspell{Fire Trap}, 7th-level wizard, 1d4+7 fire, DC 16 Reflex save half damage); 
Search DC 29; Disable Device DC 29. \textit{Cost:} 305 gp to hire NPC spellcaster.

\textbf{Fireball Trap:} CR 5; magic device; touch trigger; automatic 
reset; spell effect (\linkspell{Fireball}, 8th-level wizard, 8d6 fire, DC 14 Reflex 
save half damage); Search DC 28; Disable Device DC 28. \textit{Cost:} 12,000 gp, 
960 XP.

\textbf{Flooding Room Trap:} CR 5; mechanical; proximity trigger; automatic reset; 
no attack roll necessary (see note below); Search DC 20; Disable Device DC 25. 
\textit{Note:} Room floods in 4 rounds. \textit{Market Price:} 17,500 gp.

\textbf{Fusillade of Darts:} CR 5; mechanical; location trigger; manual reset; 
Atk +18 ranged (1d4+1, dart); multiple targets (1d8 darts per target in a 10-ft.-by-10-ft. 
area); Search DC 19; Disable Device DC 25. \textit{Market Price:} 18,000 gp.

\textbf{Moving Executioner Statue:} CR 5; mechanical; location trigger; automatic 
reset; hidden switch bypass (Search DC 25); Atk +16 melee (1d12+8/x3, greataxe); 
multiple targets (both arms attack); Search DC 25; Disable Device DC 18. \textit{Market 
Price:} 22,500 gp.

\textbf{Phantasmal Killer Trap:} CR 5; magic device; proximity 
trigger (\textit{alarm }covering the entire room); automatic reset; spell effect 
(\linkspell{Phantasmal Killer}, 7th-level wizard, DC 16 Will save for disbelief and 
DC 16 Fort save for partial effect); Search DC 29; Disable Device DC 29. \textit{Cost: 
}14,000 gp, 1,120 XP.

\textbf{Pit Trap:} CR 5; mechanical, location trigger; manual reset; DC 20 Reflex 
save avoids; 100 ft. deep (10d6, fall); Search DC 20; Disable Device DC 20. \textit{Market 
Price:} 5,000 gp.

\textbf{Poison Wall Spikes:} CR 5; mechanical; location trigger; manual reset; 
Atk +16 melee (1d8+4 plus poison, spike); multiple targets (closest target in each 
of two adjacent 5-ft. squares); poison (Medium monstrous spider venom, DC 12 Fortitude 
save resists, 1d4 Str/1d4 Str); Search DC 17; Disable Device DC 21. \textit{Market 
Price:} 12,650 gp.

\textbf{Spiked Pit Trap:} CR 5; mechanical; location trigger; manual reset; DC 
25 Reflex save avoids; 40 ft. deep (4d6, fall); multiple targets (first target 
in each of two adjacent 5-ft. squares); pit spikes (Atk +10 melee, 1d4 spikes per 
target for 1d4+4 each); Search DC 21; Disable Device DC 20. \textit{Market Price: 
}13,500 gp.

\textbf{Spiked Pit Trap (80 Ft. Deep):} CR 5; mechanical; location trigger, manual 
reset; DC 20 Reflex save avoids; 80 ft. deep (8d6, fall), pit spikes (Atk +10 melee, 
1d4 spikes for 1d4+5 each); Search DC 20; Disable Device DC 20. \textit{Market 
Price:} 5,000 gp. 

\textbf{Ungol Dust Vapor Trap:} CR 5; mechanical; location trigger; manual reset; 
gas; multiple targets (all targets in a 10-ft.-by-10-ft. room); never miss; onset 
delay (2 rounds); poison (ungol dust, DC 15 Fortitude save resists, 1 Cha/1d6 Cha 
plus 1 Cha drain); Search DC 20; Disable Device DC 16. \textit{Market Price:} 9,000 
gp.

%%%
\subsubsection{CR 6 Traps }
%%%

\textbf{Built-to-Collapse Wall:} CR 6; mechanical; proximity trigger; no reset; 
Atk +20 melee (8d6, stone blocks); multiple targets (all targets in a 10-ft.-by-10-ft. 
area); Search DC 14; Disable Device DC 16. \textit{Market Price:} 15,000 gp.

\textbf{Compacting Room:} CR 6; mechanical; timed trigger; automatic reset; hidden 
switch bypass (Search DC 25); walls move together (12d6, crush); multiple targets 
(all targets in a 10-ft.-by- 10-ft. room); never miss; onset delay (4 rounds); 
Search DC 20; Disable Device DC 22. \textit{Market Price:} 25,200 gp.

\textbf{Flame Strike Trap:} CR 6; magic device; proximity trigger 
\textit{(detect magic); }automatic reset; spell effect (\linkspell{Flame Strike}, 9th-level 
cleric, 9d6 fire, DC 17 Reflex save half damage); Search DC 30; Disable Device 
DC 30. \textit{Cost:} 22,750 gp, 1,820 XP.

\textbf{Fusillade of Spears:} CR 6; mechanical; proximity trigger; repair reset; 
Atk +21 ranged (1d8, spear); multiple targets (1d6 spears per target in a 10 ft.-by-10-ft. 
area); Search DC 26; Disable Device DC 20. \textit{Market Price:} 31,200 gp.

\textbf{Glyph of Warding (Blast):} CR 6; spell; spell trigger; 
no reset; spell effect (\linkspell{Glyph of Warding} [blast], 16th-level cleric, 8d8 
sonic, DC 14 Reflex save half damage); multiple targets (all targets within 5 ft.); 
Search DC 28; Disable Device DC 28. \textit{Cost:} 680 gp to hire NPC spellcaster.

\textbf{Lightning Bolt Trap:} CR 6; magic device; proximity trigger 
\textit{(alarm); }automatic reset; spell effect (\linkspell{Lightning Bolt}, 10th-level 
wizard, 10d6 electricity, DC 14 Reflex save half damage); Search DC 28; Disable 
Device DC 28. \textit{Cost:} 15,000 gp, 1,200 XP.

\textbf{Spiked Blocks from Ceiling:} CR 6; mechanical; location trigger; repair 
reset; Atk +20 melee (6d6, spikes); multiple targets (all targets in a 10-ft.-by-10-ft. 
area); Search DC 24; Disable Device DC 20. \textit{Market Price:} 21,600 gp.

\textbf{Spiked Pit Trap (100 Ft. Deep):} CR 6; mechanical; location trigger, manual 
reset; DC 20 Reflex save avoids; 100 ft. deep (10d6, fall); pit spikes (Atk +10 
melee, 1d4 spikes per target for 1d4+5 each); Search DC 20; Disable Device DC 20. 
\textit{Market Price:} 6,000 gp.

\textbf{Whirling Poison Blades:} CR 6; mechanical; timed trigger; automatic reset; 
hidden lock bypass (Search DC 25, Open Lock DC 30); Atk +10 melee (1d4+4/19-20 
plus poison, dagger); poison (purple worm poison, DC 24 Fortitude save resists, 
1d6 Str/2d6 Str); multiple targets (one target in each of three preselected 5-ft. 
squares); Search DC 20; Disable Device DC 20. \textit{Market Price:} 30,200 gp.

\textbf{Wide-Mouth Pit Trap:} CR 6; mechanical; location trigger, manual reset; 
DC 25 Reflex save avoids; 40 ft. deep (4d6, fall); multiple targets (all targets 
within a 10-ft.-by-10-ft. area); Search DC 26; Disable Device DC 25. \textit{Market 
Price:} 28,200 gp.

\textbf{Wyvern Arrow Trap:} CR 6; mechanical; proximity trigger; manual reset; 
Atk +14 ranged (1d8 plus poison, arrow); poison (wyvern poison, DC 17 Fortitude 
save resists, 2d6 Con/2d6 Con); Search DC 20; Disable Device DC 16. \textit{Market 
Price:} 17,400 gp. 

%%%
\subsubsection{CR 7 Traps}
%%%

\textbf{Acid Fog Trap:} CR 7; magic device; proximity trigger 
\textit{(alarm); }automatic reset; spell effect (\linkspell{Acid Fog}, 11th-level 
wizard, 2d6/round acid for 11 rounds); Search DC 31; Disable Device DC 31. \textit{Cost: 
}33,000 gp, 2,640 XP.

\textbf{Blade Barrier Trap:} CR 7; magic device; proximity trigger 
\textit{(alarm); }automatic reset; spell effect (\linkspell{Blade Barrier}, 11th-level 
cleric, 11d6 slashing, DC 19 Reflex save half damage); Search DC 31; Disable Device 
DC 31. \textit{Cost:} 33,000 gp, 2,640 XP.

\textbf{Burnt Othur Vapor Trap:} CR 7; mechanical; location trigger; repair reset; 
gas; multiple targets (all targets in a 10-ft.-by-10-ft. room); never miss; onset 
delay (3 rounds); poison (burnt othur fumes, DC 18 Fortitude save resists, 1 Con 
drain/3d6 Con); Search DC 21; Disable Device DC 21. \textit{Market Price:} 17,500 
gp.

\textbf{Chain Lightning Trap:} CR 7; magic device; proximity 
trigger \textit{(alarm); }automatic reset; spell effect (\linkspell{Chain Lightning}, 
11th-level wizard, 11d6 electricity to target nearest center of trigger area plus 
5d6 electricity to each of up to eleven secondary targets, DC 19 Reflex save half 
damage); Search DC 31; Disable Device DC 31. \textit{Cost:} 33,000 gp, 2,640 XP. 

\textbf{Black Tentacles Trap:} CR 7; magic device; proximity 
trigger \textit{(alarm); }no reset; spell effect (\linkspell{Black Tentacles}, 7th-level 
wizard, 1d4+7 tentacles, Atk +7 melee [1d6+4, tentacle]); multiple targets (up 
to six tentacles per target in each of two adjacent 5-ft. squares); Search DC 29; 
Disable Device DC 29. \textit{Cost:} 1,400 gp, 112 XP.

\textbf{Fusillade of Greenblood Oil Darts:} CR 7; mechanical; location trigger; 
manual reset; Atk +18 ranged (1d4+1 plus poison, dart); poison (greenblood oil, 
DC 13 Fortitude save resists, 1 Con/ 1d2 Con); multiple targets (1d8 darts per 
target in a 10-ft.-by-10-ft. area); Search DC 25; Disable Device DC 25. \textit{Market 
Price:} 33,000 gp.

\textbf{Lock Covered in Dragon Bile:} CR 7; mechanical; touch trigger (attached); 
no reset; poison (dragon bile, DC 26 Fortitude save resists, 3d6 Str/0); Search 
DC 27; Disable Device DC 16. \textit{Market Price:} 11,300 gp.

\textbf{Summon Monster VI Trap:} CR 7; magic device; proximity 
trigger \textit{(alarm); }no reset; spell effect (\linkspell{Summon Monster VI}, 11th-level 
wizard), Search DC 31; Disable Device DC 31. \textit{Cost:} 3,300 gp, 264 XP.

\textbf{Water-Filled Room:} CR 7; mechanical; location trigger; manual reset; multiple 
targets (all targets in a 10-ft.-by-10-ft. room); never miss; onset delay (3 rounds); 
water; Search DC 20; Disable Device DC 25. \textit{Market Price:} 21,000 gp. 

\textbf{Well-Camouflaged Pit Trap:} CR 7; mechanical; location trigger; repair 
reset; DC 25 Reflex save avoids; 70 ft. deep (7d6, fall); multiple targets (first 
target in each of two adjacent 5-ft. squares); Search DC 27; Disable Device DC 
18. \textit{Market Price:} 24,500 gp. 

%%%
\subsubsection{CR 8 Traps}
%%%

\textbf{Deathblade Wall Scythe:} CR 8; mechanical; touch trigger; manual reset; 
Atk +16 melee (2d4+8 plus poison, scythe); poison (deathblade, DC 20 Fortitude 
save resists, 1d6 Con/2d6 Con); Search DC 24; Disable Device DC 19. \textit{Market 
Price:} 31,400 gp.

\textbf{Destruction Trap:} CR 8; magic device; touch trigger 
\textit{(alarm); }automatic reset; spell effect (\linkspell{Destruction}, 13th-level 
cleric, DC 20 Fortitude save for 10d6 damage); Search DC 32; Disable Device DC 
32. \textit{Cost:} 45,500 gp, 3,640 XP.

\textbf{Earthquake Trap:} CR 8; magic device; proximity trigger 
\textit{(alarm); }automatic reset; spell effect (\linkspell{Earthquake}, 13th-level 
cleric, 65-ft. radius, DC 15 or 20 Reflex save, depending on terrain); Search DC 
32; Disable Device DC 32. \textit{Cost:} 45,500 gp, 3,640 XP.

\textbf{Insanity Mist Vapor Trap:} CR 8; mechanical; location trigger; repair reset; 
gas; never miss; onset delay (1 round); poison (insanity mist, DC 15 Fortitude 
save resists, 1d4 Wis/2d6 Wis); multiple targets (all targets in a 10-ft.-by-10-ft. 
room); Search DC 25; Disable Device DC 20. \textit{Market Price:} 23,900 gp.

\textbf{Acid Arrow Trap:} CR 8; magic device; visual trigger 
\textit{(true seeing); }automatic reset; multiple traps (two simultaneous \linkspell{Acid Arrow}
traps); Atk +9 ranged touch and +9 ranged touch; spell effect (\textit{acid 
arrow}, 18th-level wizard, 2d4 acid damage for 7 rounds); Search DC 27; Disable 
Device DC 27. \textit{Cost:} 83,500 gp, 4,680 XP. \textit{Note:} This trap is really 
two CR 6 \textit{acid arrow} traps that fire simultaneously, using the same trigger 
and reset.

\textbf{Power Word Stun Trap:} CR 8; magic device; touch trigger; 
no reset; spell effect (\linkspell{Power Word Stun}, 13th-level wizard), Search DC 
32; Disable Device DC 32. \textit{Cost:} 4,550 gp, 364 XP.

\textbf{Prismatic Spray Trap:} CR 8; magic device; proximity 
trigger \textit{(alarm); }automatic reset; spell effect (\linkspell{Prismatic Spray}, 
13th-level wizard, DC 20 Reflex, Fortitude, or Will save, depending on effect); 
Search DC 32; Disable Device DC 32. \textit{Cost:} 45,500 gp, 3,640 XP.

\textbf{Reverse Gravity Trap:} CR 8; magic device; proximity 
trigger (\textit{alarm, }10-ft. area); automatic reset; spell effect (\linkspell{Reverse Gravity},
13th-level wizard, 6d6 fall [upon hitting the ceiling of the 60-ft.- 
high room], then 6d6 fall [upon falling 60 ft. to the floor when the spell ends], 
DC 20 Reflex save avoids damage); Search DC 32; Disable Device DC 32. \textit{Cost: 
}45,500 gp, 3,640 XP.

\textbf{Well-Camouflaged Pit Trap:} CR 8; mechanical; location trigger; repair 
reset; DC 20 Reflex save avoids; 100 ft. deep (10d6, fall); Search DC 27; Disable 
Device DC 18. \textit{Market Price:} 16,000 gp.

\textbf{Word of Chaos Trap:} CR 8; magic device; proximity trigger 
\textit{(detect law); }automatic reset; spell effect (\linkspell{Word of Chaos}, 13th-level 
cleric); Search DC 32; Disable Device DC 32. \textit{Cost:} 46,000 gp, 3,680 XP.

%%%
\subsubsection{CR 9 Traps}
%%%

\textbf{Drawer Handle Smeared with Contact Poison:} CR 9; mechanical; touch trigger 
(attached); manual reset; poison (black lotus extract, DC 20 Fortitude save resists, 
3d6 Con/3d6 Con); Search DC 18; Disable Device DC 26. \textit{Market Price:} 21,600 
gp. 

\textbf{Dropping Ceiling:} CR 9; mechanical; location trigger; repair reset; ceiling 
moves down (12d6, crush); multiple targets (all targets in a 10-ft.-by-10-ft. room); 
never miss; onset delay (1 round); Search DC 20; Disable Device DC 16. \textit{Market 
Price:} 12,600 gp. 

\textbf{Incendiary Cloud Trap:} CR 9; magic device; proximity 
trigger \textit{(alarm); }automatic reset; spell effect (\linkspell{Incendiary Cloud}, 
15th-level wizard, 4d6/round for 15 rounds, DC 22 Reflex save half damage); Search 
DC 33; Disable Device DC 33. \textit{Cost:} 60,000 gp, 4,800 XP. 

\textbf{Wide-Mouth Pit Trap:} CR 9; mechanical; location trigger; manual reset; 
DC 25 Reflex save avoids; 100 ft. deep (10d6, fall); multiple targets (all targets 
within a 10-ft.-by-10-ft. area); Search DC 25; Disable Device DC 25. \textit{Market 
Price:} 40,500 gp. 

\textbf{Wide-Mouth Spiked Pit with Poisoned Spikes:} CR 9; mechanical; location 
trigger; manual reset; hidden lock bypass (Search DC 25, Open Lock DC 30); DC 
20 Reflex save avoids; 70 ft. deep (7d6, fall); multiple targets (all targets within 
a 10-ft.-by-10-ft. area); pit spikes (Atk +10 melee, 1d4 spikes per target for 
1d4+5 plus poison each); poison (giant wasp poison, DC 14 Fortitude save resists, 
1d6 Dex/1d6 Dex); Search DC 20; Disable Device DC 20. \textit{Market Price:} 11,910 
gp.

%%%
\subsubsection{CR 10 Traps}
%%%

\textbf{Crushing Room:} CR 10; mechanical; location trigger; automatic reset; walls 
move together (16d6, crush); multiple targets (all targets in a 10-ft.-by-10-ft. 
room); never miss; onset delay (2 rounds); Search DC 22; Disable Device DC 20. 
\textit{Market Price:} 29,000 gp.

\textbf{Crushing Wall Trap:} CR 10; mechanical; location trigger; automatic reset; 
no attack roll required (18d6, crush); Search DC 20; Disable Device DC 25. Market 
Price: 25,000 gp. 

\textbf{Energy Drain Trap:} CR 10; magic device; visual trigger 
(\textit{true seeing}); automatic reset; Atk +8 ranged touch; spell effect (\linkspell{Energy Drain}, 
17th-level wizard, 2d4 negative levels for 24 hours, DC 23 Fortitude save 
negates); Search DC 34; Disable Device DC 34. \textit{Cost:} 124,000 gp, 7,920 
XP.

\textbf{Forcecage and Summon Monster VII trap:} 
CR 10; magic device; proximity trigger \textit{(alarm); }automatic reset; multiple 
traps (one \linkspell{Forcecage} trap and one \linkspell{Summon Monster VII} trap that 
summons a hamatula); spell effect (\textit{Forcecage}, 13th-level wizard), spell 
effect (\textit{Summon Monster VII}, 13th-level wizard, hamatula); Search DC 32; 
Disable Device DC 32. \textit{Cost:} 241,000 gp, 7,280 XP. \textit{Note:} This 
trap is really one CR 8 trap that creates a \textit{Forcecage} and a second CR 
8 trap that summons a hamatula in the same area. If both succeed, the hamatula 
appears inside the \textit{Forcecage}. These effects are independent of each other.

\textbf{Poisoned Spiked Pit Trap:} CR 10; mechanical; location trigger; manual 
reset; hidden lock bypass (Search DC 25, Open Lock DC 30); DC 20 Reflex save avoids; 
50 ft. deep (5d6, fall); multiple targets (first target in each of two adjacent 
5-ft. squares); pit spikes (Atk +10 melee, 1d4 spikes per target for 1d4+5 plus 
poison each); poison (purple worm poison, DC 24 Fortitude save resists, 1d6 Str/2d6 
Str); Search DC 16; Disable Device DC 25. \textit{Market Price:} 19,700 gp. 

\textbf{Wail of the Banshee Trap:} CR 10; magic device; proximity 
trigger \textit{(alarm); }automatic reset; spell effect (\linkspell{Wail of the Banshee},
17th-level wizard, DC 23 Fortitude save negates); multiple targets (up to 17 creatures); 
Search DC 34; Disable Device DC 34. \textit{Cost:} 76,500 gp, 6,120 XP.

%%%%%%%%%%%%%%%%%%%%%%%%%
\subsection{Designing A Trap}
%%%%%%%%%%%%%%%%%%%%%%%%%

\textbf{Mechanical Traps:} Simply select the elements you want the trap to have 
and add up the adjustments to the trap's Challenge Rating that those elements require 
(see Table: CR Modifiers for Mechanical Traps) to arrive at the trap's final CR. 
From the CR you can derive the DC of the Craft (trapmaking) checks a character 
must make to construct the trap.

\textbf{Magic Traps:} As with mechanical traps, you don't have to do anything other 
than decide what elements you want and then determine the CR of the resulting trap 
(see Table: CR Modifiers for Magic Traps). If a player character wants to design 
and construct a magic trap, he must have the Craft Wondrous Item feat. In addition, 
he must be able to cast the spell or spells that the trap requires -- or, failing 
that, he must be able to hire an NPC to cast the spells for him.

%%%
\subsubsection{Challenge Rating of a Trap}
%%%

\begin{table}[htb]
\rowcolors{1}{white}{offyellow}
\caption{CR Modifiers for Mechanical Traps}
\centering
\begin{tabular}{l l}
\textbf{Feature} & \textbf{CR Modifier}\\
\textbf{Search DC}&\\
15 or lower &-1\\
25-29 &+1\\
30 or higher &+2\\
\textbf{Disable Device DC}&\\
15 or lower &-1\\
25-29 &+1\\
30 or higher &+2\\
\textbf{Reflex Save DC (Pit or Other Save-Dependent Trap)}&\\
15 or lower &-1\\
16-24 &+0\\
25-29 &+1\\
30 or higher &+2\\
\textbf{Attack Bonus (Melee or Ranged Attack Trap)}&\\
+0 or lower &-2\\
+1 to +5 &-1\\
+6 to +14 &+0\\
+15 to +19 &+1\\
+20 to +24 &+2\\
\textbf{Damage/Effect}&\\
Average damage &+1/7 points\textsuperscript{*}\\
\textbf{Miscellaneous Features}&\\
Alchemical device &Level of spell mimicked\\
Liquid &+5\\
Multiple target &+1 (or 0 if never miss)\\
Onset delay 1 round &+3\\
Onset delay 2 rounds &+2\\
Onset delay 3 rounds &+1\\
Onset delay 4+ rounds &-1\\
Poison & By poison (see below)\\
Pit spikes &+1\\
Touch attack &+1\\
\multicolumn{2}{l}{\textsuperscript{*} Rounded to the nearest multiple of 7 (round up for an average that lies exactly between two numbers).}\\
\end{tabular}
\end{table}

\begin{table}[htb]
\rowcolors{1}{white}{offyellow}
\caption{Poison Modifers for Mechanical Trap CRs}
\centering
\begin{tabular}{l c l c}
Black adder venom &+1 &Large scorpion venom &+3\\
Black lotus extract &+8 &Malyss root paste &+3\\
Bloodroot &+1 &Medium spider venom &+2\\
Blue whinnis &+1 &Nitharit &+4\\
Burnt othur fumes &+6 &Purple worm poison &+4\\
Deathblade &+5 &Sassone leaf residue &+3\\
Dragon bile &+6 &Shadow essence &+3\\
Giant wasp poison &+3 &Small centipede poison &+1\\
Greenblood oil &+1 &Terinav root &+5\\
Insanity mist &+4 &Ungol dust &+3\\
Wyvern poison &+5 & &\\
\end{tabular}
\end{table}

\begin{table}[htb]
\rowcolors{1}{white}{offyellow}
\caption{CR Modifiers for Magic Traps}
\centering
\begin{tabular}{l l}
\textbf{Feature} & \textbf{Cost Modifier}\\
Highest-level spell & +Spell Level OR +1 per 7 points of average damage per round\textsuperscript{*}\\
\multicolumn{2}{l}{\textsuperscript{*}See the note following Table: CR Modifiers for Mechanical Traps.}\\
\end{tabular}
\end{table}

To calculate the Challenge Rating of a trap, add all the CR modifiers (see the 
tables below) to the base CR for the trap type.

\textbf{Mechanical Trap:} The base CR for a mechanical trap is 0. If your final 
CR is 0 or lower, add features until you get a CR of 1 or higher.

\textbf{Magic Trap:} For a spell trap or magic device trap, the base CR is 1. The 
highest-level spell used modifies the CR (see Table: CR Modifiers for Magic Traps).

\textbf{Average Damage:} If a trap (either mechanical or magic) does hit point 
damage, calculate the average damage for a successful hit and round that value 
to the nearest multiple of 7. Use this value to adjust the Challenge Rating of 
the trap, as indicated on the tables below. Damage from poisons and pit spikes 
does not count toward this value, but damage from a high strength rating and extra 
damage from multiple attacks does.

For a magic trap, only one modifier applies to the CR -- either the level of the 
highest-level spell used in the trap, or the average damage figure, whichever is 
larger.

\textbf{Multiple Traps:} If a trap is really two or more connected traps that affect 
approximately the same area, determine the CR of each one separately.

\textit{Multiple Dependent Traps:} If one trap depends on the success of the other 
(that is, you can avoid the second trap altogether by not falling victim to the 
first), they must be treated as separate traps.

\textit{Multiple Independent Traps:} If two or more traps act independently (that 
is, none depends on the success of another to activate), use their CRs to determine 
their combined Encounter Level as though they were monsters\textit{. }The resulting 
Encounter Level is the CR for the combined traps.

%%%
\subsubsection{Mechanical Trap Cost}
%%%

The base cost of a mechanical trap is 1,000 gp. Apply all the modifiers from Table: 
Cost Modifiers for Mechanical Traps for the various features you've added to the 
trap to get the modified base cost.

The final cost is equal to (modified base cost x Challenge Rating) + extra costs. 
The minimum cost for a mechanical trap is (CR x 100) gp.

After you've multiplied the modified base cost by the Challenge Rating, add the 
price of any alchemical items or poison you incorporated into the trap. If the 
trap uses one of these elements and has an automatic reset, multiply the poison 
or alchemical item cost by 20 to provide an adequate supply of doses.

\textbf{Multiple Traps:} If a trap is really two or more connected traps, determine 
the final cost of each separately, then add those values together. This holds for 
both multiple dependent and multiple independent traps (see the previous section).

%%%
\subsubsection{Magic Device Trap Cost}
%%%

Building a magic device trap involves the expenditure of experience points as well 
as gold pieces, and requires the services of a spellcaster. Table: Cost Modifiers 
for Magic Device Traps summarizes the cost information for magic device traps. 
If the trap uses more than one spell (for instance, a sound or visual trigger spell 
in addition to the main spell effect), the builder must pay for them all (except 
\textit{alarm}, which is free unless it must be cast by an NPC; see below).

The costs derived from Table: Cost Modifiers for Magic Device Traps assume that 
the builder is casting the necessary spells himself (or perhaps some other PC is 
providing the spells for free). If an NPC spellcaster must be hired to cast them 
those costs must be factored in as well.

A magic device trap takes one day to construct per 500 gp of its cost.

\begin{table}[htb]
\rowcolors{1}{white}{offyellow}
\caption{Cost Modifiers for Magic Device Traps}
\centering
\begin{tabular}{l l}
\textbf{Feature} & \textbf{Cost Modifier}\\
Alarm spell used in trigger & -\\
\textbf{One-Shot Trap}&\\
\hspace{.2cm}Each spell used in trap &\shortstack{+50 gp x caster level x spell level,\\+4 XP x caster level x spell level}\\
&\\
\hspace{.2cm}Material components &+ Cost of all material components\\
\hspace{.2cm}XP components &+ Total of XP components x 5 gp\\
\textbf{Automatic Reset Trap}&\\
\hspace{.2cm}Each spell used in trap &\shortstack{+500 gp x caster level x spell level,\\+40 XP x caster level x spell level}\\
&\\
\hspace{.2cm}Material components &+ Cost of all material components x 100 gp\\
\hspace{.2cm}XP components &+ Total of XP components x 500 gp\\
\end{tabular}
\end{table}

%%%
\subsubsection{Spell Trap Cost}
%%%

A spell trap has a cost only if the builder must hire an NPC spellcaster to cast 
it.

%%%
\subsubsection{Craft DCs for Mechanical Traps}
%%%

\begin{table}[htb]
\rowcolors{1}{white}{offyellow}
\caption{Base Mechanical Craft (Trapmaking) DCs}
\centering
\begin{tabular}{l r}
\textbf{Trap CR} & \textbf{Base Craft (Trapmaking) DC}\\
1-3 & 20\\
4-6 & 25\\
7-10 & 30\\
\end{tabular}
\end{table}

\begin{table}[htb]
\rowcolors{1}{white}{offyellow}
\caption{Mechanical Craft (Trapmaking) DC Modifiers}
\centering
\begin{tabular}{l r}
\textbf{Additional Components} & \textbf{DC Modifier}\\
Proximity Trigger & +5\\
Automatic Reset & +5\\
\end{tabular}
\end{table}

Once you know the Challenge Rating of a trap determine the Craft (trapmaking) DC 
by referring to the table and the modifiers given below.

\textbf{Making the Checks:} To determine how much progress a character makes on 
building a trap each week, that character makes a Craft (trapmaking) check. See 
the Craft skill description for details on Craft checks and the circumstances that 
can affect them.


\begin{table}[htb]
\rowcolors{1}{white}{offyellow}
\caption{Cost Modifiers for Mechanical Traps}
\centering
\begin{tabular}{l l}
\textbf{Feature} & \textbf{Cost Modifier}\\
\textbf{Trigger Type}&\\
Location &-\\
Proximity &+1,000 gp\\
Touch &-\\
Touch (attached) &-100 gp\\
Timed &+1,000 gp\\
\textbf{Reset Type}&\\
No reset &-500 gp\\
Repair &-200 gp\\
Manual &-\\
Automatic &+500 gp (or 0 if trap has timed trigger)\\
\textbf{Bypass Type}&\\
Lock &+100 gp (Open Lock DC 30)\\
Hidden switch &+200 gp (Search DC 25)\\
Hidden lock &+300 gp (Open Lock DC 30, Search DC 25)\\
\textbf{Search DC}&\\
19 or lower &-100 gp x (20 - DC)\\
20 &-\\
21 or higher &+200 gp x (DC - 20)\\
\textbf{Disable Device DC}&\\
19 or lower &-100 gp x (20 - DC)\\
20 &-\\
21 or higher &+200 gp x (DC - 20)\\
\textbf{Reflex Save DC (Pit or Other Save-Dependent Trap)}&\\
19 or lower &-100 gp x (20 - DC)\\
20 &-\\
21 or higher &+300 gp x (DC - 20)\\
\textbf{Attack Bonus (Melee or Ranged Attack Trap)}&\\
+9 or lower &-100 gp x (10 - bonus)\\
+10 &-\\
+11 or higher &+200 gp x (bonus - 10)\\
\textbf{Damage Bonus}&\\
High strength rating (ranged attack trap) &+100 gp x bonus (max +4)\\
High Strength bonus (melee attack trap) &+100 gp x bonus (max +8)\\
\textbf{Miscellaneous Features}&\\
Never miss &+1,000 gp\\
Poison &Cost of poison\textsuperscript{*}\\
Alchemical item &Cost of item\textsuperscript{*}\\
\multicolumn{2}{l}{\textsuperscript{*} Multiply cost by 20 if trap features automatic reset.}\\
\end{tabular}
\end{table}

\clearpage

%%%%%%%%%%%%%%%%%%%%%%%%%%%%%%%%%%%%%%%%%%%%%%%%%%
\section{The Planes}\index{Planes}
%%%%%%%%%%%%%%%%%%%%%%%%%%%%%%%%%%%%%%%%%%%%%%%%%%

%%%%%%%%%%%%%%%%%%%%%%%%%
\subsection{What Is A Plane?}
%%%%%%%%%%%%%%%%%%%%%%%%%

The planes of existence are different realities with interwoven connections. Except 
for rare linking points, each plane is effectively its own universe with its own 
natural laws. 

The planes break down into a number of general types: the Material Plane, the Transitive 
Planes, the Inner Planes, the Outer Planes, and the demiplanes.

\textbf{\gameterm{Material Plane}:} The Material Plane tends to be the most Earthlike of all 
planes and operates under the same set of natural laws that our own real world 
does. This is the default plane for most adventures.

\textbf{\gameterm{Transitive Planes}:} These three planes have one important common characteristic: 
Each is used to get from one place to another. The Astral Plane is a conduit to 
all other planes, while the Ethereal Plane and the Plane of Shadow both serve as 
means of transportation within the Material Plane they're connected to. These planes 
have the strongest regular interaction with the Material Plane and are often accessed 
by using various spells. They have native inhabitants as well.

\textbf{\gameterm{Inner Planes}:} These six planes are manifestations of the basic building 
blocks of the universe. Each is made up of a single type of energy or element that 
overwhelms all others. The natives of a particular Inner Plane are made of the 
same energy or element as the plane itself.

\textbf{\gameterm{Outer Planes}:} The deities live on the Outer Planes, as do creatures such 
as celestials, demons, and devils. Each of the Outer Planes has an alignment, representing 
a particular moral or ethical outlook, and the natives of each plane tend to behave 
in agreement with that plane's alignment. The Outer Planes are also the final resting 
place of souls from the Material Plane, whether that final rest takes the form 
of calm introspection or eternal damnation.

\textbf{\gameterm{Demiplanes}:} This catch-all category covers all extradimensional spaces 
that function like planes but have measurable size and limited access. Other kinds 
of planes are theoretically infinite in size, but a demiplane might be only a few 
hundred feet across.

%%%%%%%%%%%%%%%%%%%%%%%%%
\subsection{Planar Traits}
%%%%%%%%%%%%%%%%%%%%%%%%%

Each plane of existence has its own properties -- the natural laws of its universe.

Planar traits are broken down into a number of general areas.

All planes have the following kinds of traits.

\textbf{Physical Traits:} These traits determine the laws of physics and nature 
on the plane, including how gravity and time function.

\textbf{Elemental and Energy Traits:} These traits determine the dominance of particular 
elemental or energy forces.

\textbf{Alignment Traits:} Just as characters may be lawful neutral or chaotic 
good, many planes are tied to a particular moral or ethical outlook.

\textbf{Magic Traits:} Magic works differently from plane to plane, and magic traits 
set the boundaries for what it can and can't do.

%%%
\subsubsection{Physical Traits}
%%%

The two most important natural laws set by physical traits are how gravity works 
and how time passes. Other physical traits pertain to the size and shape of a plane 
and how easily a plane's nature can be altered.

%%%
\subsubsection{Gravity}
%%%

The direction of gravity's pull may be unusual, and it might 
even change directions within the plane itself.

\textit{Normal Gravity:} Most planes have gravity similar to that of the Material 
Plane. The usual rules for ability scores, carrying capacity, and encumbrance apply. 
Unless otherwise noted in a description, it is assumed every plane has the normal 
gravity trait.

\textit{Heavy Gravity:} The gravity on a plane with this trait is much more intense 
than on the Material Plane. As a result, Balance, Climb, Jump, Ride, Swim, and 
Tumble checks incur a -2 circumstance penalty, as do all attack rolls. All item 
weights are effectively doubled, which might affect a character's speed. Weapon 
ranges are halved. A character's Strength and Dexterity scores are not affected. 
Characters who fall on a heavy gravity plane take 1d10 points of damage for each 
10 feet fallen, to a maximum of 20d10 points of damage.

\textit{Light Gravity:} The gravity on a plane with this trait is less intense 
than on the Material Plane. As a result, creatures find that they can lift more, 
but their movements tend to be ungainly. Characters on a plane with the light gravity 
trait take a -2 circumstance penalty on attack rolls and Balance, Ride, Swim, and 
Tumble checks. All items weigh half as much. Weapon ranges double, and characters 
gain a +2 circumstance bonus on Climb and Jump checks.

Strength and Dexterity don't change as a result of light gravity, but what you 
can do with such scores does change. These advantages apply to travelers from other 
planes as well as natives.

Falling characters on a light gravity plane take 1d4 points of damage for each 
10 feet of the fall (maximum 20d4).

\textit{No Gravity:} Individuals on a plane with this trait merely float in space, 
unless other resources are available to provide a direction for gravity's pull.

\textit{Objective Directional Gravity:} The strength of gravity on a plane with 
this trait is the same as on the Material Plane, but the direction is not the traditional 
"down" toward the ground. It may be down toward any solid object, at an angle 
to the surface of the plane itself, or even upward.

In addition, objective directional gravity may change from place to place. The 
direction of "down" may vary.

\textit{Subjective Directional Gravity:} The strength of gravity on a plane with 
this trait is the same as on the Material Plane, but each individual chooses the 
direction of gravity's pull. Such a plane has no gravity for unattended objects 
and non-sentient creatures. This sort of environment can be very disorienting to 
the newcomer, but is common on "weightless" planes.

Characters on a plane with subjective directional gravity can move normally along 
a solid surface by imagining "down" near their feet. If suspended in midair, 
a character "flies" by merely choosing a "down" direction and "falling" that 
way. Under such a procedure, an individual "falls" 150 feet in the first round 
and 300 feet in each succeeding round. Movement is straight-line only. In order 
to stop, one has to slow one's movement by changing the designated "down" direction 
(again, moving 150 feet in the new direction in the first round and 300 feet per 
round thereafter).

It takes a DC 16 Wisdom check to set a new direction of gravity as a free action; 
this check can be made once per round. Any character who fails this Wisdom check 
in successive rounds receives a +6 bonus on subsequent checks until he or she succeeds.

%%%
\subsubsection{Time}
%%%

The rate of time's passage can vary on different planes, though 
it remains constant within any particular plane. Time is always subjective for 
the viewer. The same subjectivity applies to various planes. Travelers may discover 
that they'll pick up or lose time while moving among the planes, but from their 
point of view, time always passes naturally.

\textit{Normal Time:} This trait describes the way time passes on the Material 
Plane. One hour on a plane with normal time equals one hour on the Material Plane. 
Unless otherwise noted in a description, every plane has the normal time trait.

\textit{Timeless:} On planes with this trait, time still passes, but the effects 
of time are diminished. How the timeless trait can affect certain activities or 
conditions such as hunger, thirst, aging, the effects of poison, and healing varies 
from plane to plane.

The danger of a timeless plane is that once one leaves such a plane for one where 
time flows normally, conditions such as hunger and aging do occur retroactively. 

\textit{Flowing Time:} On some planes, time can flow faster or slower. One may 
travel to another plane, spend a year there, then return to the Material Plane 
to find that only six seconds have elapsed. Everything on the plane returned to 
is only a few seconds older. But for that traveler and the items, spells, and effects 
working on him, that year away was entirely real.

When designating how time works on planes with flowing time, put the Material Plane's 
flow of time first, followed by the same flow in the other plane. 

\textit{Erratic Time:} Some planes have time that slows down and speeds up, so 
an individual may lose or gain time as he moves between the two planes. The following 
is provided as an example.

\begin{table}[htb]
\rowcolors{1}{white}{offyellow}
\caption{Erratic Time Passage}
\centering
\begin{tabular}{ccc}
\textbf{d\%} & \textbf{Time on Material Plane} & \textbf{Time on Erratic Time Plane}\\
01-10 & 1 day & 1 round\\
11-40 & 1 day & 1 hour\\
41-60 & 1 day & 1 day\\
61-90 & 1 hour & 1 day\\
91-100 & 1 round & 1 day\\
\end{tabular}
\end{table}

To the denizens of such a plane, time flows naturally and the shift is unnoticed.

If a plane is timeless with respect to magic, any spell cast with a non-instantaneous 
duration is permanent until dispelled.

%%%
\subsubsection{Shape and Size}
%%%

Planes come in a variety of sizes and shapes. Most planes 
are infinite, or at least so large that they may as well be infinite.

\textit{Infinite:} Planes with this trait go on forever, though they may have finite 
components within them. Or they may consist of ongoing expanses in two directions, 
like a map that stretches out infinitely.

\textit{Finite Shape:} A plane with this trait has defined edges or borders. These 
borders may adjoin other planes or hard, finite borders such as the edge of the 
world or a great wall. Demiplanes are often finite.

\textit{Self-Contained Shape:} On planes with this trait, the borders wrap in on 
themselves, depositing the traveler on the other side of the map. A spherical plane 
is an example of a self-contained, finite plane, but there can be cubes, toruses, 
and flat planes with magical edges that teleport the traveler to an opposite edge 
when he crosses them. 

Some demiplanes are self-contained.

%%%
\subsubsection{Morphic Traits}
%%%

This trait measures how easily the basic nature of a plane 
can be changed. Some planes are responsive to sentient thought, while others can 
be manipulated only by extremely powerful creatures. And some planes respond to 
physical or magical efforts.

\textit{Alterable Morphic:} On a plane with this trait, objects remain where they 
are (and what they are) unless affected by physical force or magic. You can change 
the immediate environment as a result of tangible effort. 

\textit{Highly Morphic:} On a plane with this trait, features of the plane change 
so frequently that it's difficult to keep a particular area stable. Such planes 
may react dramatically to specific spells, sentient thought, or the force of will. 
Others change for no reason. 

\textit{Magically Morphic:} Specific spells can alter the basic material of a plane 
with this trait.

\textit{Divinely Morphic:} Specific unique beings (deities or similar great powers) 
have the ability to alter objects, creatures, and the landscape on planes with 
this trait. Ordinary characters find these planes similar to alterable planes in 
that they may be affected by spells and physical effort. But the deities may cause 
these areas to change instantly and dramatically, creating great kingdoms for themselves. 

\textit{Static:} These planes are unchanging. Visitors cannot affect living residents 
of the plane, nor objects that the denizens possess. Any spells that would affect 
those on the plane have no effect unless the plane's static trait is somehow removed 
or suppressed. Spells cast before entering a plane with the static trait remain 
in effect, however.

Even moving an unattended object within a static plane requires a DC 16 Strength 
check. Particularly heavy objects may be impossible to move.

\textit{Sentient:} These planes are ones that respond to a single thought -- that 
of the plane itself. Travelers would find the plane's landscape changing as a result 
of what the plane thought of the travelers, either becoming more or less hospitable 
depending on its reaction.

%%%
\subsubsection{Elemental and Energy Traits}
%%%

Four basic elements and two types of energy together make up everything. The elements 
are earth, air, fire, and water. The types of energy are positive and negative.

The Material Plane reflects a balancing of those elements and energies; all are 
found there. Each of the Inner Planes is dominated by one element or type of energy. 
Other planes may show off various aspects of these elemental traits. Many planes 
have no elemental or energy traits; these traits are noted in a plane's description 
only when they are present.

\textit{Air-Dominant:} Mostly open space, planes with this trait have just a few 
bits of floating stone or other elements. They usually have a breathable atmosphere, 
though such a plane may include clouds of acidic or toxic gas. Creatures of the 
earth subtype are uncomfortable on air-dominant planes because they have little 
or no natural earth to connect with. They take no actual damage, however.

\textit{Earth-Dominant:} Planes with this trait are mostly solid. Travelers who 
arrive run the risk of suffocation if they don't reach a cavern or other pocket 
within the earth. Worse yet, individuals without the ability to burrow are entombed 
in the earth and must dig their way out (5 feet per turn). Creatures of the air 
subtype are uncomfortable on earth dominant planes because these planes are tight 
and claustrophobic to them. But they suffer no inconvenience beyond having difficulty 
moving.

\textit{Fire-Dominant:} Planes with this trait are composed of flames that continually 
burn without consuming their fuel source. Fire-dominant planes are extremely hostile 
to Material Plane creatures, and those without resistance or immunity to fire are 
soon immolated.

Unprotected wood, paper, cloth, and other flammable materials catch fire almost 
immediately, and those wearing unprotected flammable clothing catch on fire. In 
addition, individuals take 3d10 points of fire damage every round they are on a 
fire-dominant plane. Creatures of the water subtype are extremely uncomfortable 
on fire-dominant planes. Those that are made of water take double damage each round.

\textit{Water-Dominant:} Planes with this trait are mostly liquid. Visitors who 
can't breathe water or reach a pocket of air will likely drown. Creatures of the 
fire subtype are extremely uncomfortable on water-dominant planes. Those made of 
fire take 1d10 points of damage each round.

\textit{Positive-Dominant:} An abundance of life characterizes planes with this 
trait. The two kinds of positive-dominant traits are minor positive-dominant and 
major positive-dominant. A minor positive-dominant plane is a riotous explosion 
of life in all its forms. Colors are brighter, fires are hotter, noises are louder, 
and sensations are more intense as a result of the positive energy swirling through 
the plane. All individuals in a positive-dominant plane gain fast healing 2 as 
an extraordinary ability.

Major positive-dominant planes go even further. A creature on a major positive-dominant 
plane must make a DC 15 Fortitude save to avoid being blinded for 10 rounds by 
the brilliance of the surroundings. Simply being on the plane grants fast healing 
5 as an extraordinary ability. In addition, those at full hit points gain 5 additional 
temporary hit points per round. These temporary hit points fade 1d20 rounds after 
the creature leaves the major positive- dominant plane. However, a creature must 
make a DC 20 Fortitude save each round that its temporary hit points exceed its 
normal hit point total. Failing the saving throw results in the creature exploding 
in a riot of energy, killing it.

\textit{Negative-Dominant:} Planes with this trait are vast, empty reaches that 
suck the life out of travelers who cross them. They tend to be lonely, haunted 
planes, drained of color and filled with winds bearing the soft moans of those 
who died within them. As with positive-dominant planes, negative-dominant planes 
can be either minor or major. On minor negative-dominant planes, living creatures 
take 1d6 points of damage per round. At 0 hit points or lower, they crumble into 
ash.

Major negative-dominant planes are even more severe. Each round, those within must 
make a DC 25 Fortitude save or gain a negative level. A creature whose negative 
levels equal its current levels or Hit Dice is slain, becoming a wraith. The \linkspell{Death Ward}
spell protects a traveler from the damage and energy drain of a negative-dominant 
plane.

%%%
\subsubsection{Alignment Traits}
%%%

Some planes have a predisposition to a certain alignment. Most of the inhabitants 
of these planes also have the plane's particular alignment, even powerful creatures 
such as deities. In addition, creatures of alignments contrary to the plane have 
a tougher time dealing with its natives and situations.

The alignment trait of a plane affects social interactions there. Characters who 
follow other alignments than most of the inhabitants do may find life more difficult.

Alignment traits have multiple components. First are the moral (good or evil) and 
ethical (lawful or chaotic) components; a plane can have either a moral component, 
an ethical component, or one of each. Second, the specific alignment trait indicates 
whether each moral or ethical component is mildly or strongly evident.

\textit{Good-Aligned/Evil-Aligned:} These planes have chosen a side in the battle 
of good versus evil. No plane can be both good-aligned and evil-aligned.

\textit{Law-Aligned/Chaos-Aligned:} Law versus chaos is the key struggle for these 
planes and their residents. No plane can be both law-aligned and chaos-aligned.

\vspace{12pt}
Each part of the moral/ethical alignment trait has a descriptor, either "mildly" 
or "strongly," to show how powerful the influence of alignment is on the plane.

\textit{Mildly Aligned:} Creatures who have an alignment opposite that of a mildly 
aligned plane take a -2 circumstance penalty on all Charisma-based checks.

\textit{Strongly Aligned:} On planes that are strongly aligned, a -2 circumstance 
penalty applies on all Charisma-based checks made by all creatures not of the plane's 
alignment. In addition, the -2 penalty affects all Intelligence-based and Wisdom-based 
checks, too.

The penalties for the moral and ethical components of the alignment trait do stack.

\textit{Neutral-Aligned:} A mildly neutral-aligned plane does not apply a circumstance 
penalty to anyone.

The Material Plane is considered mildly neutral-aligned, though it may contain 
high concentrations of evil or good, law or chaos in places.

A strongly neutral-aligned plane would stand in opposition to all other moral and 
ethical principles: good, evil, law, and chaos. Such a plane may be more concerned 
with the balance of the alignments than with accommodating and accepting alternate 
points of view. In the same fashion as for other strongly aligned planes, strongly 
neutral-aligned planes apply a -2 circumstance penalty to Intelligence-, Wisdom-, 
or Charisma-based checks by any creature that isn't neutral. The penalty is applied 
twice (once for law/chaos, and once for good/evil), so neutral good, neutral evil, 
lawful neutral, and chaotic neutral creatures take a -2 penalty and lawful good, 
chaotic good, chaotic evil, and lawful evil creatures take a -4 penalty.

%%%
\subsubsection{Magic Traits}
%%%

A plane's magic trait describes how magic works on the plane compared to how it 
works on the Material Plane. Particular locations on a plane (such as those under 
the direct control of deities) may be pockets where a different magic trait applies.

\textit{Normal Magic:} This magic trait means that all spells and supernatural 
abilities function as written. Unless otherwise noted in a description, every plane 
has the normal magic trait.

\textit{Wild Magic:} On a plane with the wild magic trait spells and spell-like 
abilities function in radically different and sometimes dangerous ways. Any spell 
or spell-like ability used on a wild magic plane has a chance to go awry. The caster 
must make a level check (DC 15 + the level of the spell or effect) for the magic 
to function normally. For spell-like abilities, use the level or HD of the creature 
employing the ability for the caster level check and the level of the spell-like 
ability to set the DC for the caster level check. Failure on this check means that 
something strange happens; roll d\% and consult the following table.

\begin{table}[htb]
\rowcolors{1}{white}{offyellow}
\caption{Wild Magic Effects}
\centering
\begin{tabular}{c p{14cm}}
\textbf{d\%} & \textbf{Effect}\\
01-19 & Spell rebounds on caster with normal effect. If the spell cannot affect the caster, it simply fails.\\
20-23 & A circular pit 15 feet wide opens under the caster's feet; it is 10 feet deep per level of the caster.\\
24-27 & The spell fails, but the target or targets of the spell are pelted with a rain of small objects (anything from flowers to rotten fruit), which disappear upon striking. The barrage continues for 1 round. During this time the targets are blinded and must make Concentration checks (DC 15 + spell level) to cast spells.\\
28-31 & The spell affects a random target or area. Randomly choose a different target from among those in range of the spell or center the spell at a random place within range of the spell. To generate direction randomly, roll 1d8 and count clockwise around the compass, starting with south. To generate range randomly, roll 3d6. Multiply the result by 5 feet for close range spells, 20 feet for medium range spells, or 80 feet for long range spells.\\
32-35 & The spell functions normally, but any material components are not consumed. The spell is not expended from the caster's mind (a spell slot or prepared spell can be used again). An item does not lose charges, and the effect does not count against an item's or spell-like ability's use limit.\\
36-39 & The spell does not function. Instead, everyone (friend or foe) within 30 feet of the caster receives the effect of a heal spell.\\
40-43 & The spell does not function. Instead, a deeper darkness and a silence effect cover a 30-foot radius around the caster for 2d4 rounds.\\
44-47 & The spell does not function. Instead, a reverse gravity effect covers a 30-foot radius around the caster for 1 round.\\
48-51 & The spell functions, but shimmering colors swirl around the caster for 1d4 rounds. Treat this a glitterdust effect with a save DC of 10 + the level of the spell that generated this result.\\
52-59 & Nothing happens. The spell does not function. Any material components are used up. The spell or spell slot is used up, and charges or uses from an item are used up.\\
60-71 & Nothing happens. The spell does not function. Any material components are not consumed. The spell is not expended from the caster's mind (a spell slot or prepared spell can be used again). An item does not lose charges, and the effect does not count against an item's or spell-like ability's use limit.\\
72-98 & The spell functions normally.\\
99-100 & The spell functions strongly. Saving throws against the spell incur a -2 penalty. The spell has the maximum possible effect, as if it were cast with the Maximize Spell feat. If the spell is already maximized with the feat, there is no further effect.\\
\end{tabular}
\end{table}

\textit{Impeded Magic:} Particular spells and spell-like abilities are more difficult 
to cast on planes with this trait, often because the nature of the plane interferes 
with the spell.

To cast an impeded spell, the caster must make a Spellcraft check (DC 20 + the 
level of the spell). If the check fails, the spell does not function but is still 
lost as a prepared spell or spell slot. If the check succeeds, the spell functions 
normally.

\textit{Enhanced Magic:} Particular spells and spell-like abilities are easier 
to use or more powerful in effect on planes with this trait than they are on the 
Material Plane.

Natives of a plane with the enhanced magic trait are aware of which spells and 
spell-like abilities are enhanced, but planar travelers may have to discover this 
on their own.

If a spell is enhanced, certain metamagic feats can be applied to it without changing 
the spell slot required or the casting time. Spellcasters on the plane are considered 
to have that feat or feats for the purpose of applying them to that spell. Spellcasters 
native to the plane must gain the feat or feats normally if they want to use them 
on other planes as well.

\textit{Limited Magic:} Planes with this trait permit only the use of spells and 
spell-like abilities that meet particular qualifications.

Magic can be limited to effects from certain schools or subschools, to effects 
with certain descriptors, or to effects of a certain level (or any combination 
of these qualities). Spells and spell-like abilities that don't meet the qualifications 
simply don't work.

\textit{Dead Magic:} These planes have no magic at all. A plane with the dead magic 
trait functions in all respects like an \linkspell{Antimagic Field} spell. Divination 
spells cannot detect subjects within a dead magic plane, nor can a spellcaster 
use \linkspell{Plane Shift} or another spell to move in or out. The only exception to 
the "no magic" rule is permanent planar portals, which still function normally.

%%%%%%%%%%%%%%%%%%%%%%%%%
\subsection{How Planes Interact}
%%%%%%%%%%%%%%%%%%%%%%%%%

\textbf{Separate Planes:} Two planes that are separate do not overlap or directly 
connect to each other. They are like planets in different orbits. The only way 
to get from one separate plane to the other is to go through a third plane.

\textbf{\gameterm{Coterminous Planes}:} Planes that touch at specific points are coterminous. 
Where they touch, a connection exists, and travelers can leave one reality behind 
and enter the other.

\textbf{\gameterm{Coexistent Planes}:} If a link between two planes can be created at any 
point, the two planes are coexistent. These planes overlap each other completely. 
A coexistent plane can be reached from anywhere on the plane it overlaps. When 
moving on a coexistent plane, it is often possible to see into or interact with 
the plane it coexists with. 

%%%
\subsubsection{Layered Planes}
%%%

Infinities may be broken into smaller infinities, and planes into smaller, related 
planes. These layers are effectively separate planes of existence, and each layer 
can have its own planar traits. Layers are connected to each other through a variety 
of planar gates, natural vortices, paths, and shifting borders.

Access to a layered plane from elsewhere usually happens on a specific layer: the 
first layer of the plane, which can be either the top layer or the bottom layer, 
depending on the specific plane. Most fixed access points (such as portals and 
natural vortices) reach this layer, which makes it the gateway for other layers 
of the plane. The \linkspell{Plane Shift} spell also deposits the spellcaster on the 
first layer of the plane.

%%%%%%%%%%%%%%%%%%%%%%%%%
\subsection{Plane Descriptions}
%%%%%%%%%%%%%%%%%%%%%%%%%

%%%
\subsubsection{The Material Plane}\index{Material Plane}
%%%

The Material Plane is the center of most cosmologies and defines what is considered 
normal.

The Material Plane has the following traits:

\begin{itemize}
\item Normal gravity.
\item Normal Time
\item Alterable morphic.
\item No Elemental or Energy Traits (specific locations may have these traits, however)
\item Mildly neutral-aligned.
\item Normal magic. 
\end{itemize}

%%%
\subsubsection{The Ethereal Plane}\index{Ethereal Plane}
%%%

The Ethereal Plane is coexistent with the Material Plane and often other planes 
as well. The Material Plane itself is visible from the Ethereal Plane, but it appears 
muted and indistinct, its colors blurring into each other and its edges turning 
fuzzy.

While it is possible to see into the Material Plane from the Ethereal Plane, the 
Ethereal Plane is usually invisible to those on the Material Plane. Normally, creatures 
on the Ethereal Plane cannot attack creatures on the Material Plane, and vice versa. 
A traveler on the Ethereal Plane is invisible, incorporeal, and utterly silent 
to someone on the Material Plane. 

The Ethereal Plane is mostly empty of structures and impediments. However, the 
plane has its own inhabitants. Some of these are other ethereal travelers, but 
the ghosts found here pose a particular peril to those who walk the fog. 

It has the following traits.

\begin{itemize}
\item No gravity.
\item Alterable morphic. The plane contains little to alter, however.
\item Mildly neutral-aligned.
\item Normal magic. Spells function normally on the Ethereal Plane, though they do not 
cross into the Material Plane. 
\end{itemize}

The only exceptions are spells and spell-like abilities that have the force descriptor 
and abjuration spells that affect ethereal beings. Spellcasters on the Material 
Plane must have some way to detect foes on the Ethereal Plane before targeting 
them with force-based spells, of course. While it's possible to hit ethereal enemies 
with a force spell cast on the Material Plane, the reverse isn't possible. No magical 
attacks cross from the Ethereal Plane to the Material Plane, including force attacks.

%%%
\subsubsection{The Plane of Shadow}\index{Plane of Shadow}
%%%

The Plane of Shadow is a dimly lit dimension that is both coterminous to 
and coexistent with the Material Plane. It overlaps the Material Plane much as 
the Ethereal Plane does, so a planar traveler can use the Plane of Shadow to cover 
great distances quickly.

The Plane of Shadow is also coterminous to other planes. With the right spell, 
a character can use the Plane of Shadow to visit other realities.

The Plane of Shadow is a world of black and white; color itself has been bleached 
from the environment. It is otherwise appears similar to the Material Plane.

Despite the lack of light sources, various plants, animals, and humanoids call 
the Plane of Shadow home.

The Plane of Shadow is magically morphic, and parts continually flow onto other 
planes. As a result, creating a precise map of the plane is next to impossible, 
despite the presence of landmarks.

The Plane of Shadow has the following traits.

\begin{itemize}
\item Magically morphic. Certain spells modify the base material of the Plane of Shadow. 
The utility and power of these spells within the Plane of Shadow make them particularly 
useful for explorers and natives alike.
\item Mildly neutral-aligned.
\item Enhanced magic. Spells with the shadow descriptor are enhanced on the Plane of 
Shadow. Such spells are cast as though they were prepared with the Maximize Spell 
feat, though they don't require the higher spell slots. Furthermore, specific spells become more powerful on the Plane of Shadow. \linkspell{Shadow Conjuration} and \linkspell{Shadow Evocation} spells are 30\% as powerful as the conjurations and evocations they mimic (as opposed to 20\%). \linkspell{Greater Shadow Conjuration} and \linkspell{Greater Shadow Evocation} are 70\% as powerful (not 60\%), and a \linkspell{Shades} spell conjures at 90\% of the power of the original (not 80\%).
\item Impeded magic. Spells that use or generate light or fire may fizzle when cast on 
the Plane of Shadow. A spellcaster attempting a spell with the light or fire descriptor 
must succeed on a Spellcraft check (DC 20 + the level of the spell). Spells that 
produce light are less effective in general, because all light sources have their 
ranges halved on the Plane of Shadow.
\end{itemize}

Despite the dark nature of the Plane of Shadow, spells that produce, use, or manipulate 
darkness are unaffected by the plane.

%%%
\subsubsection{The Astral Plane}\index{Astral Plane}
%%%

The Astral Plane is the space between the planes. When a character moves through 
an interplanar portal or projects her spirit to a different plane of existence, 
she travels through the Astral Plane. Even spells that allow instantaneous movement 
across a plane briefly touch the Astral Plane.

The Astral Plane is a great, endless sphere of clear silvery sky, both above and 
below. Occasional bits of solid matter can be found here, but most of the Astral 
Plane is an endless, open domain.

Both planar travelers and refugees from other planes call the Astral Plane home. 

The Astral Plane has the following traits.

\begin{itemize}
\item Subjective directional gravity.
\item Timeless. Age, hunger, thirst, poison, and natural healing don't function in the 
Astral Plane, though they resume functioning when the traveler leaves the Astral 
Plane.
\item Mildly neutral-aligned.
\item Enhanced magic. All spells and spell-like abilities used within the Astral Plane 
may be employed as if they were improved by the \linkfeat{Quicken Spell} feat. Already quickened 
spells and spell-like abilities are unaffected, as are spells from magic items. 
Spells so quickened are still prepared and cast at their unmodified level. As with 
the Quicken Spell feat, only one quickened spell can be cast per round.
\end{itemize}

%%%
\subsubsection{Elemental Plane of Air}\index{Elemental Plane of Air}
%%%

The Elemental Plane of Air is an empty plane, consisting of sky above and sky below.

The Elemental Plane of Air is the most comfortable and survivable of the Inner 
Planes, and it is the home of all manner of airborne creatures. Indeed, flying 
creatures find themselves at a great advantage on this plane. While travelers without 
flight can survive easily here, they are at a disadvantage.

The Elemental Plane of Air has the following traits.

\begin{itemize}
\item Subjective directional gravity. Inhabitants of the plane determine their own "down" 
direction. Objects not under the motive force of others do not move.
\item Air-dominant.
\item Enhanced magic. Spells and spell-like abilities that use, manipulate, or create 
air (including spells of the Air domain) are both empowered and enlarged (as if 
the Empower Spell and Enlarge Spell metamagic feats had been used on them, but 
the spells don't require higher-level slots).
\item Impeded magic. Spells and spell-like abilities that use or create earth (including 
spells of the Earth domain and spells that summon earth elementals or outsiders 
with the earth subtype) are impeded.
\end{itemize}

%%%
\subsubsection{Elemental Plane of Earth}\index{Elemental Plane of Earth}
%%%

The Elemental Plane of Earth is a solid place made of rock, soil, and stone. An 
unwary and unprepared traveler may find himself entombed within this vast solidity 
of material and have his life crushed into nothingness, his powdered remains a 
warning to any foolish enough to follow.

Despite its solid, unyielding nature, the Elemental Plane of Earth is varied in 
its consistency, ranging from relatively soft soil to veins of heavier and more 
valuable metal. 

The Elemental Plane of Earth has the following traits.

\begin{itemize}
\item Earth-dominant.
\item Enhanced magic. Spells and spell-like abilities that use, manipulate, or create 
earth or stone (including those of the Earth domain) are both empowered and extended 
(as if the Empower Spell and Extend Spell metamagic feats had been used on them, 
but the spells don't require higher-level slots). Spells and spell-like abilities 
that are already empowered or extended are unaffected by this benefit.
\item Impeded magic. Spells and spell-like abilities that use or create air (including 
spells of the Air domain and spells that summon air elementals or outsiders with 
the air subtype) are impeded.
\end{itemize}

%%%
\subsubsection{Elemental Plane of Fire}\index{Elemental Plane of Fire}
%%%

Everything is alight on the Elemental Plane of Fire. The ground is nothing more 
than great, evershifting plates of compressed flame. The air ripples with the heat 
of continual firestorms, and the most common liquid is magma, not water. The oceans 
are made of liquid flame, and the mountains ooze with molten lava. Fire survives 
here without need for fuel or air, but flammables brought onto the plane are consumed 
readily. 

The Elemental Plane of Fire has the following traits.

\begin{itemize}
\item Fire-dominant.
\item Enhanced magic. Spells and spell-like abilities with the fire descriptor are both 
maximized and enlarged (as if the Maximize Spell and Enlarge Spell had been used 
on them, but the spells don't require higher-level slots). Spells and spell-like 
abilities that are already maximized or enlarged are unaffected by this benefit.
\item Impeded magic. Spells and spell-like abilities that use or create water (including 
spells of the Water domain and spells that summon water elementals or outsiders 
with the water subtype) are impeded. 
\end{itemize}

%%%
\subsubsection{Elemental Plane of Water}\index{Elemental Plane of Water}
%%%

The Elemental Plane of Water is a sea without a floor or a surface, an entirely 
fluid environment lit by a diffuse glow. It is one of the more hospitable of the 
Inner Planes once a traveler gets past the problem of breathing the local medium.

The eternal oceans of this plane vary between ice cold and boiling hot, between 
saline and fresh. They are perpetually in motion, wracked by currents and tides. 
The plane's permanent settlements form around bits of flotsam and jetsam suspended 
within this endless liquid. These settlements drift on the tides of the Elemental 
Plane of Water.

The Elemental Plane of Water has the following traits.

\begin{itemize}
\item Subjective directional gravity. The gravity here works similar to that of the Elemental 
Plane of Air. But sinking or rising on the Elemental Plane of Water is slower (and 
less dangerous) than on the Elemental Plane of Air.
\item Water-dominant.
\item Enhanced magic. Spells and spell-like abilities that use or create water are both 
extended and enlarged (as if the Extend Spell and Enlarge Spell metamagic feats 
had been used on them, but the spells don't require higher-level slots). Spells 
and spell-like abilities that are already extended or enlarged are unaffected by 
this benefit.
\item Impeded magic. Spells and spell-like abilities with the fire descriptor (including 
spells of the Fire domain) are impeded. 
\end{itemize}

%%%
\subsubsection{Negative Energy Plane}\index{Negative Energy Plane}
%%%

To an observer, there's little to see on the Negative Energy Plane. It is a dark, 
empty place, an eternal pit where a traveler can fall until the plane itself steals 
away all light and life. The Negative Energy Plane is the most hostile of the Inner 
Planes, and the most uncaring and intolerant of life. Only creatures immune to 
its life-draining energies can survive there. 

The Negative Energy Plane has the following traits.

\begin{itemize}
\item Subjective directional gravity.
\item Major negative-dominant. Some areas within the plane have only the minor negative-dominant 
trait, and these islands tend to be inhabited.
\item Enhanced magic. Spells and spell-like abilities that use negative energy are maximized 
(as if the Maximize Spell metamagic feat had been used on them, but the spells 
don't require higher-level slots). Spells and spell-like abilities that are already 
maximized are unaffected by this benefit. Class abilities that use negative energy, 
such as rebuking and controlling undead, gain a +10 bonus on the roll to determine 
Hit Dice affected. 
\item Impeded magic. Spells and spell-like abilities that use positive energy, including 
\textit{cure} spells, are impeded. Characters on this plane take a -10 penalty 
on Fortitude saving throws made to remove negative levels bestowed by an energy 
drain attack.
\end{itemize}

Random Encounters: Because the Negative Energy Plane is virtually devoid of creatures, 
random encounters on the plane are exceedingly rare.

%%%
\subsubsection{Positive Energy Plane}\index{Positive Energy Plane}
%%%

The Positive Energy Plane has no surface and is akin to the Elemental Plane of 
Air with its wide-open nature. However, every bit of this plane glows brightly 
with innate power. This power is dangerous to mortal forms, which are not made 
to handle it. Despite the beneficial effects of the plane, it is one of the most 
hostile of the Inner Planes. An unprotected character on this plane swells with 
power as positive energy is force-fed into her. Then, her mortal frame unable to 
contain that power, she immolates as if she were a small planet caught at the edge 
of a supernova. Visits to the Positive Energy Plane are brief, and even then travelers 
must be heavily protected.

The Positive Energy Plane has the following traits.

\begin{itemize}
\item Subjective directional gravity.
\item Major positive-dominant. Some regions of the plane have the minor positive-dominant 
trait instead, and those islands tend to be inhabited.
\item Enhanced magic. Spells and spell-like abilities that use positive energy, including 
\textit{cure} spells, are maximized (as if the Maximize Spell metamagic feat had 
been used on them, but the spells don't require higher-level slots). Spells and 
spell-like abilities that are already maximized are unaffected by this benefit. 
Class abilities that use positive energy, such as turning and destroying undead, 
gain a +10 bonus on the roll to determine Hit Dice affected. (Undead are almost 
impossible to find on this plane, however.)
\item Impeded magic. Spells and spell-like abilities that use negative energy (including 
\textit{inflict} spells) are impeded.
\end{itemize}

Random Encounters: Because the Positive Energy Plane is virtually devoid of creatures, 
random encounters on the plane are exceedingly rare.



\chapter{Your Role in the Campaign}

%\input{phb/your-role/intro}
\section{Types of Leadership}
A character moving into a leadership position can be a natural way for a character to advance as a game progresses. There are several ways a character can do this.

\ability{Cohort}{Having a cohort is very similar to having a secondary character. You have direct control over your cohort, just as you have control over your own character. Your cohort, if you have one, always has a CR that is 2 less than your character's level. A cohort increases in power, gaining a level or hit dice, when you gain a level. You may only ever have one cohort at a time, if a source would grant you more than one cohort, you instead gain a companion. Cohorts are usually gained by taking [Leadership] feats.}

\ability{Companion}{Companions play supporting roles to your own character. You have direct control of your companions as long as they are with your character. If for some reason one of your companions is separated from your character, the companion continues to act in your best interest, typically following any orders you have given it. Your companions always have a CR equal to your level -4 or to \sfrac{1}{2} of your level (round down), whichever is greater. Companions are usually gained through class features.}

\ability{Followers}{Your followers are people that take orders from you, possibly as members of an organization you lead or have a leadership position within. Instead of keeping track of all of your followers individually, they are represented by your Leadership Score (see below). By spending points of your Leadership Score (and temporarily reducing it) you can have your followers accomplish tasks for you.}

\section{Leadership Score}

Your Leadership Score represents the amount of resources and manpower your followers can produce. Your Leadership Score has a maximum value equal to your level, and is replenished by an amount equal to your Charisma bonus each week (minimum 1). You only have one Leadership Score, if a source would grant you more than one Leadership Score, your score is instead increased by 4. Below are the tasks you can accomplish by spending your Leadership Score.

When you first gain a leadership score, you should select a location to serve as the base for your organization. The location of your base affects the DC of and the the time it takes perform certain actions.

%\begin{wraptable}{o}{.6\linewidth}
\begin{table}[h!]
\centering
\caption{Uses of Leadership}
\rowcolors{1}{colorone}{colortwo}
\begin{tabu}to \linewidth{X X}
\header Action & Leadership Cost \\ \hline
Gather Intelligence & 1\\
Labor & Varies, see text\\
Personal Retinue & Equal to EL of Followers\\
Provide Service & Varies\\
Skill Check Equivalent & 1 per +5 Bonus\\ \hline
\end{tabu}
\end{table}
%\end{wraptable}

\ability{Gather Intelligence}{You can use your followers to find out about a person place or thing. If successful you learn a single peice of information about the object of your investigation at the end of the week, and an additional peice of information for every 5 points by which your check exceeds the DC. On a failed check, you gain a piece of information that turns out to be false. The base DC for this check is 10, modified by the conditions shown on the table below. You have a bonus to the check equal to you level.
	\begin{awesomelist}
		\item \ability{Person}{A successful check indicates that you learn one thing of your choice about a person: their current location, where they plan to be during the next week, if there is currently a plot against that person, the truth about one rumor about the person.}
		\item \ability{Place}{A successful check allows you to learn one thing about the place: presence of people or monsters and a general indication of how dangerous the area is, the number of a specific group of people that you already know the presence of in the area, the presence or absence of ambush points.}
		\item \ability{Thing}{A successful check lets you learn one of the following about an object: the history of the object, the previous owner of the object, the function of the object, the activation method for the object (if magical).}
	\end{awesomelist}
	
\begin{table}[h!]
\centering
\caption{Gather Intelligence DC Modifiers}
\rowcolors{1}{colorone}{colortwo}
\begin{tabu}to \linewidth{X[3] X || X[3] X || X[3] X}
\header Person is... & DC & Place is... & DC & Thing is... & DC \\ \hline
Friendly  & -5 & Friendly Territory & -5 & In Your Posession & -5 \\
An Enemy  & +5 & Hostile Territory  & +5 & Not Specific      & +5 \\
Secretive & +5 & Remote             & +5 & Medium or Major   & +5 \\
In Hiding & +5 & Guarded            & +5 & An Artifact       & +10 \\ \hline
\end{tabu}
\end{table}
}

\ability{Labor}{You can have your followers work to build or craft something for one week. The amount of work done is based on how much of your leadership score you spend, and is equivalent to a given number of people working for one week. This cost recurs weekly if you have your followers continue to labor. If the task would normally require some sort of skill check to complete, you must arrange for that to be done separately.}

%\begin{wraptable}{o}{.4\linewidth}
\begin{table}[h!]
\centering
\caption{Cost of Labor}
\rowcolors{1}{colorone}{colortwo}
\begin{tabu}{l l}
\header Points Spent & Number of Laborers \\ \hline
1 & 1 \\
2 & 3 \\
3 & 5 \\
4 & 10 \\
5 & 25 \\
6 & 55 \\
7 & 110 \\
8 & 225 \\
9 & 450 \\
10 & 900 \\
11 & 1,800 \\
12 & 3,750 \\
13 & 7,500 \\
14 & 15,000 \\
15 & 30,000 \\
16 & 60,000 \\
17 & 125,000 \\
18 & 250,000 \\
19 & 500,000 \\
20 & 1,000,000 \\
+1 & x2 \\ \hline
\end{tabu}
\end{table}
%\end{wraptable}

\ability{Personal Retinue}{You can take followers with you on adventures. The cost to do this is equal to the encounter level (EL) of the followers you bring with you, up to a maximum of \sfrac{1}{2} of your level. The followers remain with you for one week, and the cost recurs for each week they remain with you. If some of them are killed or otherwise rendered unfit for service, a one-time penalty equal to half of the cost to bring them is incurred. If \emph{all} of them are killed or otherwise rendered unfit for service, the penalty is instead equal to the full cost to bring them. This can temporarily reduce your Leadership Score to a negative value.}

\ability{Skill Check Equivalent}{You can have your followers perform some task that is equivalent to a skill check. The skill may be any relevant skill, and the bonus is equal to your character level. For every additional point you spend, the check is made at an additional +5 bonus, up to a maximum of double your character level.}

\subsection{Followers and Location}

When you first gain your leadership score, you should select a place to serve as the base of operations for your organization. Tasks take additional time to start based on how distant the task is to take place from your base. This amount of time does not necessarily equate to the time it takes for your followers to physically reach the location, but could be the time it takes to get in touch with with local contacts or contractors.

\begin{table}[h!]
\centering
\caption{Distance and Time}
\rowcolors{1}{colorone}{colortwo}
\begin{tabu}{l l}
\header Distance to Task & Extra Time Required \\ \hline
Local & No Extra Time \\
Neighboring Province (100 mi.) & 1 Week \\
Distant Province (250 mi.) &  2 Weeks \\
Neighboring Country (500 mi.) & 1 Month \\
Distant Country (1000 mi.) & 2 Months \\
Another Continent (3000 mi) & 3 Months \\
Another Plane & 4 Months \\ \hline
\end{tabu}
\end{table}

\subsection{Rushing Your Followers}

By spending an extra 50\% of the cost of a use of your leadership (round up), you can reduce the time it takes to complete the task from a number of weeks, to an equivalent number of days.

\section{Replacing Cohorts and companions}

Sometimes followers, companions, and even cohorts might die. Or maybe the one you have just isn't cutting it anymore. Often the requirements involve some amount of in-game time passing, and this assumes that the character spends some minimal amount of time between adventures looking for suitable replacements. Unless there are extenuating circumstances, the DM should try to follow these guidelines or work out another suitable means of replacing the minion. The requirements to replace them are as follows.

\ability{Replacing a Cohort}{A cohort can be replaced any time you gain a level, or after some period of in-game time (typically a month). Your DM may also allow you to immediately replace your cohort with an appropriate NPC that is already in the campaign.}

\ability{Replacing a Companion}{Typically the steps necessary to replace a companion are given by the feat or class feature that granted it. If no method of replacement is specified, a companion can usually be replaced after one week of in-game time.}

\section{Converting Followers}

Sometimes you might want to convert a specific NPC into a follower. You can try to turn an NPC that is friendly to you with at least one minute of effort and a successful Charisma check with a DC of 10 + the NPC's CR. If the NPC is already the follower of another character the DC is increased by the Charisma bonus of the character they are already a follower of. A failure carries no penalty, but you cannot try to convert the same NPC again.
%\input{phb/your-role/owning-a-business}
%\input{phb/your-role/going-to-war}
\chapter{Combat}
\input{phb/combat/combat}
%%%%%%%%%%%%%%%%%%%%%%%%%%%%%%%%%%%%%%%%%%%%%%%%%%
\section{Movement, Position, and Distance}
%%%%%%%%%%%%%%%%%%%%%%%%%%%%%%%%%%%%%%%%%%%%%%%%%%

Miniatures are on the 30mm scale -- a miniature figure of a six-foot-tall human 
is approximately 30mm tall. A square on the battle grid is 1 inch across, representing 
a 5-foot-by-5-foot area.

%%%%%%%%%%%%%%%%%%%%%%%%%
\subsection{Tactical Movement In Combat}\index{Movement}
%%%%%%%%%%%%%%%%%%%%%%%%%

%%%
\subsubsection{How Far Can Your Character Move?}
%%%

Your speed is determined by your race and your armor (see Table: Tactical Speed). 
Your speed while unarmored is your base land speed.

\textbf{\gameterm{Encumbrance}:} A character encumbered by carrying a large amount of gear, 
treasure, or fallen comrades may move slower than normal.

\textbf{\gameterm{Hampered Movement}:} Difficult terrain, obstacles, or poor visibility can 
hamper movement.

\textbf{Movement in Combat:} Generally, you can move your speed in a round and 
still do something (take a move action and a standard action).

If you do nothing but move (that is, if you use both of your actions in a round 
to move your speed), you can move double your speed.

If you spend the entire round running, you can move quadruple your speed. If you 
do something that requires a full round you can only take a 5-foot step.

\textbf{Bonuses to Speed:} A barbarian has a +10 foot bonus to his speed (unless 
he's wearing heavy armor). Experienced monks also have higher speed (unless they're 
wearing armor of any sort). In addition, many spells and magic items can affect 
a character's speed. Always apply any modifiers to a character's speed before adjusting 
the character's speed based on armor or encumbrance, and remember that multiple 
bonuses of the same type to a character's speed don't stack.

\begin{table}[htb]
\rowcolors{1}{white}{offyellow}
\caption{Tactical Speed}
\centering
\begin{tabular}{l l l}
\textbf{Race} & \textbf{No Armor or Light Armor} & \textbf{Medium or Heavy Armor}\\
Human, Elf, Half-Elf, Half-Orc & 30ft (6 squares) & 20ft (4 squares)\\
Dwarf & 20ft (4 squares) & 20ft (4 squares)\\
Halfling & 20ft (4 squares) & 15ft (3 squares)\\
\end{tabular}
\end{table}

%%%
\subsubsection{Measuring Distance}
%%%

\textbf{Diagonals:}\index{Movement!Diagonals} When measuring distance, the first diagonal counts as 1 square, 
the second counts as 2 squares, the third counts as 1, the fourth as 2, and so 
on.

You can't move diagonally past a corner (even by taking a 5-foot step). You can 
move diagonally past a creature, even an opponent.

You can also move diagonally past other impassable obstacles, such as pits.

\textbf{Closest Creature:} When it's important to determine the closest square 
or creature to a location, if two squares or creatures are equally close, randomly 
determine which one counts as closest by rolling a die.

%%%
\subsubsection{Moving through a Square}
%%%

\textbf{Friend:} You can move through a square occupied by a friendly character, 
unless you are charging. When you move through a square occupied by a friendly 
character, that character doesn't provide you with cover.

\textbf{Opponent:} You can't move through a square occupied by an opponent, unless 
the opponent is helpless. You can move through a square occupied by a helpless 
opponent without penalty. (Some creatures, particularly very large ones, may present 
an obstacle even when helpless. In such cases, each square you move through counts 
as 2 squares.)

\textbf{Ending Your Movement:} You can't end your movement in the same square as 
another creature unless it is helpless.

\textbf{Overrun:} During your movement you can attempt to move through a square 
occupied by an opponent.

\textbf{Tumbling:} A trained character can attempt to tumble through a square occupied 
by an opponent (see the Tumble skill).

\textbf{Very Small Creature:} A Fine, Diminutive, or Tiny creature can move into 
or through an occupied square. The creature provokes attacks of opportunity when 
doing so.

\textbf{Square Occupied by Creature Three Sizes Larger or Smaller:} Any creature 
can move through a square occupied by a creature three size categories larger than 
it is.

A big creature can move through a square occupied by a creature three size categories 
smaller than it is.

\textbf{Designated Exceptions:} Some creatures break the above rules. A creature 
that completely fills the squares it occupies cannot be moved past, even with the 
\linkskill{Tumble} skill or similar special abilities.

%%%
\subsubsection{Terrain and Obstacles}
%%%

\textbf{\gameterm{Difficult Terrain}:} Difficult terrain hampers movement. Each square of 
difficult terrain counts as 2 squares of movement. (Each diagonal move into a difficult 
terrain square counts as 3 squares.) You can't run or charge across difficult terrain.

If you occupy squares with different kinds of terrain, you can move only as fast 
as the most difficult terrain you occupy will allow.

Flying and incorporeal creatures are not hampered by difficult terrain.

\textbf{Obstacles:} Like difficult terrain, obstacles can hamper movement. If an 
obstacle hampers movement but doesn't completely block it each obstructed square 
or obstacle between squares counts as 2 squares of movement. You must pay this 
cost to cross the barrier, in addition to the cost to move into the square on the 
other side. If you don't have sufficient movement to cross the barrier and move 
into the square on the other side, you can't cross the barrier. Some obstacles 
may also require a skill check to cross.

On the other hand, some obstacles block movement entirely. A character can't move 
through a blocking obstacle.

Flying and incorporeal creatures can avoid most obstacles

\textbf{\gameterm{Squeezing}:} In some cases, you may have to squeeze into or through an area 
that isn't as wide as the space you take up. You can squeeze through or into a 
space that is at least half as wide as your normal space. Each move into or through 
a narrow space counts as if it were 2 squares, and while squeezed in a narrow space 
you take a -4 penalty on attack rolls and a -4 penalty to AC.

When a Large creature (which normally takes up four squares) squeezes into a space 
that's one square wide, the creature's miniature figure occupies two squares, centered 
on the line between the two squares. For a bigger creature, center the creature 
likewise in the area it squeezes into.

A creature can squeeze past an opponent while moving but it can't end its movement 
in an occupied square.

To squeeze through or into a space less than half your space's width, you must 
use the Escape Artist skill. You can't attack while using Escape Artist to squeeze 
through or into a narrow space, you take a -4 penalty to AC, and you lose any Dexterity 
bonus to AC.

%%%
\subsubsection{Special Movement Rules}
%%%

These rules cover special movement situations.

\textbf{Accidentally Ending Movement in an Illegal Space:} Sometimes a character 
ends its movement while moving through a space where it's not allowed to stop. 
When that happens, put your miniature in the last legal position you occupied, 
or the closest legal position, if there's a legal position that's closer.

\textbf{Double Movement Cost:} When your movement is hampered in some way, your 
movement usually costs double. For example, each square of movement through difficult 
terrain counts as 2 squares, and each diagonal move through such terrain counts 
as 3 squares (just as two diagonal moves normally do).

If movement cost is doubled twice, then each square counts as 4 squares (or as 
6 squares if moving diagonally). If movement cost is doubled three times, then 
each square counts as 8 squares (12 if diagonal) and so on. This is an exception 
to the general rule that two doublings are equivalent to a tripling.

\textbf{Minimum Movement:} Despite penalties to movement, you can take a full-round 
action to move 5 feet (1 square) in any direction, even diagonally. (This rule 
doesn't allow you to move through impassable terrain or to move when all movement 
is prohibited.) Such movement provokes attacks of opportunity as normal (despite 
the distance covered, this move isn't a 5-foot step).

%%%%%%%%%%%%%%%%%%%%%%%%%
\subsection{Big And Little Creatures In Combat}
%%%%%%%%%%%%%%%%%%%%%%%%%

Creatures smaller than Small or larger than Medium have special rules relating 
to position. 

\textbf{Tiny, Diminutive, and Fine Creatures:} Very small creatures take up less 
than 1 square of space. This means that more than one such creature can fit into 
a single square. A Tiny creature typically occupies a space only 2-1/2 feet across, 
so four can fit into a single square. Twenty-five Diminutive creatures or 100 Fine 
creatures can fit into a single square. Creatures that take up less than 1 square 
of space typically have a natural reach of 0 feet, meaning they can't reach into 
adjacent squares. They must enter an opponent's square to attack in melee. This 
provokes an attack of opportunity from the opponent. You can attack into your own 
square if you need to, so you can attack such creatures normally. Since they have 
no natural reach, they do not threaten the squares around them. You can move past 
them without provoking attacks of opportunity. They also can't flank an enemy.

\textbf{Large, Huge, Gargantuan, and Colossal Creatures:} Very large creatures 
take up more than 1 square.

Creatures that take up more than 1 square typically have a natural reach of 10 
feet or more, meaning that they can reach targets even if they aren't in adjacent 
squares.

Unlike when someone uses a reach weapon, a creature with greater than normal natural 
reach (more than 5 feet) still threatens squares adjacent to it. A creature with 
greater than normal natural reach usually gets an attack of opportunity against 
you if you approach it, because you must enter and move within the range of its 
reach before you can attack it. (This attack of opportunity is not provoked if 
you take a 5-foot step.)

Large or larger creatures using reach weapons can strike up to double their natural 
reach but can't strike at their natural reach or less. 

\begin{table}[htb]
\rowcolors{1}{white}{offyellow}
\caption{Creature Size and Scale}
\centering
\begin{tabular}{lcc}
\textbf{Creature Size} & \textbf{Space\textsuperscript{1}} & \textbf{Natural Reach\textsuperscript{1}}\\
Fine & 1/2ft & 0ft\\
Diminutive & 1ft & 0ft\\
Tiny & 2.5ft & 0ft\\
Small & 5ft & 5ft\\
Medium & 5ft & 5ft\\
Large (long) & 10ft & 5ft\\
Large (tall) & 10ft & 10ft\\
Huge (long) & 15ft & 10ft\\
Huge (tall) & 15ft & 15ft\\
Gargantuan (long) & 20ft & 15ft\\
Gargantuan (tall) & 20ft & 20ft\\
Colossal (long) & 30ft & 20ft\\
Colossal (tall) & 30ft & 30ft\\
\multicolumn{3}{p{6cm}}{\textsuperscript{1} These values are typical for creatures of the indicated size. Some exceptions exist.}\\
\end{tabular}
\end{table}

%%%%%%%%%%%%%%%%%%%%%%%%%%%%%%%%%%%%%%%%%%%%%%%%%%
\section{Combat Modifiers}
%%%%%%%%%%%%%%%%%%%%%%%%%%%%%%%%%%%%%%%%%%%%%%%%%%

%%%%%%%%%%%%%%%%%%%%%%%%%
\subsection{Favorable And Unfavorable Conditions}
%%%%%%%%%%%%%%%%%%%%%%%%%

\begin{table}[htb]
\rowcolors{1}{white}{offyellow}
\caption{Attack Roll Modifiers}
\centering
\begin{tabular}{p{7cm}cc}
\textbf{Attacker is \ldots{}} & \textbf{Melee} & \textbf{Ranged}\\
Dazzled & -1 & -1\\
Entangled & -2\textsuperscript{1} & -2\textsuperscript{1}\\
Flanking defender & +2 & --\\
Invisible & +2\textsuperscript{2} & +2\textsuperscript{2}\\
On higher ground & +1 & +0\\
Prone & -4 & --\textsuperscript{3}\\
Shaken or frightened & -2 & -2\\
Squeezing through a space & -4 & -4\\
\multicolumn{3}{p{10cm}}{\textsuperscript{1} An entangled character also takes a -4 penalty to Dexterity, which may affect his attack roll.}\\
\multicolumn{3}{p{10cm}}{\textsuperscript{2} The defender loses any Dexterity bonus to AC. This bonus doesn't apply if the target is blinded.}\\
\multicolumn{3}{p{10cm}}{\textsuperscript{3} Most ranged weapons can't be used while the attacker is prone, but you can use a crossbow or shuriken while prone at no penalty.}\\
\end{tabular}
\end{table}

%perhaps convert to tablularx
\begin{table}[htb]
\rowcolors{1}{white}{offyellow}\mcinherit
\caption{Armor Class Modifiers}
\centering
\begin{tabular}{p{8.5cm}cc}
\textbf{Defender is \ldots{}} & \textbf{Melee} & \textbf{Ranged}\\
Behind cover & +4 & +4\\
Blinded & -2\textsuperscript{1} & -2\textsuperscript{1}\\
Concealed or invisible & \multicolumn{2}{c}{-- See Concealment --}\\
Cowering & -2\textsuperscript{1} & -2\textsuperscript{1}\\
Entangled & +0\textsuperscript{2} & +0\textsuperscript{2}\\
Flat-footed (such as surprised, balancing, climbing) & +0\textsuperscript{1} & +0\textsuperscript{1}\\
Grappling (but attacker is not) & +0\textsuperscript{1} & +0\textsuperscript{1,3}\\
Helpless (such as paralyzed, sleeping, or bound) & -4\textsuperscript{4} & +0\textsuperscript{4}\\
Kneeling or sitting & -2 & +2\\
Pinned & -4\textsuperscript{4} & +0\textsuperscript{4}\\
Prone & -4 & +4\\
Squeezing through a space & -4 & -4\\
Stunned & -2\textsuperscript{1} & -2\textsuperscript{1}\\
\multicolumn{3}{p{12cm}}{\textsuperscript{1} The defender loses any Dexterity bonus to AC.}\\
\multicolumn{3}{p{12cm}}{\textsuperscript{2} An entangled character takes a -4 penalty to Dexterity.}\\
\multicolumn{3}{p{12cm}}{\textsuperscript{3} Roll randomly to see which grappling combatant you strike. That defender loses any Dexterity bonus to AC.}\\
\multicolumn{3}{p{12cm}}{\textsuperscript{4} Treat the defender's Dexterity as 0 (-5 modifier). Rogues can sneak attack helpless or pinned defenders.}\\
\end{tabular}
\end{table}

%%%%%%%%%%%%%%%%%%%%%%%%%
\subsection{Cover}\index{Cover}
%%%%%%%%%%%%%%%%%%%%%%%%%

To determine whether your target has cover from your ranged attack, choose a corner 
of your square. If any line from this corner to any corner of the target's square 
passes through a square or border that blocks line of effect or provides cover, 
or through a square occupied by a creature, the target has cover (+4 to AC).

When making a melee attack against an adjacent target, your target has cover if 
any line from your square to the target's square goes through a wall (including 
a low wall). When making a melee attack against a target that isn't adjacent to 
you (such as with a reach weapon), use the rules for determining cover from ranged 
attacks.

\textbf{Low Obstacles and Cover:} A low obstacle (such as a wall no higher than 
half your height) provides cover, but only to creatures within 30 feet (6 squares) 
of it. The attacker can ignore the cover if he's closer to the obstacle than his 
target.

\textbf{Cover and Attacks of Opportunity:} You can't execute an attack of opportunity 
against an opponent with cover relative to you.

\textbf{Cover and Reflex Saves:} Cover grants you a +2 bonus on Reflex saves against 
attacks that originate or burst out from a point on the other side of the cover 
from you. Note that spread effects can extend around corners and thus negate this 
cover bonus.

\textbf{Cover and Hide Checks:} You can use cover to make a Hide check. Without 
cover, you usually need concealment (see below) to make a Hide check.

\textbf{Soft Cover:} Creatures, even your enemies, can provide you with cover against 
ranged attacks, giving you a +4 bonus to AC. However, such soft cover provides 
no bonus on Reflex saves, nor does soft cover allow you to make a Hide check.

\textbf{Big Creatures and Cover:} Any creature with a space larger than 5 feet 
(1 square) determines cover against melee attacks slightly differently than smaller 
creatures do. Such a creature can choose any square that it occupies to determine 
if an opponent has cover against its melee attacks. Similarly, when making a melee 
attack against such a creature, you can pick any of the squares it occupies to 
determine if it has cover against you.

\textbf{Total Cover:} If you don't have line of effect to your target he is considered 
to have total cover from you. You can't make an attack against a target that has 
total cover.

\textbf{Varying Degrees of Cover:} In some cases, cover may provide a greater bonus 
to AC and Reflex saves. In such situations the normal cover bonuses to AC and Reflex 
saves can be doubled (to +8 and +4, respectively). A creature with this improved 
cover effectively gains improved evasion against any attack to which the Reflex 
save bonus applies. Furthermore, improved cover provides a +10 bonus on Hide checks.

%%%%%%%%%%%%%%%%%%%%%%%%%
\subsection{Concealment}\index{Concealment}
%%%%%%%%%%%%%%%%%%%%%%%%%

To determine whether your target has concealment from your ranged attack, choose 
a corner of your square. If any line from this corner to any corner of the target's 
square passes through a square or border that provides concealment, the target 
has concealment.

When making a melee attack against an adjacent target, your target has concealment 
if his space is entirely within an effect that grants concealment. When making 
a melee attack against a target that isn't adjacent to you use the rules for determining 
concealment from ranged attacks.

In addition, some magical effects provide concealment against all attacks, regardless 
of whether any intervening concealment exists.

\textbf{Concealment Miss Chance:} Concealment gives the subject of a successful 
attack a 20\% chance that the attacker missed because of the concealment. If the 
attacker hits, the defender must make a miss chance percentile roll to avoid being 
struck. Multiple concealment conditions do not stack.

\textbf{Concealment and Hide Checks:} You can use concealment to make a Hide check. 
Without concealment, you usually need cover to make a Hide check.

\textbf{Total Concealment:} If you have line of effect to a target but not line 
of sight he is considered to have total concealment from you. You can't attack 
an opponent that has total concealment, though you can attack into a square that 
you think he occupies. A successful attack into a square occupied by an enemy with 
total concealment has a 50\% miss chance (instead of the normal 20\% miss chance 
for an opponent with concealment).

You can't execute an attack of opportunity against an opponent with total concealment, 
even if you know what square or squares the opponent occupies.

\textbf{Ignoring Concealment:} Concealment isn't always effective. A shadowy area 
or darkness doesn't provide any concealment against an opponent with darkvision. 
Characters with low-light vision can see clearly for a greater distance with the 
same light source than other characters. Although invisibility provides total concealment, 
sighted opponents may still make Spot checks to notice the location of an invisible 
character. An invisible character gains a +20 bonus on Hide checks if moving, or 
a +40 bonus on Hide checks when not moving (even though opponents can't see you, 
they might be able to figure out where you are from other visual clues).

\textbf{Varying Degrees of Concealment:} Certain situations may provide more or 
less than typical concealment, and modify the miss chance accordingly.

%%%%%%%%%%%%%%%%%%%%%%%%%
\subsection{Flanking}\index{Flanking}
%%%%%%%%%%%%%%%%%%%%%%%%%

When making a melee attack, you get a +2 flanking bonus if your opponent is threatened 
by a character or creature friendly to you on the opponent's opposite border or 
opposite corner.

When in doubt about whether two friendly characters flank an opponent in the middle, 
trace an imaginary line between the two friendly characters' centers. If the line 
passes through opposite borders of the opponent's space (including corners of those 
borders), then the opponent is flanked.

\textit{Exception:} If a flanker takes up more than 1 square, it gets the flanking 
bonus if any square it occupies counts for flanking.

Only a creature or character that threatens the defender can help an attacker get 
a flanking bonus.

Creatures with a reach of 0 feet can't flank an opponent.

%%%%%%%%%%%%%%%%%%%%%%%%%
\subsection{Helpless Defenders}\index{Helpless}
%%%%%%%%%%%%%%%%%%%%%%%%%

A helpless opponent is someone who is bound, sleeping, paralyzed, unconscious, 
or otherwise at your mercy.

\textbf{Regular Attack:} A helpless character takes a -4 penalty to AC against 
melee attacks, but no penalty to AC against ranged attacks.

A helpless defender can't use any Dexterity bonus to AC. In fact, his Dexterity 
score is treated as if it were 0 and his Dexterity modifier to AC as if it were 
-5 (and a rogue can sneak attack him).

\textbf{\gameterm{Coup de Grace}:} As a full-round action, you can use a melee weapon to deliver 
a coup de grace to a helpless opponent. You can also use a bow or crossbow, provided 
you are adjacent to the target.

You automatically hit and score a critical hit. If the defender survives the damage, 
he must make a Fortitude save (DC 10 + damage dealt) or die. A rogue also gets 
her extra sneak attack damage against a helpless opponent when delivering a coup 
de grace.

Delivering a coup de grace provokes attacks of opportunity from threatening opponents.

You can't deliver a coup de grace against a creature that is immune to critical 
hits. You can deliver a coup de grace against a creature with total concealment, 
but doing this requires two consecutive full-round actions (one to "find" the 
creature once you've determined what square it's in, and one to deliver the coup 
de grace).

%%%%%%%%%%%%%%%%%%%%%%%%%%%%%%%%%%%%%%%%%%%%%%%%%%
\section{Special Attacks}
%%%%%%%%%%%%%%%%%%%%%%%%%%%%%%%%%%%%%%%%%%%%%%%%%%

\begin{table}[htb]
\rowcolors{1}{white}{offyellow}\mcinherit
\caption{Special Attacks}
\centering
\begin{tabular}{ll}
\textbf{Special Attack} & \textbf{Brief Description}\\
Aid Another & Grant an ally a +2 bonus on attacks or AC\\
Bull Rush & Push an opponent back 5 feet or more\\
Charge & Move up to twice your speed and attack with +2 bonus\\
Disarm & Knock a weapon from your opponent's hands\\
Feint & Negate your opponent's Dex bonus to AC\\
Grapple & Wrestle with an opponent\\
Overrun & Plow past or over an opponent as you move\\
Sunder & Strike an opponent's weapon or shield\\
Throw splash weapon & Throw container of dangerous liquid at target\\
Trip & Trip an opponent\\
Turn (rebuke) undead & Channel positive (or negative) energy to turn away (or awe) undead\\
Two-weapon Fighting & Fight with a weapon in each hand\\
\end{tabular}
\end{table}

%%%%%%%%%%%%%%%%%%%%%%%%%
\subsection{Aid Another}\index{Aid Another}
%%%%%%%%%%%%%%%%%%%%%%%%%

In melee combat, you can help a friend attack or defend by distracting or interfering 
with an opponent. If you're in position to make a melee attack on an opponent that 
is engaging a friend in melee combat, you can attempt to aid your friend as a standard 
action. You make an attack roll against AC 10. If you succeed, your friend gains 
either a +2 bonus on his next attack roll against that opponent or a +2 bonus to 
AC against that opponent's next attack (your choice), as long as that attack comes 
before the beginning of your next turn. Multiple characters can aid the same friend, 
and similar bonuses stack.

You can also use this standard action to help a friend in other ways, such as when 
he is affected by a spell, or to assist another character's skill check.

%%%%%%%%%%%%%%%%%%%%%%%%%
\subsection{Bull Rush}\index{Bull Rush}
%%%%%%%%%%%%%%%%%%%%%%%%%

You can make a bull rush as a standard action (an attack) or as part of a charge 
(see Charge, below). When you make a bull rush, you attempt to push an opponent 
straight back instead of damaging him. You can only bull rush an opponent who is 
one size category larger than you, the same size, or smaller.

\textbf{Initiating a Bull Rush:} First, you move into the defender's space. Doing 
this provokes an attack of opportunity from each opponent that threatens you, including 
the defender. (If you have the Improved Bull Rush feat, you don't provoke an attack 
of opportunity from the defender.) Any attack of opportunity made by anyone other 
than the defender against you during a bull rush has a 25\% chance of accidentally 
targeting the defender instead, and any attack of opportunity by anyone other than 
you against the defender likewise has a 25\% chance of accidentally targeting you. 
(When someone makes an attack of opportunity, make the attack roll and then roll 
to see whether the attack went astray.) 

Second, you and the defender make opposed Strength checks. You each add a +4 bonus 
for each size category you are larger than Medium or a -4 penalty for each size 
category you are smaller than Medium. You get a +2 bonus if you are charging. The 
defender gets a +4 bonus if he has more than two legs or is otherwise exceptionally 
stable.

\textbf{Bull Rush Results:} If you beat the defender's Strength check result, you 
push him back 5 feet. If you wish to move with the defender, you can push him back 
an additional 5 feet for each 5 points by which your check result is greater than 
the defender's check result. You can't, however, exceed your normal movement limit. 
(\textit{Note:} The defender provokes attacks of opportunity if he is moved. So 
do you, if you move with him. The two of you do not provoke attacks of opportunity 
from each other, however.)

If you fail to beat the defender's Strength check result, you move 5 feet straight 
back to where you were before you moved into his space. If that space is occupied, 
you fall prone in that space.

%%%%%%%%%%%%%%%%%%%%%%%%%
\subsection{Charge}\index{Charge}
%%%%%%%%%%%%%%%%%%%%%%%%%

Charging is a special full-round action that allows you to move up to twice your 
speed and attack during the action. However, it carries tight restrictions on how 
you can move.

\textbf{Movement During a Charge:} You must move before your attack, not after. 
You must move at least 10 feet (2 squares) and may move up to double your speed 
directly toward the designated opponent.

You must have a clear path toward the opponent, and nothing can hinder your movement 
(such as difficult terrain or obstacles). Here's what it means to have a clear 
path. First, you must move to the closest space from which you can attack the opponent. 
(If this space is occupied or otherwise blocked, you can't charge.) Second, if 
any line from your starting space to the ending space passes through a square that 
blocks movement, slows movement, or contains a creature (even an ally), you can't 
charge. (Helpless creatures don't stop a charge.)

If you don't have line of sight to the opponent at the start of your turn, you 
can't charge that opponent.

You can't take a 5-foot step in the same round as a charge.

If you are able to take only a standard action or a move action on your turn, you 
can still charge, but you are only allowed to move up to your speed (instead of 
up to double your speed). You can't use this option unless you are restricted to 
taking only a standard action or move action on your turn.

\textbf{Attacking on a Charge:} After moving, you may make a single melee attack. 
You get a +2 bonus on the attack roll. and take a -2 penalty to your AC until the 
start of your next turn.

A charging character gets a +2 bonus on the Strength check made to bull rush an 
opponent (see Bull Rush, above).

Even if you have extra attacks, such as from having a high enough base attack bonus 
or from using multiple weapons, you only get to make one attack during a charge.

\textbf{Lances and Charge Attacks:} A lance deals double damage if employed by 
a mounted character in a charge.

\textbf{Weapons Readied against a Charge:} Spears, tridents, and certain other 
piercing weapons deal double damage when readied (set) and used against a charging 
character.

%%%%%%%%%%%%%%%%%%%%%%%%%
\subsection{Disarm}\index{Disarm}
%%%%%%%%%%%%%%%%%%%%%%%%%

As a melee attack, you may attempt to disarm your opponent. If you do so with a 
weapon, you knock the opponent's weapon out of his hands and to the ground. If 
you attempt the disarm while unarmed, you end up with the weapon in your hand.

If you're attempting to disarm a melee weapon, follow the steps outlined here. 
If the item you are attempting to disarm isn't a melee weapon the defender may 
still oppose you with an attack roll, but takes a penalty and can't attempt to 
disarm you in return if your attempt fails.

\textbf{Step 1:} Attack of Opportunity. You provoke an attack of opportunity from 
the target you are trying to disarm. (If you have the Improved Disarm feat, you 
don't incur an attack of opportunity for making a disarm attempt.) If the defender's 
attack of opportunity deals any damage, your disarm attempt fails.

\textbf{Step 2:} Opposed Rolls. You and the defender make opposed attack rolls 
with your respective weapons. The wielder of a two-handed weapon on a disarm attempt 
gets a +4 bonus on this roll, and the wielder of a light weapon takes a -4 penalty. 
(An unarmed strike is considered a light weapon, so you always take a penalty when 
trying to disarm an opponent by using an unarmed strike.) If the combatants are 
of different sizes, the larger combatant gets a bonus on the attack roll of +4 
per difference in size category. If the targeted item isn't a melee weapon, the 
defender takes a -4 penalty on the roll.

\textbf{Step Three:} Consequences. If you beat the defender, the defender is disarmed. 
If you attempted the disarm action unarmed, you now have the weapon. If you were 
armed, the defender's weapon is on the ground in the defender's square.

If you fail on the disarm attempt, the defender may immediately react and attempt 
to disarm you with the same sort of opposed melee attack roll. His attempt does 
not provoke an attack of opportunity from you. If he fails his disarm attempt, 
you do not subsequently get a free disarm attempt against him.

\textit{Note:} A defender wearing spiked gauntlets can't be disarmed. A defender 
using a weapon attached to a locked gauntlet gets a +10 bonus to resist being disarmed.

%%%
\subsubsection{Grabbing Items}
%%%

You can use a disarm action to snatch an item worn by the target. If you want to 
have the item in your hand, the disarm must be made as an unarmed attack.

If the item is poorly secured or otherwise easy to snatch or cut away the attacker 
gets a +4 bonus. Unlike on a normal disarm attempt, failing the attempt doesn't 
allow the defender to attempt to disarm you. This otherwise functions identically 
to a disarm attempt, as noted above.

You can't snatch an item that is well secured unless you have pinned the wearer 
(see Grapple). Even then, the defender gains a +4 bonus on his roll to resist the 
attempt.

%%%%%%%%%%%%%%%%%%%%%%%%%
\subsection{Feint}\index{Feint}
%%%%%%%%%%%%%%%%%%%%%%%%%

Feinting is a standard action. To feint, make a Bluff check opposed by a Sense 
Motive check by your target. The target may add his base attack bonus to this Sense 
Motive check. If your Bluff check result exceeds your target's Sense Motive check 
result, the next melee attack you make against the target does not allow him to 
use his Dexterity bonus to AC (if any). This attack must be made on or before your 
next turn.

When feinting in this way against a nonhumanoid you take a -4 penalty. Against 
a creature of animal Intelligence (1 or 2), you take a -8 penalty. Against a nonintelligent 
creature, it's impossible.

Feinting in combat does not provoke attacks of opportunity.

\textbf{Feinting as a Move Action:} With the Improved Feint feat, you can attempt 
a feint as a move action instead of as a standard action.

%%%%%%%%%%%%%%%%%%%%%%%%%
\subsection{Grapple}\index{Grapple}
%%%%%%%%%%%%%%%%%%%%%%%%%

%%%
\subsubsection{Grapple Checks}
%%%

Repeatedly in a grapple, you need to make opposed grapple checks against an opponent. 
A grapple check is like a melee attack roll. Your attack bonus on a grapple check 
is: Base attack bonus + Strength modifier + special size modifier

\textbf{Special Size Modifier:} The special size modifier for a grapple check is 
as follows: Colossal +16, Gargantuan +12, Huge +8, Large +4, Medium +0, Small -4, 
Tiny -8, Diminutive -12, Fine -16. Use this number in place of the normal size 
modifier you use when making an attack roll.

%%%
\subsubsection{Starting a Grapple}
%%%

To start a grapple, you need to grab and hold your target. Starting a grapple requires 
a successful melee attack roll. If you get multiple attacks, you can attempt to 
start a grapple multiple times (at successively lower base attack bonuses).

\textbf{Step 1:} Attack of Opportunity. You provoke an attack of opportunity from 
the target you are trying to grapple. If the attack of opportunity deals damage, 
the grapple attempt fails. (Certain monsters do not provoke attacks of opportunity 
when they attempt to grapple, nor do characters with the Improved Grapple feat.) 
If the attack of opportunity misses or fails to deal damage, proceed to Step 2.

\textbf{Step 2:} Grab. You make a melee touch attack to grab the target. If you 
fail to hit the target, the grapple attempt fails. If you succeed, proceed to Step 3.

\textbf{Step 3:} Hold. Make an opposed grapple check as a free action.

If you succeed, you and your target are now grappling, and you deal damage to the 
target as if with an unarmed strike.

If you lose, you fail to start the grapple. You automatically lose an attempt to 
hold if the target is two or more size categories larger than you are.

In case of a tie, the combatant with the higher grapple check modifier wins. If 
this is a tie, roll again to break the tie.

\textbf{Step 4:} Maintain Grapple. To maintain the grapple for later rounds, you 
must move into the target's space. (This movement is free and doesn't count as 
part of your movement in the round.)

Moving, as normal, provokes attacks of opportunity from threatening opponents, 
but not from your target.

If you can't move into your target's space, you can't maintain the grapple and 
must immediately let go of the target. To grapple again, you must begin at Step 1.

%%%
\subsubsection{Grappling Consequences}
%%%

While you're grappling, your ability to attack others and defend yourself is limited.

\textbf{No Threatened Squares:} You don't threaten any squares while grappling.

\textbf{No Dexterity Bonus:} You lose your Dexterity bonus to AC (if you have one) 
against opponents you aren't grappling. (You can still use it against opponents 
you are grappling.)

\textbf{No Movement:} You can't move normally while grappling. You may, however, 
make an opposed grapple check (see below) to move while grappling.

%%%
\subsubsection{If You're Grappling}
%%%

When you are grappling (regardless of who started the grapple), you can perform 
any of the following actions. Some of these actions take the place of an attack 
(rather than being a standard action or a move action). If your base attack bonus 
allows you multiple attacks, you can attempt one of these actions in place of each 
of your attacks, but at successively lower base attack bonuses.

\textbf{Activate a Magic Item:} You can activate a magic item, as long as the item 
doesn't require a spell completion trigger. You don't need to make a grapple check 
to activate the item.

\textbf{Attack Your Opponent:} You can make an attack with an unarmed strike, natural 
weapon, or light weapon against another character you are grappling. You take a 
-4 penalty on such attacks.

You can't attack with two weapons while grappling, even if both are light weapons.

\textbf{Cast a Spell:} You can attempt to cast a spell while grappling or even 
while pinned (see below), provided its casting time is no more than 1 standard 
action, it has no somatic component, and you have in hand any material components 
or focuses you might need. Any spell that requires precise and careful action\textit{ 
}is impossible to cast while grappling or being pinned. If the spell is one that 
you can cast while grappling, you must make a Concentration check (DC 20 + spell 
level) or lose the spell. You don't have to make a successful grapple check to 
cast the spell.

\textbf{Damage Your Opponent:} While grappling, you can deal damage to your opponent 
equivalent to an unarmed strike. Make an opposed grapple check in place of an attack. 
If you win, you deal nonlethal damage as normal for your unarmed strike (1d3 points 
for Medium attackers or 1d2 points for Small attackers, plus Strength modifiers). 
If you want to deal lethal damage, you take a -4 penalty on your grapple check.

\textit{Exception:} Monks deal more damage on an unarmed strike than other characters, 
and the damage is lethal. However, they can choose to deal their damage as nonlethal 
damage when grappling without taking the usual -4 penalty for changing lethal damage 
to nonlethal damage.

\textbf{Draw a Light Weapon:} You can draw a light weapon as a move action with 
a successful grapple check.

\textbf{Escape from Grapple:} You can escape a grapple by winning an opposed grapple 
check in place of making an attack. You can make an Escape Artist check in place 
of your grapple check if you so desire, but this requires a standard action. If 
more than one opponent is grappling you, your grapple check result has to beat 
all their individual check results to escape. (Opponents don't have to try to hold 
you if they don't want to.) If you escape, you finish the action by moving into 
any space adjacent to your opponent(s).

\textbf{Move:} You can move half your speed (bringing all others engaged in the 
grapple with you) by winning an opposed grapple check. This requires a standard 
action, and you must beat all the other individual check results to move the grapple.

\textit{Note:} You get a +4 bonus on your grapple check to move a pinned opponent, 
but only if no one else is involved in the grapple.

\textbf{Retrieve a Spell Component:} You can produce a spell component from your 
pouch while grappling by using a full-round action. Doing so does not require a 
successful grapple check.

\textbf{Pin Your Opponent:} You can hold your opponent immobile for 1 round by 
winning an opposed grapple check (made in place of an attack). Once you have an 
opponent pinned, you have a few options available to you (see below).

\textbf{Break Another's Pin:} If you are grappling an opponent who has another 
character pinned, you can make an opposed grapple check in place of an attack. 
If you win, you break the hold that the opponent has over the other character. 
The character is still grappling, but is no longer pinned.

\textbf{Use Opponent's Weapon:} If your opponent is holding a light weapon, you 
can use it to attack him. Make an opposed grapple check (in place of an attack). 
If you win, make an attack roll with the weapon with a -4 penalty (doing this doesn't 
require another action).

You don't gain possession of the weapon by performing this action.

%%%
\subsubsection{If You're Pinning an Opponent}
%%%

You can attempt to damage your opponent with an opposed grapple check, you can 
attempt to use your opponent's weapon against him, or you can attempt to move the 
grapple (all described above). At your option, you can prevent a pinned opponent 
from speaking.

You can use a disarm action to remove or grab away a well secured object worn by 
a pinned opponent, but he gets a +4 bonus on his roll to resist your attempt (see 
Disarm).

You may voluntarily release a pinned character as a free action; if you do so, 
you are no longer considered to be grappling that character (and vice versa).

You can't draw or use a weapon (against the pinned character or any other character), 
escape another's grapple, retrieve a spell component, pin another character, or 
break another's pin while you are pinning an opponent.

%%%
\subsubsection{If You're Pinned by an Opponent}
%%%

When an opponent has pinned you, you are held immobile (but not helpless) for 1 
round. While you're pinned, you take a -4 penalty to your AC against opponents 
other than the one pinning you. At your opponent's option, you may also be unable 
to speak. On your turn, you can try to escape the pin by making an opposed grapple 
check in place of an attack. You can make an Escape Artist check in place of your 
grapple check if you want, but this requires a standard action. If you win, you 
escape the pin, but you're still grappling.

%%%
\subsubsection{Joining a Grapple}
%%%

If your target is already grappling someone else, you can use an attack to start 
a grapple, as above, except that the target doesn't get an attack of opportunity 
against you, and your grab automatically succeeds. You still have to make a successful 
opposed grapple check to become part of the grapple.

If there are multiple opponents involved in the grapple, you pick one to make the 
opposed grapple check against.

%%%
\subsubsection{Multiple Grapplers}
%%%

Several combatants can be in a single grapple. Up to four combatants can grapple 
a single opponent in a given round. Creatures that are one or more size categories 
smaller than you count for half, creatures that are one size category larger than 
you count double, and creatures two or more size categories larger count quadruple.

When you are grappling with multiple opponents, you choose one opponent to make 
an opposed check against. The exception is an attempt to escape from the grapple; 
to successfully escape, your grapple check must beat the check results of each 
opponent.

%%%%%%%%%%%%%%%%%%%%%%%%%
\subsection{Mounted Combat}\index{Mounted Combat}
%%%%%%%%%%%%%%%%%%%%%%%%%

\textbf{Horses in Combat:} Warhorses and warponies can serve readily as combat 
steeds. Light horses, ponies, and heavy horses, however, are frightened by combat. 
If you don't dismount, you must make a DC 20 Ride check each round as a move action 
to control such a horse. If you succeed, you can perform a standard action after 
the move action. If you fail, the move action becomes a full round action and you 
can't do anything else until your next turn.

Your mount acts on your initiative count as you direct it. You move at its speed, 
but the mount uses its action to move.

A horse (not a pony) is a Large creature and thus takes up a space 10 feet (2 squares) 
across. For simplicity, assume that you share your mount's space during combat.

\textbf{Combat while Mounted:} With a DC 5 Ride check, you can guide your mount 
with your knees so as to use both hands to attack or defend yourself. This is a 
free action.

When you attack a creature smaller than your mount that is on foot, you get the 
+1 bonus on melee attacks for being on higher ground. If your mount moves more 
than 5 feet, you can only make a single melee attack. Essentially, you have to 
wait until the mount gets to your enemy before attacking, so you can't make a full 
attack. Even at your mount's full speed, you don't take any penalty on melee attacks 
while mounted.

If your mount charges, you also take the AC penalty associated with a charge. If 
you make an attack at the end of the charge, you receive the bonus gained from 
the charge. When charging on horseback, you deal double damage with a lance (see 
Charge).

You can use ranged weapons while your mount is taking a double move, but at a -4 
penalty on the attack roll. You can use ranged weapons while your mount is running 
(quadruple speed), at a -8 penalty. In either case, you make the attack roll when 
your mount has completed half its movement. You can make a full attack with a ranged 
weapon while your mount is moving. Likewise, you can take move actions normally

\textbf{Casting Spells while Mounted:} You can cast a spell normally if your mount 
moves up to a normal move (its speed) either before or after you cast. If you have 
your mount move both before and after you cast a spell, then you're casting the 
spell while the mount is moving, and you have to make a Concentration check due 
to the vigorous motion (DC 10 + spell level) or lose the spell. If the mount is 
running (quadruple speed), you can cast a spell when your mount has moved up to 
twice its speed, but your Concentration check is more difficult due to the violent 
motion (DC 15 + spell level).

\textbf{If Your Mount Falls in Battle:} If your mount falls, you have to succeed 
on a DC 15 Ride check to make a soft fall and take no damage. If the check fails, 
you take 1d6 points of damage.

\textbf{If You Are Dropped:} If you are knocked unconscious, you have a 50\% chance 
to stay in the saddle (or 75\% if you're in a military saddle). Otherwise you fall 
and take 1d6 points of damage.

Without you to guide it, your mount avoids combat.

%%%%%%%%%%%%%%%%%%%%%%%%%
\subsection{Overrun}\index{Overrun}
%%%%%%%%%%%%%%%%%%%%%%%%%

You can attempt an overrun as a standard action taken during your move. (In general, 
you cannot take a standard action during a move; this is an exception.) With an 
overrun, you attempt to plow past or over your opponent (and move through his square) 
as you move. You can only overrun an opponent who is one size category larger than 
you, the same size, or smaller. You can make only one overrun attempt per round.

If you're attempting to overrun an opponent, follow these steps.

\textbf{Step 1:} Attack of Opportunity. Since you begin the overrun by moving into 
the defender's space, you provoke an attack of opportunity from the defender.

\textbf{Step 2:} Opponent Avoids? The defender has the option to simply avoid you. 
If he avoids you, he doesn't suffer any ill effect and you may keep moving (You 
can always move through a square occupied by someone who lets you by.) The overrun 
attempt doesn't count against your actions this round (except for any movement 
required to enter the opponent's square). If your opponent doesn't avoid you, move 
to Step 3.

\textbf{Step 3:} Opponent Blocks? If your opponent blocks you, make a Strength 
check opposed by the defender's Dexterity or Strength check (whichever ability 
score has the higher modifier). A combatant gets a +4 bonus on the check for every 
size category he is larger than Medium or a -4 penalty for every size category 
he is smaller than Medium. The defender gets a +4 bonus on his check if he has 
more than two legs or is otherwise more stable than a normal humanoid. If you win, 
you knock the defender prone. If you lose, the defender may immediately react and 
make a Strength check opposed by your Dexterity or Strength check (including the 
size modifiers noted above, but no other modifiers) to try to knock you prone.

\textbf{Step 4:} Consequences. If you succeed in knocking your opponent prone, 
you can continue your movement as normal. If you fail and are knocked prone in 
turn, you have to move 5 feet back the way you came and fall prone, ending your 
movement there. If you fail but are not knocked prone, you have to move 5 feet 
back the way you came, ending your movement there. If that square is occupied, 
you fall prone in that square.

\textbf{Improved Overrun:} If you have the Improved Overrun feat, your target may 
not choose to avoid you.

\textbf{Mounted Overrun (Trample):} If you attempt an overrun while mounted, your 
mount makes the Strength check to determine the success or failure of the overrun 
attack (and applies its size modifier, rather than yours). If you have the Trample 
feat and attempt an overrun while mounted, your target may not choose to avoid 
you, and if you knock your opponent prone with the overrun, your mount may make 
one hoof attack against your opponent.

%%%%%%%%%%%%%%%%%%%%%%%%%
\subsection{Sunder}\index{Sunder}
%%%%%%%%%%%%%%%%%%%%%%%%%

You can use a melee attack with a slashing or bludgeoning weapon to strike a weapon 
or shield that your opponent is holding. If you're attempting to sunder a weapon 
or shield, follow the steps outlined here. (Attacking held objects other than weapons 
or shields is covered below.)

\textbf{Step 1:} Attack of Opportunity. You provoke an attack of opportunity from 
the target whose weapon or shield you are trying to sunder. (If you have the Improved 
Sunder feat, you don't incur an attack of opportunity for making the attempt.)

\textbf{Step 2:} Opposed Rolls. You and the defender make opposed attack rolls 
with your respective weapons. The wielder of a two-handed weapon on a sunder attempt 
gets a +4 bonus on this roll, and the wielder of a light weapon takes a -4 penalty. 
If the combatants are of different sizes, the larger combatant gets a bonus on 
the attack roll of +4 per difference in size category.

\textbf{Step 3:} Consequences. If you beat the defender, roll damage and deal it 
to the weapon or shield. See Table: Common Armor, Weapon, and Shield Hardness and 
Hit Points to determine how much damage you must deal to destroy the weapon or 
shield.

If you fail the sunder attempt, you don't deal any damage.

\textit{Sundering a Carried or Worn Object:} You don't use an opposed attack roll 
to damage a carried or worn object. Instead, just make an attack roll against the 
object's AC. A carried or worn object's AC is equal to 10 + its size modifier + 
the Dexterity modifier of the carrying or wearing character. Attacking a carried 
or worn object provokes an attack of opportunity just as attacking a held object 
does. To attempt to snatch away an item worn by a defender rather than damage it, 
see Disarm. You can't sunder armor worn by another character.

%%%%%%%%%%%%%%%%%%%%%%%%%
\subsection{Throw Splash Weapon}\index{Splash Weapons}
%%%%%%%%%%%%%%%%%%%%%%%%%

A splash weapon is a ranged weapon that breaks on impact, splashing or scattering 
its contents over its target and nearby creatures or objects. To attack with a 
splash weapon, make a ranged touch attack against the target. Thrown weapons require 
no weapon proficiency, so you don't take the -4 nonproficiency penalty. A hit deals 
direct hit damage to the target, and splash damage to all creatures within 5 feet 
of the target.

You can instead target a specific grid intersection. Treat this as a ranged attack 
against AC 5. However, if you target a grid intersection, creatures in all adjacent 
squares are dealt the splash damage, and the direct hit damage is not dealt to 
any creature. (You can't target a grid intersection occupied by a creature, such 
as a Large or larger creature; in this case, you're aiming at the creature.)

If you miss the target (whether aiming at a creature or a grid intersection), roll 
1d8. This determines the misdirection of the throw, with 1 being straight back 
at you and 2 through 8 counting clockwise around the grid intersection or target 
creature. Then, count a number of squares in the indicated direction equal to the 
range increment of the throw.

After you determine where the weapon landed, it deals splash damage to all creatures 
in adjacent squares.

%%%%%%%%%%%%%%%%%%%%%%%%%
\subsection{Trip}\index{Trip}
%%%%%%%%%%%%%%%%%%%%%%%%%

You can try to trip an opponent as an unarmed melee attack. You can only trip an 
opponent who is one size category larger than you, the same size, or smaller.

\textbf{Making a Trip Attack:} Make an unarmed melee touch attack against your 
target. This provokes an attack of opportunity from your target as normal for unarmed 
attacks.

If your attack succeeds, make a Strength check opposed by the defender's Dexterity 
or Strength check (whichever ability score has the higher modifier). A combatant 
gets a +4 bonus for every size category he is larger than Medium or a -4 penalty 
for every size category he is smaller than Medium. The defender gets a +4 bonus 
on his check if he has more than two legs or is otherwise more stable than a normal 
humanoid. If you win, you trip the defender. If you lose, the defender may immediately 
react and make a Strength check opposed by your Dexterity or Strength check to 
try to trip you.

\textit{Avoiding Attacks of Opportunity:} If you have the Improved Trip feat, or 
if you are tripping with a weapon (see below), you don't provoke an attack of opportunity 
for making a trip attack.

\textbf{Being Tripped (Prone):} A tripped character is prone. Standing up is a 
move action.

\textbf{Tripping a Mounted Opponent:} You may make a trip attack against a mounted 
opponent. The defender may make a Ride check in place of his Dexterity or Strength 
check. If you succeed, you pull the rider from his mount.

\textbf{Tripping with a Weapon:} Some weapons can be used to make trip attacks. 
In this case, you make a melee touch attack with the weapon instead of an unarmed 
melee touch attack, and you don't provoke an attack of opportunity.

If you are tripped during your own trip attempt, you can drop the weapon to avoid 
being tripped.

%%%%%%%%%%%%%%%%%%%%%%%%%
\subsection{Turn or Rebuke Undead}\index{Turn Undead}\index{Rebuke Undead}
%%%%%%%%%%%%%%%%%%%%%%%%%

Good clerics and paladins and some neutral clerics can channel positive energy, 
which can halt, drive off (rout), or destroy undead.

Evil clerics and some neutral clerics can channel negative energy, which can halt, 
awe (rebuke), control (command), or bolster undead.

Regardless of the effect, the general term for the activity is "turning." When 
attempting to exercise their divine control over these creatures, characters make 
turning checks.

%%%
\subsubsection{Turning Checks}
%%%

Turning undead is a supernatural ability that a character can perform as a standard 
action. It does not provoke attacks of opportunity.

You must present your holy symbol to turn undead. Turning is considered an attack.

\textbf{Times per Day:} You may attempt to turn undead a number of times per day 
equal to 3 + your Charisma modifier. You can increase this number by taking the 
Extra Turning feat.

\textbf{Range:} You turn the closest turnable undead first, and you can't turn 
undead that are more than 60 feet away or that have total cover relative to you. 
You don't need line of sight to a target, but you do need line of effect.

\textbf{Turning Check:} The first thing you do is roll a turning check to see how 
powerful an undead creature you can turn. This is a Charisma check (1d20 + your 
Charisma modifier). Table: Turning Undead gives you the Hit Dice of the most powerful 
undead you can affect, relative to your level. On a given turning attempt, you 
can turn no undead creature whose Hit Dice exceed the result on this table.

\textbf{Turning Damage:} If your roll on Table: Turning Undead is high enough to 
let you turn at least some of the undead within 60 feet, roll 2d6 + your cleric 
level + your Charisma modifier for turning damage. That's how many total Hit Dice 
of undead you can turn.

If your Charisma score is average or low, it's possible to roll fewer Hit Dice 
of undead turned than indicated on Table: Turning Undead.

You may skip over already turned undead that are still within range, so that you 
do not waste your turning capacity on them.

\textbf{Effect and Duration of Turning:} Turned undead flee from you by the best 
and fastest means available to them. They flee for 10 rounds (1 minute). If they 
cannot flee, they cower (giving any attack rolls against them a +2 bonus). If you 
approach within 10 feet of them, however, they overcome being turned and act normally. 
(You can stand within 10 feet without breaking the turning effect -- you just can't 
approach them.) You can attack them with ranged attacks (from at least 10 feet 
away), and others can attack them in any fashion, without breaking the turning 
effect.

\textbf{Destroying Undead:} If you have twice as many levels (or more) as the undead 
have Hit Dice, you destroy any that you would normally turn.

\begin{table}[htb]
\rowcolors{1}{white}{offyellow}\mcinherit
\caption{Turning Undead}
\centering
\begin{tabular}{c c}
\multicolumn{1}{p{3.5cm}}{\textbf{Turning Check Result}} & \multicolumn{1}{p{5cm}}{\textbf{Most Powerful Undead Affected (Maximum Hit Dice)}}\\
0 or lower & Cleric's Level -4\\
1-3 & Cleric's Level -3\\
4-6 & Cleric's Level -2\\
7-9 & Cleric's Level -1\\
10-12 & Cleric's Level\\
13-15 & Cleric's Level +1\\
16-18 & Cleric's Level +2\\
19-21 & Cleric's Level +3\\
22 or higher & Cleric's Level +4\\
\end{tabular}
\end{table}

%%%
\subsubsection{Evil Clerics and Undead}
%%%

Evil clerics channel negative energy to rebuke (awe) or command (control) undead 
rather than channeling positive energy to turn or destroy them. An evil cleric 
makes the equivalent of a turning check. Undead that would be turned are rebuked 
instead, and those that would be destroyed are commanded.

\textbf{Rebuked:} A rebuked undead creature cowers as if in awe (attack rolls against 
the creature get a +2 bonus). The effect lasts 10 rounds.

\textbf{Commanded:} A commanded undead creature is under the mental control of 
the evil cleric. The cleric must take a standard action to give mental orders to 
a commanded undead. At any one time, the cleric may command any number of undead 
whose total Hit Dice do not exceed his level. He may voluntarily relinquish command 
on any commanded undead creature or creatures in order to command new ones.

\textbf{Dispelling Turning:} An evil cleric may channel negative energy to dispel 
a good cleric's turning effect. The evil cleric makes a turning check as if attempting 
to rebuke the undead. If the turning check result is equal to or greater than the 
turning check result that the good cleric scored when turning the undead, then 
the undead are no longer turned. The evil cleric rolls turning damage of 2d6 + 
cleric level + Charisma modifier to see how many Hit Dice worth of undead he can 
affect in this way (as if he were rebuking them).

\textbf{Bolstering Undead:} An evil cleric may also bolster undead creatures against 
turning in advance. He makes a turning check as if attempting to rebuke the undead, 
but the Hit Dice result on Table: Turning Undead becomes the undead creatures' 
effective Hit Dice as far as turning is concerned (provided the result is higher 
than the creatures' actual Hit Dice). The bolstering lasts 10 rounds. An evil undead 
cleric can bolster himself in this manner.

%%%
\subsubsection{Neutral Clerics and Undead}
%%%

A cleric of neutral alignment can either turn undead but not rebuke them, or rebuke 
undead but not turn them. See Turn or Rebuke Undead for more information.

Even if a cleric is neutral, channeling positive energy is a good act and channeling 
negative energy is evil.

%%%
\subsubsection{Paladins and Undead}
%%%

Beginning at 4th level, paladins can turn undead as if they were clerics of three 
levels lower than they actually are.

%%%
\subsubsection{Turning Other Creatures}
%%%

Some clerics have the ability to turn creatures other than undead.

The turning check result is determined as normal.

%%%%%%%%%%%%%%%%%%%%%%%%%
\subsection{Two-Weapon Fighting}\index{Two-Weapon Fighting}
%%%%%%%%%%%%%%%%%%%%%%%%%

If you wield a second weapon in your off hand, you can get one extra attack per 
round with that weapon. You suffer a -6 penalty with your regular attack or attacks 
with your primary hand and a -10 penalty to the attack with your off hand when 
you fight this way. You can reduce these penalties in two ways:

\begin{itemize*}
\item If your off-hand weapon is light, the penalties are reduced by 2 each. (An unarmed strike is always considered light.)
\item The \linkfeat{Two-Weapon Fighting} feat lessens the primary hand penalty by 2, and the off-hand penalty by 6.
\end{itemize*}

Table: Two-Weapon Fighting Penalties summarizes the interaction of all these factors.

\begin{table}[htb]
\rowcolors{1}{white}{offyellow}\mcinherit
\caption{Two-Weapon Fighting Penalties}
\centering
\begin{tabular}{l c c}
\textbf{Circumstances} & \textbf{Primary Hand} & \textbf{Off Hand}\\
Normal penalties & -6 & -10\\
Off-hand weapon is light & -4 & -8\\
Two-Weapon Fighting feat & -4 & -4\\
\shortstack{Off-hand weapon is light and\\Two-Weapon Fighting feat} & -2 & -2\\
\end{tabular}
\end{table}

\textbf{Double Weapons}: You can use a double weapon to make an extra attack with 
the off-hand end of the weapon as if you were fighting with two weapons. The penalties 
apply as if the off-hand end of the weapon were a light weapon.

\textbf{Thrown Weapons:} The same rules apply when you throw a weapon from each 
hand. Treat a dart or shuriken as a light weapon when used in this manner, and 
treat a bolas, javelin, net, or sling as a one-handed weapon.

%%%%%%%%%%%%%%%%%%%%%%%%%%%%%%%%%%%%%%%%%%%%%%%%%%
\section{Special Initiative Actions}
%%%%%%%%%%%%%%%%%%%%%%%%%%%%%%%%%%%%%%%%%%%%%%%%%%

Here are ways to change when you act during combat by altering your place in the 
initiative order.

%%%%%%%%%%%%%%%%%%%%%%%%%
\subsection{Delay}\index{Delay Action}\index{Initiative!Delay}
%%%%%%%%%%%%%%%%%%%%%%%%%

By choosing to delay, you take no action and then act normally on whatever initiative 
count you decide to act. When you delay, you voluntarily reduce your own initiative 
result for the rest of the combat. When your new, lower initiative count comes 
up later in the same round, you can act normally. You can specify this new initiative 
result or just wait until some time later in the round and act then, thus fixing 
your new initiative count at that point.

You never get back the time you spend waiting to see what's going to happen. You 
can't, however, interrupt anyone else's action (as you can with a readied action).

\textbf{Initiative Consequences of Delaying:} Your initiative result becomes the 
count on which you took the delayed action. If you come to your next action and 
have not yet performed an action, you don't get to take a delayed action (though 
you can delay again).

If you take a delayed action in the next round, before your regular turn comes 
up, your initiative count rises to that new point in the order of battle, and you 
do not get your regular action that round.

%%%%%%%%%%%%%%%%%%%%%%%%%
\subsection{Ready}\index{Readied Action}\index{Initiative!Readied Action}
%%%%%%%%%%%%%%%%%%%%%%%%%

The ready action lets you prepare to take an action later, after your turn is over 
but before your next one has begun. Readying is a standard action. It does not 
provoke an attack of opportunity (though the action that you ready might do so).

\textbf{Readying an Action:} You can ready a standard action, a move action, or 
a free action. To do so, specify the action you will take and the conditions under 
which you will take it. Then, any time before your next action, you may take the 
readied action in response to that condition. The action occurs just before the 
action that triggers it. If the triggered action is part of another character's 
activities, you interrupt the other character. Assuming he is still capable of 
doing so, he continues his actions once you complete your readied action. Your 
initiative result changes. For the rest of the encounter, your initiative result 
is the count on which you took the readied action, and you act immediately ahead 
of the character whose action triggered your readied action.

You can take a 5-foot step as part of your readied action, but only if you don't 
otherwise move any distance during the round. 

\textbf{Initiative Consequences of Readying:} Your initiative result becomes the 
count on which you took the readied action. If you come to your next action and 
have not yet performed your readied action, you don't get to take the readied action 
(though you can ready the same action again). If you take your readied action in 
the next round, before your regular turn comes up, your initiative count rises 
to that new point in the order of battle, and you do not get your regular action 
that round.

\textbf{Distracting Spellcasters:} You can ready an attack against a spellcaster 
with the trigger "if she starts casting a spell." If you damage the spellcaster, 
she may lose the spell she was trying to cast (as determined by her Concentration 
check result).

\textbf{Readying to Counterspell:} You may ready a counterspell against a spellcaster 
(often with the trigger "if she starts casting a spell"). In this case, when 
the spellcaster starts a spell, you get a chance to identify it with a Spellcraft 
check (DC 15 + spell level). If you do, and if you can cast that same spell (are 
able to cast it and have it prepared, if you prepare spells), you can cast the 
spell as a counterspell and automatically ruin the other spellcaster's spell. Counterspelling 
works even if one spell is divine and the other arcane.

A spellcaster can use \linkspell{Dispel Magic} to counterspell another spellcaster, 
but it doesn't always work.

\textbf{Readying a Weapon against a Charge:} You can ready certain piercing weapons, 
setting them to receive charges. A readied weapon of this type deals double damage 
if you score a hit with it against a charging character.

%%%%%%%%%%%%%%%%%%%%%%%%%%%%%%%%%%%%%%%%%%%%%%%%%%
\section{Conditions}
%%%%%%%%%%%%%%%%%%%%%%%%%%%%%%%%%%%%%%%%%%%%%%%%%%

If more than one condition affects a character, apply them all. If certain effects 
can't combine, apply the most severe effect.

%%%
\subsubsection{Ability Damaged}\index{Ability Damage}
%%%

The character has temporarily lost 1 or more ability 
score points. Lost points return at a rate of 1 per day unless noted otherwise 
by the condition dealing the damage. A character with Strength 0 falls to the ground 
and is helpless. A character with Dexterity 0 is paralyzed. A character with Constitution 
0 is dead. A character with Intelligence, Wisdom, or Charisma 0 is unconscious. 
Ability damage is different from penalties to ability scores, which go away when 
the conditions causing them go away.

%%%
\subsubsection{Ability Drained}\index{Ability Drain}
%%%

The character has permanently lost 1 or more ability 
score points. The character can regain these points only through magical means. 
A character with Strength 0 falls to the ground and is helpless. A character with 
Dexterity 0 is paralyzed. A character with Constitution 0 is dead. A character 
with Intelligence, Wisdom, or Charisma 0 is unconscious. 

%%%
\subsubsection{Blinded}\index{Blind}
%%%

The character cannot see. He takes a -2 penalty to Armor Class, 
loses his Dexterity bonus to AC (if any), moves at half speed, and takes a -4 penalty 
on \linkskill{Search} checks and on most Strength- and Dexterity-based skill checks. All checks 
and activities that rely on vision (such as reading and \linkskill{Spot} checks) automatically 
fail. All opponents are considered to have total concealment (50\% miss chance) 
to the blinded character. Characters who remain blinded for a long time grow accustomed 
to these drawbacks and can overcome some of them.

%%%
\subsubsection{Blown Away}\index{Blown Away}
%%%

Depending on its size, a creature can be blown away by winds 
of high velocity. A creature on the ground that is blown away is knocked down and 
rolls 1d4 x 10 feet, taking 1d4 points of nonlethal damage per 10 feet. A flying 
creature that is blown away is blown back 2d6 x 10 feet and takes 2d6 points of 
nonlethal damage due to battering and buffering. 

%%%
\subsubsection{Checked}\index{Checked}
%%%

Prevented from achieving forward motion by an applied force, 
such as wind. Checked creatures on the ground merely stop. Checked flying creatures 
move back a distance specified in the description of the effect.

%%%
\subsubsection{Confused}\index{Confused}
%%%

A confused character's actions are 
determined by rolling d\% at the beginning of his turn: 01-10, attack caster with 
melee or ranged weapons (or close with caster if attacking is not possible); 11-20, 
act normally; 21-50, do nothing but babble incoherently; 51-70, flee away from 
caster at top possible speed; 71-100, attack nearest creature (for this purpose, 
a familiar counts as part of the subject's self). A confused character 
who can't carry out the indicated action does nothing but babble incoherently. 
Attackers are not at any special advantage when attacking a confused character. 
Any confused character who is attacked automatically attacks its attackers 
on its next turn, as long as it is still confused when its turn comes. 
A confused character does not make attacks of opportunity against any 
creature that it is not already devoted to attacking (either because of its most 
recent action or because it has just been attacked).

%%%
\subsubsection{Cowering}\index{Cowering}
%%%

The character is frozen in fear and can take no actions. A cowering 
character takes a -2 penalty to Armor Class and loses her Dexterity bonus (if any).

%%%
\subsubsection{Dazed}\index{Dazed}
%%%

The creature is unable to act normally. A dazed creature can take 
no actions, but has no penalty to AC.

A dazed condition typically lasts 1 round.

%%%
\subsubsection{Dazzled}\index{Dazzled}
%%%

The creature is unable to see well because of overstimulation 
of the eyes. A dazzled creature takes a -1 penalty on attack rolls, \linkskill{Search} checks, 
and \linkskill{Spot} checks.

%%%
\subsubsection{Dead}\index{Dead}
%%%

The character's hit points are reduced to -10, his Constitution 
drops to 0, or he is killed outright by a spell or effect. The character's soul 
leaves his body. Dead characters cannot benefit from normal or magical healing, 
but they can be restored to life via magic. A dead body decays normally unless 
magically preserved, but magic that restores a dead character to life also restores 
the body either to full health or to its condition at the time of death (depending 
on the spell or device). Either way, resurrected characters need not worry about 
rigor mortis, decomposition, and other conditions that affect dead bodies.

%%%
\subsubsection{Deafened}\index{Deafened}
%%%

A deafened character cannot hear. She takes a -4 penalty on 
initiative checks, automatically fails \linkskill{Listen} checks, and has a 20\% chance of 
spell failure when casting spells with verbal components. Characters who remain 
deafened for a long time grow accustomed to these drawbacks and can overcome some 
of them.

%%%
\subsubsection{Disabled}\index{Disabled}
%%%

A character with 0 hit points, or one who has negative hit points 
but has become stable and conscious, is disabled. A disabled character may take 
a single move action or standard action each round (but not both, nor can she take 
full-round actions). She moves at half speed. Taking move actions doesn't risk 
further injury, but performing any standard action (or any other action the DM 
deems strenuous, including some free actions such as casting a quickened spell) 
deals 1 point of damage after the completion of the act. Unless the action increased 
the disabled character's hit points, she is now in negative hit points and dying.

A disabled character with negative hit points recovers hit points naturally if 
she is being helped. Otherwise, each day she has a 10\% chance to start recovering 
hit points naturally (starting with that day); otherwise, she loses 1 hit point. 
Once an unaided character starts recovering hit points naturally, she is no longer 
in danger of losing hit points (even if her current hit points are negative).

%%%
\subsubsection{Dying}\index{Dying}
%%%

A dying character is unconscious and near death. She has -1 to 
-9 current hit points. A dying character can take no actions and is unconscious. 
At the end of each round (starting with the round in which the character dropped 
below 0 hit points), the character rolls d\% to see whether she becomes stable. 
She has a 10\% chance to become stable. If she does not, she loses 1 hit point. 
If a dying character reaches -10 hit points, she is dead.

%%%
\subsubsection{Energy Drained}\index{Energy Drain}
%%%

The character gains one or more negative levels, which 
might permanently drain the character's levels. If the subject has at least as 
many negative levels as Hit Dice, he dies. Each negative level gives a creature 
the following penalties: -1 penalty on attack rolls, saving throws, skill checks, 
ability checks; loss of 5 hit points; and -1 to effective level (for determining 
the power, duration, DC, and other details of spells or special abilities). In 
addition, a spellcaster loses one spell or spell slot from the highest spell level 
castable.

%%%
\subsubsection{Entangled}\index{Entangled}
%%%

The character is ensnared. Being entangled impedes movement, 
but does not entirely prevent it unless the bonds are anchored to an immobile object 
or tethered by an opposing force. An entangled creature moves at half speed, cannot 
run or charge, and takes a -2 penalty on all attack rolls and a -4 penalty to Dexterity. 
An entangled character who attempts to cast a spell must make a Concentration check 
(DC 15 + the spell's level) or lose the spell. 

%%%
\subsubsection{Exhausted}\index{Exhausted}
%%%

An exhausted character moves at half speed and takes a -6 penalty 
to Strength and Dexterity. After 1 hour of complete rest, an exhausted character 
becomes fatigued. A fatigued character becomes exhausted by doing something else 
that would normally cause fatigue.

%%%
\subsubsection{Fascinated}\index{Fascinated}
%%%

A fascinated creature is entranced by a supernatural or spell 
effect. The creature stands or sits quietly, taking no actions other than to pay 
attention to the fascinating effect, for as long as the effect lasts. It takes 
a -4 penalty on skill checks made as reactions, such as \linkskill{Listen} and \linkskill{Spot} checks. 
Any potential threat, such as a hostile creature approaching, allows the fascinated 
creature a new saving throw against the fascinating effect. Any obvious threat, 
such as someone drawing a weapon, casting a spell, or aiming a ranged weapon at 
the fascinated creature, automatically breaks the effect. A fascinated creature's 
ally may shake it free of the spell as a standard action. 

%%%
\subsubsection{Fatigued}\index{Fatigued}
%%%

A fatigued character can neither run nor charge and takes a 
-2 penalty to Strength and Dexterity. Doing anything that would normally cause 
fatigue causes the fatigued character to become exhausted. After 8 hours of complete 
rest, fatigued characters are no longer fatigued. 

%%%
\subsubsection{Flat-Footed}\index{Flat-Footed}
%%%

A character who has not yet acted during a combat is flat-footed, 
not yet reacting normally to the situation. A flat-footed character loses his Dexterity 
bonus to AC (if any) and cannot make attacks of opportunity. 

%%%
\subsubsection{Frightened}\index{Frightened}
%%%

A frightened creature flees from the source of its fear as 
best it can. If unable to flee, it may fight. A frightened creature takes a -2 
penalty on all attack rolls, saving throws, skill checks, and ability checks. A 
frightened creature can use special abilities, including spells, to flee; indeed, 
the creature must use such means if they are the only way to escape. 

Frightened is like shaken, except that the creature must flee if possible.
\linksec{Panicked} is a more extreme state of fear.

%%%
\subsubsection{Grappling}\index{Grappling}
%%%

Engaged in wrestling or some other form of hand-to-hand struggle 
with one or more attackers. A grappling character can undertake only a limited 
number of actions. He does not threaten any squares, and loses his Dexterity bonus 
to AC (if any) against opponents he isn't grappling.

%%%
\subsubsection{Helpless}\index{Helpless}
%%%

A helpless character is paralyzed, \textit{held}, bound, sleeping, 
unconscious, or otherwise completely at an opponent's mercy. A helpless target 
is treated as having a Dexterity of 0 (-5 modifier). Melee attacks against a helpless 
target get a +4 bonus (equivalent to attacking a prone target). Ranged attacks 
gets no special bonus against helpless targets. Rogues can sneak attack helpless 
targets. 

As a full-round action, an enemy can use a melee weapon to deliver a coup de grace 
to a helpless foe. An enemy can also use a bow or crossbow, provided he is adjacent 
to the target. The attacker automatically hits and scores a critical hit. (A rogue 
also gets her sneak attack damage bonus against a helpless foe when delivering 
a coup de grace.) If the defender survives, he must make a Fortitude save (DC 10 
+ damage dealt) or die. 

Delivering a \gameterm{Coup de Grace} provokes attacks of opportunity. 

Creatures that are immune to critical hits do not take critical damage, nor do 
they need to make Fortitude saves to avoid being killed by a coup de grace.

%%%
\subsubsection{Incorporeal}\index{Incorporeal}
%%%

Having no physical body. Incorporeal creatures are immune 
to all nonmagical attack forms. They can be harmed only by other incorporeal creatures, 
+1 or better magic weapons, spells, spell-like effects, or supernatural effects. 

%%%
\subsubsection{Invisible}\index{Invisible}
%%%

Visually undetectable. An invisible creature gains a +2 bonus 
on attack rolls against sighted opponents, and ignores its opponents' Dexterity 
bonuses to AC (if any). (See \linksec{Invisibility}, under Special Abilities.)

%%%
\subsubsection{Knocked Down}\index{Knocked Down}
%%%

Depending on their size, creatures can be knocked down by 
winds of high velocity. Creatures on the ground are knocked prone by the force 
of the wind. Flying creatures are instead blown back 1d6 x 10 feet.

%%%
\subsubsection{Nauseated}\index{Nauseated}
%%%

Experiencing stomach distress. Nauseated creatures are unable 
to attack, cast spells, concentrate on spells, or do anything else requiring attention. 
The only action such a character can take is a single move action per turn.

%%%
\subsubsection{Panicked}\index{Panicked}
%%%

A panicked creature must drop anything it holds and flee at 
top speed from the source of its fear, as well as any other dangers it encounters, 
along a random path. It can't take any other actions. In addition, the creature 
takes a -2 penalty on all saving throws, skill checks, and ability checks. If cornered, 
a panicked creature cowers and does not attack, typically using the total defense 
action in combat. A panicked creature can use special abilities, including spells, 
to flee; indeed, the creature must use such means if they are the only way to escape.

Panicked is a more extreme state of fear than \linksec{Shaken} or \linksec{Frightened}.

%%%
\subsubsection{Paralyzed}\index{Paralyzed}
%%%

A paralyzed character is frozen in place and unable to move 
or act. A paralyzed character has effective Dexterity and Strength scores of 0 
and is helpless, but can take purely mental actions. A winged creature flying in 
the air at the time that it becomes paralyzed cannot flap its wings and falls. 
A paralyzed swimmer can't swim and may drown. A creature can move through a space 
occupied by a paralyzed creature -- ally or not. Each square occupied by a paralyzed 
creature, however, counts as 2 squares.

%%%
\subsubsection{Petrified}\index{Petrified}
%%%

A petrified character has been turned to stone and is considered 
unconscious. If a petrified character cracks or breaks, but the broken pieces are 
joined with the body as he returns to flesh, he is unharmed. If the character's 
petrified body is incomplete when it returns to flesh, the body is likewise incomplete 
and there is some amount of permanent hit point loss and/or debilitation.

%%%
\subsubsection{Pinned}\index{Pinned}
%%%

Held immobile (but not helpless) in a grapple.

%%%
\subsubsection{Prone}\index{Prone}
%%%

The character is on the ground. An attacker who is prone has a 
-4 penalty on melee attack rolls and cannot use a ranged weapon (except for a crossbow). 
A defender who is prone gains a +4 bonus to Armor Class against ranged attacks, 
but takes a -4 penalty to AC against melee attacks.

Standing up is a move-equivalent action that provokes an attack of opportunity.

%%%
\subsubsection{Shaken}\index{Shaken}
%%%

A shaken character takes a -2 penalty on attack rolls, saving 
throws, skill checks, and ability checks.

Shaken is a less severe state of fear than \linksec{Frightened} or \linksec{Panicked}.

%%%
\subsubsection{Sickened}\index{Sickened}
%%%

The character takes a -2 penalty on all attack rolls, weapon 
damage rolls, saving throws, skill checks, and ability checks.

%%%
\subsubsection{Stable}\index{Stable}
%%%

A character who was dying but who has stopped losing hit points 
and still has negative hit points is stable. The character is no longer dying, 
but is still unconscious. If the character has become stable because of aid from 
another character (such as a \linkskill{Heal} check or magical healing), then the character 
no longer loses hit points. He has a 10\% chance each hour of becoming conscious 
and disabled (even though his hit points are still negative). 

If the character became stable on his own and hasn't had help, he is still at risk 
of losing hit points. Each hour, he has a 10\% chance of becoming conscious and 
disabled. Otherwise he loses 1 hit point.

%%%
\subsubsection{Staggered}\index{Staggered}
%%%

A character whose nonlethal damage exactly equals his current 
hit points is staggered. A staggered character may take a single move action or 
standard action each round (but not both, nor can she take full-round actions).

A character whose current hit points exceed his nonlethal damage is no longer staggered; 
a character whose nonlethal damage exceeds his hit points becomes unconscious.

%%%
\subsubsection{Stunned}\index{Stunned}
%%%

A stunned creature drops everything held, can't take actions, 
takes a -2 penalty to AC, and loses his Dexterity bonus to AC (if any).

%%%
\subsubsection{Turned}\index{Turned}
%%%

Affected by a turn undead attempt. Turned undead flee for 10 rounds 
(1 minute) by the best and fastest means available to them. If they cannot flee, 
they cower.

%%%
\subsubsection{Unconscious}\index{Unconscious}
%%%

Knocked out and helpless. Unconsciousness can result from 
having current hit points between -1 and -9, or from nonlethal damage in excess 
of current hit points.

\chapter{Magic}
\section{Casting Spells}
foo
\section{How To Read A Spell Description}
foo
\section{Arcane Spells}
foo
\section{Divine Spells}
foo
\section{Special Abilities and Spells}
foo
\section{Spell Lists}
foo
\input{phb/12-magic-items}

%%%%%%%%%%%%%%%%%%%%%%%%%%%%%%%%%%%%%%%%%%%%%%%%%%
%%%%%%%%%%%%%%%%%%%%%%%%%%%%%%%%%%%%%%%%%%%%%%%%%%
\appendix
%%%%%%%%%%%%%%%%%%%%%%%%%%%%%%%%%%%%%%%%%%%%%%%%%%
%%%%%%%%%%%%%%%%%%%%%%%%%%%%%%%%%%%%%%%%%%%%%%%%%%
\appendixpage

\makeatletter
\renewcommand{\@makechapterhead}[1]{%
\vspace*{0 pt}{
\raggedright \normalfont \fontsize{32}{32} \selectfont \bfseries
\ifnum \value{secnumdepth}>-1
  \if@mainmatter \vspace{-8pt} {\fontsize{20}{20} \selectfont Appendix \thechapter:}\\[8pt]
  \fi%
\fi
\hspace{0.65cm} #1\par\nobreak\vspace{20 pt}
}}
\makeatother

\clearpage

%% Appendix Chapters Here

%%%%%%%%%%%%%%%%%%%%%%%%
%%Spell Formatting
%%%%%%%%%%%%%%%%%%%%%%%%

\newenvironment{spellwriteup}{
\vspace{0pt}
\begin{minipage}[h]{\linewidth}
}{
\end{minipage}
}

\newenvironment{spellinformation}{
\begin{wraptable}[10]{R}{.5\linewidth}
\begin{tabular}{|r p{1.75in}|}
\hline
}{
\hline
\end{tabular}
\end{wraptable}
}

\newcommand{\spellhead}[4]{\belowpdfbookmark{#1}{spell:#1}\paragraph{\Large#1}\large\textbf{#2 (#3) [#4]}\normalsize \\}

%\paragraph{\Large#1}\large\textbf{#2}

\newcommand{\spellschool}[3]{\multicolumn{2}{l}{\textbf{#1 (#2) [#3]}}\\}
\newcommand{\spelllevel}[1]{\textbf{Level:} & #1\\}
\newcommand{\spellcomp}[1]{\textbf{Components:} & #1\\}
\newcommand{\spellcast}[1]{\textbf{Casting Time:} & #1\\}
\newcommand{\spellrange}[1]{\textbf{Range:} & #1\\}
\newcommand{\spelleffect}[1]{\textbf{Effect:} & #1\\}
\newcommand{\spellarea}[1]{\textbf{Area:} & #1\\}
\newcommand{\spelltarget}[1]{\textbf{Target:} & #1\\}
\newcommand{\spelltargets}[1]{\textbf{Targets:} & #1\\}
\newcommand{\spellduration}[1]{\textbf{Duration:} & #1\\}
\newcommand{\spellsave}[1]{\textbf{Saving Throw:} & #1\\}
\newcommand{\spellsr}[1]{\textbf{Spell Resistance:} & #1\\}

\newcommand{\component}[1]{\textit{Material Component:} #1}
\newcommand{\focus}[1]{\textit{Focus:} #1}

%%%%%%%%%%%%%%%%%%%%%%%%

\chapter{Spells}
\section{Spells, A through Z}
\spellentry{Acid Arrow}

Conjuration (Creation) [Acid]

\textbf{Level:} Sor/Wiz 2

\textbf{Components:} V, S, M, F

\textbf{Casting Time:} 1 standard action

\textbf{Range:} Long (400 ft. + 40 ft./level)

\textbf{Effect:} One arrow of acid

\textbf{Duration:} 1 round + 1 round per three levels

\textbf{Saving Throw:} None

\textbf{Spell Resistance:} No

A magical arrow of acid springs from your hand and speeds to its target. You must 
succeed on a ranged touch attack to hit your target. The arrow deals 2d4 points 
of acid damage with no splash damage. For every three caster levels (to a maximum 
of 18th), the acid, unless somehow neutralized, lasts for another round, dealing 
another 2d4 points of damage in that round.

\textit{Material Component:} Powdered rhubarb leaf and an adder's stomach.

\textit{Focus:} A dart.


\spellentry{Acid Fog}

Conjuration (Creation) [Acid]

\textbf{Level:} Sor/Wiz 6, Water 7

\textbf{Components:} V, S, M/DF

\textbf{Casting Time:} 1 standard action

\textbf{Range:} Medium (100 ft. + 10 ft./level)

\textbf{Effect:} Fog spreads in 20-ft. radius, 20 ft. high

\textbf{Duration:} 1 round/level

\textbf{Saving Throw:} None

\textbf{Spell Resistance:} No

Acid Fog creates a billowing mass of misty vapors similar to that produced 
by a \linkspell{Solid Fog} spell. In addition to slowing creatures down and obscuring 
sight, this spell's vapors are highly acidic. Each round on your turn, starting 
when you cast the spell, the fog deals 2d6 points of acid damage to each creature 
and object within it.

\textit{Arcane Material Component:} A pinch of dried, powdered peas combined with 
powdered animal hoof.




\chapter{Prestige Classes}
\section{Prestige Class Basics}
\section{WhatClasses}

\chapter{Monsters}
\section{Reading a Monster Entry}
\section{Monsters, A though Z}

\chapter{NPC Classes}
\section{Adept}
foo
\section{Aristocrat}
foo
\section{Commoner}
foo
\section{Expert}
foo
\section{Warrior}
foo

\input{tome-srd-ogl}

\clearpage
\phantomsection
\listoftables

\clearpage
\phantomsection
\printindex

\end{document}