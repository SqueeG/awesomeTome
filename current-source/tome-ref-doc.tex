\documentclass[10pt,oneside,onecolumn,openany,final]{memoir}
\setstocksize{11in}{8.5in}

\usepackage[toc,lot,lof]{multitoc}
\usepackage[top=.5in, bottom=.5in, left=.75in, right=.75in]{geometry}
\usepackage{graphicx} \graphicspath{{./images/}}
\usepackage{longtable}
\usepackage{mdwlist}
\usepackage{microtype} \DisableLigatures{encoding = *, family = *}
\usepackage{multicol}
\usepackage{textcomp}
\usepackage[normalem]{ulem}
\usepackage{wrapfig}
\usepackage{xtab}
\usepackage{enumerate}
\usepackage{phonetic}
\usepackage{bbding}
\usepackage{linearb}
\usepackage{cypriot}
\usepackage{tipa}
\usepackage{xfrac}
\usepackage{appendix}
\usepackage{xparse}
\usepackage{letltxmacro}
\usepackage{makeidx} \makeindex
\usepackage[table,dvipsnames]{xcolor}
\definecolor{offyellow}{RGB}{255,255,128}
\definecolor{links}{RGB}{200,0,50}
\usepackage{placeins}
\usepackage{floatflt}
\usepackage{anyfontsize}
\usepackage{colortbl}
\usepackage{tabularx}
\usepackage{mdframed}
\usepackage{longtable}
\usepackage{tabu}
\usepackage{afterpage}

\usepackage{caption}

%% Font
\usepackage[T1]{fontenc}
\usepackage[bitstream-charter]{mathdesign}
\usepackage{aurical}

\usepackage[colorlinks=true,linkcolor=blue,urlcolor=links,pdfstartview={XYZ null null 1.00},bookmarksdepth=2]{hyperref}
%%%%%%%%%%%%%%%%%%%%%%%%%
%%%% End of Import tion %%%%%%%%%%%
%%%%%%%%%%%%%%%%%%%%%%%%%


%%%%%%%%%%%%%%%%%%%%%%%%%%%%%%%%%%%%%%%%%%%%%%%%%%
%%%%%%%%%%%%%%%%%%%%%%%%%%%%%%%%%%%%%%%%%%%%%%%%%%
%%% Revised Commands
%%%%%%%%%%%%%%%%%%%%%%%%%%%%%%%%%%%%%%%%%%%%%%%%%%
%%%%%%%%%%%%%%%%%%%%%%%%%%%%%%%%%%%%%%%%%%%%%%%%%%
\makeatletter

%fiddles with how chapter titles are displayed
\renewcommand{\@makechapterhead}[1]{%
\vspace*{0 pt}{%
\raggedright \normalfont \fontsize{32}{32} \selectfont \bfseries%
\ifnum \value{secnumdepth}>-1%
  \if@mainmatter \vspace{-8pt} {\fontsize{20}{20} \selectfont Chapter \thechapter:}\\[8pt]%
  \fi%
\fi
\hspace{0.65cm} #1\par\nobreak\vspace{20 pt}%
}}

%makes paragraphs show up closer together
\renewcommand{\paragraph}{%
\@startsection{paragraph}{4}%
{\z@}{1.0ex \@plus 1ex \@minus 0.2ex}{-1em} % wtf is an 'ex' anyways?
{\normalfont\normalsize\bfseries}%
}

%lets multicolumn have the proper background colors as defined by rowcolors
\let\oldmc\multicolumn
\newcommand{\mcinherit}{% Activate \multicolumn inheritance
  \renewcommand{\multicolumn}[3]{%
    \oldmc{##1}{##2}{\ifodd\rownum \@oddrowcolor\else\@evenrowcolor\fi ##3}%
  }}

\makeatother

%add labels within sections, subsections, and subsubsections
\LetLtxMacro{\oldsection}{\section}
\renewcommand{\section}[1]{\oldsection{#1}\label{sec:#1}}

\LetLtxMacro{\oldsubsection}{\subsection}
\renewcommand{\subsection}[1]{\oldsubsection{#1}\label{sec:#1}}

\LetLtxMacro{\oldsubsubsection}{\subsubsection}
\renewcommand{\subsubsection}[1]{\oldsubsubsection{#1}\label{sec:#1}}

%only put chapters and sections into the TOC
\setcounter{secnumdepth}{1}
%makes a subsubsection start off indented.
\setlength{\beforesubsubsecskip}{-\beforesubsubsecskip}



%%%%%%%%%%%%%%%%%%%%%%%%%%%%%%%%%%%%%%%%%%%%%%%%%%
%%%%%%%%%%%%%%%%%%%%%%%%%%%%%%%%%%%%%%%%%%%%%%%%%%
%%% Table Formatting
%%%%%%%%%%%%%%%%%%%%%%%%%%%%%%%%%%%%%%%%%%%%%%%%%%
%%%%%%%%%%%%%%%%%%%%%%%%%%%%%%%%%%%%%%%%%%%%%%%%%%
\newcolumntype{L}[1]{>{\raggedright\let\newline\\\arraybackslash\hspace{0pt}}m{#1}} %New type of column 'L' that is ragged-right, behaves like a paragraph, and allows manual definition of width like a 'p' column.
\newcolumntype{C}[1]{>{\centering\let\newline\\\arraybackslash\hspace{0pt}}m{#1}}  %New type of column 'C' that is centered, behaves like a paragraph, and allows manual definition of width like a 'p' column.
\newcolumntype{R}[1]{>{\raggedleft\let\newline\\\arraybackslash\hspace{0pt}}m{#1}}  %New type of column 'R' that is ragged-left, behaves like a paragraph, and allows manual definition of width like a 'p' column.
\newcommand{\header}{\rowcolor{headercolor}}
%when inserted in a row, makes that row the color headercolor



%%%%%%%%%%%%%%%%%%%%%%%%%%%%%%%%%%%%%%%%%%%%%%%%%%
%%%%%%%%%%%%%%%%%%%%%%%%%%%%%%%%%%%%%%%%%%%%%%%%%%
%%% List Formatting
%%%%%%%%%%%%%%%%%%%%%%%%%%%%%%%%%%%%%%%%%%%%%%%%%%
%%%%%%%%%%%%%%%%%%%%%%%%%%%%%%%%%%%%%%%%%%%%%%%%%%

\let\olditemize\itemize
\renewcommand{\itemize}{
  \olditemize
  \setlength{\itemsep}{1pt}
  \setlength{\parskip}{0pt}
  \setlength{\parsep}{0pt}
}
%fixes itemize spacing

\let\olddescription\description
\renewcommand{\description}{
  \olddescription
  \setlength{\itemsep}{1pt}
  \setlength{\parskip}{0pt}
  \setlength{\parsep}{0pt}
}
%fixes description spacing

\let\oldenumerate\enumerate
\renewcommand{\enumerate}{
  \oldenumerate
  \setlength{\itemsep}{1pt}
  \setlength{\parskip}{0pt}
  \setlength{\parsep}{0pt}
}
%fixes enumerate spacing

\newcommand{\descability}[2]{\item[#1:] #2}


%%%%%%%%%%%%%%%%%%%%%%%%%%%%%%%%%%%%%%%%%%%%%%%%%%
%%%%%%%%%%%%%%%%%%%%%%%%%%%%%%%%%%%%%%%%%%%%%%%%%%
%%% New Commands
%%%%%%%%%%%%%%%%%%%%%%%%%%%%%%%%%%%%%%%%%%%%%%%%%%
%%%%%%%%%%%%%%%%%%%%%%%%%%%%%%%%%%%%%%%%%%%%%%%%%%


%%%%%%%%%%%%%%%%%%%%%%%%
%%Basic Formatting
%%%%%%%%%%%%%%%%%%%%%%%%

\newcommand{\originallineskip}{\baselineskip}
 %A command that is equal to the original \baselineskip of the doc, in case we change it for a section and want to change it back later

\newcommand{\ability}[2]{\textbf{#1:} #2} 
%The \ability{#1}{#2} command from legacy-source. Should rarely be directly used, changes to this will cascade into other new commands that use its functionality

\newcommand{\shortability}[2]{\noindent\textbf{#1} #2\\}
%A specialized version of the \ability command

\newcommand{\itemspace}{\setlength{\itemsep}{-1mm}\setlength{\topsep}{-1mm} }
%A command from legacy-source for compatabilty

\newcommand{\listone}{\begin{list}{$\bullet$}{\itemspace}}

\newcommand{\listtwo}{\begin{list}{$\triangleright$}{\itemspace}}
%A type of list from legacy sorce

\newcommand{\listnum}{\begin{list}{\textbf{\arabic{counter}}:}{\usecounter{counter}}}

\newcommand{\spell}[1]{\emph{\MakeLowercase{#1}}}
%makes spell name lowercase italics.

\setlength{\parindent}{12pt}
%sets the indentation of all paragraphs in the work

\newcommand{\half}[0]{\ensuremath{\sfrac{1}{2}} }

\newcommand{\third}[0]{\ensuremath{\sfrac{1}{3}} }

\newcommand{\fourth}[0]{\ensuremath{\sfrac{1}{4}} }




%%%%%%%%%%%%%%%%%%%%%%%%
%%Logic
%%%%%%%%%%%%%%%%%%%%%%%%
\newcommand{\testempty}{\empty}
\newcommand{\isempty}{\empty}
%Two commands that can be compared to one another for \ifx logic tests. \isempty should never be changed. If \testempty holds a value of anything but empty, the test should return false.

\newcounter{counter}




%%%%%%%%%%%%%%%%%%%%%%%%
%%Colors
%%%%%%%%%%%%%%%%%%%%%%%%
\colorlet{colorone}{white}
\colorlet{colortwo}{gray!15}
\colorlet{headercolor}{gray!50}
\colorlet{tablecolorone}{gray!40}
\colorlet{tablecolortwo}{gray!20}





%%%%%%%%%%%%%%%%%%%%%%%%
%%Sectioning
%%%%%%%%%%%%%%%%%%%%%%%%
\newcommand{\classentry}[1]{\newpage \section{#1} \label{class:#1} \renewcommand{\class}{#1} \index{#1 (class)} \renewcommand{\testempty}{\isempty}}
%Starts a new page, creates a section with the name of the class (#1), sets \class to be the name of the class, indexes the class.

\newcommand{\raceentry}[1]{\subsection{#1} \label{race:#1} \renewcommand{\race}{#1}}
%\newcommand{\raceentry}[1]{\oldsection{#1}\index{#1 (race)}\label{race:#1}}

\newcommand{\Requirements}{\oldsubsubsection*{Requirements}}

\newcommand{\Basics}{\oldsubsubsection*{Basics}}

\newcommand{\ClassFeatures}{\oldsubsubsection*{Class Features}}

\newcommand{\skillentry}[2]{\oldsubsection[#1]{#1 #2}\index{#1 (skill)}\label{skill:#1}}




%%%%%%%%%%%%%%%%%%%%%%%%
%%Race Chapter
%%%%%%%%%%%%%%%%%%%%%%%%
\newcommand{\race}{placeholder}




%%%%%%%%%%%%%%%%%%%%%%%%
%%Class Chapter
%%%%%%%%%%%%%%%%%%%%%%%%

\newcommand{\class}{placeholder}
%Holds the classes name, as defined by \classentry

\newcommand{\quot}[1]{
	\vspace{-8pt}
	\noindent\emph{#1}\medskip}
%Displays a flavor quote.}

\newenvironment{classpreamble}{
\begin{table}[h]
\rowcolors{1}{colorone}{colortwo}
\begin{tabu} to \textwidth {X}
}{
\end{tabu}
\end{table}
}

\newcommand{\desc}[1]{  #1 \\}
\newcommand{\playingaclass}[1]{\ability{Playing a \class : }{#1}\\}
\newcommand{\hitdie}[1]{\ability{Hit Die: }{#1}\\}
\newcommand{\alignment}[1]{\ability{Alignment: }{#1}\\}
\newcommand{\races}[1]{\ability{Races: }{#1}\\}
\newcommand{\startinggold}[1]{\ability{Starting Gold: }{#1}\\}
\newcommand{\startingage}[1]{\ability{Starting Age: }{#1}\\}  
\newcommand{\skillpoints}[1]{\ability{Skill Points per Level: }{#1 + Intelligence Bonus}\\}
\newcommand{\classskills}[1]{\ability{Class Skills: }{The {\class}'s class skills (and the key ability for each skill) are #1}\\}

\newcommand{\startclassfeatures}{
 \smallskip\noindent All of the following are class features of the \class ~class.}
%place before actual class features entries.

\newcommand{\proficiencies}[1]{
 \ability{Weapon and Armor Proficiencies:}{The \class ~is proficient with #1}}
%Displays proficiencies with minimal input, implimentation looks like \proficiencies{the proficiencies}

\newcommand{\classfeature}[2]{
  \ability{#1}{#2}}
%No functional difference from \ability currently

%%%%Class Table Commands

\newcommand{\gbab}{\empty}
\newcommand{\mbab}{\empty}
\newcommand{\fort}{\empty}
\newcommand{\refl}{\empty}
\newcommand{\will}{\empty}
%Creates new commands for use in \ifx statements for formatting purposes.

\newcommand{\goodbab}{\renewcommand{\gbab}{\empty}\renewcommand{\mbab}{a}}
\newcommand{\modebab}{\renewcommand{\gbab}{a}\renewcommand{\mbab}{\empty}}
\newcommand{\poorbab}{\renewcommand{\gbab}{a}\renewcommand{\mbab}{a}}
%A set of commands to tell LaTeX what BAB progression the class has. Only one should be called per class.

\newcommand{\goodfor}{\renewcommand{\fort}{\empty}}
\newcommand{\poorfor}{\renewcommand{\fort}{a}}
%A set of commands to tell LaTeX what Fortitude progression the class has. Only one should be called per class.

\newcommand{\goodref}{\renewcommand{\refl}{\empty}}
\newcommand{\poorref}{\renewcommand{\refl}{a}}
%A set of commands to tell LaTeX what Reflex progression the class has. Only one should be called per class.

\newcommand{\goodwil}{\renewcommand{\will}{\empty}}
\newcommand{\poorwil}{\renewcommand{\will}{a}}
%A set of commands to tell LaTeX what Will progression the class has. Only one should be called per class.

\newenvironment{classtable}[1]
{
\begin{table}[htb]
\centering
\rowcolors{1}{colorone}{colortwo}
\begin{tabu} to \textwidth {p{.275in} l p{0.275in} p{0.275in} p{0.275in} X l l l l}
\rowcolor{headercolor} Level & Base Attack & Fort. & Ref. & Will & Special #1\\
}{
\end{tabu}
\end{table}
}
%A a new environment that sets up the class tables. Include the \level commands between \begin{classtable}.

\newenvironment{minorcastingclasstable}
{
\table[htb]
\center
\rowcolors{1}{colorone}{colortwo}
\tabularx{\textwidth}{p{.275in}lp{0.275in}p{0.275in}p{0.275in}Xccccccc}
\rowcolor{headercolor} & & & & & &\multicolumn{7}{c}{Spells Per Day (By Level)} \\
\rowcolor{headercolor} Level & Base Attack & Fort. & Ref. & Will & Special &0&1&2&3&4&5&6\\
}{
\endtabularx
\endcenter
\endtable
}
%A a new environment similar to classtable, but with columns for a minor (zero through six) spell slot progression.

\newenvironment{fullcastingclasstable}
{
\table[htb]
\center
\rowcolors{1}{colorone}{colortwo}
\tabularx{\textwidth}{p{.275in}lp{0.275in}p{0.275in}p{0.275in}Xcccccccccc}
\rowcolor{headercolor} & & & & & &\multicolumn{10}{c}{Spells Per Day (By Level)} \\
\rowcolor{headercolor} Level & Base Attack & Fort. & Ref. & Will & Special &0&1&2&3&4&5&6&7&8&9\\
}{
\endtabularx
\endcenter
\endtable
}
%Another environment for class tables, this one for full (0 through 9) spell slot progression.

\newcommand{\levelone}[1]{
1st  & \ifx\gbab\isempty +1 \else\ifx\mbab\isempty +0 \else +0 \fi \fi
	 & \ifx\fort\isempty +2 \else +0 \fi
	 & \ifx\refl\isempty +2 \else +0 \fi
	 & \ifx\will\isempty +2 \else +0 \fi
	 & #1 \\}
%A command that declares a table row within the class feature table.

\newcommand{\leveltwo}[1]{
2nd  & \ifx\gbab\isempty +2 \else\ifx\mbab\isempty +1 \else +1 \fi \fi
	 & \ifx\fort\isempty +3 \else +0 \fi
	 & \ifx\refl\isempty +3 \else +0 \fi
	 & \ifx\will\isempty +3 \else +0 \fi
	 & #1 \\}
%A command that declares a table row within the class feature table.

\newcommand{\levelthree}[1]{
3rd  & \ifx\gbab\isempty +3 \else\ifx\mbab\isempty +2 \else +1 \fi \fi
	 & \ifx\fort\isempty +3 \else +1 \fi
	 & \ifx\refl\isempty +3 \else +1 \fi
	 & \ifx\will\isempty +3 \else +1 \fi
	 & #1 \\}
%A command that declares a table row within the class feature table.

\newcommand{\levelfour}[1]{
4th  & \ifx\gbab\isempty +4 \else\ifx\mbab\isempty +3 \else +2 \fi \fi
	 & \ifx\fort\isempty +4 \else +1 \fi
	 & \ifx\refl\isempty +4 \else +1 \fi
	 & \ifx\will\isempty +4 \else +1 \fi
	 & #1 \\}
%A command that declares a table row within the class feature table.

\newcommand{\levelfive}[1]{
5th  & \ifx\gbab\isempty +5 \else\ifx\mbab\isempty +3 \else +2 \fi \fi
	 & \ifx\fort\isempty +4 \else +1 \fi
	 & \ifx\refl\isempty +4 \else +1 \fi
	 & \ifx\will\isempty +4 \else +1 \fi
	 & #1 \\}
%A command that declares a table row within the class feature table.
	 
\newcommand{\levelsix}[1]{
6th  & \ifx\gbab\isempty +6/+1 \else\ifx\mbab\isempty +4 \else +3 \fi \fi
	 & \ifx\fort\isempty +5 \else +2 \fi
	 & \ifx\refl\isempty +5 \else +2 \fi
	 & \ifx\will\isempty +5 \else +2 \fi
	 & #1 \\}
%A command that declares a table row within the class feature table.

\newcommand{\levelseven}[1]{
7th  & \ifx\gbab\isempty +7/+2 \else\ifx\mbab\isempty +5 \else +3 \fi \fi
	 & \ifx\fort\isempty +5 \else +2 \fi
	 & \ifx\refl\isempty +5 \else +2 \fi
	 & \ifx\will\isempty +5 \else +2 \fi
	 & #1 \\}
%A command that declares a table row within the class feature table.
	 
\newcommand{\leveleight}[1]{
8th  & \ifx\gbab\isempty +8/+3 \else\ifx\mbab\isempty +6/+1 \else +4 \fi \fi
	 & \ifx\fort\isempty +6 \else +2 \fi
	 & \ifx\refl\isempty +6 \else +2 \fi
	 & \ifx\will\isempty +6 \else +2 \fi
	 & #1 \\}
%A command that declares a table row within the class feature table.
	 
\newcommand{\levelnine}[1]{
9th  & \ifx\gbab\isempty +9/+4 \else\ifx\mbab\isempty +6/+1 \else +4 \fi \fi
	 & \ifx\fort\isempty +6 \else +3 \fi
	 & \ifx\refl\isempty +6 \else +3 \fi
	 & \ifx\will\isempty +6 \else +3 \fi
	 & #1 \\}
%A command that declares a table row within the class feature table.
	 
\newcommand{\levelten}[1]{
10th & \ifx\gbab\isempty +10/+5 \else\ifx\mbab\isempty +7/+2 \else +5 \fi \fi
	 & \ifx\fort\isempty +7 \else +3 \fi
	 & \ifx\refl\isempty +7 \else +3 \fi
	 & \ifx\will\isempty +7 \else +3 \fi
	 & #1 \\}
%A command that declares a table row within the class feature table.
	 
\newcommand{\leveleleven}[1]{
11th & \ifx\gbab\isempty +11/+6/+6 \else\ifx\mbab\isempty +8/+3 \else +5 \fi \fi
	 & \ifx\fort\isempty +7 \else +3 \fi
	 & \ifx\refl\isempty +7 \else +3 \fi
	 & \ifx\will\isempty +7 \else +3 \fi
	 & #1 \\}
%A command that declares a table row within the class feature table.
	 
\newcommand{\leveltwelve}[1]{
12th & \ifx\gbab\isempty +12/+7/+7 \else\ifx\mbab\isempty +9/+4 \else +6/+1 \fi \fi
	 & \ifx\fort\isempty +8 \else +4 \fi
	 & \ifx\refl\isempty +8 \else +4 \fi
	 & \ifx\will\isempty +8 \else +4 \fi
	 & #1 \\}
%A command that declares a table row within the class feature table.
	 
\newcommand{\levelthirteen}[1]{
13th & \ifx\gbab\isempty +13/+8/+8 \else\ifx\mbab\isempty +9/+4 \else +6/+1 \fi \fi
	 & \ifx\fort\isempty +8 \else +4 \fi
	 & \ifx\refl\isempty +8 \else +4 \fi
	 & \ifx\will\isempty +8 \else +4 \fi
	 & #1 \\}
%A command that declares a table row within the class feature table.
	 
\newcommand{\levelfourteen}[1]{
14th & \ifx\gbab\isempty +14/+9/+9 \else\ifx\mbab\isempty +10/+5 \else +7/+2 \fi \fi
	 & \ifx\fort\isempty +9 \else +4 \fi
	 & \ifx\refl\isempty +9 \else +4 \fi
	 & \ifx\will\isempty +9 \else +4 \fi
	 & #1 \\}
%A command that declares a table row within the class feature table.
	 
\newcommand{\levelfifteen}[1]{
15th & \ifx\gbab\isempty +15/+10/+10 \else\ifx\mbab\isempty +11/+6/+6 \else +7/+2 \fi \fi
	 & \ifx\fort\isempty +9 \else +5 \fi
	 & \ifx\refl\isempty +9 \else +5 \fi
	 & \ifx\will\isempty +9 \else +5 \fi
	 & #1 \\}
%A command that declares a table row within the class feature table.
	 
\newcommand{\levelsixteen}[1]{
16th & \ifx\gbab\isempty +16/+11/+11/+11 \else\ifx\mbab\isempty +12/+7/+7 \else +8/+3 \fi \fi
	 & \ifx\fort\isempty +10 \else +5 \fi
	 & \ifx\refl\isempty +10 \else +5 \fi
	 & \ifx\will\isempty +10 \else +5 \fi
	 & #1 \\}
%A command that declares a table row within the class feature table.
	 
\newcommand{\levelseventeen}[1]{
17th & \ifx\gbab\isempty +17/+12/+12/+12 \else\ifx\mbab\isempty +12/+7/+7 \else +8/+3 \fi \fi
	 & \ifx\fort\isempty +10 \else +5 \fi
	 & \ifx\refl\isempty +10 \else +5 \fi
	 & \ifx\will\isempty +10 \else +5 \fi
	 & #1 \\}
%A command that declares a table row within the class feature table.
	 
\newcommand{\leveleighteen}[1]{
18th & \ifx\gbab\isempty +18/+13/+13/+13 \else\ifx\mbab\isempty +13/+8/+8 \else +9/+4 \fi \fi
	 & \ifx\fort\isempty +11 \else +6 \fi
	 & \ifx\refl\isempty +11 \else +6 \fi
	 & \ifx\will\isempty +11 \else +6 \fi
	 & #1 \\}
%A command that declares a table row within the class feature table.
	 
\newcommand{\levelnineteen}[1]{
19th & \ifx\gbab\isempty +19/+14/+14/+14 \else\ifx\mbab\isempty +14/+9/+9 \else +9/+4 \fi \fi
	 & \ifx\fort\isempty +11 \else +6 \fi
	 & \ifx\refl\isempty +11 \else +6 \fi
	 & \ifx\will\isempty +11 \else +6 \fi
	 & #1 \\}
%A command that declares a table row within the class feature table.
	 
\newcommand{\leveltwenty}[1]{
20th & \ifx\gbab\isempty +20/+15/+15/+15 \else\ifx\mbab\isempty +15/+10/+10 \else +10/+5 \fi \fi
	 & \ifx\fort\isempty +12 \else +6 \fi
	 & \ifx\refl\isempty +12 \else +6 \fi
	 & \ifx\will\isempty +12 \else +6 \fi
	 & #1 \\}
%A command that declares a table row within the class feature table.

\newmdenv[hidealllines=true,backgroundcolor=gray!20]{optionbox}

\newcommand{\option}[1]{
  \renewmdenv[hidealllines=true,backgroundcolor=colorone]{optionbox}
   \begin{optionbox}\noindent{#1}\end{optionbox}
   ~\\*
}

\newenvironment{optional}{
\colorlet{colortwo}{white}
\colorlet{colorone}{gray!15}
}


%%%%%%%%%%%%%%%%%%%%%%%%
%%Unsorted Commands
%%%%%%%%%%%%%%%%%%%%%%%%
\newcommand{\tagline}[1]{\vspace{-6pt} \textit{#1} \medskip}

\newcommand{\gameterm}[1]{#1\index{#1}}

\NewDocumentCommand\featentry{m+g}{%
  \IfNoValueTF{#2}
    {\oldsubsubsection[#1]{#1 [General]}\label{feat:#1}}%no second arg, general feat
    {\oldsubsubsection[#1]{#1 [#2]}\label{feat:#1}}%second arg, special type of feat
}

\newcommand{\spellentry}[1]{\oldsubsubsection{#1}\label{spell:#1}}

\NewDocumentCommand\linkrace{m+g}{%
  \IfNoValueTF{#2}
    {\hyperref[race:#1]{#1}}%no second arg, display is same as link
    {\hyperref[race:#1]{#2}}%second arg, link to first and display second
}
\NewDocumentCommand\linkclass{m+g}{%
  \IfNoValueTF{#2}
    {\hyperref[class:#1]{#1}}%no second arg, display is same as link
    {\hyperref[class:#1]{#2}}%second arg, link to first and display second
}
\NewDocumentCommand\linkskill{m+g}{%
  \IfNoValueTF{#2}
    {\hyperref[skill:#1]{#1}}%no second arg, display is same as link
    {\hyperref[skill:#1]{#2}}%second arg, link to first and display second
}
\NewDocumentCommand\linkfeat{m+g}{%
  \IfNoValueTF{#2}
    {\hyperref[feat:#1]{#1}}%no second arg, display is same as link
    {\hyperref[feat:#1]{#2}}%second arg, link to first and display second
}
\NewDocumentCommand\linkspell{m+g}{%
  \IfNoValueTF{#2}
    {\hyperref[spell:#1]{#1}}%no second arg, display is same as link
    {\hyperref[spell:#1]{#2}}%second arg, link to first and display second
}
\NewDocumentCommand\linkcondition{m+g}{%
  \IfNoValueTF{#2}
    {\hyperref[condition:#1]{#1}}%no second arg, display is same as link
    {\hyperref[condition:#1]{#2}}%second arg, link to first and display second
}
\NewDocumentCommand\linkability{m+g}{%
  \IfNoValueTF{#2}
    {\hyperref[ability:#1]{#1}}%no second arg, display is same as link
    {\hyperref[ability:#1]{#2}}%second arg, link to first and display second
}
\NewDocumentCommand\linksec{m+g}{%
  \IfNoValueTF{#2}
    {\hyperref[sec:#1]{#1}}%no second arg, display is same as link
    {\hyperref[sec:#1]{#2}}%second arg, link to first and display second
}

\begin{document}

%%%%%%%%%%%%%%%%%%%%%%%%%%%%%%%%%%%%%%%%%%%%%%%%%%
%%%%%%%%%%%%%%%%%%%%%%%%%%%%%%%%%%%%%%%%%%%%%%%%%%
%%% Title Page
%%%%%%%%%%%%%%%%%%%%%%%%%%%%%%%%%%%%%%%%%%%%%%%%%%
%%%%%%%%%%%%%%%%%%%%%%%%%%%%%%%%%%%%%%%%%%%%%%%%%%
\thispagestyle{empty}
\begin{center}
\textsc{\Large}\\[0.25cm]
\rule{\linewidth}{0.5mm} \\[0.70cm]
\fontsize{30}{30} \selectfont Tome Reference Document\\[.30cm]
\fontsize{16}{18} \selectfont \guillemotleft{} For that game we all known and love \guillemotright{}\\
\rule{\linewidth}{0.5mm} \\[0.6cm]
%\includegraphics[clip,trim=5cm 2cm 9cm 1cm,width=\linewidth]{OldBookArt--MapImages-173.jpg}
\vfill
{\large \textit{This material is Open Game Content, and is licensed for public use under the terms of the Open Game License v1.0a.}\\
\today}
\end{center}

\pagebreak
\sffamily
\pagestyle{plain}
\raggedbottom

%%%%%%%%%%%%%%%%%%%%%%%%%%%%%%%%%%%%%%%%%%%%%%%%%%
%%%%%%%%%%%%%%%%%%%%%%%%%%%%%%%%%%%%%%%%%%%%%%%%%%
%%% Table of Contents
%%%%%%%%%%%%%%%%%%%%%%%%%%%%%%%%%%%%%%%%%%%%%%%%%%
%%%%%%%%%%%%%%%%%%%%%%%%%%%%%%%%%%%%%%%%%%%%%%%%%%
\renewcommand{\contentsname}{Table of Contents}
\setcounter{tocdepth}{1}
\tableofcontents

%%%%%%%%%%%%%%%%%%%%%%%%%%%%%%%%%%%%%%%%%%%%%%%%%%
%%%%%%%%%%%%%%%%%%%%%%%%%%%%%%%%%%%%%%%%%%%%%%%%%%
%%% Main Content %%%
%%%%%%%%%%%%%%%%%%%%%%%%%%%%%%%%%%%%%%%%%%%%%%%%%%
%%%%%%%%%%%%%%%%%%%%%%%%%%%%%%%%%%%%%%%%%%%%%%%%%%

%% Primary Chapters Here

\clearpage

\chapter{Introduction}
\section{What is a Role-playing Game?}
foo
\section{What You Need To Play}
foo
\section{The Core Mechanic}
foo
\section{Creating a Character}
foo

\chapter{Races}
%need a race basics section here
%\begin{racebox}
\raceentry{Aasimar}{``My ancestors were more beautiful than you can imagine."}

Aasimar are humans that have a beautiful outsider, usually but not always a celestial, somewhere in their ancestry.

\racedescription{Aasimar look like especially beautiful humans, though they sometimes bear vestiges of their ancestry that denote them as being different (strangely colored eyes, silver-blonder or white hair, slightly `off' facial features).}

\racepersonality{Though mostly human, an aasimar's immortal heritage influences their mental development. Aasimar tend toward more extreme personalities, being especially quiet and introspective or particularly loud and boisterous. Most aasimar are very opinionated, and have strongly held beliefs.}

\racesociety{Aasimar are typically born and raised in human societies, and gain the same customs of that culture}

\racealignment{Most aasimar are the descendants of celestials, and tend towards the good alignments. Rarely, an aasimar might instead have an infernal heritage, being the descendant of an erinyes or succubus. Such aasimar instead tend towards an evil alignment.}

\racedatastart
\begin{racetable}
\type{Outsider (Native and Human Subtype)}
\size{Medium}
\scores{+2 Wisdom, +2 Charisma}
\speed{30}
\senses{Standard}
\autolanguages{Common}
\bonuslanguages{Abyssal, Aquan, Auran, Celestial, Formian, Ignan, Slaad, Sylvan, Terran.}
\favoredclasses{Paladin and Sorcerer}
\end{racetable}

\vspace{\baselineskip}
\agetable{20}{+1d6}{+2d6}{+3d6}

\vspace{\baselineskip}
\begin{heightweighttable}
\male{4' 7"}{+2d8}{90 lb.}{x(2d4)}
\female{4' 5"}{+2d8}{80 lb.}{x(2d4)}
\end{heightweighttable}
\racedataend

\racialtraits{
\racetrait{Inner Light \sla}{An Aasimar with a Charisma of at least 10 may cast \spell{light} once per day, with a caster level equal to their character level.}
\racetrait{Keen Senses}{+2 bonus to Spot, and Listen checks.}
}
%\end{racebox}
%\begin{racebox}
\raceentry{Drow}{``Time to die for the Spider Queen."}

No description until we either obtain or write one.

\racedescription{NYW}

\racepersonality{NYW}

\racesociety{NYW}

\racealignment{NYW}

\racedatastart
\begin{racetable}
\type{Humanoid (Elf Subtype)}
\size{Medium}
\scores{+2 Dexterity, -2 Constitution}
\speed{30}
\senses{Darkvision 120'}
\autolanguages{Elvish}
\bonuslanguages{Abyssal, Beholder, Common, Draconic, Drow Sign Language, Dwarvish, Gnome, Kuo-Toa, Terran, Undercommon}
\favoredclasses{Cleric and Wizard}
\end{racetable}

\vspace{\baselineskip}
\agetable{20}{+1d6}{+2d6}{+3d6}

\vspace{\baselineskip}
\begin{heightweighttable}
\male{4' 7"}{+2d8}{90 lb.}{x(2d4)}
\female{4' 5"}{+2d8}{80 lb.}{x(2d4)}
\end{heightweighttable}
\racedataend

\racialtraits{
\racetrait{Daylight Sensitivity}{While in brightly lit surroundings (such as a daylight spell), a Drow suffers a -2 penalty to attack rolls and precision-based skill checks.}
\racetrait{Innate Magic}{Drow with a Charisma of at least 10 may cast deeper darkness (duration 4 hours), and fairie fire as spell-like abilities with a caster level equal to their character level once per day each.}
\racetrait{Magic Resistant}{+2 bonus to saving throws against spells and spell-like abilities.}
\racetrait{Skill Bonus}{+2 bonus to Spot and Listen checks.}
\racetrait{Elven Trance}{Drow never sleep and are immune to sleep effects. Drow must still perform their 4 hour daily trance to stay coherent and rested.}
\racetrait{Interesting Times}{Drow live an exceedingly interesting life and every Drow has proficiency with the rapier and an exotic ranged weapon of their choice.}
}
%\end{racebox}
\raceentry{Dwarf}
\quot{``I remember that...''}

\listone
		\item Medium Size
		\item 20' movement
		\item Humanoid Type (Dwarf Subtype)
		\item +2 Constitution, -2 Charisma
		\item Dwarves can move up to their full speed even when wearing medium or heavy armor or when carrying a medium or heavy load
		\item Darkvision: Dwarves can see up to 60 feet in the dark.
		\item Stonecunning: This ability grants a dwarf a +2 racial bonus on Search checks to notice unusual stonework, such as sliding walls, stonework traps, new construction (even when built to match the old), unsafe stone surfaces, shaky stone ceilings, and the like. Something that isn’t stone but that is disguised as stone also counts as unusual stonework. A dwarf who merely comes within 10 feet of unusual stonework can make a Search check as if he were actively searching, and a dwarf can use the Search skill to find stonework traps as a rogue can. A dwarf can also intuit depth, sensing his approximate depth underground as naturally as a human can sense which way is up.
		\item Weapon Familiarity: Dwarves may treat dwarven waraxes and dwarven urgroshes as martial weapons, rather than exotic weapons.
		\item Stability: A dwarf gains a +4 bonus on ability checks made to resist being bull rushed or tripped when standing on the ground (but not when climbing, flying, riding, or otherwise not standing firmly on the ground).
		\item +2 racial bonus on saving throws against poison.
		\item +2 racial bonus on saving throws against spells and spell-like effects.
		\item +1 racial bonus on attack rolls against orcs and goblinoids.
		\item +4 dodge bonus to Armor Class against monsters of the giant type. Any time a creature loses its Dexterity bonus (if any) to Armor Class, such as when it’s caught flat-footed, it loses its dodge bonus, too.
		\item +2 racial bonus on Appraise checks that are related to stone or metal items.
		\item +2 racial bonus on Craft checks that are related to stone or metal.
		\item Favored class: Fighter
		\item Automatic Languages: Common and Dwarven.
		\item Bonus Languages: Giant, Gnome, Goblin, Orc, Terran, and Undercommon.
\end{list}

\raceentry{Elf}
\quot{``You shall never harm my beautiful trees!''}

\listone
		\item Medium Size
		\item Humanoid Type (Elf Subtype)
		\item 30' movement
		\item +2 Dexterity, -2 Constitution
		\item Low Light Vision: Elves can see twice as far as a human in poor lighting.
		\item Weapon Proficiency: Elves are proficient with the longsword, rapier, longbow (including composite longbow), and shortbow (including composite shortbow).
		\item +2 racial bonus on Listen, Search, and Spot checks. An elf who merely passes within 5 feet of a secret or concealed door is entitled to a Search check to notice it as if she were actively looking for it.
		\item Favored Class: Wizard.
		\item Automatic Languages: Common and Elven. 
		\item Bonus Languages: Draconic, Gnoll, Gnome, Goblin, Orc, and Sylvan.
\end{list}
\raceentry{Feytouched}
\quot{``All my life, I have never fit in. Not in town, not in the forest. In some integral fashion I am unlike those around me, and I believe it is my fate to live and die alone."}

\listone
		\item Medium Size
    \item Fey Type
    \item 30' movement
    \item Low-Light Vision: Feytouched can twice as far as a human in poor lighting.
    \item +2 Dexterity, +2 Charisma, -2 Constitution. Feytouched are graceful and those which are not beautiful are terrifying, but they are fragile like flowers.
    \item Immunity to [Compulsion] Effects
    \item Magic Affinity: Every Feytouched is different, and marked by the signature magics of the fey in a different manner. Every Feytouched has one spell that can be used once per day as a spell-like ability. This spell is chosen at 1st level and cannot be changed. Any 1st level Illusion or Enchantment spell from the Sorcerer/Wizard list is fair game, and the save DC is Charisma-based.
    \item Favored Class: Bard
    \item Feytouched speak Common and Sylvan. Bonus Languages may be selected from the following list:
      Aquan, Auran, Elvish, Draconic, Dwarvish, Druidic, Goblin, Gnoll, Gnome, Halfling.
\end{list}
\begin{racebox}
\raceentry{Gnome}{``What's that you say little mole? Kobolds in the well!?"}
\begin{multicols}{2}

\begin{racetable}
\type{Humanoid (Gnome subtype)}
\size{Small}
\scores{+2 Constitution, --2 Strength}
\speed{20}
\senses{Low Light Vision}
\autolanguages{Common and Gnome}
\bonuslanguages{Draconic, Dwarven, Elven, Giant, Goblin, and Orc. In addition, a gnome can speak with a burrowing mammal (a badger, fox, rabbit, or the like).}
\favoredclasses{Bard}
\end{racetable}

Nothing until we find or write it.

\racedescription{NYW}

\racepersonality{NYW}

\racesociety{NYW}

\racealignment{NYW}

\racialtraits{
\racetrait{Weapon Familiarity}{Gnomes may treat gnome hooked hammers as martial weapons rather than exotic weapons.}
\racetrait{+2 racial bonus on saving throws against illusions.}
\racetrait{Add +1 to the Difficulty Class for all saving throws against illusion spells cast by gnomes. This adjustment stacks with those from similar effects.}
\racetrait{+1 racial bonus on attack rolls against kobolds and goblinoids.}
\racetrait{+4 dodge bonus to Armor Class against monsters of the giant type.}
\racetrait{+2 racial bonus on Listen checks.}
\racetrait{+2 racial bonus on Craft (alchemy) checks.}
\racetrait{Spell-Like Abilities: 1/day—speak with animals (burrowing mammal only, duration 1 minute). A gnome with a Charisma score of at least 10 also has the following spell-like abilities: 1/day—dancing lights, ghost sound, prestidigitation. Caster level 1st; save DC 10 + gnome’s Cha modifier + spell level.} % I'm pretty sure there's a much better way to format this, say by using the \sla used for aasimar. That Cha req goofs it up.
}


%\columnbreak

\vspace{\baselineskip}
\agetable{40}{+4d6}{+6d6}{+9d6}

\vspace{\baselineskip}
\begin{heightweighttable}
\male{3' 0"}{+2d4}{40 lb.}{x(1)}
\female{2' 10"}{+2d4}{35 lb.}{x(1)}
\end{heightweighttable}
\end{multicols}
\end{racebox}
\raceentry{Goblin}
\quot{``You weren't hired to think. You were hired because you have opposable thumbs."}

\listone
    \item Small Size
    \item 30' movement (despite small size).
    \item Humanoid Type (Goblinoid subtype)
    \item Darkvision: Goblins can see up to 60 feet in the dark.
    \item +2 Dexterity, -2 Strength, -2 Charisma
    \item +4 bonus to Move Silently and Ride checks.
    \item Bonus Feat: Mounted Combat
    \item Goblins benefit from an ancient pact with the Worgs, and every Goblin receives a +2 bonus to any Bluff, Diplomacy, Handle Animal, Sense Motive, or Survival check made with respect to a Worg.
    \item Favored Classes: Rogue and Wizard
    \item Automatic Languages: Common, Goblin
    \item Bonus Languages: Draconic, Elvish, Dwarvish, Giant, Gnoll, Infernal, Orcish, Undercommon, and Worg.
\end{list}
\raceentry{Half-Elf}{``I don't fit in anywhere, please, listen to me cry.''}

\listone
		\item Medium Size
		\item 30' Movement
		\item Humanoid Type
		\item Low-Light Vision: Half-Elves can see twice as humans in poor lighting.
		\item Immunity to sleep spells and similar magical effects, and a +2 racial bonus on saving throws against enchantment spells or effects.
		\item +1 racial bonus on Listen, Search, and Spot checks.
		\item +2 racial bonus on Diplomacy and Gather Information checks.
		\item Elven Blood: For all effects related to race, a half-elf is considered an elf.
		\item Favored Class: Any
		\item Automatic Languages: Common and Elven.
		\item Bonus Languages: Any (other than secret languages, such as Druidic).
\end{list}
\raceentry{Half-Orc}
\quot{``I don't fit in anywhere, but you may be surprised to know that this dagger fits all kinds of places."}

\listone
    \item Medium Size
    \item 30' movement
    \item Humanoid Type (Orc and Human subtype)
    \item Darkvision: Half-Orcs can see up to 60 feet in the dark.
    \item +2 Strength
    \item +2 to Intimidate, Gather Information, and Survival checks.
    \item Favored Classes: Assassin and Barbarian
    \item Automatic Languages: Orc, Common
    \item Bonus Languages: Any.
\end{list}
\raceentry{Halfling}{``Where are we going Mr. Frodo?''}

\listone
		\item Small Size
		\item 20' movement
		\item +2 Dexterity, -2 Strength
		\item +2 racial bonus on Climb, Jump, Listen, and Move Silently checks.
		\item +1 racial bonus on all saving throws.
		\item +2 morale bonus on saving throws against fear: This bonus stacks with the halfling’s +1 bonus on saving throws in general.
		\item +1 racial bonus on attack rolls with thrown weapons and slings.
		\item Favored Class: Rogue
		\item Automatic Languages: Common and Halfling.
		\item Bonus Languages: Dwarven, Elven, Gnome, Goblin, and Orc.
\end{list}		

\raceentry{Hobgoblin}
\quot{``That's some tough talk from a man who wears a basket on his head."}

\listone
    \item Medium Size
    \item 30' movement.
    \item Humanoid Type (Goblinoid subtype)
    \item Darkvision: Hobgoblins can see up to 60 feet in the dark.
    \item +2 Dexterity, +2 Constitution
    \item +4 bonus to Move Silently checks.
    \item Favored Classes: Fighter and Samurai
    \item Automatic Languages: Common, Goblin
    \item Bonus Languages: Draconic, Elvish, Dwarvish, Giant, Gnoll, Ignan, Infernal, Orcish.
\end{list}
\raceentry{Human}
\quot{``Yeah, I'm pretty normal.''}

\listone
	\item Medium Size
	\item 30' movement.
	\item Humanoid Type (Human subtype)
	\item 1 extra feat at 1st level.
	\item 4 extra skill points at 1st level and 1 extra skill point at each additional level.
	\item Favored Class: Any. When determining whether a multiclass human takes an experience point penalty, his or her highest-level class does not count.
	\item Automatic Language: Common. 
	\item Bonus Languages: Any (other than secret languages, such as Druidic). See the Speak Language skill.
\end{list}
\raceentry{Kobold}
\quot{``Aieeeeeeeee!''}

\listone
		\item Small Size
		\item 30' movement (despite small size)
		\item Humanoid Type (Reptilian subtype)
		\item Darkvision: Kobolds can see up to 60 feet in the dark.
		\item -4 Strength, +2 Dexterity, -2 Constitution
		\item Racial Skills: A kobold character has a +2 racial bonus on Craft (trapmaking), Profession (miner), and Search checks.
		\item +1 natural armor bonus.
		\item Light sensitivity: Kobolds are dazzled in bright sunlight or within the radius of a daylight spell. 
		\item Favored Class: Sorcerer.
		\item Automatic Languages: Draconic.
		\item Bonus Languages: Common, Undercommon.
\end{list}
\raceentry{Orc}
\quot{``Waaarrrggghhhh!"}

\listone
    \item Medium Size
    \item 30' movement.
    \item Humanoid Type (Orc subtype)
    \item Darkvision: Orcs can see up to 60 feet in the dark.
    \item +4 Strength, -2 Intelligence, -2 Charisma, -2 Wisdom
    \item Daylight Sensitivity: While in brightly lit surroundings (such as a daylight spell), an Orc suffers the dazzled condition and is thus at a -1 penalty to attack rolls and precision-based skill checks.
    \item +2 bonus to saving throws vs. Poison and Disease.
    \item Immunity to ingested poisons.
    \item +2 to Jump and Survival checks.
    \item Favored Classes: Barbarian and Cleric
    \item Automatic Languages: Orc, Common
    \item Bonus Languages: Dwarvish, Elvish, Giant, Gnoll, Goblin, Sylvan, Undercommon.
\end{list}
\raceentry{Tiefling}
\quot{``My ancestors were more evil than you will ever know, but let's see how I compare.''}

\listone
    \item Medium Size
    \item 30' movement.
    \item Outsider Type (Native and Human subtype)
    \item Darkvision: Tieflings can see up to 60 feet in the dark.
    \item +2 Dexterity, +2 Intelligence, -2 Charisma
    \item Tieflings with a Charisma of at least 10 may cast darkness as a spell-like ability with a caster level equal to their character level once per day.
    \item +2 bonus to Bluff, Hide, and Move Silently checks.
    \item Favored Classes: Rogue and True Fiend
    \item Automatic Languages: Common
    \item Bonus Languages: Abyssal, Aquan, Auran, Celestial, Formian, Ignan, Slaad, Sylvan, Terran.
\end{list}

\chapter{Classes}
\section{Class Basics}
foo
%%%%%%%%%%%%%%%%%%%%%%%%%%%%%%%%%%%%%%%%%%%%%%%%%%
\classentry{Assassin}
%%%%%%%%%%%%%%%%%%%%%%%%%%%%%%%%%%%%%%%%%%%%%%%%%%

%%%%%%%%%%%%%%%%%%%%%%%%%
\Requirements
%%%%%%%%%%%%%%%%%%%%%%%%%

To qualify to become an assassin, a character must fulfill all the following criteria.

\textbf{Alignment:} Any evil.

\textbf{Skills:} \linkskill{Disguise} 4 ranks, \linkskill{Hide} 8 ranks, \linkskill{Move Silently} 8 ranks.

\textbf{Special:} The character must kill someone for no other reason than to join 
the assassins.

%%%%%%%%%%%%%%%%%%%%%%%%%
\Basics
%%%%%%%%%%%%%%%%%%%%%%%%%

\textbf{Hit Die:} d6.

\textbf{Class Skills}

The assassin's class skills (and the key ability for each skill) are \linkskill{Balance} (Dex), 
\linkskill{Bluff} (Cha), \linkskill{Climb} (Str), \linkskill{Craft} (Int), \linkskill{Decipher Script} (Int), \linkskill{Diplomacy} (Cha), 
\linkskill{Disable Device} (Int), \linkskill{Disguise} (Cha), \linkskill{Escape Artist} (Dex), \linkskill{Forgery} (Int),
\linkskill{Gather Information} (Cha), \linkskill{Hide} (Dex), \linkskill{Intimidate} (Cha), \linkskill{Jump} (Str), \linkskill{Listen} (Wis), \linkskill{Move Silently} (Dex), \linkskill{Open Lock} (Dex), \linkskill{Search} (Int), \linkskill{Sense Motive} (Wis), \linkskill{Sleight of Hand} 
(Dex), \linkskill{Spot} (Wis), \linkskill{Swim} (Str), \linkskill{Tumble} (Dex), \linkskill{Use Magic Device} (Cha), and \linkskill{Use Rope} 
(Dex). 

\textbf{Skill Points at Each Level:} 4 + Int modifier.

\begin{table}[htb]
\rowcolors{1}{white}{offyellow}
\caption{The Assassin}
\centering
\begin{tabular}{*{6}{l}*{4}{c}}
\textbf{Level} & \textbf{BAB} & \textbf{Fort} & \textbf{Reflex} & \textbf{Will} & \textbf{Special} & \textbf{1st} & \textbf{2nd} & \textbf{3rd} & \textbf{4th} \\
1st & +0 & +0 & +2 & +0 & Sneak Attack +1d6, Death Attack, Poison Use, Spells & 0 & - & - & - \\
2nd & +1 & +0 & +3 & +0 & +1 save against poison, Uncanny Dodge & 1 & - & - & - \\
3rd & +2 & +1 & +3 & +1 & Sneak Attack +2d6 & 2 & 0 & - & - \\
4th & +3 & +1 & +4 & +1 & +2 save against poison & 3 & 1 & - & - \\
5th & +3 & +1 & +4 & +1 & Sneak Attack +3d6, Improved Uncanny Dodge & 3 & 2 & 0 & - \\
6th & +4 & +2 & +5 & +2 & +3 save against poison & 3 & 3 & 1 & - \\
7th & +5 & +2 & +5 & +2 & Sneak Attack +4d6 & 3 & 3 & 2 & 0 \\
8th & +6 & +2 & +6 & +2 & +4 save against poison, Hide In Plain Sight & 3 & 3 & 3 & 1 \\
9th & +6 & +3 & +6 & +3 & Sneak Attack +5d6 & 3 & 3 & 3 & 2 \\
10th & +7 & +3 & +7 & +3 & +5 save against poison & 3 & 3 & 3 & 3 \\
\end{tabular}
\end{table}

%%%%%%%%%%%%%%%%%%%%%%%%%
\ClassFeatures
%%%%%%%%%%%%%%%%%%%%%%%%%

All of the following are Class Features of the assassin prestige class.

\textbf{Weapon and Armor Proficiency:} Assassins are proficient with the crossbow 
(hand, light, or heavy), dagger (any type), dart, rapier, sap, shortbow (normal 
and composite), and short sword. Assassins are proficient with light armor but 
not with shields.

\textbf{Sneak Attack:} This is exactly like the rogue ability of the same name. 
The extra damage dealt increases by +1d6 every other level (2nd, 4th, 6th, 8th, 
and 10th). If an assassin gets a sneak attack bonus from another source the bonuses 
on damage stack.

\textbf{Death Attack:} If an assassin studies his victim for 3 rounds and then 
makes a sneak attack with a melee weapon that successfully deals damage, the sneak 
attack has the additional effect of possibly either paralyzing or killing the target 
(assassin's choice). While studying the victim, the assassin can undertake other 
actions so long as his attention stays focused on the target and the target does 
not detect the assassin or recognize the assassin as an enemy. If the victim of 
such an attack fails a Fortitude save (DC 10 + the assassin's class level + the 
assassin's Int modifier) against the kill effect, she dies. If the saving throw 
fails against the paralysis effect, the victim is rendered helpless and unable 
to act for 1d6 rounds plus 1 round per level of the assassin. If the victim's saving 
throw succeeds, the attack is just a normal sneak attack. Once the assassin has 
completed the 3 rounds of study, he must make the death attack within the next 
3 rounds.

If a death attack is attempted and fails (the victim makes her save) or if the 
assassin does not launch the attack within 3 rounds of completing the study, 3 
new rounds of study are required before he can attempt another death attack.

\textbf{Poison Use:} Assassins are trained in the use of poison and never risk 
accidentally poisoning themselves when applying poison to a blade.

\textbf{Spells:} Beginning at 1st level, an assassin gains the ability to cast 
a number of arcane spells. To cast a spell, an assassin must have an Intelligence 
score of at least 10 + the spell's level, so an assassin with an Intelligence of 
10 or lower cannot cast these spells. Assassin bonus spells are based on Intelligence, 
and saving throws against these spells have a DC of 10 + spell level + the assassin's 
Intelligence bonus. When the assassin gets 0 spells per day of a given spell level 
he gains only the bonus spells he would be entitled to based on his Intelligence 
score for that spell level.

The assassin's spell list appears below. An assassin casts spells just as a bard 
does.

Upon reaching 6th level, at every even-numbered level after that (8th and 10th), 
an assassin can choose to learn a new spell in place of one he already knows. The 
new spell's level must be the same as that of the spell being exchanged, and it 
must be at least two levels lower than the highest-level assassin spell the assassin 
can cast. An assassin may swap only a single spell at any given level, and must 
choose whether or not to swap the spell at the same time that he gains new spells 
known for that level.

\begin{table}[htb]
\rowcolors{1}{white}{offyellow}\mcinherit
\caption{Assassin Spells Known}
\centering
\begin{tabular}{l*{4}{c}}
\textbf{Level} & \textbf{1st} & \textbf{2nd} & \textbf{3rd} & \textbf{4th}\\
1st & 2\textsuperscript{1} & - & - & -\\
2nd & 3 & - & - & -\\
3rd & 3 & 2\textsuperscript{1} & - & -\\
4th & 4 & 3 & - & -\\
5th & 4 & 3 & 2\textsuperscript{1} & -\\
6th & 4 & 4 & 3 & -\\
7th & 4 & 4 & 3 & 2\textsuperscript{1}\\
8th & 4 & 4 & 4 & 3\\
9th & 4 & 4 & 4 & 3\\
10th & 4 & 4 & 4 & 4\\
\multicolumn{5}{p{7cm}}{\textsuperscript{1}Provided the assassin has sufficient Intelligence to have a bonus spell of this level.}\\
\end{tabular}
\end{table}

\textbf{Save Bonus against Poison:} The assassin gains a natural saving throw bonus 
to all poisons gained at 2nd level that increases by +1 for every two additional 
levels the assassin gains.

\textbf{Uncanny Dodge (Ex):} Starting at 2nd level, an assassin retains his Dexterity 
bonus to AC (if any) regardless of being caught flat-footed or struck by an invisible 
attacker. (He still loses any Dexterity bonus to AC if immobilized.)

If a character gains uncanny dodge from a second class the character automatically 
gains improved uncanny dodge (see below).

\textbf{Improved Uncanny Dodge (Ex):} At 5th level, an assassin can no longer be 
flanked, since he can react to opponents on opposite sides of him as easily as 
he can react to a single attacker. This defense denies rogues the ability to use 
flank attacks to sneak attack the assassin. The exception to this defense is that 
a rogue at least four levels higher than the assassin can flank him (and thus sneak 
attack him).

If a character gains uncanny dodge (see above) from a second class the character 
automatically gains improved uncanny dodge, and the levels from those classes stack 
to determine the minimum rogue level required to flank the character.

\textbf{Hide in Plain Sight (Su):} At 8th level, an assassin can use the Hide skill 
even while being observed. As long as he is within 10 feet of some sort of shadow, 
an assassin can hide himself from view in the open without having anything to actually 
hide behind.

He cannot, however, hide in his own shadow.

%%%
\subsubsection{Assassin Spell List}
%%%

Assassins choose their spells from the following list:

\textbf{1st Level:} \linkspell{Disguise Self}, \linkspell{Detect Poison}, \linkspell{Feather Fall}, \linkspell{Ghost Sound}, \linkspell{Jump}, \linkspell{Obscuring Mist}, \linkspell{Sleep}, \linkspell{True Strike}.

\textbf{2nd Level:} \linkspell{Alter Self}, \linkspell{Cat's Grace}, \linkspell{Darkness}, \linkspell{Fox's Cunning}, \linkspell{Illusory Script}, \linkspell{Invisibility}, \linkspell{Pass Without Trace}, \linkspell{Spider Climb}, \linkspell{Undetectable Alignment}.

\textbf{3rd Level:} \linkspell{Deep Slumber}, \linkspell{Deeper Darkness}, \linkspell{False Life}, \linkspell{Magic Circle Against Good}, \linkspell{Misdirection}, \linkspell{Nondetection}.

\textbf{4th Level:} \linkspell{Clairaudience/Clairvoyance}, \linkspell{Dimension Door}, \linkspell{Freedom of Movement}, \linkspell{Glibness}, \linkspell{Greater Invisibility}, \linkspell{Locate Creature}, \linkspell{Modify Memory}, \linkspell{Poison}.


\classname{Barbarian} \label{class:barbarian}
\vspace{-8pt}
\quot{"My name is Sharptooth of the Wolf Tribe. Your women, lands, and riches are mine."}

\ability{Playing a Barbarian:}{Playing a Barbarian is actually very easy. In general, you hit things, and they fall down. A Barbarian's action in almost any circumstance can plausibly be "I hit it with my great axe!" As such, a Barbarian character can be a good method to introduce a new player to the game or kill some orcs when you've had a few glasses of brew.}

\ability{Alignment:}{Every alignment has its share of Barbarians, however more Barbarians are of Chaotic alignment than of Lawful Alignment.}

\ability{Races:}{Anybody can become a barbarian, and in areas with little in the way of civilization, a lot of people do.}

\ability{Starting Gold:}{4d6x10 gp (140 gold)}

\ability{Starting Age:}{As Barbarian.}

\ability{Hit Die:}{d12}

\ability{Class Skills:}{The Barbarian's class skills (and the key ability for each skill) are Balance (Dex), Climb (Str), Hide (Dex), Intimidate (Cha), Jump (Str), Knowledge: Nature (Int), Listen (Wis), Move Silently (Dex), Sense Motive (Wis), Spot (Wis), Survival (Wis), and Swim (Str).}

\ability{Skills/Level:}{4 + Intelligence Bonus}

\begin{table}[htb]
\begin{small}
\begin{tabular}{lp{3cm}p{0.7cm}p{0.7cm}p{0.7cm}l}
Level  &Base Attack Bonus &Fort Save &Ref Save &Will Save &Special\\
1st &+1 &+2 &+0 &+0 &Rage, Fast Healing 1\\
2nd &+2 &+3 &+0 &+0 &Rage Dice +1d6, Combat Movement +5'\\
3rd &+3 &+3 &+1 &+1 &Battle Hardened\\
4th &+4 &+4 &+1 &+1 &Rage Dice +2d6, Combat Movement +10'\\
5th &+5 &+4 &+1 &+1 &Sidestep Hazards , Fast Healing 5\\
6th &+6/+1 &+5 &+2 &+2 &Rage Dice +3d6, Combat Movement +15'\\
7th &+7/+2 &+5 &+2 &+2 &Great Blows\\
8th &+8/+3 &+6 &+2 &+2 &Rage Dice +4d6, Combat Movement +20'\\
9th &+9/+4 &+6 &+3 &+3 &Great Life\\
10th &+10/+5 &+7 &+3 &+3 &Rage Dice +5d6, Combat Movement +25', Fast Healing 10\\
11th &+11/+6/+6 &+7 &+3 &+3 &Call the Horde\\
12th &+12/+7/+7 &+8 &+4 &+4 &Rage Dice +6d6, Combat Movement +30'\\
13th &+13/+8/+8 &+8 &+4 &+4 &Watched by Totems\\
14th &+14/+9/+9 &+9 &+4 &+4 &Rage Dice +7d6, Combat Movement +35'\\
15th &+15/+10/+10 &+9 &+5 &+5 &Primal Assault, Fast Healing 15\\
16th &+16/+11/+11/+11 &+10 &+5 &+5 &Rage Dice +8d6, Combat Movement +40'\\
17th &+17/+12/+12/+12 &+10 &+5 &+5 &Savagery\\
18th &+18/+13/+13/+13 &+11 &+6 &+6 &Rage Dice +9d6, Combat Movement +45'\\
19th &+19/+14/+14/+14 &+11 &+6 &+6 &One With The Beast\\
20th &+20/+15/+15/+15 &+12 &+6 &+6 &Rage Dice +10d6, Combat Movement +50', Fast Healing 20\\
\end{tabular}
\end{small}
\end{table}

\smallskip\noindent All of the following are Class Features of the Barbarian class.


\ability{Weapon and Armor Proficiency:}{Barbarians are proficient with simple weapons, martial weapons, light armor, medium armor and with shields.}

\ability{Rage (Ex):}{When doing melee damage to a foe or being struck by a foe, a Barbarian may choose to enter a Rage as an immediate action. While Raging, a Barbarian gains a +2 morale bonus to hit and damage in melee combat and may apply any Rage Dice he has to his melee damage rolls. He also gains a +2 to saves, a -2 to AC, and he gains DR X/-- with "X" being equal to half his Barbarian level +2 (rounded down). For example, a 1st level Barbarian has DR 3/-- while Raging and a 10th level Barbarian has DR 7/-- while Raging. While Raging, a Barbarian may not cast spells, activate magic items, use spell-like abilities, or drop his weapons or shield. Rage lasts until he has neither struck an enemy for three consecutive rounds nor suffered damage from an enemy for three consecutive rounds. He may voluntarily end a Rage as a full-round action.}

\ability{Fast Healing:}{Barbarians shrug off wounds that would cripple a lesser man, and have learned to draw upon deep reserves of energy and stamina. At 1st level, they gain Fast Healing 1. At 5th level this becomes Fast Healing 5, Fast Healing 10 at 10th level, Fast Healing 15 at 15th level, and Fast Healing 20 at 20th level. This healing only applies while he is not raging. \smallskip

If a Barbarian ever multiclasses, he permanently loses this ability. A multiclass character does not gain this ability.  A character with 4 or more levels of Barbarian gains this ability even if multiclassed.}

\ability{Rage Dice:}{While Raging, a Barbarian may add these dice of damage to each of his melee attacks. These dice are not multiplied by damage multipliers, and are not applied to any bonus attacks beyond those granted by Base Attack Bonus. These dice are not sneak attack dice, and do not count as sneak attack dice for the prerequisites of prestige classes or feats.}

\ability{Combat Movement:}{While Raging, a Barbarian moves faster in combat, and may add his Combat Movement to his speed when he takes a move action to move.}

\ability{Battle Hardened:}{At 3th level, a Raging Barbarian's mind has been closed off from distractions by the depths of his bloodlust and battle fury. While Raging, he may use his Fortitude Save in place of his Will Save. If he is under the effects of a compulsion or fear effect, he may act normally while Raging as if he was inside a \spell{protection from evil} effect.}

\ability{Sidestep Hazards (Ex):}{At 5th level, a Raging Barbarian learns to sidestep hazards with an intuitive and primal danger sense. While Raging, he may use his Fortitude Save in place of his Reflex Save.}

\ability{Great Blows (Ex):}{At 7th level, a Raging Barbarian's melee attacks are Great Blows. Any enemy struck by the Barbarian's melee or thrown weapon attacks must make a Fort Save or be stunned for one round. No enemy can be targeted by this ability more than once a round, and the save DC for this ability is 10 + half the Barbarian's HD + his Constitution modifier.}

\ability{Great Life (Ex):}{While Raging, a 9th level Barbarian is immune to nonlethal damage, death effects, stunning, critical hits, negative levels, and ability damage (but not ability drain).}

\ability{Call the Horde (Ex):}{An 11th level Barbarian becomes a hero of his people. He gains the Command feat as a bonus feat, but his followers must be Barbarians. In campaigns that do not use Leadership feats, he instead gains a +2 unnamed bonus to all saves.}

\ability{Watched by Totems (Ex):}{At 13th level, a Barbarian may immediately reroll any failed save. He may do this no more than once per failed save.}

\ability{Primal Assault (Ex):}{At 15th level, a Raging Barbarian may choose to radiate an effect similar to an \spell{antimagic field} when he enters a Rage, with a caster level equal to his HD. Unlike a normal antimagic field, this effect does not suppress magic effects on him or the effects of magic items he is wearing or holding.}

\ability{Savagery (Ex):}{At 17th level, a Raging Barbarian may take a full round action to make a normal melee attack that has an additional effect similar to a \spell{mordenkainen's disjunction}. Unlike a normal \spell{mordenkainen's disjunction}, this effect only targets a single item or creature struck.}

\ability{One With The Beast:}{At 19th level, a Barbarian no longer needs to be in a Rage to use any Barbarian ability.}

%\input{Bard}
\classentry{Cleric}
\modebab
\goodfor
\poorref
\goodwil
\quot{``Fear my righteous shining holy beacon of... righteousness?''}

\desc{\class s are the holy (or unholy) warriors, standing fast against the darkness (or light). They are also made of cheese, and thus a prime target for minmaxxers.}

\playingaclass{The \class ~class can fit many different playstyles, but all \class s should have a high Wisdom score.}

\alignment{A \class 's alignment must be within one step of his deity's (that is, it may be one step away on either the lawful-chaotic axis or the good-evil axis, but not both). A \class ~may not be neutral unless his deity's alignment is also neutral.}

\races{Any.}

\startinggold{5d4x10gp (125 Gold)}

\startingage{ <-starting age, often written as a class reference like "As Rogue."-> }

\hitdie{d8}

\classskills{Concentration (Con), Craft (Int), Diplomacy (Cha), Heal (Wis), Knowledge (arcana) (Int), Knowledge (history) (Int), Knowledge (religion) (Int), Knowledge (the planes) (Int), Profession (Wis), and Spellcraft (Int).}

\ability{Domains and Class Skills}{A \class ~who chooses the Animal or Plant domain adds Knowledge (nature) (Int) to the \class ~class skills listed above. A \class ~who chooses the Knowledge domain adds all Knowledge (Int) skills to the list. A \class ~who chooses the Travel domain adds Survival (Wis) to the list. A \class ~who chooses the Trickery domain adds Bluff (Cha), Disguise (Cha), and Hide (Dex) to the list. See Deity, Domains, and Domain Spells, below, for more information.}

\skillpoints{2}

\begin{fullcastingclasstable}
\levelone{Turn or Rebuke Undead  &3 &1 &- &- &- &- &- &- &- &- }
\leveltwo{&4 &2 &- &- &- &- &- &- &- &- }
\levelthree{&4 &2 &1 &- &- &- &- &- &- &- }
\levelfour{&5 &3 &2 &- &- &- &- &- &- &- }
\levelfive{&5 &3 &2 &1 &- &- &- &- &- &- }
\levelsix{&5 &3 &3 &2 &- &- &- &- &- &- }
\levelseven{&6 &4 &3 &2 &1 &- &- &- &- &- }
\leveleight{&6 &4 &3 &3 &2 &- &- &- &- &- }
\levelnine{&6 &4 &4 &3 &2 &1 &- &- &- &- }
\levelten{&6 &4 &4 &3 &3 &2 &- &- &- &- }
\leveleleven{&6 &5 &4 &4 &3 &2 &1 &- &- &- }
\leveltwelve{&6 &5 &4 &4 &3 &3 &2 &- &- &- }
\levelthirteen{&6 &5 &5 &4 &4 &3 &2 &1 &- &- }
\levelfourteen{&6 &5 &5 &4 &4 &3 &3 &2 &- &- }
\levelfifteen{&6 &5 &5 &5 &4 &4 &3 &2 &1 &- }
\levelsixteen{&6 &5 &5 &5 &4 &4 &3 &3 &2 &- }
\levelseventeen{&6 &5 &5 &5 &5 &4 &4 &3 &2 &1 }
\leveleighteen{&6 &5 &5 &5 &5 &4 &4 &3 &3 &2 }
\levelnineteen{&6 &5 &5 &5 &5 &5 &4 &4 &3 &3 }
\leveltwenty{&6 &5 &5 &5 &5 &5 &4 &4 &4 &4 }
\end{fullcastingclasstable}

%Theres no way the table would fit with the '+1's in the spells per day section, so I took it out, the text is pretty clear that you get an extra spell slot for your domain.  Feel free to break it into two tables if you think its best.

\startclassfeatures

\proficiencies{all types of armor (light, medium, and heavy), and with shields (except tower shields).

\smallskip\noindent A \class ~who chooses the War domain receives the Weapon Focus feat related to his deity's weapon as a bonus feat. He also receives the appropriate Martial Weapon Proficiency feat as a bonus feat, if the weapon falls into that category.}

\classfeature{Aura (Ex)}{A \class ~of a chaotic, evil, good, or lawful deity has a particularly powerful aura corresponding to the deity's alignment (see the detect evil spell for details). \class s who don't worship a specific deity but choose the Chaotic, Evil, Good, or Lawful domain have a similarly powerful aura of the corresponding alignment.}

\classfeature{Spells}{A \class ~casts divine spells, which are drawn from the \class ~spell list. However, his alignment may restrict him from casting certain spells opposed to his moral or ethical beliefs; see Chaotic, Evil, Good, and Lawful Spells, below. A \class ~must choose and prepare his spells in advance (see below). To prepare or cast a spell, a \class ~must have a Wisdom score equal to at least 10 + the spell level. The Difficulty Class for a saving throw against a \class 's spell is 10 + the spell level + the \class 's Wisdom modifier. Like other spellcasters, a \class ~can cast only a certain number of spells of each spell level per day. His base daily spell allotment is given on Table: The \class. In addition, he receives bonus spells per day if he has a high Wisdom score. 

\smallskip\noindent A \class ~also gets one domain spell of each spell level he can cast, starting at 1st level. When a \class ~prepares a spell in a domain spell slot, it must come from one of his two domains (see Deities, Domains, and Domain Spells, below). \class smeditate or pray for their spells. Each \class ~must choose a time at which he must spend 1 hour each day in quiet contemplation or supplication to regain his daily allotment of spells. 

\smallskip\noindent Time spent resting has no effect on whether a \class ~can prepare spells. A \class ~may prepare and cast any spell on the \class ~spell list, provided that he can cast spells of that level, but he must choose which spells to prepare during his daily meditation.}

\classfeature{Deity, Domains, and Domain Spells}{A \class 's deity influences his alignment, what magic he can perform, his values, and how others see him. A \class ~chooses two domains from among those belonging to his deity. A \class ~can select an alignment domain (Chaos, Evil, Good, or Law) only if his alignment matches that domain. If a \class ~is not devoted to a particular deity, he still selects two domains to represent his spiritual inclinations and abilities. The restriction on alignment domains still applies. Each domain gives the \class ~access to a domain spell at each spell level he can cast, from 1st on up, as well as a granted power. The \class ~gets the granted powers of both the domains selected. With access to two domain spells at a given spell level, a \class ~prepares one or the other each day in his domain spell slot. If a domain spell is not on the \class ~spell list, a \class ~can prepare it only in his domain spell slot.}

\classfeature{Spontaneous Casting}{A good \class ~(or a neutral \class ~of a good deity) can channel stored spell energy into healing spells that the \class ~did not prepare ahead of time. The \class ~can ``lose'' any prepared spell that is not a domain spell in order to cast any cure spell of the same spell level or lower (a cure spell is any spell with ``cure'' in its name). An evil \class ~(or a neutral \class ~of an evil deity), can't convert prepared spells to cure spells but can convert them to inflict spells (an inflict spell is one with ``inflict'' in its name). A \class ~who is neither good nor evil and whose deity is neither good nor evil can convert spells to either cure spells or inflict spells (player's choice). Once the player makes this choice, it cannot be reversed. This choice also determines whether the \class ~turns or commands undead (see below).}

\classfeature{Chaotic, Evil, Good, and Lawful Spells}{A \class ~can't cast spells of an alignment opposed to his own or his deity's (if he has one). Spells associated with particular alignments are indicated by the chaos, evil, good, and law descriptors in their spell descriptions.}

\classfeature{Turn or Rebuke Undead (Su)}{Any \class, regardless of alignment, has the power to affect undead creatures by channeling the power of his faith through his holy (or unholy) symbol (see Turn or Rebuke Undead). A good \class ~(or a neutral \class ~who worships a good deity) can turn or destroy undead creatures. An evil \class ~(or a neutral \class ~who worships an evil deity) instead rebukes or commands such creatures. A neutral \class ~of a neutral deity must choose whether his turning ability functions as that of a good \class ~or an evil \class. Once this choice is made, it cannot be reversed. This decision also determines whether the \class ~can cast spontaneous cure or inflict spells (see above). A \class ~may attempt to turn undead a number of times per day equal to 3 + his Charisma modifier. A \class ~with 5 or more ranks in Knowledge (religion) gets a +2 bonus on turning checks against undead.}

\classfeature{Bonus Languages}{A \class s bonus language options include Celestial, Abyssal, and Infernal (the languages of good, chaotic evil, and lawful evil outsiders, respectively). These choices are in addition to the bonus languages available to the character because of his race.}

\classfeature{Ex-\class ~s}{A \class ~who grossly violates the code of conduct required by his god loses all spells and class features, except for armor and shield proficiencies and proficiency with simple weapons. He cannot thereafter gain levels as a \class ~of that god until he atones (see the atonement spell description).}

%\input{Druid}
\classname{Fighter} \label{class:fighter}
\vspace{-8pt}
\quot{"I've seen this kind of fire-breathing chicken-demon before. We're going to need more rope. Also a bigger cart."}

\desc{The Fighter is a versatile combatant who is able to actively disrupt the activities of his enemies. Fighters represent plucky heroes and grizzled veterans, but they always appear to surmount impossible odds. Which means in retrospect that the odds weren't all that impossible. At least, not for someone with a Fighter's talents.}

\ability{Playing a Fighter:}{Fighters are often handed to beginning players in order to help them learn the ropes. This is a cruel practice that dates back to when the Fighter was explicitly a weak class that players were forced to play to the (quit proximate) death if for whatever reason they didn't roll well enough on their stats to play a real character. The Fighter described here is not the hazing ritual of old, but it is a more complicated character than many others, being the non-magical equivalent to the Wizard. Beginning characters should probably be given a Barbarian, Conduit, or Rogue character to introduce them to the game mechanics of D\&D.}

\desc{A Fighter has an answer for virtually any circumstance and a great deal of adaptability and flexibility, and benefits greatly from being played by a player who actually knows how far a Roper's strands or a Beholder's rays reach. The Fighter character is archetypically a character who uses her opponent's limitations against them, and it really slows down play if the player needs to have those limitations explained during combat. As such, a full classed Fighter is recommended for experienced players of the game.}

\desc{That being said, a level or two of Fighter can give some breadth and resilience to almost any martial build, and makes a good multiclassing dip even (sometimes especially) for inexperienced players.}

\ability{Alignment:}{Every alignment has its share of Fighters, however more Fighters are of Lawful alignment than of Chaotic Alignment.}

\ability{Races:}{Every humanoid race has warriors, but actual Fighters are rarer in societies that don't value logistics and planning. So while there are many Fighters among the Hobgoblins, Dwarves, and Fire Giants, a Fighter is rarely seen among the ranks of the Orcs, Gnomes, or Ogres.}

\ability{Starting Gold:}{6d6x10 gp (210 gold)}

\ability{Starting Age:}{As Fighter.}

\ability{Hit Die:}{d10}

\ability{Class Skills:}{The Fighter's class skills (and the key ability for each skill) are Balance (Dex), Bluff (Cha), Climb (Str), Craft (Int), Diplomacy (Cha), Escape Artist (Dex), Handle Animal (Cha), Intimidate (Cha), Jump (Str), Knowledge (all skills individually) (Int), Listen (Wis), Move Silently (Dex), Profession (Wis), Ride (Dex), Sense Motive (Wis), Spot (Wis), Survival (Wis), Swim (Str), Tumble (Dex), and Use Rope (Dex).}

\ability{Skills/Level:}{6 + Intelligence Bonus}

\begin{table}[tbh]
\begin{small}
\begin{tabular}{lp{3cm}p{0.7cm}p{0.7cm}p{0.7cm}l}
Level  &Base Attack Bonus &Fort Save &Ref Save &Will Save &Special\\
1st &+1 &+2 &+2 &+2 &Weapons Training, Combat Focus\\
2nd &+2 &+3 &+3 &+3 &Bonus Feat\\
3rd &+3 &+3 &+3 &+3 &Problem Solver, Pack Mule\\
4th &+4 &+4 &+4 &+4 &Bonus Feat\\
5th &+5 &+4 &+4 &+4 &Logistics Mastery, Active Assualt\\
6th &+6/+1 &+5 &+5 &+5 &Bonus Feat\\
7th &+7/+2 &+5 &+5 &+5 &Forge Lore, Improved Delay\\
8th &+8/+3 &+6 &+6 &+6 &Bonus Feat\\
9th &+9/+4 &+6 &+6 &+6 &Foil Action\\
10th &+10/+5 &+7 &+7 &+7 &Bonus Feat\\
11th &+11/+6/+6 &+7 &+7 &+7 &Lunging Attacks\\
12th &+12/+7/+7 &+8 &+8 &+8 &Bonus Feat\\
13th &+13/+8/+8 &+8 &+8 &+8 &Array of Stunts\\
14th &+14/+9/+9 &+9 &+9 &+9 &Bonus Feat\\
15th &+15/+10/+10 &+9 &+9 &+9 &Greater Combat Focus\\
16th &+16/+11/+11/+11 &+10 &+10 &+10 &Bonus Feat\\
17th &+17/+12/+12/+12 &+10 &+10 &+10 &Improved Foil Action\\
18th &+18/+13/+13/+13 &+11 &+11 &+11 &Bonus Feat\\
19th &+19/+14/+14/+14 &+11 &+11 &+11 &Intense Focus, Supreme Combat Focus\\
20th &+20/+15/+15/+15 &+12 &+12 &+12 &Bonus Feat\\
\end{tabular}
\end{small}
\end{table}


\smallskip\noindent All of the following are Class Features of the Fighter class.

\ability{Weapon and Armor Proficiency:}{Fighters are proficient with all simple and Martial Weapons. Fighters are proficient with Light, Medium, and Heavy Armor and with Shields and Great Shields.}

\ability{Weapons Training (Ex):}{Fighters train obsessively with armor and weapons of all kinds, and using a new weapon is easy and fun. By practicing with a weapon he is not proficient with for a day, a Fighter may permanently gain proficiency with that weapon by succeeding at an Intelligence check DC 10 (you may not take 10 on this check).}

\ability{Combat Focus (Ex):}{A Fighter is at his best when the chips are down and everything is going to Baator in a handbasket. When the world is on fire, a Fighter keeps his head better than anyone. If the Fighter is in a situation that is stressful and/or dangerous enough that he would normally be unable to "take 10" on skill checks, he may spend a Swift Action to gain Combat Focus. A Fighter may end his Combat Focus at any time to reroll any die roll he makes, and if not used it ends on its own after a number of rounds equal to his Base Attack Bonus.}

\ability{Problem Solver (Ex):}{A Fighter of 3rd level can draw upon his intense and diverse training to respond to almost any situation. As a Swift action, he may choose any [Combat] feat he meets the prerequisites for and use it for a number of rounds equal to his base attack bonus. This ability may be used once per hour.}

\ability{Pack Mule (Ex):}{Fighters are used to long journeys with a heavy pack and the use of a wide variety of weaponry and equipment. A 3rd level Fighter suffers no penalties for carrying a medium load, and may retrieve stored items from his person without provoking an attack of opportunity.}

\ability{Logistics Mastery (Ex):}{Fighters are excellent and efficient logisticians. When a Fighter reaches 5th level, he gains a bonus to his Command Rating equal to one third his Fighter Level.}

\ability{Active Assault (Ex):}{A 5th level Fighter can flawlessly place himself where he is most needed in combat. He may take a 5 foot step as an immediate action. This is in addition to any other movement he takes during his turn, even another 5 foot step.}

\ability{Forge Lore:}{A 7th level Fighter can produce magical weapons and equipment as if he had a Caster Level equal to his ranks in Craft.}

\ability{Improved Delay (Ex):}{A Fighter of 7th level may delay his action in one round without compromising his Initiative in the next round. In addition, a Fighter may interrupt another action with his delayed action like it was a readied action (though he does not have to announce his intentions before hand).}

\ability{Foil Action (Ex):}{A 9th level Fighter may attempt to monkeywrench any action an opponent is taking. The Fighter may throw sand into a beholder's eye, bat aside a key spell component, or strike a weapon hand with a thrown object, but the result is the same: the opponent's action is wasted, and any spell slots, limited ability uses, or the like used to power it are expended. A Fighter must be within 30 feet of his opponent to use this ability, and must hit with a touch attack or ranged touch attack. Using Foil Action is an Immediate action. At 17th level, Foil Action may be used at up to 60 feet.}

\ability{Lunging Attacks (Ex):}{The battlefield is an extremely dangerous place, and 11th level Fighters are expected to hold off Elder Elementals, Hezrous, and Hamatulas. Fighters of this level may add 5 feet to the reach of any of their weapons.}

\ability{Array of Stunts (Ex):}{A 13th level Fighter may take one extra Immediate Action between his turns without sacrificing a Swift action during his next turn.}

\ability{Greater Combat Focus (Ex):}{At 15th level, a Fighter may voluntarily expend his Combat Focus as a non-action to suppress any status effect or ongoing spell effect on himself for his Base Attack Bonus in rounds.}

\ability{Intense Focus (Ex):}{A 19th level Fighter may take an extra Swift Action each round (in addition to the extra Immediate Action he can take from Array of Stunts).}

\ability{Supreme Combat Focus (Ex):}{A 19th level Fighter may expend his Combat Focus as a non-action to take 20 on any die roll. He must elect to use Supreme Combat Focus before rolling the die.}

\classentry{Knight}
\goodbab
\poorfor
\poorref
\goodwil
\quot{``Do you hear me you big lizard? You unhand that young man this instant!''}

\desc{Knights are more than a social position; in fact many knights don't have any social standing at all. These knight errants uphold the values of honor, and make a name for themselves adventuring.}

\playingaclass{A Knight has the potential to dish out tremendous damage to a single opponent, and it is tempting to think of them as monster killers. However, it is best to realize in advance that the Knight does not often realize their tremendous damage output. The threat of the Knight's Designate Opponent ability is just that -- a threat. A Knight excels at defensive tasks, and attacking a Knight is often one of the least effective options an opponent might exercise.

So by making it be a logical combat action for your opponents to attack your party's defensive expert, you've really contributed a lot to the party. A Knight can take a lot of the heat off the rest of the party. So don't get frustrated if enemies constantly interrupt your Designate Opponent action -- that's the whole point. A Knight's role is to protect others, and the best way you can do that is to provide a legitimate threat to your opponents.}

\alignment{Many Knights are Lawful. But not all of them. You have to maintain your code of conduct, but plenty of Chaotic creatures can do that too.}

\races{Knights require a fairly social background to receive their training. After all, a solitary creature generally has little use for honor. As such, while Knights often spend tremendous amounts of time far from civilization, they are almost exclusively recruited from the ranks of races that are highly urban in nature.}

\startinggold{6d6x10 gp (210 gold)}

\startingage{ <-starting age, often written as a class reference like "As Rogue."-> }

\hitdie{d12}

\classskills{ Climb (Str), Craft (Int), Diplomacy (Cha), Handle Animal (Cha), Intimidate (Cha), Jump (Str), Knowledge (History, Nobility, and Geography) (Int), Listen (Wis), Perform (Cha), Ride (Dex), Sense Motive (Wis), Spot (Wis), and Swim (Str).}

\skillpoints{4}

\begin{classtable}{}
\levelone{Designate Opponent, Mounted Combat, Code of Conduct}
\leveltwo{Damage Reduction}
\levelthree{Energy Resistance, Speak to Animals}
\levelfour{Immunity to Fear, Knightly Spirit}
\levelfive{Command}
\levelsix{Defend Others, Quick Recovery}
\levelseven{Bastion of Defense, Draw Fire}
\leveleight{Mettle, Spell Shield}
\levelnine{Sacrifice}
\levelten{Knightly Order}
\end{classtable}

\startclassfeatures

\proficiencies{simple weapons and Martial Weapons. Knights are proficient with Light, Medium, and Heavy Armor, Shields and Great Shields.}

\classfeature{Designate Opponent (Ex)}{As a Swift Action, a Knight may mark an opponent as their primary foe. This foe must be within medium range and be able to hear the Knight's challenge. If the target creature inflicts ay damage on the Knight before the Knight's next turn, the attempt fails. Otherwise, any attacks the Knight uses against the opponent during her next turn inflict an extra d6 of damage for each Knight level. This effect ends at the end of her next turn, or when she has struck her opponent a number of times equal to the number of attacks normally allotted her by her Base Attack Bonus. \smallskip

\emph{Example: Vayn is a 6th level Knight presently benefiting from a haste spell, granting her an extra attack during a Full Attack action. On her turn she designates an Ettin as her primary opponent, and the Ettin declines to attack her during the ensuing turn. When her next turn comes up, she uses a Full Attack and attacks 3 times. The first two hits inflict an extra 6d6 of damage, and then she designates the Ettin as her opponent again. It won't soon ignore her!}}

\classfeature{Mounted Combat}{A Knight gains Mounted Combat as a bonus feat at 1st level. If she already has Mounted Combat, she may gain any Combat feat she meets the prerequisites for instead.}

\classfeature{Code of Conduct}{A Knight must fight with honor even when her opponents do not. Indeed, a Knight subscribes to honor to a degree far more than that which is strictly considered necessary by other honorable characters. Actions which even hint at the appearance of impropriety are anathema to the Knight:

\begin{awesomelist}
    \item A Knight must not accept undue assistance from allies even in combat. A Knight must refuse bonuses from Aid Another actions.
    \item A Knight must refrain from the use poisons of any kind, even normally acceptable poisons such as blade toxins.
    \item A Knight may not voluntarily change shape, whether she is impersonating a specific creature or not.
    \item A Knight may not sell Magic Items.
\end{awesomelist}

A Knight who fails to abide by her code of conduct loses the ability to use any of her Knightly abilities which require actions until she atones.}

\classfeature{Damage Reduction (Ex)}{A Knight trains to suffer the unbearable with chivalry and grace. At 2nd level, she gains Damage Reduction of X/-, where X is half her Knight level, rounded down.}

\classfeature{Energy Resistance (Ex)}{A Knight may protect herself from energy types that she expects. As a Swift Action, a 3rd level Knight may grant herself Energy Resistance against any energy type she chooses equal to her Knight Level plus her Shield Bonus. This energy resistance lasts until she spends a Swift Action to choose another Energy type or her Shield bonus is reduced.}

\classfeature{Speak to Animals (Ex)}{A Knight can make herself understood by beasts. Her steed always seems to be able to catch the thrust of anything she says. A 3rd level Knight gains a bonus to any of her Ride and Handle Animal checks equal to half her Knight Level. In addition, there is no limit to how many tricks she can teach a creature, and her her Handle Animal checks are not penalized for attempting to get a creature to perform a trick it does not know.}

\classfeature{Immunity to Fear (Ex)}{At 4th level, a Knight becomes immune to [Fear] effects.}

\classfeature{Knightly Spirit (Ex)}{As a Move Equivalent Action, a 4th level Knight may restore any amount of attribute damage or drain that she has suffered.}

\classfeature{Command}{A Knight gains Command as a bonus feat at level 5.}

\classfeature{Defend Others (Ex)}{A 6th level Knight may use her own body to defend others. Any ally adjacent to the Knight gains Evasion, though she does not.}

\classfeature{Quick Recovery (Ex)}{If a 6th level Knight is stunned or dazed during her turn, that condition ends at the end of that turn. \smallskip

\emph{Example: Vayn is hit by a mindblast and would be stunned for 7 turns. She misses her next action and then shakes off the condition ready to fight.}}

\classfeature{Bastion of Defense (Ex)}{A 7th level Knight can defend others with great facility. All adjacent allies except the Knight gain a +2 Dodge bonus to their Armor Class and Reflex Saves.}

\classfeature{Draw Fire (Ex)}{A 7th level Knight can exploit the weaknesses of unintelligent opponents. With a Swift Action, she may pique the interest of any mindless opponent within medium range. That creature must make a Willpower Save (DC 10 + \half\  Hit Dice + Constitution Modifier) or spend all of its actions moving towards or attacking the Knight. This effect ends after a number of rounds equal to the Knight's class level.}

\classfeature{Mettle (Ex)}{An 8th level Knight who succeeds at a Fortitude Partial or Willpower Partial save takes no effect as if she had immunity. \smallskip

\emph{For example, if Vayn was hit with an inflict wounds spell and made her saving throw, she would take no damage instead of the partial effect in the spell description (half damage in this case).}}

\classfeature{Spell Shield (Ex)}{An 8th level Knight gains Spell Resistance of 5 + her character level. This Spell Resistance is increased by her shield bonus to AC if she has one.}

\classfeature{Sacrifice (Ex)}{As an immediate action, a 9th level Knight may make herself the target of an attack or targeted effect that targets any creature within her reach.}

\classfeature{Knightly Order}{What is a powerful Knight without a descriptive adjective? Upon reaching 10th level, a Knight must join or found a Knightly order. From this point on, she may ignore one of the prerequisites for joining a Knightly Order prestige class. In addition, becoming a member of an order has special meaning for a 10th level Knight, and she gains an ability related to the order she joins. Some sample orders are listed below:}

\begin{awesomelist}
    \item \ability{Angelic Knight}{The Angelic Knights are a transformational order that attempts to live by the precepts of the upper planes. An Angelic Knight gains wings that allow her to fly at double her normal speed with perfect maneuverability. Also an Angelic Knight benefits from protection from evil at all times.}
    \item \ability{Bane Knight}{The Bane Knights stand for running around burning the countryside with extreme burning. Bane Knights are immune to fire and do not have to breathe. In addition, a Bane Knight may set any unattended object on fire with a Swift Action at up to Medium Range.}
    \item \ability{Chaos Knight}{Chaos Knights stand for madness and Giant Frog. With the powers of Giant Frog, they can Giant Frog. Also their natural armor bonus increases by +5 and they are immune to sleep effects.}
    \item \ability{Dragon Knight}{Dedicated to the Platinum Dragon, the Dragon Knights serve love and justice in equal measure as dishes to those who need them. A Dragon Knight gains a +5 bonus to Sense Motive and any armor she wears has its enhancement bonus increased to +5 (it also gains a platinum sheen in the process, and as a side effect a Dragon Knight is never dirty for more than a few seconds).}
    \item \ability{Elemental Knight}{The Elemental Knights may be dedicated to a particular element, or somehow dedicated to all of them. An Elemental Knight can planeshift at will to any Inner plane or the Prime Material plane. Also, she is immune to stunning and always benefits from attune form when on any Inner Plane.}
    \item \ability{Fey Knight}{Using the powers of the Sprites, the Fey Knight has many fairy strengths. Firstly, she gains DR 10/Iron. Also, any of her attacks may do non-lethal damage at any time if this is desired. Also she never ages and does not need to drink.}
    \item \ability{Great Knight}{Clad in opulent armor, the Great Knight cares only for her own power. The Great Knight gains a +4 bonus on Disarm or Sunder tests, and gains a +4 Profane bonus to her Strength.}
    \item \ability{Hell Knight}{Forged in the sulphurous clouds of Baator, the Hell Knight is bathed in an evil radiance. In addition to being granted a ceremonial weapon made of green steel, a Hell Knight gains the coveted see in darkness ability of the Baatorians. Also, she has an inherent ability to know what every creature within 60' of finds most repugnant.}
    \item \ability{Imperial Knight}{The great Empire needs champions able to unswervingly support its interests, and the Imperial Knight is one of the best. She may impose a zone of truth at will as a Supernatural ability, and all of her attacks are Lawfully aligned. Also, she continuously benefits from \spell{magic circle against chaos}.}
\end{awesomelist}


%%%%%%%%%%%%%%%%%%%%%%%%%%%%%%%%%%%%%%%%%%%%%%%%%%
\classentry{Monk}
%%%%%%%%%%%%%%%%%%%%%%%%%%%%%%%%%%%%%%%%%%%%%%%%%%

\textbf{Alignment:} Any lawful.

\textbf{Hit Die:} d8.

\textbf{Class Skills}

The monk's class skills (and the key ability for each skill) are \linkskill{Balance} (Dex), 
\linkskill{Climb} (Str), \linkskill{Concentration} (Con), \linkskill{Craft} (Int), \linkskill{Diplomacy} (Cha), \linkskill{Escape Artist} (Dex), 
\linkskill{Hide} (Dex), \linkskill{Jump} (Str), \linkskill{Knowledge} (arcana) (Int), \linkskill{Knowledge} (religion) (Int), \linkskill{Listen} 
(Wis), \linkskill{Move Silently} (Dex), \linkskill{Perform} (Cha), \linkskill{Profession} (Wis), \linkskill{Sense Motive} (Wis), 
\linkskill{Spot} (Wis), \linkskill{Swim} (Str), and \linkskill{Tumble} (Dex).

\textbf{Skill Points at 1st Level:} (4 + Int modifier) x4.

\textbf{Skill Points at Each Additional Level:} 4 + Int modifier.

\begin{table}[htb]
\rowcolors{1}{white}{offyellow}
\caption{The Monk}
\centering
\begin{tabular}{*{7}{l}}
\textbf{Level} & \textbf{BAB} & \textbf{Fort} & \textbf{Reflex} & \textbf{Will} & \textbf{Special} & \textbf{Speed Bonus}\\
1st & +0 & +2 & +2 & +2 & Bonus Feat, Flurry Of Blows, Unarmed Strike & +0ft\\
2nd & +1 & +3 & +3 & +3 & Bonus Feat, Evasion & +0ft\\
3rd & +2 & +3 & +3 & +3 & Still Mind & +10ft\\
4th & +3 & +4 & +4 & +4 & Ki Strike (Magic), Slow Fall 20ft & +10ft\\
5th & +3 & +4 & +4 & +4 & Purity Of Body & +10ft\\
6th & +4 & +5 & +5 & +5 & Bonus Feat, Slow Fall 30ft & +20ft\\
7th & +5 & +5 & +5 & +5 & Wholeness of Body & +20ft\\
8th & +6 & +6 & +6 & +6 & Slow Fall 40ft & +20ft\\
9th & +6 & +6 & +6 & +6 & Improved Evasion & +30ft\\
10th & +7 & +7 & +7 & +7 & Ki Strike (Lawful), Slow Fall 50ft & +30ft\\
11th & +8 & +7 & +7 & +7 & Diamond Body, Greater Flurry & +30ft\\
12th & +9 & +8 & +8 & +8 & Abundant Step, Slow Fall 60ft & +40ft\\
13th & +9 & +8 & +8 & +8 & Diamond Soul & +40ft\\
14th & +10 & +9 & +9 & +9 & Slow Fall 70ft & +40ft\\
15th & +11 & +9 & +9 & +9 & Quivering Palm & +50ft\\
16th & +12 & +10 & +10 & +10 & Ki Strike (Adamantine), Slow Fall 80ft & +50ft\\
17th & +12 & +10 & +10 & +10 & Timeless Body, Tongue of The Sun And Moon & +50ft\\
18th & +13 & +11 & +11 & +11 & Slow Fall 90ft & +60ft\\
19th & +14 & +11 & +11 & +11 & Empty Body & +60ft\\
20th & +15 & +12 & +12 & +12 & Perfect Self, Slow Fall Any Distance & +60ft\\
\end{tabular}
\end{table}

%%%%%%%%%%%%%%%%%%%%%%%%%
\ClassFeatures
%%%%%%%%%%%%%%%%%%%%%%%%%

All of the following are class features of the monk.

\textbf{Weapon and Armor Proficiency:} Monks are proficient with club, crossbow 
(light or heavy), dagger, handaxe, javelin, kama, nunchaku, quarterstaff, sai, 
shuriken, siangham, and sling.

Monks are not proficient with any armor or shields.

When wearing armor, using a shield, or carrying a medium or heavy load, a monk 
loses her AC bonus, as well as her fast movement and flurry of blows abilities.

\textbf{AC Bonus (Ex):} When unarmored and unencumbered, the monk adds her Wisdom 
bonus (if any) to her AC. In addition, a monk gains a +1 bonus to AC at 5th level. 
This bonus increases by 1 for every five monk levels thereafter (+2 at 10th, +3 
at 15th, and +4 at 20th level).

These bonuses to AC apply even against touch attacks or when the monk is flat-footed. 
She loses these bonuses when she is immobilized or helpless, when she wears any 
armor, when she carries a shield, or when she carries a medium or heavy load.

\textbf{Flurry of Blows (Ex):} When unarmored, a monk may strike with a flurry 
of blows at the expense of accuracy. When doing so, she may make one extra attack 
in a round at her highest base attack bonus, but this attack takes a -2 penalty, 
as does each other attack made that round. The resulting modified base attack bonuses 
are shown in the Flurry of Blows Attack Bonus column on Table: The Monk. This penalty 
applies for 1 round, so it also affects attacks of opportunity the monk might make 
before her next action. When a monk reaches 5th level, the penalty lessens to -1, 
and at 9th level it disappears. A monk must use a full attack action to strike 
with a flurry of blows.

When using flurry of blows, a monk may attack only with unarmed strikes or with 
special monk weapons (kama, nunchaku, quarterstaff, sai, shuriken, and siangham). 
She may attack with unarmed strikes and special monk weapons interchangeably as 
desired. When using weapons as part of a flurry of blows, a monk applies her Strength 
bonus (not Str bonus x1-1/2 or x1/2) to her damage rolls for all successful 
attacks, whether she wields a weapon in one or both hands. The monk can't use any 
weapon other than a special monk weapon as part of a flurry of blows.

In the case of the quarterstaff, each end counts as a separate weapon for the purpose 
of using the flurry of blows ability. Even though the quarterstaff requires two 
hands to use, a monk may still intersperse unarmed strikes with quarterstaff strikes, 
assuming that she has enough attacks in her flurry of blows routine to do so. 

When a monk reaches 11th level, her flurry of blows ability improves. In addition 
to the standard single extra attack she gets from flurry of blows, she gets a second 
extra attack at her full base attack bonus.

\textbf{Unarmed Strike:} At 1st level, a monk gains Improved Unarmed Strike as 
a bonus feat. A monk's attacks may be with either fist interchangeably or even 
from elbows, knees, and feet. This means that a monk may even make unarmed strikes 
with her hands full. There is no such thing as an off-hand attack for a monk striking 
unarmed. A monk may thus apply her full Strength bonus on damage rolls for all 
her unarmed strikes.

Usually a monk's unarmed strikes deal lethal damage, but she can choose to deal 
nonlethal damage instead with no penalty on her attack roll. She has the same choice 
to deal lethal or nonlethal damage while grappling.

A monk's unarmed strike is treated both as a manufactured weapon and a natural 
weapon for the purpose of spells and effects that enhance or improve either manufactured 
weapons or natural weapons.

A monk also deals more damage with her unarmed strikes than a normal person of their size would, 
as shown on Table: Monk Unarmed Damage.

\begin{table}[htb]
\rowcolors{1}{white}{offyellow}
\caption{Monk Unarmed Damage}
\centering
\begin{tabular}{l c c c}
\textbf{Level} & \textbf{Small} & \textbf{Medium} & \textbf{Large}\\
1st-3rd & 1d4 & 1d6 & 1d8 \\
4th-7th & 1d6 & 1d8 & 2d6 \\
8th-11th & 1d8 & 1d10 & 2d8 \\
12th-15th & 1d10 & 2d6 & 3d6 \\
16th-19th & 2d6 & 2d8 & 3d8 \\
20th & 2d8 & 2d10 & 4d8 \\
\end{tabular}
\end{table}

\textbf{Bonus Feat:} At 1st level, a monk may select either \linkfeat{Improved Grapple} or 
\linkfeat{Stunning Fist} as a bonus feat. At 2nd level, she may select either \linkfeat{Combat Reflexes} 
or \linkfeat{Deflect Arrows} as a bonus feat. At 6th level, she may select either \linkfeat{Improved Disarm}
or \linkfeat{Improved Trip} as a bonus feat. A monk need not have any of the prerequisites 
normally required for these feats to select them.

\textbf{Evasion (Ex):} At 2nd level or higher if a monk makes a successful Reflex 
saving throw against an attack that normally deals half damage on a successful 
save, she instead takes no damage. Evasion can be used only if a monk is wearing 
light armor or no armor. A helpless monk does not gain the benefit of evasion.

\textbf{Fast Movement (Ex):} At 3rd level, a monk gains an enhancement bonus to 
her speed, as shown on Table: The Monk. A monk in armor or carrying a medium or 
heavy load loses this extra speed.

\textbf{Still Mind (Ex):} A monk of 3rd level or higher gains a +2 bonus on saving 
throws against spells and effects from the school of enchantment.

\textbf{Ki Strike (Su):} At 4th level, a monk's unarmed attacks 
are empowered with \textit{ki}. Her unarmed attacks are treated as magic weapons 
for the purpose of dealing damage to creatures with damage reduction. Ki 
strike improves with the character's monk level. At 10th level, her unarmed attacks 
are also treated as lawful weapons for the purpose of dealing damage to creatures 
with damage reduction. At 16th level, her unarmed attacks are treated as adamantine 
weapons for the purpose of dealing damage to creatures with damage reduction and 
bypassing hardness.

\textbf{Slow Fall (Ex):} At 4th level or higher, a monk within arm's reach of a 
wall can use it to slow her descent. When first using this ability, she takes damage 
as if the fall were 20 feet shorter than it actually is. The monk's ability to 
slow her fall (that is, to reduce the effective distance of the fall when next 
to a wall) improves with her monk level until at 20th level she can use a nearby 
wall to slow her descent and fall any distance without harm.

\textbf{Purity of Body (Ex):} At 5th level, a monk gains immunity to all diseases 
except for supernatural and magical diseases.

\textbf{Wholeness of Body (Su):} At 7th level or higher, a monk can heal her own 
wounds. She can heal a number of hit points of damage equal to twice her current 
monk level each day, and she can spread this healing out among several uses.

\textbf{Improved Evasion (Ex):} At 9th level, a monk's evasion ability improves. 
She still takes no damage on a successful Reflex saving throw against attacks, 
but henceforth she takes only half damage on a failed save. A helpless monk does 
not gain the benefit of improved evasion.

\textbf{Diamond Body (Su):} At 11th level, a monk gains immunity to poisons of 
all kinds.

\textbf{Abundant Step (Su):} At 12th level or higher, a monk can slip magically 
between spaces, as if using the spell \linkspell{Dimension Door}, once per day. Her 
caster level for this effect is one-half her monk level (rounded down).

\textbf{Diamond Soul (Ex):} At 13th level, a monk gains spell resistance equal 
to her current monk level + 10. In order to affect the monk with a spell, a spellcaster 
must get a result on a caster level check (1d20 + caster level) that equals or 
exceeds the monk's spell resistance.

\textbf{Quivering Palm (Su):} Starting at 15th level, a monk can set up vibrations 
within the body of another creature that can thereafter be fatal if the monk so 
desires. She can use this quivering palm attack once a week, and she must announce 
her intent before making her attack roll. Constructs, oozes, plants, undead, incorporeal 
creatures, and creatures immune to critical hits cannot be affected. Otherwise, 
if the monk strikes successfully and the target takes damage from the blow, the 
quivering palm attack succeeds. Thereafter the monk can try to slay the victim 
at any later time, as long as the attempt is made within a number of days equal 
to her monk level. To make such an attempt, the monk merely wills the target to 
die (a free action), and unless the target makes a Fortitude saving throw (DC 10 
+ 1/2 the monk's level + the monk's Wis modifier), it dies. If the saving throw 
is successful, the target is no longer in danger from that particular quivering 
palm attack, but it may still be affected by another one at a later time.

\textbf{Timeless Body (Ex):} Upon attaining 17th level, a monk no longer takes 
penalties to her ability scores for aging and cannot be magically aged. Any such 
penalties that she has already taken, however, remain in place. Bonuses still accrue, 
and the monk still dies of old age when her time is up.

\textbf{Tongue of the Sun and Moon (Ex):} A monk of 17th level or higher can speak 
with any living creature.

\textbf{Empty Body (Su):} At 19th level, a monk gains the ability to assume an 
ethereal state for 1 round per monk level per day, as though using the spell \linkspell{Etherealness}. 
She may go ethereal on a number of different occasions during any single day, as 
long as the total number of rounds spent in an ethereal state does not exceed her 
monk level.

\textbf{Perfect Self:} At 20th level, a monk becomes a magical creature. She is 
forevermore treated as an outsider rather than as a humanoid (or whatever the monk's 
creature type was) for the purpose of spells and magical effects. Additionally, 
the monk gains damage reduction 10/magic, which allows her to ignore the first 
10 points of damage from any attack made by a nonmagical weapon or by any natural 
attack made by a creature that doesn't have similar damage reduction. Unlike other 
outsiders, the monk can still be brought back from the dead as if she were a member 
of her previous creature type.

%%%%%%%%%%%%%%%%%%%%%%%%%
\subsection{Ex-Monks}
%%%%%%%%%%%%%%%%%%%%%%%%%

A monk who becomes nonlawful cannot gain new levels as a monk but retains all monk 
abilities.

Like a member of any other class, a monk may be a multiclass character, but multiclass 
monks face a special restriction. A monk who gains a new class or (if already multiclass) 
raises another class by a level may never again raise her monk level, though she 
retains all her monk abilities.
%\input{Paladin}
%\input{Ranger}
%\input{Rogue}
%\input{Samurai}
%\input{Sorcerer}
\classentry{Templar}

\newcommand{\vow}[6]{
\subsubsection{Vow of #1}
\quot{``#2''}
\listone
	\item \ability{First: }{#3}
	\item \ability{Second: }{#4}
	\item \ability{Third: }{#5}
\end{list}
\vspace{8pt}
\ability{Roleplaying Ideas: }{#6}
}

\newcommand{\faith}[7]{
\subsubsection{#1}
#2
\begin{list}{\textbf{\arabic{counter}}:~}{\itemspace\usecounter{counter}}
	\item #3
	\item #4
	\item #5
	\item #6
	\item #7
\end{list}
\vspace{8pt}
}

\goodbab
\goodfor
\poorref
\goodwil
\quot{``Nobody is more dangerous than he who imagines himself pure in heart, for his purity, by definition, is unassailable.''}

\begin{classpreamble}
\desc{Every religion has clerics, those tasked with performing the duties of the religion. Many also have faithful members who leave their homes to travel distant lands, spreading the word of their god or pantheon. Templars are ordained warriors tasked with spreading the faith and defending the faithful, while also beating down the foes of a deity.
\newline
Templars are the militant arm of their church and/or cause. They are often guards of sacred places, dispatched away from the temples as agents of higher powers, or simply wander to share the virtues of their philosophy and ideal with others. Initially able and zealous warriors combining martial abilities with the power of their deity, they eventually become an active sword or shield for their deity, with high levels of offensive prowess and devastating crowd control. Whether as a bodyguard or a support character, they often find themselves in the ranks of adventuring parties who can make use of the talents.
\newline
A templar generally exemplifies a particular ideology of life, and associated nomenclature may depend on the side with which he aligns himself. A good templar, for instance, might assume the title of paladin while those who embrace evil are often known as blackguards and those who serve neutrality are called gray wardens. What truly differentiates these characters are the vows that they swear to uphold.}
\playingaclass{Templars value Charisma greatly, as it allows them to better convince those they encounter of the importance of their deity and provides force to their spells. They also value Strength as it allows them to beat up those who steadfastly refuse to believe and get in the way of the templar's work. Constitution is often the third most important ability for a templar, as it allows them to stand longer in the fray.}
\alignment{Any, though a templar may only select a deity who allows worshipers of the templar's alignment. Conversely, a templar of a specific deity is limited to only those alignments which would be allowed by the deity for a follower. Templars without a patron deity may select any alignment they like.}
\races{Any. Every race that has deities has templars to spread their teachings.}
\startinggold{3d10x10 gp (165 gp).}
%\startingage{Moderate}
\hitdie{d10}
\classskills{Appraise (Int), Climb (Str), Concentration (Con), Craft (Int), Heal (Wis), Intimidate (Cha), Jump (Str), Knowledge (nobility and royalty) (religion) (Int), Listen (Wis), Ride (Dex), Sense Motive (Wis), Speak Language (None), Spellcraft (Int), Swim (Str).}
\skillpoints{4}
\end{classpreamble}

\afterpage{
\begin{minorcastingclasstable}
\levelone{Divine Vow (Once Vowed), Vow of Piety (Once Vowed)& 						2&-&-&-&-&-&-}
\leveltwo{Avenger of the Faith (Primary)& 															3&-&-&-&-&-&-}
\levelthree{Divine Vow (Once Vowed)& 																3&2&-&-&-&-&-}
\levelfour{Avenger of the Faith (Secondary)& 														3&2&-&-&-&-&-}
\levelfive{Divine Vow (Once Vowed)& 																	3&3&2&-&-&-&-}
\levelsix{Avenger of the Faith (Primary), Arms of the Faithful& 							3&3&2&-&-&-&-}
\levelseven{Divine Vow (Twice Vowed), Vow of Piety (Twice Vowed)& 				3&3&3&2&-&-&-}
\leveleight{Avenger of the Faith (Secondary), Inquisitor& 									3&3&3&2&-&-&-}
\levelnine{Divine Vow (Twice Vowed)& 																3&3&3&2&-&-&-}
\levelten{Avenger of the Faith (Primary)& 															3&3&3&3&2&-&-}
\leveleleven{Divine Vow (Twice Vowed)& 															3&3&3&3&2&-&-}
\leveltwelve{Avenger of the Faith (Secondary), Sustained by Faith& 					3&3&3&3&2&-&-}
\levelthirteen{Divine Vow (Thrice Vowed)& 															3&3&3&3&3&2&-}
\levelfourteen{Avenger of the Faith (Primary)& 													4&3&3&3&3&2&-}
\levelfifteen{Divine Vow (Thrice Vowed), Undying Faith (as raise dead)& 			4&4&3&3&3&2&-}
\levelsixteen{Avenger of the Faith (Secondary)& 													4&4&4&3&3&3&2}
\levelseventeen{Divine Vow (Thrice Vowed)& 														4&4&4&4&3&3&2}
\leveleighteen{Avenger of the Faith (Primary), Undying Faith (as resurrection)&	4&4&4&4&4&3&3}
\levelnineteen{Divine Vow (Thrice Vowed)& 														4&4&4&4&4&4&3}
\leveltwenty{Avenger of the Faith (Secondary), All Things Are Possible& 			4&4&4&4&4&4&4}
\end{minorcastingclasstable}}

\startclassfeatures

\proficiencies{simple and martial weapons, all forms of armor, and all shields.}

\classfeature{Spells:}{A templar cast divine spells, which are drawn from the list below and supplemented by their deity's domains (see Vow of Piety). His caster level for these spells is equal to his class level. The save DCs for these spells are equal to 10 + the spell's level + his Charisma modifier. A templar must have a charisma score of at least 10 + the spell's level in order to cast the spell.
\newline
A Templar know all of the spells on his class list, and may cast any of them without preparation so long as he has an appropriate spell slot available and an charisma score of at least 10 + the spell's level. His maximum available slots per day are determined by his class level (as seen on Table: The Templar), and he gains bonus slots from his charisma score.
\newline
In order to receive their spell slots, the templar must pray for 1 hour without interruption in a place free from distractions or noise. At the end of this time, he receives his spell slots. After praying, the templar cannot pray again until one whole day (24 hours) has passed.
A templar’s spells are more for utility than combat efficacy, either allowing him to better solve problems through non-violent means or enhancing his combat abilities past even their already formidable limits.}

\classfeature{Code of Conduct (Ex):}{Like any other character, a templar does what he must to uphold the duties given to him by an organization of which he is a part, even if that organization is as loose as his alignment group. But let’s face it; sometimes even the good and honorable knight may want to lie about his identity or consort with unscrupulous characters in order to root out the evil, demonic cult. And evil knights can be obsessed with battle, honor, and battling with honor. A templar is not specifically prohibited from acts that lie outside of their alignment or run counter to their deity's wishes. Many aspire to these things and most follow them, but not all do so and no templar is punished for being found slightly wanting. Templars who actively displease or betray their deity may still be stripped of their powers and dismissed, however.}

\classfeature{Divine Vow (Su):}{A templar’s code is somewhat variable; different deities and philosophies extol different virtues that a templar must try to uphold. But more than that, each templar is permitted to extol these virtues in slightly different ways. The vows a templar makes are a representation of his personal or religious code, and determine which aspects he attempts to uphold most strongly. These vows grant him extraordinary powers (the nature of which vary based on the vows he takes). These are detailed in the section on divine vows below.
\listone
	\item At 1st level the templar gains the Vow of Piety and one other rank 1 vow of their choice. At every odd-numbered class level thereafter the templar may take a new vow, but he may not advance one of his existing vows beyond rank 1 at this time.
	\item At 7th level, he reaffirms his Vow of Piety and gains a second domain. He may also reaffirm any other vow which he already possesses to gain the rank 2 ability. A vow that has been reaffirmed in this way is known as "twice vowed." Instead of reaffirming a rank 1 vow, he may instead select two new vows at rank 1. He may not advance a vow beyond rank 2 at this time.
	\item At 13th level, he may reaffirm any other vow in which he already possesses the rank 2 ability to gain the rank 3 ability. A vow that has been reaffirmed in this way is known as "thrice vowed." Instead of reaffirming a rank 2 vow, he may instead select two rank 1 vows at advance to rank 2, or may select a new vow to gain both the rank 1 and rank 2 benefits.
\end{list}}

\classfeature{Avenger of the Faith:}{A templar trains himself in multiple forms of combat, so as to serve as both the weapon and shield of their church or ideals. Starting at second level, he chooses a primary combat form (see Avenger of the Faith Styles) for which he gains the corresponding abilities at 2nd level and every four class levels thereafter. At 4th level, he chooses his secondary style, and gains the benefits thereof at each 4 class levels.}

\classfeature{Arms of the Faithful (Ex):}{At sixth level a templar gains Craft Magic Arms and Armor as a bonus feat. When crafting any magic items with this feat, they are treated as having access to the spells of the war domain in addition to those on their class list. If they already possess Craft Magic Arms and Armor, they may select another item creation feat for which they qualify.}

\classfeature{Inquisitor (Su):}{An eigth level templar can detect the alignments of any creature that he can see as a swift action. He instantly gains all information about their alignment as if he had spent three rounds concentrating on them with the appropriate spells. If the creature is warded, the templar may make a caster level check against the warding spell to gain the information if such a check is allowed by the ward. In addition, all the templar’s attacks are automatically considered aligned (good or evil, lawful or chaotic, etc. based on his alignment) for the purposes of overcoming damage reduction.}

\classfeature{Sustained by Faith (Ex):}{An eleventh level templar gains everything they need to live from their relationship with their deity. They no longer need to eat, drink, breathe, or sleep. They can still do these things if they want to of course.}

\classfeature{Undying Faith (Su):}{Fifteenth level templars are extremely difficult to kill. The templar may elect to gain the benefit of a raise dead spell at any time within 1 minute of being killed. If they do, their return is announced by a powerful flash of light (as a daylight spell) for 1 round. Instead of the normal level loss, they instead suffer 2 points of Charisma burn. Once used, they may not return from the dead in this way for 24 hours; a templar who dies twice in a day will need someone else to bring them back to continue their work. At eighteenth level, this ability improves to offer the benefit of a resurrection spell instead, though the templar only returns with half of their maximum hit points.}

\classfeature{All Things Are Possible (Sp):}{The prayers of a twentieth level templar are taken very seriously. Once per day they may cast miracle as a spell-like ability, though they must still spend experience points if the effect would require them from a spellcaster casting it.}

\subsection{Ex-Templars}

A templar who wishes to pursue other classes is welcome to do so. There are no multiclssing restrictions against the templar.
A templar who willingly leaves his faith or who is cast out loses all spells, spell-like, and supernatural abilities, as well as any ability stemming from one of their vows. They may return to the faith if a ranking member casts an atonement for them. They may also pursue a new faith entirely. They must still find a member of the faith to atone them, however. When joining a new faith in this way, the templar loses all of their old vows. They may swear a new one each day until they have reached the level allotted them based on their level.

\subsection{Vows}
\quot{``So many vows, they make you swear and swear. Defend the King, obey the King. Obey your father. Protect the innocent. Defend the weak. What if your father despises the King? What if the King massacres the innocent? It's too much. No matter what you do, you're forsaking one vow or another.''}

\vow{Charity
}{A bone to the dog is not charity. Charity is the bone shared with the dog, when you are just as hungry as the dog.
}{Once per round on your turn you may aid another as a free action.
}{Once per round when you are targeted by a spell with an effect beneficial to you, you may allow another creature within Close Range to also gain the benefits of that spell. The spell must also be beneficial to the creature you wish to share it with (interpreted at the DM's discretion), or the sharing fails.
}{An ally within Close range of you may use your spell slots to cast a spell of an equivalent or lower spell level, so long as you possess the minimum charisma score to use the slot yourself. Your ally may use this slot to cast any spell that they have prepared or that they know (in the case of spontaneous casters), using your slot instead of their own. Your ally may also cast spells from your spell list, even if they would not normally be capable of casting divine spells. Anyone casting a spell in this fashion uses their own attributes, feats, and character level to determine the effects and DC of the spell. They do not need to meet the minimum charisma score requirement for a particular spell level cast from your list, but they must be of a sufficient level that they would be able to use the spell slot were they a templar of the same level.
}{Perhaps your church decrees that its members must give aid to others, or maybe you give out of the goodness of your heart. You are the quintessential selfless knight, giving to others without necessarily thinking of your own gains. There are times when you may give up more important things than money; the truest sacrifice a templar can make is to offer their own life in the service of their cause.}

\subsection{Avenger of the Faith Styles}
As there are many different vows that a templar can swear, so to are there different combat styles that they may practice. A templar selects one of these styles as their primary style and another as a secondary. They are both then advanced as the templar gains levels.

\faith{Charger
}{A charger is a very straightforward templar. They see their foes, and they run or ride out to meet them. This generally leads to the defeat of their foes.
}{\ability{Knight Errant (Ex):}{A charger needs to work around the limitations of the bulky armor that is so often part of his attire. You no longer suffer penalties to your base speed from wearing medium or heavy armor. You also gain additional benefits while charging. You may make 1 turn up to 90 degrees as part of your charge action, though you must still travel at least 10 feet in a straight line immediately before you attack a target. Additionally, you are not required to move to the closest space to your opponent during a charge, and may make your charge attack when your opponent is in any of your threatened spaces. This would allow you to take a charge attack while running past an opponent, but this movement would provoke attacks of opportunity as normal.}
}{\ability{Cataphract (Ex):}{When charging you gain a +4 bonus to your attack roll instead of the normal +2 and you may make a full attack on a charge. You also may charge up to three times your normal base speed when you make a charge as a full-round action. If you would only be limited to a partial charge, you may move twice your base speed as part of that action. You may not make a full-attack when you perform a partial charge, however. This benefit also applies while you are mounted.}
}{\ability{Charge of Necessity (Su):}{While charging or running, you gain the benefit or air walk for the round, until the start of your next turn. If you do not continue running or charging at the start of the next round, you instead fall to the ground under the effect of feather fall. If you begin a fall from other circumstances you do not benefit from this effect. This benefit also applies while you are mounted.}
}{\ability{Charge of Glory (Ex):}{You can trample over those who fall before your charge, continuing to seek more blood. If you destroy an effect in your path, render a charged opponent unconscious or dead, or otherwise clear the way forward while charging you may continue the charge along the same path (following all normal restrictions as they apply) up to your full allowed distance. You may make additional attacks against those in your way along this additional distance as if they were your intended charge target. This benefit also applies while you are mounted.}
}{\ability{Charge of Destruction (Su):}{When a foe is struck with your charge attack and killed, they are destroyed utterly as if they had been immolated or disintegrated. Further, while charging or running you may leave behind a blade barrier as you leaves each space. The wall need not be continuous, and may have as many or as few breaks in it as you desire. This wall deals 15d6 points of damage, has a save DC of 16 + the templar's Charisma modifer, and dissipates at the start of your next turn. This benefit also applies while you are mounted.}
}
\classentry{Wizard}
\poorbab
\poorfor
\poorref
\goodwil
\quot{``Don't make this wizard mad, don't make this wizard pissed, I can kill a hill giant with a flick of my wrist!'''}

\desc{ }

\playingaclass{Wizards primarily rely on a high Intelligence to learn and cast their spells. Some wizards also rely on having a decent Dexterity score to deliver touch attacks.}

\alignment{Though the study and practice involved in wizardry tends to attract practitioners of a lawful alignment, people of any alignment may become wizards.}

\races{Becoming a wizard takes years of study, and wizards are typically represented by races that are more long lived and by societies that are well organized. That said, people of any race may become wizards.}

\startinggold{3d4x10 gp (75 Gold)}

\startingage{Something needs to go here when we decide what even goes here.}

\hitdie{d4}

\classskills{Concentration (Con), Craft (Int), Decipher Script (Int), Knowledge (all skills, taken individually) (Int), Profession (Wis), and Spellcraft (Int).}

\skillpoints{2}

\begin{fullcastingclasstable}
\levelone{Summon familiar, Scribe Scroll &3 &1 &- &- &- &- &- &- &- &-}
\leveltwo{&4 &2 &- &- &- &- &- &- &- &-}
\levelthree{&4 &2 &1 &- &- &- &- &- &- &-}
\levelfour{&4 &3 &2 &- &- &- &- &- &- &-}
\levelfive{Bonus Feat&4 &3 &2 &1 &- &- &- &- &- &-}
\levelsix{&4 &3 &3 &2 &- &- &- &- &- &-}
\levelseven{&4 &4 &3 &2 &1 &- &- &- &- &-}
\leveleight{&4 &4 &3 &3 &2 &- &- &- &- &-}
\levelnine{&4 &4 &4 &3 &2 &1 &- &- &- &-}
\levelten{&4 &4 &4 &3 &3 &2 &- &- &- &-}
\leveleleven{&4 &4 &4 &4 &3 &2 &1 &- &- &-}
\leveltwelve{&4 &4 &4 &4 &3 &3 &2 &- &- &-}
\levelthirteen{&4 &4 &4 &4 &4 &3 &2 &1 &- &-}
\levelfourteen{&4 &4 &4 &4 &4 &3 &3 &2 &- &-}
\levelfifteen{&4 &4 &4 &4 &4 &4 &3 &2 &1 &-}
\levelsixteen{&4 &4 &4 &4 &4 &4 &3 &3 &2 &-}
\levelseventeen{&4 &4 &4 &4 &4 &4 &4 &3 &2 &1}
\leveleighteen{&4 &4 &4 &4 &4 &4 &4 &3 &3 &2}
\levelnineteen{&4 &4 &4 &4 &4 &4 &4 &4 &3 &3}
\leveltwenty{&4 &4 &4 &4 &4 &4 &4 &4 &4 &4}
\end{fullcastingclasstable}

\startclassfeatures

\proficiencies{the club, dagger, heavy crossbow, light crossbow, and quarterstaff, but not with any type of armor or shield. Armor of any type interferes with a wizard's movements, which can cause her spells with somatic components to fail.}

\classfeature{Spells}{A wizard casts arcane spells which are drawn from the sorcerer/ wizard spell list. A wizard must choose and prepare her spells ahead of time (see below).
To learn, prepare, or cast a spell, the wizard must have an Intelligence score equal to at least 10 + the spell level. The Difficulty Class for a saving throw against a wizard's spell is 10 + the spell level + the wizard's Intelligence modifier.

Like other spellcasters, a wizard can cast only a certain number of spells of each spell level per day. Her base daily spell allotment is given on Table: The Wizard. In addition, she receives bonus spells per day if she has a high Intelligence score.

Unlike a bard or sorcerer, a wizard may know any number of spells. She must choose and prepare her spells ahead of time by getting a good night's sleep and spending 1 hour studying her spellbook. While studying, the wizard decides which spells to prepare.
Bonus Languages: A wizard may substitute Draconic for one of the bonus languages available to the character because of her race.}

\classfeature{Familiar}{A wizard can obtain a familiar in exactly the same manner as a sorcerer can. See the sorcerer description and the information on Familiars below for details.}

\classfeature{Scribe Scroll}{At 1st level, a wizard gains Scribe Scroll as a bonus feat.}

\classfeature{Bonus Feats}{At 5th, 10th, 15th, and 20th level, a wizard gains a bonus feat. At each such opportunity, she can choose a metamagic feat, an item creation feat, or Spell Mastery. The wizard must still meet all prerequisites for a bonus feat, including caster level minimums.

These bonus feats are in addition to the feat that a character of any class gets from advancing levels. The wizard is not limited to the categories of item creation feats, metamagic feats, or Spell Mastery when choosing these feats.}

\classfeature{Spellbooks}{A wizard must study her spellbook each day to prepare her spells. She cannot prepare any spell not recorded in her spellbook, except for read magic, which all wizards can prepare from memory.

A wizard begins play with a spellbook containing all 0-level wizard spells (except those from her prohibited school or schools, if any; see School Specialization, below) plus three 1st-level spells of your choice. For each point of Intelligence bonus the wizard has, the spellbook holds one additional 1st-level spell of your choice. At each new wizard level, she gains two new spells of any spell level or levels that she can cast (based on her new wizard level) for her spellbook. At any time, a wizard can also add spells found in other wizards' spellbooks to her own.}

\subsubsection{School Specialization}

A school is one of eight groupings of spells, each defined by a common theme. If desired, a wizard may specialize in one school of magic (see below). Specialization allows a wizard to cast extra spells from her chosen school, but she then never learns to cast spells from some other schools.

A specialist wizard can prepare one additional spell of her specialty school per spell level each day. She also gains a +2 bonus on Spellcraft checks to learn the spells of her chosen school.

The wizard must choose whether to specialize and, if she does so, choose her specialty at 1st level. At this time, she must also give up two other schools of magic (unless she chooses to specialize in divination; see below), which become her prohibited schools.

A wizard can never give up divination to fulfill this requirement.

Spells of the prohibited school or schools are not available to the wizard, and she can't even cast such spells from scrolls or fire them from wands. She may not change either her specialization or her prohibited schools later.

The eight schools of arcane magic are abjuration, conjuration, divination, enchantment, evocation, illusion, necromancy, and transmutation.

Spells that do not fall into any of these schools are called universal spells.

\begin{awesomelist}
	\item \ability{Abjuration}{Spells that protect, block, or banish. An abjuration specialist is called an abjurer.}
	\item \ability{Conjuration}{Spells that bring creatures or materials to the caster. A conjuration specialist is called a conjurer.}
	\item \ability{Divination}{Spells that reveal information. A divination specialist is called a diviner. Unlike the other specialists, a diviner must give up only one other school.}
	\item \ability{Enchantment}{Spells that imbue the recipient with some property or grant the caster power over another being. An enchantment specialist is called an enchanter.}
	\item \ability{Evocation}{Spells that manipulate energy or create something from nothing. An evocation specialist is called an evoker.}
	\item \ability{Illusion}{Spells that alter perception or create false images. An illusion specialist is called an illusionist.}
	\item \ability{Necromancy}{Spells that manipulate, create, or destroy life or life force. A necromancy specialist is called a necromancer.}
	\item \ability{Transmutation}{Spells that transform the recipient physically or change its properties in a more subtle way. A transmutation specialist is called a transmuter.}
	\item \ability{Universal}{Not a school, but a category for spells that all wizards can learn. A wizard cannot select universal as a specialty school or as a prohibited school. Only a limited number of spells fall into this category.}
\end{awesomelist}

\subsubsection{Familiars}

A familiar is a normal animal that gains new powers and becomes a magical beast when summoned to service by a sorcerer or wizard. It retains the appearance, Hit Dice, base attack bonus, base save bonuses, skills, and feats of the normal animal it once was, but it is treated as a magical beast instead of an animal for the purpose of any effect that depends on its type. Only a normal, unmodified animal may become a familiar. An animal companion cannot also function as a familiar.

A familiar also grants special abilities to its master (a sorcerer or wizard), as given on the table below. These special abilities apply only when the master and familiar are within 1 mile of each other.

Levels of different classes that are entitled to familiars stack for the purpose of determining any familiar abilities that depend on the master's level.

\begin{table}[h]
\rowcolors{1}{colorone}{colortwo}
\begin{tabu}to \textwidth{lX}
\header Familiar & Special \\ \hline
Bat & Master gains a +3 bonus on Listen checks \\
Cat & Master gains a +3 bonus on Move Silently checks \\
Hawk & Master gains a +3 bonus on Spot checks in bright light \\
Lizard & Master gains a +3 bonus on Climb checks \\
Owl & Master gains a +3 bonus on Spot checks in shadows \\
Rat & Master gains a +2 bonus on Fortitude saves \\
Raven\textsuperscript{1} & Master gains a +3 bonus on Appraise checks	\\
Snake\textsuperscript{2} & Master gains a +3 bonus on Bluff checks \\
Toad & Master gains +3 hit points \\
Weasel & Master gains a +2 bonus on Reflex saves \\ \hline
\multicolumn{2}{l}{\textsuperscript{1} A raven familiar can speak one language of its master's choice as a supernatural ability.} \\
\multicolumn{2}{l}{\textsuperscript{2} Tiny viper.} \\ \hline
\end{tabu}
\end{table}

Use the basic statistics for a creature of the familiar's kind, but make the following changes.

\ability{Hit Dice}{For the purpose of effects related to number of Hit Dice, use the master's character level or the familiar's normal HD total, whichever is higher.}

\ability{Hit Points}{The familiar has one-half the master's total hit points (not including temporary hit points), rounded down, regardless of its actual Hit Dice.}

\ability{Attacks}{Use the master's base attack bonus, as calculated from all his classes. Use the familiar's Dexterity or Strength modifier, whichever is greater, to get the familiar's melee attack bonus with natural weapons. Damage equals that of a normal creature of the familiar's kind.}

\ability{Saving Throws}{For each saving throw, use either the familiar's base save bonus (Fortitude +2, Reflex +2, Will +0) or the master's (as calculated from all his classes), whichever is better. The familiar uses its own ability modifiers to saves, and it doesn't share any of the other bonuses that the master might have on saves.}

\ability{Skills}{For each skill in which either the master or the familiar has ranks, use either the normal skill ranks for an animal of that type or the master's skill ranks, whichever are better. In either case, the familiar uses its own ability modifiers. Regardless of a familiar's total skill modifiers, some skills may remain beyond the familiar's ability to use.}

\ability{Familiar Ability Descriptions}{All familiars have special abilities (or impart abilities to their masters) depending on the master's combined level in classes that grant familiars, as shown on the table below. The abilities given on the table are cumulative.}

\begin{awesomelist}
  \item \ability{Natural Armor Adj.}{The number noted here is an improvement to the familiar's existing natural armor bonus.}
  \item \ability{Int}{The familiar's Intelligence score.}
  \item \ability{Alertness (Ex)}{While a familiar is within arm's reach, the master gains the Alertness feat.}
  \item \ability{Improved Evasion (Ex)}{When subjected to an attack that normally allows a Reflex saving throw for half damage, a familiar takes no damage if it makes a successful saving throw and half damage even if the saving throw fails.}
  \item \ability{Share Spells}{At the master's option, he may have any spell (but not any spell-like ability) he casts on himself also affect his familiar. The familiar must be within 5 feet at the time of casting to receive the benefit. If the spell or effect has a duration other than instantaneous, it stops affecting the familiar if it moves farther than 5 feet away and will not affect the familiar again even if it returns to the master before the duration expires. Additionally, the master may cast a spell with a target of ``You'' on his familiar (as a touch range spell) instead of on himself. A master and his familiar can share spells even if the spells normally do not affect creatures of the familiar's type (magical beast).}
  \item \ability{Empathic Link (Su)}{The master has an empathic link with his familiar out to a distance of up to 1 mile. The master cannot see through the familiar's eyes, but they can communicate empathically. Because of the limited nature of the link, only general emotional content can be communicated. Because of this empathic link, the master has the same connection to an item or place that his familiar does.}
  \item \ability{Deliver Touch Spells (Su)}{If the master is 3rd level or higher, a familiar can deliver touch spells for him. If the master and the familiar are in contact at the time the master casts a touch spell, he can designate his familiar as the toucher. The familiar can then deliver the touch spell just as the master could. As usual, if the master casts another spell before the touch is delivered, the touch spell dissipates.}
  \item \ability{Speak with Master (Ex)}{If the master is 5th level or higher, a familiar and the master can communicate verbally as if they were using a common language. Other creatures do not understand the communication without magical help.}
  \item \ability{Speak with Animals of Its Kind (Ex)}{If the master is 7th level or higher, a familiar can communicate with animals of approximately the same kind as itself (including dire varieties): bats with bats, rats with rodents, cats with felines, hawks and owls and ravens with birds, lizards and snakes with reptiles, toads with amphibians, weasels with similar creatures (weasels, minks, polecats, ermines, skunks, wolverines, and badgers). Such communication is limited by the intelligence of the conversing creatures.}
  \item \ability{Spell Resistance (Ex)}{If the master is 11th level or higher, a familiar gains spell resistance equal to the master's level + 5. To affect the familiar with a spell, another spellcaster must get a result on a caster level check (1d20 + caster level) that equals or exceeds the familiar's spell resistance.}
  \item \ability{Scry on Familiar (Sp)}{If the master is 13th level or higher, he may scry on his familiar (as if casting the scrying spell) once per day.}
\end{awesomelist}

\begin{table}[h]
\rowcolors{1}{colorone}{colortwo}
\begin{tabu} to \textwidth{lllX}
\header Master Class Level & Natural Armor Adj. & Int & Special \\ \hline
1st-2nd & +1 & 6 & Alertness, improved evasion, share spells, empathic link \\
3rd-4th & +2 & 7 & Deliver touch spells \\
5th-6th & +3 & 8 & Speak with master \\
7th-8th & +4 & 9 & Speak with animals of its kind \\
9th-10th & +5 & 10 & -\\
11th-12th & +6 & 11 & Spell resistance \\
13th-14th & +7 & 12 & Scry on familiar \\
15th-16th & +8 & 13 & -\\
17th-18th & +9 & 14 &  -\\
19th-20th & +10 & 15 & -\\
\end{tabu}
\end{table}

\subsubsection{Arcane Spells and Armor}

Wizards and sorcerers do not know how to wear armor effectively.

If desired, they can wear armor anyway (though they'll be clumsy in it), or they can gain training in the proper use of armor (with the various Armor Proficiency feats light, medium, and heavy and the Shield Proficiency feat), or they can multiclass to add a class that grants them armor proficiency. Even if a wizard or sorcerer is wearing armor with which he or she is proficient, however, it might still interfere with spellcasting.

Armor restricts the complicated gestures that a wizards or sorcerer must make while casting any spell that has a somatic component (most do). The armor and shield descriptions list the arcane spell failure chance for different armors and shields.

By contrast, bards not only know how to wear light armor effectively, but they can also ignore the arcane spell failure chance for such armor. A bard wearing armor heavier than light or using any type of shield incurs the normal arcane spell failure chance, even if he becomes proficient with that armor.

If a spell doesn't have a somatic component, an arcane spellcaster can cast it with no problem while wearing armor. Such spells can also be cast even if the caster's hands are bound or if he or she is grappling (although Concentration checks still apply normally). Also, the metamagic feat Still Spell allows a spellcaster to prepare or cast a spell at one spell level higher than normal without the somatic component. This also provides a way to cast a spell while wearing armor without risking arcane spell failure. 

\chapter{Skills}
\section{How Skills Work}
foo
\section{Appraise}
foo
\section{Athletics}
foo
\section{Balance}
foo
\section{Bluff}
foo
\section{Concentration}
foo
\section{Craft}
foo
\section{Decipher Script}
foo
\section{Diplomacy}
foo
\section{Disable Device}
foo
\section{Disguise}
foo
\section{Escape Artist}
foo
\section{Forgery}
foo
\section{Gather Information}
foo
\section{Handle Animal}
foo
\section{Heal}
foo
\section{Intimidate}
foo
\section{Knowledge}
foo
\section{Perception}
foo
\section{Perform}
foo
\section{Profession}
foo
\section{Ride}
foo
\section{Search}
foo
\section{Sense Motive}
foo
\section{Sleight of Hand}
foo
\section{Speak Language}
foo
\section{Spellcraft}
foo
\section{Stealth}
foo
\section{Survival}
foo
\section{Tumble}
foo
\section{Use Magic Device}
foo

\chapter{Feats}
\section{How Feats Work}
foo
\section{General Feats}
foo
\section{Combat Feats}
foo
\section{Skill Feats}
foo
\section{Metamagic Feats}
foo

\chapter{Goods and Services}
\section{The Three Economies}
foo
\section{Armor}
foo
\section{Weapons}
At their core, a weapon is just an object of a particular size and complexity that you wield against a foe in an attempt to disable them. The size, and complexity of that object are not negligible parts of it though, and are in fact the basis of an effective weapon.

\subsection{Weapon Descriptors}

\subsubsection{Melee Weapons}
Melee weapons are used for making attacks against nearby foes, and generally threaten every space within a creature's natural reach.

\textbf{Reach Weapons:} A reach weapon is a melee weapon with a much longer haft than normal, allowing its wielder to strike at targets that aren't adjacent to him or her. Most reach weapons double the wielder's natural reach, meaning that a typical Small or Medium wielder of such a weapon can attack a creature 10 feet away, but not a creature in an adjacent square. A typical Large character wielding a reach weapon of the appropriate size can attack a creature 15 or 20 feet away, but not adjacent creatures or creatures up to 10 feet away. There may be limits on how you can use a reach weapon, consult each weapons individual entry. 

\textbf{Hurled Weapons:} Some melee weapons also list a range. These may be hurled at a target, and function as a ranged weapon when they are used in this way. See "Thrown Weapons" for more information about using a hurled melee weapon in this way.

\textbf{Double Weapons:} A double weapon has a damaging head on both the ends of the weapon. A character can fight with both ends of a double weapon as if fighting with two weapons, but they incur all the normal attack penalties associated with two-weapon combat as if they were wielding a one-handed weapon and a light weapon. The character can also choose to use a double weapon two handed, attacking with only one end of it. A creature wielding a double weapon in one hand can't use it as a double weapon-only one end of the weapon can be used in any given round. 

\subsubsection{Ranged Weapons}
Ranged weapons are suited for striking distant foes, and threaten no spaces. Using a ranged weapon within a threat range generally provokes attacks of opportunity, making them well suited to firing from out of the fray.

\textbf{Projectile Weapons:} Crossbows, repeating crossbows, bows, compound bows, and slings are projectile weapons. Most projectile weapons require two hands to use (see specific weapon descriptions). A character does not add their Strength bonus on damage rolls with a projectile weapon unless it's a composite bow or sling. If the character has a penalty for low Strength, it is added to damage rolls when they use a projectile weapon other than a crossbow.

\textbf{Ammunition:} Projectile weapons use ammunition: arrows (for bows), bolts (for crossbows), or sling bullets (for slings). When using a bow, a character can draw ammunition as a free action; crossbows and slings require an action for reloading. Generally speaking, ammunition that hits its target is destroyed or rendered useless, while normal ammunition that misses has a 50\% chance of being destroyed or lost. 

Attempting to use an arrow or bolt as a melee weapon incurs a -4 non-proficiency penalty, and deals damage equal to the bow or crossbow it was designed for. Sling bullets may not be used as a melee weapon.

\textbf{Ranged weapons and Mounts:} Thrown weapons and crossbows can be used from mounts without complication (aside from the normal penalties for using ranged weapons from mounts). Bows must be at least one size category smaller than the wielder to be used on a mount.

\textbf{Thrown Weapons:} In order to use a thrown weapon properly, it must be small enough for the wielder to use one handed. Ranged weapons the same size as the wielder can be thrown with two hands, but doing so incurs a -4 penalty on the attack roll. The wielder applies his or her Strength modifier to damage dealt by thrown weapons (except for splash weapons). It is possible to throw a weapon that isn't designed to be thrown (that is, a melee weapon that doesn't have a numeric entry in the Range Increment column on Table: Weapons), but a character who does so takes a -4 penalty on the attack roll. Throwing a light or one-handed weapon is a standard action, while throwing a two-handed weapon is a full-round action. Regardless of the type of weapon, such an attack scores a threat only on a natural roll of 20 and deals double damage on a critical hit. Such a weapon has a range increment of 10 feet. Any weapon three sizes smaller than the wielder can be thrown with a 10 foot range increment without penalty. 

\subsubsection{Improvised Weapons}
Sometimes objects not crafted to be weapons nonetheless see use in combat. Because such objects are not designed for this use, any creature that uses one in combat is considered to be nonproficient with it and takes a -4 penalty on attack rolls made with that object. To determine the size category and appropriate damage for an improvised weapon, compare its relative size and damage potential to the weapon list to find a reasonable match. An improvised weapon scores a threat on a natural roll of 20 and deals double damage on a critical hit. An improvised thrown weapon has a range increment of 10 feet. Objects heaver than a character's light load cannot be used as weapons.

\subsubsection{Weapon Size}
Every weapon, like every object and creature, has a size category that indicates how different sized creatures can interact with it.

\textbf{Two-Handed:} A two-handed weapon is one that is the same size category as the wielder. Two-handed weapons must be wielded with both the primary and off hand to be effective. Attacks with a two-handed melee weapon add 1-1/2 times the character's Strength bonus to damage rolls.

\textbf{One-Handed:} A one-handed weapon is one that is one size category smaller than the wielder. One-handed weapons can be used in either the primary hand or the off hand. Attacks with a one-handed melee weapon add the wielder's Strength bonus to damage rolls if it's used in the primary hand, or 1/2 their Strength bonus if it's used in the off hand. If a one-handed melee weapon is wielded with two hands during combat, 1-1/2 times the character's Strength bonus is added to damage rolls. 

\textbf{Light:} A light weapon is one that is two or more size categories smaller than the wielder. Light weapons can be used in either the primary hand or the off hand. Attacks with a light weapon add the wielder's Strength bonus (if any) to damage rolls for melee attacks with a light weapon if it's used in the primary hand, or one-half the wielder's Strength bonus if it's used in the off hand. Using two hands to wield a light weapon gives no advantage on damage; the Strength bonus applies as though the weapon were held in the wielder's primary hand only. It is even easier to use in one's off hand than a one-handed weapon is, however, and they are well suited for two-weapon fighting styles. Light melee weapons can also be used while grappling. An unarmed strike is always considered a light weapon.

\textbf{Inappropriately Sized Weapons:} A creature can't use weapons larger than itself.

\subsubsection{Weapon Complexity}
In addition to their size, every weapon is grouped according to how difficult the weapon is to master. Every weapon falls into one of three broad categories: simple, martial, and exotic. 

\textbf{Simple Weapons:} Simple weapons are those that require the least training and practice to use effectively. Many are simple 'swing and pray' or 'point and thrust' weapons that are hard to mess up, or simply weapons that are so small that they only work in very straightforward ways. Most classes are proficient in all simple weapons.

\textbf{Martial Weapons:} Martial weapons are those that require dedication and training to use effectively. 

Because of this martial weapons that are three sizes smaller than the wielder are always treated as simple weapons for the wielder.

\textbf{Exotic Weapons:} 

\subsection{Weapon Description}
Weapon entries follow the following format.

\noindent\textbf{Cost:} This value is the weapon's cost in gold pieces (gp) or silver pieces (sp). The cost includes miscellaneous gear that goes with the weapon. The cost is the same for a Small or Medium version of the weapon, while a Large version costs twice the listed price. Versions smaller than Small cost half as much for each size category reduction.

\textbf{Damage:} Each type of weapon deals damage based on its size.

\textbf{Critical:} The entry in this column notes how the weapon is used with the rules for critical hits. The first number indicates the minimum d20 roll that will generate a critical threat. When your character confirms a threat and scores a critical hit, roll the weapon damage two, three, or four times as indicated by its critical multiplier (using all applicable modifiers to weapon damage on each roll), and add all the results together. Extra damage over and above a weapon's normal damage, such as sneak attack damage, is not multiplied when you score a critical hit.

\begin{itemize}
  \item\noindent\textbf{20x2:}{The weapon deals double damage on a critical hit.}
  \item\noindent\textbf{20x3:}{The weapon deals triple damage on a critical hit.}
  \item\noindent\textbf{x3/x4:}{One head of this double weapon deals triple damage on a critical hit. The other head deals quadruple damage on a critical hit.}
  \item\noindent\textbf{20x4:}{The weapon deals quadruple damage on a critical hit.}
  \item\noindent\textbf{19-20/x2:}{The weapon scores a threat on a natural roll of 19 or 20 (instead of just 20) and deals double damage on a critical hit. (The weapon has a threat range of 19-20.)}
  \item\noindent\textbf{18-20/x2:}{The weapon scores a threat on a natural roll of 18, 19, or 20 (instead of just 20) and deals double damage on a critical hit. (The weapon has a threat range of 18-20.)}
  \item\noindent\textbf{19-20/x3:}{The weapon scores a threat on a natural roll of 19 or 20 (instead of just 20) and deals triple damage on a critical hit. (The weapon has a threat range of 19-20.)}
\end{itemize}

\textbf{Range:} Any attack at less than this distance is not penalized for range. However, each full range increment imposes a cumulative -2 penalty on the attack roll. A thrown weapon has a maximum range of five range increments. A projectile weapon can shoot out to ten range increments.

\textbf{Type:} Weapons are classified according to the type of damage they deal: bludgeoning, piercing, or slashing. Some monsters may be resistant or immune to attacks from certain types of weapons. Some weapons deal damage of multiple types. If a weapon is of two types, the damage it deals is not half one type and half another; all of it is both types. Therefore, a creature would have to be immune to both types of damage to ignore any of the damage from such a weapon.  In other cases, a weapon can deal either of two types of damage. In a situation when the damage type is significant, the wielder can choose which type of damage to deal with such a weapon.

\textbf{Special:} Some weapons have special features. See the weapon descriptions for details.

\afterpage{\newcommand{\wepcell}[3]{\begin{tabular}{l}#1\\#2\\#3\end{tabular}}

\begin{small}
{\tabulinesep=1mm
\rowcolors{1}{colorone}{colortwo}
\begin{longtabu} to \textwidth {X[3.75, l] X[3, l] X[4, l] X[3, l] X[2, l] X[2, l] X[2, l] X[2, l] X[2, l] X[2, l]}
\header\textbf{Simple Weapons}&&&&&&&&\\
\hline
\rowcolor{colortwo}\textbf{Weapon} &\textbf{Critical} &\textbf{Type} &\textbf{Range} &\textbf{Dimin.} &\textbf{Tiny} &\textbf{Small} &\textbf{Medium} &\textbf{Large} \endhead
 Club &20x2 &Bludgeoning &Melee &1d3 &1d4  &1d6   &1d8 &2d6 \\[1ex]
 Crowssbow &19-20x2 &Piercing &120 ft. &1d4 &1d6 &1d8 &1d10 &2d8 \\[1ex]
 Gauntlet\footnotemark[1] &20x2 &Bludgeoning &Melee &- &1 &1d2 &1d3 &1d4 \\[1ex]
 Hammer &20x2 &Bludgeoning &Melee &1d3 &1d4 &1d6 &1d8 &2d6 \\[1ex]
 Longspear &20x3 &Piercing &Reach &1d3 &1d4 &1d6 &1d8 &2d6 \\[1ex]
 Morning Star &20x2 &Bludgeoning \& Piercing &Melee &1d4 &1d6 &1d8 &2d6 &3d6 \\
 Sling &20x2 &Bludgeoning &50 ft. &1 &1d2 &1d3 &1d4 &1d6 \\[1ex]
 Spear &20x3 &Piercing &Melee or 20 ft. &1d3 &1d4 &1d6 &1d8 &2d6 \\
 Staff &20x2 &Bludgeoning &Melee &\sfrac{1d2}{1d2} &\sfrac{1d3}{1d3} &\sfrac{1d4}{1d4} &\sfrac{1d6}{1d6} &\sfrac{1d8}{1d8} \\
 Spiked Gauntlet\footnotemark[1] &20x2 &Piercing and Bludgeoning &Melee &1 &1d2 &1d3 &1d4 &1d6 \\
 Unarmed\footnotemark[1] &20x2 &Bludgeoning &Melee &- &1 &1d2 &1d3 &1d4 \\[1ex] \hline
\end{longtabu} 
\vspace{-1pt}
\vspace{-1\baselineskip}
\noindent\begin{tabu}to \textwidth{X}
\cellcolor{colortwo}\footnotemark[1] The size and damage for this weapon indicates the size of the creature using it, instead of the actual size of the weapon. These weapons are always considered light weapons.\\ \hline
 \end{tabu}
%
\rowcolors{1}{colorone}{colortwo}
\begin{longtabu} to \textwidth {X[6, l] X[3, l] X[4, l] X[3, l] X[2, l] X[2, l] X[2, l] X[2, l] X[2, l] X[2, l]}
\header\textbf{Martial Weapons}&&&&&&&&&\\
\hline
\rowcolor{colortwo}\textbf{Weapon} &\textbf{Critical} &\textbf{Type} &\textbf{Range} &\textbf{Fine} &\textbf{Dimin.} &\textbf{Tiny} &\textbf{Small} &\textbf{Medium} &\textbf{Large} \endhead
 Axe &20x3 &Bludgeoning \& Slashing &Melee &1d3 &1d4 &1d6 &1d8 &1d12 &3d6 \\
 Bastard Sword &19-20x2 &Slashing or Piercing &Melee &1d3 &1d4 &1d6 &1d8 &1d10 &2d8 \\
 Bow &20x3 &Piercing &100 ft. &- &1d3 &1d4 &1d6 &1d8 &2d6 \\[1ex]
 Composite Bow &20x3 &Piercing &110 ft.&- &1d3 &1d4 &1d6 &1d8 &2d6 \\[1ex]
 Curved Sword &18-20x2 &Slashing &Melee &1d2 &1d3 &1d4 &1d6 &2d4 &2d6 \\[1ex]
 Dwarven Axe &20x3 &Bludgeoning \& Slashing &Melee &1d3 &1d4 &1d6 &1d8 &1d10 &1d12 \\
 Flail &19-20x2 &Bludgeoning &Melee &1d3 &1d4 &1d6 &1d8 &1d10 &2d6 \\[1ex]
 Glaive &20x3 &Slashing &Reach &1d3 &1d4 &1d6 &1d8 &1d10 &2d6 \\[1ex]
 Greatclub &20x2 &Bludgeoning &Melee &1d3 &1d4 &1d6 &1d8 &1d10 &2d6 \\[1ex]
 Guisarme &20x3 &Slashing &Reach &1d2 &1d3 &1d4 &1d6 &2d4 &2d6 \\[1ex]
 Halberd &20x3 &Percing or Slashing &Melee &1d3 &1d4 &1d6 &1d8 &1d10 &2d6 \\
 Pick &20x4 &Piercing &Melee &1 &1d2 &1d4 &1d6 &1d8 &1d10 \\[1ex]
 Ranseur &20x3 &Piercing &Reach &1d2 &1d3 &1d4 &1d6 &2d4 &2d6 \\[1ex]
 Sap &20x2 &Bludgeoning &Melee &1d3 &1d4 &1d6 &1d8 &1d10 &2d6 \\[1ex]
 Scythe &20x4 &Piercing or Slashing &Melee &1d2 &1d3 &1d4 &1d6 &2d4 &2d6 \\
 Shield &20x2 &Bludgeoning &Melee &1 &1d2 &1d3 &1d4 &1d6 &1d8 \\[1ex]
 Spiked Armor\footnotemark[1] &20x2 &Piercing &Melee &1 &1d2 &1d3 &1d4 &1d6 &1d8 \\[1ex]
 Spiked Shield &20x2 &Bludgeoning \& Piercing &Melee &1d3 &1d4 &1d6 &1d8 &1d10 &2d6 \\[1ex]
 Sword &19-20x2 &Slashing or Piercing &Melee &1d3 &1d4 &1d6 &1d8 &2d6 &3d6 \\
 Thinblade &19-20x3 &Piercing &Melee &1d2 &1d3 &1d4 &1d6 &2d4 &2d6 \\[1ex]
 Throwing Axe &20x2 &Bludgeoning \& Slashing &10 ft. &1d2 &1d3 &1d4 &1d6 &1d8 &1d10 \\
 Throwing Hammer &20x2 &Bludgeoning &20 ft. &1d2 &1d3 &1d4 &1d6 &1d8 &1d10  \\[1ex]
 Trident &20x2 &Piercing &Melee or 10 ft. &1d3 &1d4 &1d6 &1d8 &1d10 &2d6 \\
 Warhammer &20x3 or 20x4 &Bludgeoning or Piercing &Melee &1d3 or 1d2 &1d4 or 1d3 &1d6 or 1d4 &1d8 or 1d6 &2d6 or 1d8 &3d6 or 2d6 \\
 \hline
\end{longtabu}
%
\rowcolors{1}{colorone}{colortwo}
\begin{longtabu} to \textwidth {X[6, l] X[3, l] X[4, l] X[3, l] X[2, l] X[2, l] X[2, l] X[2, l] X[2, l] X[2, l]}
\header\textbf{Exotic Weapons}&&&&&&&&&\\
\hline
\rowcolor{colortwo}\textbf{Weapon} &\textbf{Critical} &\textbf{Type} &\textbf{Range} &\textbf{Fine} &\textbf{Dimin.} &\textbf{Tiny} &\textbf{Small} &\textbf{Medium} &\textbf{Large} \endhead
 Bolas &20x2 &Bludgeoning &10 ft. &- &1 &1d2 &1d3 &1d4 &1d6 \\[1ex]
 Dire Flail &19-20x2/ 19-20x2 &Bludgeoning &Melee &1d2/ 1d2 &1d3/ 1d3 &1d4/ 1d4 &1d6/ 1d6 &1d8/ 1d8 &1d10/ 1d10 \\
 Double Axe &20x3/ 20x3 &Bludgeoning \& Slashing &Melee &1d3/ 1d3 &1d3/ 1d3 &1d4/ 1d4 &1d6/ 1d6 &1d8/ 1d8 &1d10/ 1d10 \\
 Double Sword &19-20x2/ 19-20x2 &Piercing or Slashing &Melee &1d2/ 1d2 &1d3/ 1d3 &1d4/ 1d4 &1d6/ 1d6 &1d8/ 1d8 &1d10/ 1d10 \\
 Hook-Hammer &20x3/ 20x4 &Bludgeoning/ Piercing &Melee &- &1d2/\newline{}1 &1d3/ 1d2 &1d4/ 1d3 &1d6/ 1d4 &1d8/ 1d6 \\
 Kama &20x2 &Slashing &Melee &1d3 &1d4 &1d6 &1d8 &1d10 &2d6 \\[1ex]
 Kasurigama &20x2 &Slashing &Melee or Reach &1 &1d2 &1d3 &1d4 &1d6 &2d4 \\
 Net &N/A &N/A &Reach &- &- &- &- &- &- \\[1ex]
 Nunchaku &20x2 &Bludgeoning &Melee &1d3 &1d4 &1d6 &1d8 &1d10 &2d6 \\[1ex]
 Repeating Crossbow &19-20x2 &Piercing &120 ft. &1d3 &1d4 &1d6 &1d8 &1d10 &2d6\\[1ex]
 Sai &20x2 &Bludgeoning &Melee or 10 ft. &1d2 &1d3 &1d4 &1d6 &1d8 &1d10 \\
 Shuriken &20x2 &Piercing &10 ft &1 &1d2 &1d3 &1d4 &1d6 &1d8 \\[1ex]
 Siangham &20x2 &Piercing &Melee &1d3 &1d4 &1d6 &1d8 &1d10 &2d6 \\[1ex]
 Spiked Chain &20x2 &Piercing &Special &1d2 &1d3 &1d4 &1d6 &2d4 & \\[1ex]
 Urgrosh &20x3/\newline{}20x3 &Slashing/ Piercing &Melee &1d2/\newline{}1 &1d3/ 1d2 &1d4/ 1d3 &1d6/ 1d4 &1d8/ 1d6 &1d10/ 1d8 \\
 Whip &20x2 &Slashing &Special &- &1 &1d2 &1d3 &1d4 &1d6 \\[1ex]
 \hline
\end{longtabu}}
\end{small}}

\subsection{Individual Weapon Rules}

In addition to the qualities given on the table, some weapons have additional rules, given below.

\textbf{Bastard Sword:} A character with exotic weapon proficiency can wield a bastard sword as if they were one size larger than they are.

\textbf{Bolas:} You can use this weapon to make a ranged trip attack against an opponent. You can't be tripped during your own trip attempt when using a set of bolas. As a thrown weapon, bolas must be one size smaller than you to be used effectively.

\textbf{Bow:} Bows are projectile weapons, the range given is for a medium sized bow. For every size category larger or smaller than medium, add or subtract 30 feet from the bows range. You need at least two hands to use a bow, regardless of its size. A bow the same size as you is too unwieldy to use while you are mounted. If you have a penalty for low Strength, apply it to damage rolls when you use a Bow. If you have a bonus for high Strength, you can apply it to damage rolls when you use a composite bow (see below) but not a regular bow.

\textbf{Composite Bow:} You need at least two hands to use a composite bow, regardless of its size. You can use a composite bow up to your size while mounted. All composite bows are made with a particular minimum strength rating (that is, each requires a minimum Strength score to use with proficiency). If your Strength score is less than the strength rating of the composite bow, you can't use it. The default composite longbow requires a Strength score of 10 or higher to use. A composite longbow can be made with a high strength rating to take advantage of an above-average Strength score; this feature allows you to add your Strength bonus to damage, as long as you meet the strength rating for the bow you can add either your Strength bonus, or the strength bonus that would be derived from the bows strength rating +4, to your damage rolls, whichever is lower.

\textbf{Crossbow:} Crossbows are ranged weapons that use bolts. The range listed for the crossbow is for one of medium size, for every size category larger or smaller than medium increase or decrease the range by 40 ft. Reloading a crossbow provokes an attack of opportunity, Reloading a light and one-handed crossbows is a move action, two-handed crossbows require a full round action to reload. Reloading a crossbow requires two hands.

\textbf{Dire Flail:} A dire flail is a double weapon. You can fight with it as if fighting with two weapons, but if you do, you incur all the normal attack penalties associated with fighting with two weapons, just as if you were using a one-handed weapon and a light weapon. A creature wielding a dire flail in one hand can't use it as a double weapon— only one end of the weapon can be used in any given round. When using a dire flail, you get a +2 bonus on attack rolls made to disarm an enemy. You can also use this weapon to make trip attacks. If you are tripped during your own trip attempt, you can drop the dire flail to avoid being tripped.

\textbf{Double Axe:} A double axe is a double weapon. You can fight with it as if fighting with two weapons, but if you do, you incur all the normal attack penalties associated with fighting with two weapons, just as if you were using a one-handed weapon and a light weapon. A creature wielding an orc double axe in one hand can't use it as a double weapon-only one end of the weapon can be used in any given round.

\textbf{Double Sword:} A double sword is a double weapon. You can fight with it as if fighting with two weapons, but if you do, you incur all the normal attack penalties associated with fighting with two weapons, just as if you were using a one-handed weapon and a light weapon. A creature wielding a two-bladed sword in one hand can't use it as a double weapon-only one end of the weapon can be used in any given round.

\textbf{Dwarven Axe:} A character with exotic proficiency with a Dwarven Axe can wield one as if they were one size category larger than they are. Dwarves only need martial proficiency with them to do this.

\textbf{Flail:} With a flail, you get a +2 bonus on attack rolls made to disarm an enemy. You can also use this weapon to make trip attacks. If you are tripped during your own trip attempt, you can drop the flail to avoid being tripped.

\textbf{Gauntlet:} This metal glove lets you deal lethal damage rather than nonlethal damage with unarmed strikes. A strike with a gauntlet is otherwise considered an unarmed attack. Medium and heavy armors (except breastplate) come with gauntlets. The damage listings given are for a gauntlet made for a creature of the indicated size, instead of fo a gauntlet of the indicated size. You may not wear gauntlets made for a creature of a different size than you.

\textbf{Glaive:} A glaive has reach. The glaives reach property can only be used when it is a two-handed weapon. You can strike opponents 10 feet away with it, but you can't use it against an adjacent foe.

\textbf{Guisarme:} A guisarme has reach. The guisarmes reach property can only be used when it is a two-handed weapon. You can strike opponents 10 feet away with it, but you can't use it against an adjacent foe. You can also use it to make trip attacks. If you are tripped during your own trip attempt, you can drop the guisarme to avoid being tripped.

\textbf{Halberd:} If you use a ready action to set a halberd against a charge, you deal double damage on a successful hit against a charging character. You can use a halberd to make trip attacks. If you are tripped during your own trip attempt, you can drop the halberd to avoid being tripped.

\textbf{Hook-Hammer:} A hook-hammer is a double weapon. You can fight with it as if fighting with two weapons, but if you do, you incur all the normal attack penalties associated with fighting with two weapons, just as if you were using a one-handed weapon and a light weapon. On a medium sized hook-hammer the hammer's blunt head is a bludgeoning weapon that deals 1d6 points of damage (crit ×3) and its hook is a piercing weapon that deals 1d4 points of damage (crit ×4). You can use either head as the primary weapon. The other head is the offhand weapon. A creature wielding a gnome hook-hammer in one hand can't use it as a double weapon-only one end of the weapon can be used in any given round. You can use a hook-hammer to make trip attacks. If you are tripped during your own trip attempt, you can drop the gnome hooked hammer to avoid being tripped. Gnomes treat hook-hammers as martial weapons.

\textbf{Kusarigama:} A kusarigama has reach, so you can strike opponents 10 feet away with it. The kusarigamas reach property can only be used when it is wielded in two hands (though not necessarily a two-handed weapon). In addition, unlike most other weapons with reach, it can be used against an adjacent foe. You can make trip attacks with the chain. If you are tripped during your own trip attempt, you can drop the chain to avoid being tripped. When using a spiked chain, you get a +2 bonus on opposed attack rolls made to disarm an opponent (including the roll to avoid being disarmed if such an attempt fails).

\textbf{Longspear:} A longspear has reach. The longspears reach property can only be used when it is a two-handed weapon. You can strike opponents 10 feet away with it, but you can't use it against an adjacent foe. If you use a ready action to set a longspear against a charge, you deal double damage on a successful hit against a charging character. While mounted, you can wield a lance with one hand. A longspear couched in a military saddle deals double damage on a charge.

\textbf{Net:} A net is a reach weapon used to entangle enemies. Unlike other reach weapons, a net the same size as you can be used with one hand. When you use a net, you make a ranged touch attack against your target. If you hit, the target is entangled. An entangled creature takes a -2 penalty on attack rolls and a -4 penalty on Dexterity, can move at only half speed, and cannot charge or run. If you control the trailing rope by succeeding on an opposed Strength check while holding it, the entangled creature can move only within the limits that the rope allows. If the entangled creature attempts to cast a spell, it must make a DC 15 Concentration check or be unable to cast the spell. An entangled creature can escape with a DC 20 Escape Artist check (a full-round action). The net has 5 hit points and can be burst with a DC 25 Strength check (also a full-round action). A net is useful only against creatures within one size category of you. A net must be folded to be thrown effectively. The first time you throw your net in a fight, you make a normal ranged touch attack roll. After the net is unfolded, you take a -4 penalty on attack rolls with it. It takes 2 rounds for a proficient user to fold a net and twice that long for a nonproficient one to do so.

\textbf{Nunchaku:} The nunchaku is a special monk weapon. This designation gives a monk wielding a nunchaku special options. With a nunchaku, you get a +2 bonus on attack rolls made to disarm an enemy. Nunchakus only count as monk weapons if they are light.

\textbf{Ranseur:} A ranseur has reach. The ranseurs reach property can only be used when it is a two-handed weapon. You can strike opponents 10 feet away with it, but you can't use it against an adjacent foe. With a ranseur, you get a +2 bonus on attack rolls made to disarm an opponent.

\textbf{Repeating Crossbow:} The repeating crossbow holds 5 crossbow bolts. As long as it holds bolts, you can reload it by pulling the reloading lever (a free action). Loading a new case of 5 bolts is a full-round action that provokes attacks of opportunity. A repeating crossbow functions identically to a crossbow in all other ways.

\textbf{Sai:} With a sai, you get a +4 bonus on opposed attack rolls made to disarm an enemy. The sai is a special monk weapon. This designation gives a monk wielding a sai special options. Sais only count as monk weapons if they are light.

\textbf{Shield:} You can bash with a shield instead of using it for defense. Doing so incurs all the normal penalties for two weapon fighting. Great Shields are one size smaller than the size of creature it was designed for, normal shields are two sizes smaller.

\textbf{Shuriken:} A shuriken is a special monk weapon. This designation gives a monk wielding shuriken special options. A shuriken can't be used as a melee weapon. Although they are thrown weapons, shuriken are treated as ammunition for the purposes of drawing them as long as they are three size categories smaller than you.

\textbf{Siangham:} The siangham is a special monk weapon. This designation gives a monk wielding a siangham special options. Siangham must be light to be used as a monk weapon.

\textbf{Sickle:} A sickle can be used to make trip attacks. If you are tripped during your own trip attempt, you can drop the sickle to avoid being tripped.

\textbf{Sling:} Your Strength modifier applies to damage rolls when you use a sling, just as it does for thrown weapons. You can fire, but not load, a sling the same size as you with one hand. Loading a sling is a move action that requires two hands and provokes attacks of opportunity. You can hurl ordinary stones with a sling, but stones are not as dense or as round as bullets. Thus, such an attack deals damage as if the weapon were designed for a creature one size category smaller than you and you take a -1 penalty on attack rolls. The range given is for a sling of medium size, for every size larger or smaller than medium increase or decrease the range by 15 feet.

\textbf{Spear:} If you use a ready action to set a spear against a charge, you deal double damage on a successful hit against a charging character. A spear one size smaller than you can be used as a thrown weapon with a 20 foot range incriment.

\textbf{Spiked Armor:} You can outfit your armor with spikes, which can deal damage in a grapple or as a separate attack. The damage listed is for armor made for a creature of the given size. Spiked armor is a light weapon.

\textbf{Spiked Chain:} A spiked chain has reach, so you can strike opponents 10 feet away with it. The spiked chains reach property can only be used when it is wielded in two hands (though not necessarily a two-handed weapon). In addition, unlike most other weapons with reach, it can be used against an adjacent foe. You can make trip attacks with the chain. If you are tripped during your own trip attempt, you can drop the chain to avoid being tripped. When using a spiked chain, you get a +2 bonus on opposed attack rolls made to disarm an opponent (including the roll to avoid being disarmed if such an attempt fails).

\textbf{Spiked Gauntlet:} Your opponent cannot use a disarm action to disarm you of spiked gauntlets. An attack with a spiked gauntlet is considered an armed attack. The damage listings given are for a spiked gauntlet made for a creature of the indicated size, instead of fo a spiked gauntlet of the indicated size. You may not wear gauntlets made for a creature of a different size than you.

\textbf{Spiked Shield:} You can bash with a spiked shield instead of using it for defense. If you use a ready action to set a spear against a charge, you deal double damage on a successful hit against a charging character.

\textbf{Staff:} A staff is a double weapon. You can fight with it as if fighting with two weapons, but if you do, you incur all the normal attack penalties associated with fighting with two weapons, just as if you were using a one-handed weapon and a light weapon. A creature wielding a quarterstaff in one hand can't use it as a double weapon-only one end of the weapon can be used in any given round. The quarterstaff is a special monk weapon. This designation gives a monk wielding a staff special options.

\textbf{Trident:} This weapon can be thrown as long as it is one size category smaller than you. If you use a ready action to set a trident against a charge, you deal double damage on a successful hit against a charging character.

\textbf{Unarmed Strike:} The damage listed for each size of unarmed strike is the size of the creature using unarmed strike. You can deal leathal or non-leathal damage at your option with an unarmed strike. The damage from an unarmed strike is considered weapon damage for the purposes of effects that give you a bonus on weapon damage rolls. An unarmed strike is always considered a light weapon.

\textbf{Urgrosh:} An urgrosh is a double weapon. You can fight with it as if fighting with two weapons, but if you do, you incur all the normal attack penalties associated with fighting with two weapons, just as if you were using a one-handed weapon and a light weapon. The urgrosh's axe head is a slashing weapon that deals 1d8 points of damage. Its spear head is a piercing weapon that deals 1d6 points of damage. You can use either head as the primary weapon. The other is the off-hand weapon. A creature wielding an urgrosh in one hand can't use it as a double weapon-only one end of the weapon can be used in any given round. If you use a ready action to set an urgrosh against a charge, you deal double damage if you score a hit against a charging character. If you use an urgrosh against a charging character, the spear head is the part of the weapon that deals damage. Dwarves treat urgroshes as martial weapons.

\textbf{Warhammer:} A warhammer has two sides that can be used interchangably. One side deals bludgeoning and has a critical range of 20x3, the other deals piercing damage and has a critical range of 20x4. As a medium weapon the hammer side deals 2d6 damage and the pick side deals 1d8 damage. You can choose which side you make an attack with at the beginning of each attack. It is not a double weapon, and cannot be weilded as one. Enhancements to the weapon effect both sides.

\textbf{Whip:} A whip has a 15 foot reach and can be used to attack any creature within range, including adjacent foes. The whips reach property can only be used when it is a one-handed or light weapon. A whip deals nonlethal damage. It also deals no damage to any creature with an armor bonus of +1 or higher, or a natural armor bonus of +3 or higher. Using a whip provokes an attack of opportunity as if you had used a ranged weapon. You cannot use a whip as a two-handed weapon. You can make trip attacks with a whip. If you are tripped during your own trip attempt, you can drop the whip to avoid being tripped. When using a whip, you get a +2 bonus on opposed attack rolls made to disarm an opponent.

\subsection{Masterwork Weapons}

A masterwork weapon is a finely crafted version of a normal weapon. Wielding it 
provides a +1 enhancement bonus on attack rolls.

You can't add the masterwork quality to a weapon after it is created; it must be 
crafted as a masterwork weapon (see the Craft skill). The masterwork quality adds 
300 gp to the cost of a normal weapon (or 6 gp to the cost of a single unit of 
ammunition). Adding the masterwork quality to a double weapon costs twice the normal 
increase (+600 gp).

Masterwork ammunition is damaged (effectively destroyed) when used. The enhancement 
bonus of masterwork ammunition does not stack with any enhancement bonus of the 
projectile weapon firing it.

All magic weapons are automatically considered to be of masterwork quality. The 
enhancement bonus granted by the masterwork quality doesn't stack with the enhancement 
bonus provided by the weapon's magic.

Even though some types of armor and shields can be used as weapons, you can't create 
a masterwork version of such an item that confers an enhancement bonus on attack 
rolls. Instead, masterwork armor and shields have lessened armor check penalties.
\section{Gear}
foo
\section{Animals}
foo
\section{Services}
foo

\chapter{Description}
\section{Physical Appearance}
foo
\section{Personality}
foo
\section{Alignment}
foo
\section{Religion}
foo

\chapter{Adventuring}
\section{Overland Travel}
foo
\section{Exploration}
foo
\section{Traps}
foo
\section{Encounters}
foo

\chapter{Combat}
\input{phb/combat/combat}
%%%%%%%%%%%%%%%%%%%%%%%%%%%%%%%%%%%%%%%%%%%%%%%%%%
\section{Movement, Position, and Distance}
%%%%%%%%%%%%%%%%%%%%%%%%%%%%%%%%%%%%%%%%%%%%%%%%%%

Miniatures are on the 30mm scale -- a miniature figure of a six-foot-tall human 
is approximately 30mm tall. A square on the battle grid is 1 inch across, representing 
a 5-foot-by-5-foot area.

%%%%%%%%%%%%%%%%%%%%%%%%%
\subsection{Tactical Movement In Combat}\index{Movement}
%%%%%%%%%%%%%%%%%%%%%%%%%

%%%
\subsubsection{How Far Can Your Character Move?}
%%%

Your speed is determined by your race and your armor (see Table: Tactical Speed). 
Your speed while unarmored is your base land speed.

\textbf{\gameterm{Encumbrance}:} A character encumbered by carrying a large amount of gear, 
treasure, or fallen comrades may move slower than normal.

\textbf{\gameterm{Hampered Movement}:} Difficult terrain, obstacles, or poor visibility can 
hamper movement.

\textbf{Movement in Combat:} Generally, you can move your speed in a round and 
still do something (take a move action and a standard action).

If you do nothing but move (that is, if you use both of your actions in a round 
to move your speed), you can move double your speed.

If you spend the entire round running, you can move quadruple your speed. If you 
do something that requires a full round you can only take a 5-foot step.

\textbf{Bonuses to Speed:} A barbarian has a +10 foot bonus to his speed (unless 
he's wearing heavy armor). Experienced monks also have higher speed (unless they're 
wearing armor of any sort). In addition, many spells and magic items can affect 
a character's speed. Always apply any modifiers to a character's speed before adjusting 
the character's speed based on armor or encumbrance, and remember that multiple 
bonuses of the same type to a character's speed don't stack.

\begin{table}[htb]
\rowcolors{1}{white}{offyellow}
\caption{Tactical Speed}
\centering
\begin{tabular}{l l l}
\textbf{Race} & \textbf{No Armor or Light Armor} & \textbf{Medium or Heavy Armor}\\
Human, Elf, Half-Elf, Half-Orc & 30ft (6 squares) & 20ft (4 squares)\\
Dwarf & 20ft (4 squares) & 20ft (4 squares)\\
Halfling & 20ft (4 squares) & 15ft (3 squares)\\
\end{tabular}
\end{table}

%%%
\subsubsection{Measuring Distance}
%%%

\textbf{Diagonals:}\index{Movement!Diagonals} When measuring distance, the first diagonal counts as 1 square, 
the second counts as 2 squares, the third counts as 1, the fourth as 2, and so 
on.

You can't move diagonally past a corner (even by taking a 5-foot step). You can 
move diagonally past a creature, even an opponent.

You can also move diagonally past other impassable obstacles, such as pits.

\textbf{Closest Creature:} When it's important to determine the closest square 
or creature to a location, if two squares or creatures are equally close, randomly 
determine which one counts as closest by rolling a die.

%%%
\subsubsection{Moving through a Square}
%%%

\textbf{Friend:} You can move through a square occupied by a friendly character, 
unless you are charging. When you move through a square occupied by a friendly 
character, that character doesn't provide you with cover.

\textbf{Opponent:} You can't move through a square occupied by an opponent, unless 
the opponent is helpless. You can move through a square occupied by a helpless 
opponent without penalty. (Some creatures, particularly very large ones, may present 
an obstacle even when helpless. In such cases, each square you move through counts 
as 2 squares.)

\textbf{Ending Your Movement:} You can't end your movement in the same square as 
another creature unless it is helpless.

\textbf{Overrun:} During your movement you can attempt to move through a square 
occupied by an opponent.

\textbf{Tumbling:} A trained character can attempt to tumble through a square occupied 
by an opponent (see the Tumble skill).

\textbf{Very Small Creature:} A Fine, Diminutive, or Tiny creature can move into 
or through an occupied square. The creature provokes attacks of opportunity when 
doing so.

\textbf{Square Occupied by Creature Three Sizes Larger or Smaller:} Any creature 
can move through a square occupied by a creature three size categories larger than 
it is.

A big creature can move through a square occupied by a creature three size categories 
smaller than it is.

\textbf{Designated Exceptions:} Some creatures break the above rules. A creature 
that completely fills the squares it occupies cannot be moved past, even with the 
\linkskill{Tumble} skill or similar special abilities.

%%%
\subsubsection{Terrain and Obstacles}
%%%

\textbf{\gameterm{Difficult Terrain}:} Difficult terrain hampers movement. Each square of 
difficult terrain counts as 2 squares of movement. (Each diagonal move into a difficult 
terrain square counts as 3 squares.) You can't run or charge across difficult terrain.

If you occupy squares with different kinds of terrain, you can move only as fast 
as the most difficult terrain you occupy will allow.

Flying and incorporeal creatures are not hampered by difficult terrain.

\textbf{Obstacles:} Like difficult terrain, obstacles can hamper movement. If an 
obstacle hampers movement but doesn't completely block it each obstructed square 
or obstacle between squares counts as 2 squares of movement. You must pay this 
cost to cross the barrier, in addition to the cost to move into the square on the 
other side. If you don't have sufficient movement to cross the barrier and move 
into the square on the other side, you can't cross the barrier. Some obstacles 
may also require a skill check to cross.

On the other hand, some obstacles block movement entirely. A character can't move 
through a blocking obstacle.

Flying and incorporeal creatures can avoid most obstacles

\textbf{\gameterm{Squeezing}:} In some cases, you may have to squeeze into or through an area 
that isn't as wide as the space you take up. You can squeeze through or into a 
space that is at least half as wide as your normal space. Each move into or through 
a narrow space counts as if it were 2 squares, and while squeezed in a narrow space 
you take a -4 penalty on attack rolls and a -4 penalty to AC.

When a Large creature (which normally takes up four squares) squeezes into a space 
that's one square wide, the creature's miniature figure occupies two squares, centered 
on the line between the two squares. For a bigger creature, center the creature 
likewise in the area it squeezes into.

A creature can squeeze past an opponent while moving but it can't end its movement 
in an occupied square.

To squeeze through or into a space less than half your space's width, you must 
use the Escape Artist skill. You can't attack while using Escape Artist to squeeze 
through or into a narrow space, you take a -4 penalty to AC, and you lose any Dexterity 
bonus to AC.

%%%
\subsubsection{Special Movement Rules}
%%%

These rules cover special movement situations.

\textbf{Accidentally Ending Movement in an Illegal Space:} Sometimes a character 
ends its movement while moving through a space where it's not allowed to stop. 
When that happens, put your miniature in the last legal position you occupied, 
or the closest legal position, if there's a legal position that's closer.

\textbf{Double Movement Cost:} When your movement is hampered in some way, your 
movement usually costs double. For example, each square of movement through difficult 
terrain counts as 2 squares, and each diagonal move through such terrain counts 
as 3 squares (just as two diagonal moves normally do).

If movement cost is doubled twice, then each square counts as 4 squares (or as 
6 squares if moving diagonally). If movement cost is doubled three times, then 
each square counts as 8 squares (12 if diagonal) and so on. This is an exception 
to the general rule that two doublings are equivalent to a tripling.

\textbf{Minimum Movement:} Despite penalties to movement, you can take a full-round 
action to move 5 feet (1 square) in any direction, even diagonally. (This rule 
doesn't allow you to move through impassable terrain or to move when all movement 
is prohibited.) Such movement provokes attacks of opportunity as normal (despite 
the distance covered, this move isn't a 5-foot step).

%%%%%%%%%%%%%%%%%%%%%%%%%
\subsection{Big And Little Creatures In Combat}
%%%%%%%%%%%%%%%%%%%%%%%%%

Creatures smaller than Small or larger than Medium have special rules relating 
to position. 

\textbf{Tiny, Diminutive, and Fine Creatures:} Very small creatures take up less 
than 1 square of space. This means that more than one such creature can fit into 
a single square. A Tiny creature typically occupies a space only 2-1/2 feet across, 
so four can fit into a single square. Twenty-five Diminutive creatures or 100 Fine 
creatures can fit into a single square. Creatures that take up less than 1 square 
of space typically have a natural reach of 0 feet, meaning they can't reach into 
adjacent squares. They must enter an opponent's square to attack in melee. This 
provokes an attack of opportunity from the opponent. You can attack into your own 
square if you need to, so you can attack such creatures normally. Since they have 
no natural reach, they do not threaten the squares around them. You can move past 
them without provoking attacks of opportunity. They also can't flank an enemy.

\textbf{Large, Huge, Gargantuan, and Colossal Creatures:} Very large creatures 
take up more than 1 square.

Creatures that take up more than 1 square typically have a natural reach of 10 
feet or more, meaning that they can reach targets even if they aren't in adjacent 
squares.

Unlike when someone uses a reach weapon, a creature with greater than normal natural 
reach (more than 5 feet) still threatens squares adjacent to it. A creature with 
greater than normal natural reach usually gets an attack of opportunity against 
you if you approach it, because you must enter and move within the range of its 
reach before you can attack it. (This attack of opportunity is not provoked if 
you take a 5-foot step.)

Large or larger creatures using reach weapons can strike up to double their natural 
reach but can't strike at their natural reach or less. 

\begin{table}[htb]
\rowcolors{1}{white}{offyellow}
\caption{Creature Size and Scale}
\centering
\begin{tabular}{lcc}
\textbf{Creature Size} & \textbf{Space\textsuperscript{1}} & \textbf{Natural Reach\textsuperscript{1}}\\
Fine & 1/2ft & 0ft\\
Diminutive & 1ft & 0ft\\
Tiny & 2.5ft & 0ft\\
Small & 5ft & 5ft\\
Medium & 5ft & 5ft\\
Large (long) & 10ft & 5ft\\
Large (tall) & 10ft & 10ft\\
Huge (long) & 15ft & 10ft\\
Huge (tall) & 15ft & 15ft\\
Gargantuan (long) & 20ft & 15ft\\
Gargantuan (tall) & 20ft & 20ft\\
Colossal (long) & 30ft & 20ft\\
Colossal (tall) & 30ft & 30ft\\
\multicolumn{3}{p{6cm}}{\textsuperscript{1} These values are typical for creatures of the indicated size. Some exceptions exist.}\\
\end{tabular}
\end{table}

%%%%%%%%%%%%%%%%%%%%%%%%%%%%%%%%%%%%%%%%%%%%%%%%%%
\section{Combat Modifiers}
%%%%%%%%%%%%%%%%%%%%%%%%%%%%%%%%%%%%%%%%%%%%%%%%%%

%%%%%%%%%%%%%%%%%%%%%%%%%
\subsection{Favorable And Unfavorable Conditions}
%%%%%%%%%%%%%%%%%%%%%%%%%

\begin{table}[htb]
\rowcolors{1}{white}{offyellow}
\caption{Attack Roll Modifiers}
\centering
\begin{tabular}{p{7cm}cc}
\textbf{Attacker is \ldots{}} & \textbf{Melee} & \textbf{Ranged}\\
Dazzled & -1 & -1\\
Entangled & -2\textsuperscript{1} & -2\textsuperscript{1}\\
Flanking defender & +2 & --\\
Invisible & +2\textsuperscript{2} & +2\textsuperscript{2}\\
On higher ground & +1 & +0\\
Prone & -4 & --\textsuperscript{3}\\
Shaken or frightened & -2 & -2\\
Squeezing through a space & -4 & -4\\
\multicolumn{3}{p{10cm}}{\textsuperscript{1} An entangled character also takes a -4 penalty to Dexterity, which may affect his attack roll.}\\
\multicolumn{3}{p{10cm}}{\textsuperscript{2} The defender loses any Dexterity bonus to AC. This bonus doesn't apply if the target is blinded.}\\
\multicolumn{3}{p{10cm}}{\textsuperscript{3} Most ranged weapons can't be used while the attacker is prone, but you can use a crossbow or shuriken while prone at no penalty.}\\
\end{tabular}
\end{table}

%perhaps convert to tablularx
\begin{table}[htb]
\rowcolors{1}{white}{offyellow}\mcinherit
\caption{Armor Class Modifiers}
\centering
\begin{tabular}{p{8.5cm}cc}
\textbf{Defender is \ldots{}} & \textbf{Melee} & \textbf{Ranged}\\
Behind cover & +4 & +4\\
Blinded & -2\textsuperscript{1} & -2\textsuperscript{1}\\
Concealed or invisible & \multicolumn{2}{c}{-- See Concealment --}\\
Cowering & -2\textsuperscript{1} & -2\textsuperscript{1}\\
Entangled & +0\textsuperscript{2} & +0\textsuperscript{2}\\
Flat-footed (such as surprised, balancing, climbing) & +0\textsuperscript{1} & +0\textsuperscript{1}\\
Grappling (but attacker is not) & +0\textsuperscript{1} & +0\textsuperscript{1,3}\\
Helpless (such as paralyzed, sleeping, or bound) & -4\textsuperscript{4} & +0\textsuperscript{4}\\
Kneeling or sitting & -2 & +2\\
Pinned & -4\textsuperscript{4} & +0\textsuperscript{4}\\
Prone & -4 & +4\\
Squeezing through a space & -4 & -4\\
Stunned & -2\textsuperscript{1} & -2\textsuperscript{1}\\
\multicolumn{3}{p{12cm}}{\textsuperscript{1} The defender loses any Dexterity bonus to AC.}\\
\multicolumn{3}{p{12cm}}{\textsuperscript{2} An entangled character takes a -4 penalty to Dexterity.}\\
\multicolumn{3}{p{12cm}}{\textsuperscript{3} Roll randomly to see which grappling combatant you strike. That defender loses any Dexterity bonus to AC.}\\
\multicolumn{3}{p{12cm}}{\textsuperscript{4} Treat the defender's Dexterity as 0 (-5 modifier). Rogues can sneak attack helpless or pinned defenders.}\\
\end{tabular}
\end{table}

%%%%%%%%%%%%%%%%%%%%%%%%%
\subsection{Cover}\index{Cover}
%%%%%%%%%%%%%%%%%%%%%%%%%

To determine whether your target has cover from your ranged attack, choose a corner 
of your square. If any line from this corner to any corner of the target's square 
passes through a square or border that blocks line of effect or provides cover, 
or through a square occupied by a creature, the target has cover (+4 to AC).

When making a melee attack against an adjacent target, your target has cover if 
any line from your square to the target's square goes through a wall (including 
a low wall). When making a melee attack against a target that isn't adjacent to 
you (such as with a reach weapon), use the rules for determining cover from ranged 
attacks.

\textbf{Low Obstacles and Cover:} A low obstacle (such as a wall no higher than 
half your height) provides cover, but only to creatures within 30 feet (6 squares) 
of it. The attacker can ignore the cover if he's closer to the obstacle than his 
target.

\textbf{Cover and Attacks of Opportunity:} You can't execute an attack of opportunity 
against an opponent with cover relative to you.

\textbf{Cover and Reflex Saves:} Cover grants you a +2 bonus on Reflex saves against 
attacks that originate or burst out from a point on the other side of the cover 
from you. Note that spread effects can extend around corners and thus negate this 
cover bonus.

\textbf{Cover and Hide Checks:} You can use cover to make a Hide check. Without 
cover, you usually need concealment (see below) to make a Hide check.

\textbf{Soft Cover:} Creatures, even your enemies, can provide you with cover against 
ranged attacks, giving you a +4 bonus to AC. However, such soft cover provides 
no bonus on Reflex saves, nor does soft cover allow you to make a Hide check.

\textbf{Big Creatures and Cover:} Any creature with a space larger than 5 feet 
(1 square) determines cover against melee attacks slightly differently than smaller 
creatures do. Such a creature can choose any square that it occupies to determine 
if an opponent has cover against its melee attacks. Similarly, when making a melee 
attack against such a creature, you can pick any of the squares it occupies to 
determine if it has cover against you.

\textbf{Total Cover:} If you don't have line of effect to your target he is considered 
to have total cover from you. You can't make an attack against a target that has 
total cover.

\textbf{Varying Degrees of Cover:} In some cases, cover may provide a greater bonus 
to AC and Reflex saves. In such situations the normal cover bonuses to AC and Reflex 
saves can be doubled (to +8 and +4, respectively). A creature with this improved 
cover effectively gains improved evasion against any attack to which the Reflex 
save bonus applies. Furthermore, improved cover provides a +10 bonus on Hide checks.

%%%%%%%%%%%%%%%%%%%%%%%%%
\subsection{Concealment}\index{Concealment}
%%%%%%%%%%%%%%%%%%%%%%%%%

To determine whether your target has concealment from your ranged attack, choose 
a corner of your square. If any line from this corner to any corner of the target's 
square passes through a square or border that provides concealment, the target 
has concealment.

When making a melee attack against an adjacent target, your target has concealment 
if his space is entirely within an effect that grants concealment. When making 
a melee attack against a target that isn't adjacent to you use the rules for determining 
concealment from ranged attacks.

In addition, some magical effects provide concealment against all attacks, regardless 
of whether any intervening concealment exists.

\textbf{Concealment Miss Chance:} Concealment gives the subject of a successful 
attack a 20\% chance that the attacker missed because of the concealment. If the 
attacker hits, the defender must make a miss chance percentile roll to avoid being 
struck. Multiple concealment conditions do not stack.

\textbf{Concealment and Hide Checks:} You can use concealment to make a Hide check. 
Without concealment, you usually need cover to make a Hide check.

\textbf{Total Concealment:} If you have line of effect to a target but not line 
of sight he is considered to have total concealment from you. You can't attack 
an opponent that has total concealment, though you can attack into a square that 
you think he occupies. A successful attack into a square occupied by an enemy with 
total concealment has a 50\% miss chance (instead of the normal 20\% miss chance 
for an opponent with concealment).

You can't execute an attack of opportunity against an opponent with total concealment, 
even if you know what square or squares the opponent occupies.

\textbf{Ignoring Concealment:} Concealment isn't always effective. A shadowy area 
or darkness doesn't provide any concealment against an opponent with darkvision. 
Characters with low-light vision can see clearly for a greater distance with the 
same light source than other characters. Although invisibility provides total concealment, 
sighted opponents may still make Spot checks to notice the location of an invisible 
character. An invisible character gains a +20 bonus on Hide checks if moving, or 
a +40 bonus on Hide checks when not moving (even though opponents can't see you, 
they might be able to figure out where you are from other visual clues).

\textbf{Varying Degrees of Concealment:} Certain situations may provide more or 
less than typical concealment, and modify the miss chance accordingly.

%%%%%%%%%%%%%%%%%%%%%%%%%
\subsection{Flanking}\index{Flanking}
%%%%%%%%%%%%%%%%%%%%%%%%%

When making a melee attack, you get a +2 flanking bonus if your opponent is threatened 
by a character or creature friendly to you on the opponent's opposite border or 
opposite corner.

When in doubt about whether two friendly characters flank an opponent in the middle, 
trace an imaginary line between the two friendly characters' centers. If the line 
passes through opposite borders of the opponent's space (including corners of those 
borders), then the opponent is flanked.

\textit{Exception:} If a flanker takes up more than 1 square, it gets the flanking 
bonus if any square it occupies counts for flanking.

Only a creature or character that threatens the defender can help an attacker get 
a flanking bonus.

Creatures with a reach of 0 feet can't flank an opponent.

%%%%%%%%%%%%%%%%%%%%%%%%%
\subsection{Helpless Defenders}\index{Helpless}
%%%%%%%%%%%%%%%%%%%%%%%%%

A helpless opponent is someone who is bound, sleeping, paralyzed, unconscious, 
or otherwise at your mercy.

\textbf{Regular Attack:} A helpless character takes a -4 penalty to AC against 
melee attacks, but no penalty to AC against ranged attacks.

A helpless defender can't use any Dexterity bonus to AC. In fact, his Dexterity 
score is treated as if it were 0 and his Dexterity modifier to AC as if it were 
-5 (and a rogue can sneak attack him).

\textbf{\gameterm{Coup de Grace}:} As a full-round action, you can use a melee weapon to deliver 
a coup de grace to a helpless opponent. You can also use a bow or crossbow, provided 
you are adjacent to the target.

You automatically hit and score a critical hit. If the defender survives the damage, 
he must make a Fortitude save (DC 10 + damage dealt) or die. A rogue also gets 
her extra sneak attack damage against a helpless opponent when delivering a coup 
de grace.

Delivering a coup de grace provokes attacks of opportunity from threatening opponents.

You can't deliver a coup de grace against a creature that is immune to critical 
hits. You can deliver a coup de grace against a creature with total concealment, 
but doing this requires two consecutive full-round actions (one to "find" the 
creature once you've determined what square it's in, and one to deliver the coup 
de grace).

%%%%%%%%%%%%%%%%%%%%%%%%%%%%%%%%%%%%%%%%%%%%%%%%%%
\section{Special Attacks}
%%%%%%%%%%%%%%%%%%%%%%%%%%%%%%%%%%%%%%%%%%%%%%%%%%

\begin{table}[htb]
\rowcolors{1}{white}{offyellow}\mcinherit
\caption{Special Attacks}
\centering
\begin{tabular}{ll}
\textbf{Special Attack} & \textbf{Brief Description}\\
Aid Another & Grant an ally a +2 bonus on attacks or AC\\
Bull Rush & Push an opponent back 5 feet or more\\
Charge & Move up to twice your speed and attack with +2 bonus\\
Disarm & Knock a weapon from your opponent's hands\\
Feint & Negate your opponent's Dex bonus to AC\\
Grapple & Wrestle with an opponent\\
Overrun & Plow past or over an opponent as you move\\
Sunder & Strike an opponent's weapon or shield\\
Throw splash weapon & Throw container of dangerous liquid at target\\
Trip & Trip an opponent\\
Turn (rebuke) undead & Channel positive (or negative) energy to turn away (or awe) undead\\
Two-weapon Fighting & Fight with a weapon in each hand\\
\end{tabular}
\end{table}

%%%%%%%%%%%%%%%%%%%%%%%%%
\subsection{Aid Another}\index{Aid Another}
%%%%%%%%%%%%%%%%%%%%%%%%%

In melee combat, you can help a friend attack or defend by distracting or interfering 
with an opponent. If you're in position to make a melee attack on an opponent that 
is engaging a friend in melee combat, you can attempt to aid your friend as a standard 
action. You make an attack roll against AC 10. If you succeed, your friend gains 
either a +2 bonus on his next attack roll against that opponent or a +2 bonus to 
AC against that opponent's next attack (your choice), as long as that attack comes 
before the beginning of your next turn. Multiple characters can aid the same friend, 
and similar bonuses stack.

You can also use this standard action to help a friend in other ways, such as when 
he is affected by a spell, or to assist another character's skill check.

%%%%%%%%%%%%%%%%%%%%%%%%%
\subsection{Bull Rush}\index{Bull Rush}
%%%%%%%%%%%%%%%%%%%%%%%%%

You can make a bull rush as a standard action (an attack) or as part of a charge 
(see Charge, below). When you make a bull rush, you attempt to push an opponent 
straight back instead of damaging him. You can only bull rush an opponent who is 
one size category larger than you, the same size, or smaller.

\textbf{Initiating a Bull Rush:} First, you move into the defender's space. Doing 
this provokes an attack of opportunity from each opponent that threatens you, including 
the defender. (If you have the Improved Bull Rush feat, you don't provoke an attack 
of opportunity from the defender.) Any attack of opportunity made by anyone other 
than the defender against you during a bull rush has a 25\% chance of accidentally 
targeting the defender instead, and any attack of opportunity by anyone other than 
you against the defender likewise has a 25\% chance of accidentally targeting you. 
(When someone makes an attack of opportunity, make the attack roll and then roll 
to see whether the attack went astray.) 

Second, you and the defender make opposed Strength checks. You each add a +4 bonus 
for each size category you are larger than Medium or a -4 penalty for each size 
category you are smaller than Medium. You get a +2 bonus if you are charging. The 
defender gets a +4 bonus if he has more than two legs or is otherwise exceptionally 
stable.

\textbf{Bull Rush Results:} If you beat the defender's Strength check result, you 
push him back 5 feet. If you wish to move with the defender, you can push him back 
an additional 5 feet for each 5 points by which your check result is greater than 
the defender's check result. You can't, however, exceed your normal movement limit. 
(\textit{Note:} The defender provokes attacks of opportunity if he is moved. So 
do you, if you move with him. The two of you do not provoke attacks of opportunity 
from each other, however.)

If you fail to beat the defender's Strength check result, you move 5 feet straight 
back to where you were before you moved into his space. If that space is occupied, 
you fall prone in that space.

%%%%%%%%%%%%%%%%%%%%%%%%%
\subsection{Charge}\index{Charge}
%%%%%%%%%%%%%%%%%%%%%%%%%

Charging is a special full-round action that allows you to move up to twice your 
speed and attack during the action. However, it carries tight restrictions on how 
you can move.

\textbf{Movement During a Charge:} You must move before your attack, not after. 
You must move at least 10 feet (2 squares) and may move up to double your speed 
directly toward the designated opponent.

You must have a clear path toward the opponent, and nothing can hinder your movement 
(such as difficult terrain or obstacles). Here's what it means to have a clear 
path. First, you must move to the closest space from which you can attack the opponent. 
(If this space is occupied or otherwise blocked, you can't charge.) Second, if 
any line from your starting space to the ending space passes through a square that 
blocks movement, slows movement, or contains a creature (even an ally), you can't 
charge. (Helpless creatures don't stop a charge.)

If you don't have line of sight to the opponent at the start of your turn, you 
can't charge that opponent.

You can't take a 5-foot step in the same round as a charge.

If you are able to take only a standard action or a move action on your turn, you 
can still charge, but you are only allowed to move up to your speed (instead of 
up to double your speed). You can't use this option unless you are restricted to 
taking only a standard action or move action on your turn.

\textbf{Attacking on a Charge:} After moving, you may make a single melee attack. 
You get a +2 bonus on the attack roll. and take a -2 penalty to your AC until the 
start of your next turn.

A charging character gets a +2 bonus on the Strength check made to bull rush an 
opponent (see Bull Rush, above).

Even if you have extra attacks, such as from having a high enough base attack bonus 
or from using multiple weapons, you only get to make one attack during a charge.

\textbf{Lances and Charge Attacks:} A lance deals double damage if employed by 
a mounted character in a charge.

\textbf{Weapons Readied against a Charge:} Spears, tridents, and certain other 
piercing weapons deal double damage when readied (set) and used against a charging 
character.

%%%%%%%%%%%%%%%%%%%%%%%%%
\subsection{Disarm}\index{Disarm}
%%%%%%%%%%%%%%%%%%%%%%%%%

As a melee attack, you may attempt to disarm your opponent. If you do so with a 
weapon, you knock the opponent's weapon out of his hands and to the ground. If 
you attempt the disarm while unarmed, you end up with the weapon in your hand.

If you're attempting to disarm a melee weapon, follow the steps outlined here. 
If the item you are attempting to disarm isn't a melee weapon the defender may 
still oppose you with an attack roll, but takes a penalty and can't attempt to 
disarm you in return if your attempt fails.

\textbf{Step 1:} Attack of Opportunity. You provoke an attack of opportunity from 
the target you are trying to disarm. (If you have the Improved Disarm feat, you 
don't incur an attack of opportunity for making a disarm attempt.) If the defender's 
attack of opportunity deals any damage, your disarm attempt fails.

\textbf{Step 2:} Opposed Rolls. You and the defender make opposed attack rolls 
with your respective weapons. The wielder of a two-handed weapon on a disarm attempt 
gets a +4 bonus on this roll, and the wielder of a light weapon takes a -4 penalty. 
(An unarmed strike is considered a light weapon, so you always take a penalty when 
trying to disarm an opponent by using an unarmed strike.) If the combatants are 
of different sizes, the larger combatant gets a bonus on the attack roll of +4 
per difference in size category. If the targeted item isn't a melee weapon, the 
defender takes a -4 penalty on the roll.

\textbf{Step Three:} Consequences. If you beat the defender, the defender is disarmed. 
If you attempted the disarm action unarmed, you now have the weapon. If you were 
armed, the defender's weapon is on the ground in the defender's square.

If you fail on the disarm attempt, the defender may immediately react and attempt 
to disarm you with the same sort of opposed melee attack roll. His attempt does 
not provoke an attack of opportunity from you. If he fails his disarm attempt, 
you do not subsequently get a free disarm attempt against him.

\textit{Note:} A defender wearing spiked gauntlets can't be disarmed. A defender 
using a weapon attached to a locked gauntlet gets a +10 bonus to resist being disarmed.

%%%
\subsubsection{Grabbing Items}
%%%

You can use a disarm action to snatch an item worn by the target. If you want to 
have the item in your hand, the disarm must be made as an unarmed attack.

If the item is poorly secured or otherwise easy to snatch or cut away the attacker 
gets a +4 bonus. Unlike on a normal disarm attempt, failing the attempt doesn't 
allow the defender to attempt to disarm you. This otherwise functions identically 
to a disarm attempt, as noted above.

You can't snatch an item that is well secured unless you have pinned the wearer 
(see Grapple). Even then, the defender gains a +4 bonus on his roll to resist the 
attempt.

%%%%%%%%%%%%%%%%%%%%%%%%%
\subsection{Feint}\index{Feint}
%%%%%%%%%%%%%%%%%%%%%%%%%

Feinting is a standard action. To feint, make a Bluff check opposed by a Sense 
Motive check by your target. The target may add his base attack bonus to this Sense 
Motive check. If your Bluff check result exceeds your target's Sense Motive check 
result, the next melee attack you make against the target does not allow him to 
use his Dexterity bonus to AC (if any). This attack must be made on or before your 
next turn.

When feinting in this way against a nonhumanoid you take a -4 penalty. Against 
a creature of animal Intelligence (1 or 2), you take a -8 penalty. Against a nonintelligent 
creature, it's impossible.

Feinting in combat does not provoke attacks of opportunity.

\textbf{Feinting as a Move Action:} With the Improved Feint feat, you can attempt 
a feint as a move action instead of as a standard action.

%%%%%%%%%%%%%%%%%%%%%%%%%
\subsection{Grapple}\index{Grapple}
%%%%%%%%%%%%%%%%%%%%%%%%%

%%%
\subsubsection{Grapple Checks}
%%%

Repeatedly in a grapple, you need to make opposed grapple checks against an opponent. 
A grapple check is like a melee attack roll. Your attack bonus on a grapple check 
is: Base attack bonus + Strength modifier + special size modifier

\textbf{Special Size Modifier:} The special size modifier for a grapple check is 
as follows: Colossal +16, Gargantuan +12, Huge +8, Large +4, Medium +0, Small -4, 
Tiny -8, Diminutive -12, Fine -16. Use this number in place of the normal size 
modifier you use when making an attack roll.

%%%
\subsubsection{Starting a Grapple}
%%%

To start a grapple, you need to grab and hold your target. Starting a grapple requires 
a successful melee attack roll. If you get multiple attacks, you can attempt to 
start a grapple multiple times (at successively lower base attack bonuses).

\textbf{Step 1:} Attack of Opportunity. You provoke an attack of opportunity from 
the target you are trying to grapple. If the attack of opportunity deals damage, 
the grapple attempt fails. (Certain monsters do not provoke attacks of opportunity 
when they attempt to grapple, nor do characters with the Improved Grapple feat.) 
If the attack of opportunity misses or fails to deal damage, proceed to Step 2.

\textbf{Step 2:} Grab. You make a melee touch attack to grab the target. If you 
fail to hit the target, the grapple attempt fails. If you succeed, proceed to Step 3.

\textbf{Step 3:} Hold. Make an opposed grapple check as a free action.

If you succeed, you and your target are now grappling, and you deal damage to the 
target as if with an unarmed strike.

If you lose, you fail to start the grapple. You automatically lose an attempt to 
hold if the target is two or more size categories larger than you are.

In case of a tie, the combatant with the higher grapple check modifier wins. If 
this is a tie, roll again to break the tie.

\textbf{Step 4:} Maintain Grapple. To maintain the grapple for later rounds, you 
must move into the target's space. (This movement is free and doesn't count as 
part of your movement in the round.)

Moving, as normal, provokes attacks of opportunity from threatening opponents, 
but not from your target.

If you can't move into your target's space, you can't maintain the grapple and 
must immediately let go of the target. To grapple again, you must begin at Step 1.

%%%
\subsubsection{Grappling Consequences}
%%%

While you're grappling, your ability to attack others and defend yourself is limited.

\textbf{No Threatened Squares:} You don't threaten any squares while grappling.

\textbf{No Dexterity Bonus:} You lose your Dexterity bonus to AC (if you have one) 
against opponents you aren't grappling. (You can still use it against opponents 
you are grappling.)

\textbf{No Movement:} You can't move normally while grappling. You may, however, 
make an opposed grapple check (see below) to move while grappling.

%%%
\subsubsection{If You're Grappling}
%%%

When you are grappling (regardless of who started the grapple), you can perform 
any of the following actions. Some of these actions take the place of an attack 
(rather than being a standard action or a move action). If your base attack bonus 
allows you multiple attacks, you can attempt one of these actions in place of each 
of your attacks, but at successively lower base attack bonuses.

\textbf{Activate a Magic Item:} You can activate a magic item, as long as the item 
doesn't require a spell completion trigger. You don't need to make a grapple check 
to activate the item.

\textbf{Attack Your Opponent:} You can make an attack with an unarmed strike, natural 
weapon, or light weapon against another character you are grappling. You take a 
-4 penalty on such attacks.

You can't attack with two weapons while grappling, even if both are light weapons.

\textbf{Cast a Spell:} You can attempt to cast a spell while grappling or even 
while pinned (see below), provided its casting time is no more than 1 standard 
action, it has no somatic component, and you have in hand any material components 
or focuses you might need. Any spell that requires precise and careful action\textit{ 
}is impossible to cast while grappling or being pinned. If the spell is one that 
you can cast while grappling, you must make a Concentration check (DC 20 + spell 
level) or lose the spell. You don't have to make a successful grapple check to 
cast the spell.

\textbf{Damage Your Opponent:} While grappling, you can deal damage to your opponent 
equivalent to an unarmed strike. Make an opposed grapple check in place of an attack. 
If you win, you deal nonlethal damage as normal for your unarmed strike (1d3 points 
for Medium attackers or 1d2 points for Small attackers, plus Strength modifiers). 
If you want to deal lethal damage, you take a -4 penalty on your grapple check.

\textit{Exception:} Monks deal more damage on an unarmed strike than other characters, 
and the damage is lethal. However, they can choose to deal their damage as nonlethal 
damage when grappling without taking the usual -4 penalty for changing lethal damage 
to nonlethal damage.

\textbf{Draw a Light Weapon:} You can draw a light weapon as a move action with 
a successful grapple check.

\textbf{Escape from Grapple:} You can escape a grapple by winning an opposed grapple 
check in place of making an attack. You can make an Escape Artist check in place 
of your grapple check if you so desire, but this requires a standard action. If 
more than one opponent is grappling you, your grapple check result has to beat 
all their individual check results to escape. (Opponents don't have to try to hold 
you if they don't want to.) If you escape, you finish the action by moving into 
any space adjacent to your opponent(s).

\textbf{Move:} You can move half your speed (bringing all others engaged in the 
grapple with you) by winning an opposed grapple check. This requires a standard 
action, and you must beat all the other individual check results to move the grapple.

\textit{Note:} You get a +4 bonus on your grapple check to move a pinned opponent, 
but only if no one else is involved in the grapple.

\textbf{Retrieve a Spell Component:} You can produce a spell component from your 
pouch while grappling by using a full-round action. Doing so does not require a 
successful grapple check.

\textbf{Pin Your Opponent:} You can hold your opponent immobile for 1 round by 
winning an opposed grapple check (made in place of an attack). Once you have an 
opponent pinned, you have a few options available to you (see below).

\textbf{Break Another's Pin:} If you are grappling an opponent who has another 
character pinned, you can make an opposed grapple check in place of an attack. 
If you win, you break the hold that the opponent has over the other character. 
The character is still grappling, but is no longer pinned.

\textbf{Use Opponent's Weapon:} If your opponent is holding a light weapon, you 
can use it to attack him. Make an opposed grapple check (in place of an attack). 
If you win, make an attack roll with the weapon with a -4 penalty (doing this doesn't 
require another action).

You don't gain possession of the weapon by performing this action.

%%%
\subsubsection{If You're Pinning an Opponent}
%%%

You can attempt to damage your opponent with an opposed grapple check, you can 
attempt to use your opponent's weapon against him, or you can attempt to move the 
grapple (all described above). At your option, you can prevent a pinned opponent 
from speaking.

You can use a disarm action to remove or grab away a well secured object worn by 
a pinned opponent, but he gets a +4 bonus on his roll to resist your attempt (see 
Disarm).

You may voluntarily release a pinned character as a free action; if you do so, 
you are no longer considered to be grappling that character (and vice versa).

You can't draw or use a weapon (against the pinned character or any other character), 
escape another's grapple, retrieve a spell component, pin another character, or 
break another's pin while you are pinning an opponent.

%%%
\subsubsection{If You're Pinned by an Opponent}
%%%

When an opponent has pinned you, you are held immobile (but not helpless) for 1 
round. While you're pinned, you take a -4 penalty to your AC against opponents 
other than the one pinning you. At your opponent's option, you may also be unable 
to speak. On your turn, you can try to escape the pin by making an opposed grapple 
check in place of an attack. You can make an Escape Artist check in place of your 
grapple check if you want, but this requires a standard action. If you win, you 
escape the pin, but you're still grappling.

%%%
\subsubsection{Joining a Grapple}
%%%

If your target is already grappling someone else, you can use an attack to start 
a grapple, as above, except that the target doesn't get an attack of opportunity 
against you, and your grab automatically succeeds. You still have to make a successful 
opposed grapple check to become part of the grapple.

If there are multiple opponents involved in the grapple, you pick one to make the 
opposed grapple check against.

%%%
\subsubsection{Multiple Grapplers}
%%%

Several combatants can be in a single grapple. Up to four combatants can grapple 
a single opponent in a given round. Creatures that are one or more size categories 
smaller than you count for half, creatures that are one size category larger than 
you count double, and creatures two or more size categories larger count quadruple.

When you are grappling with multiple opponents, you choose one opponent to make 
an opposed check against. The exception is an attempt to escape from the grapple; 
to successfully escape, your grapple check must beat the check results of each 
opponent.

%%%%%%%%%%%%%%%%%%%%%%%%%
\subsection{Mounted Combat}\index{Mounted Combat}
%%%%%%%%%%%%%%%%%%%%%%%%%

\textbf{Horses in Combat:} Warhorses and warponies can serve readily as combat 
steeds. Light horses, ponies, and heavy horses, however, are frightened by combat. 
If you don't dismount, you must make a DC 20 Ride check each round as a move action 
to control such a horse. If you succeed, you can perform a standard action after 
the move action. If you fail, the move action becomes a full round action and you 
can't do anything else until your next turn.

Your mount acts on your initiative count as you direct it. You move at its speed, 
but the mount uses its action to move.

A horse (not a pony) is a Large creature and thus takes up a space 10 feet (2 squares) 
across. For simplicity, assume that you share your mount's space during combat.

\textbf{Combat while Mounted:} With a DC 5 Ride check, you can guide your mount 
with your knees so as to use both hands to attack or defend yourself. This is a 
free action.

When you attack a creature smaller than your mount that is on foot, you get the 
+1 bonus on melee attacks for being on higher ground. If your mount moves more 
than 5 feet, you can only make a single melee attack. Essentially, you have to 
wait until the mount gets to your enemy before attacking, so you can't make a full 
attack. Even at your mount's full speed, you don't take any penalty on melee attacks 
while mounted.

If your mount charges, you also take the AC penalty associated with a charge. If 
you make an attack at the end of the charge, you receive the bonus gained from 
the charge. When charging on horseback, you deal double damage with a lance (see 
Charge).

You can use ranged weapons while your mount is taking a double move, but at a -4 
penalty on the attack roll. You can use ranged weapons while your mount is running 
(quadruple speed), at a -8 penalty. In either case, you make the attack roll when 
your mount has completed half its movement. You can make a full attack with a ranged 
weapon while your mount is moving. Likewise, you can take move actions normally

\textbf{Casting Spells while Mounted:} You can cast a spell normally if your mount 
moves up to a normal move (its speed) either before or after you cast. If you have 
your mount move both before and after you cast a spell, then you're casting the 
spell while the mount is moving, and you have to make a Concentration check due 
to the vigorous motion (DC 10 + spell level) or lose the spell. If the mount is 
running (quadruple speed), you can cast a spell when your mount has moved up to 
twice its speed, but your Concentration check is more difficult due to the violent 
motion (DC 15 + spell level).

\textbf{If Your Mount Falls in Battle:} If your mount falls, you have to succeed 
on a DC 15 Ride check to make a soft fall and take no damage. If the check fails, 
you take 1d6 points of damage.

\textbf{If You Are Dropped:} If you are knocked unconscious, you have a 50\% chance 
to stay in the saddle (or 75\% if you're in a military saddle). Otherwise you fall 
and take 1d6 points of damage.

Without you to guide it, your mount avoids combat.

%%%%%%%%%%%%%%%%%%%%%%%%%
\subsection{Overrun}\index{Overrun}
%%%%%%%%%%%%%%%%%%%%%%%%%

You can attempt an overrun as a standard action taken during your move. (In general, 
you cannot take a standard action during a move; this is an exception.) With an 
overrun, you attempt to plow past or over your opponent (and move through his square) 
as you move. You can only overrun an opponent who is one size category larger than 
you, the same size, or smaller. You can make only one overrun attempt per round.

If you're attempting to overrun an opponent, follow these steps.

\textbf{Step 1:} Attack of Opportunity. Since you begin the overrun by moving into 
the defender's space, you provoke an attack of opportunity from the defender.

\textbf{Step 2:} Opponent Avoids? The defender has the option to simply avoid you. 
If he avoids you, he doesn't suffer any ill effect and you may keep moving (You 
can always move through a square occupied by someone who lets you by.) The overrun 
attempt doesn't count against your actions this round (except for any movement 
required to enter the opponent's square). If your opponent doesn't avoid you, move 
to Step 3.

\textbf{Step 3:} Opponent Blocks? If your opponent blocks you, make a Strength 
check opposed by the defender's Dexterity or Strength check (whichever ability 
score has the higher modifier). A combatant gets a +4 bonus on the check for every 
size category he is larger than Medium or a -4 penalty for every size category 
he is smaller than Medium. The defender gets a +4 bonus on his check if he has 
more than two legs or is otherwise more stable than a normal humanoid. If you win, 
you knock the defender prone. If you lose, the defender may immediately react and 
make a Strength check opposed by your Dexterity or Strength check (including the 
size modifiers noted above, but no other modifiers) to try to knock you prone.

\textbf{Step 4:} Consequences. If you succeed in knocking your opponent prone, 
you can continue your movement as normal. If you fail and are knocked prone in 
turn, you have to move 5 feet back the way you came and fall prone, ending your 
movement there. If you fail but are not knocked prone, you have to move 5 feet 
back the way you came, ending your movement there. If that square is occupied, 
you fall prone in that square.

\textbf{Improved Overrun:} If you have the Improved Overrun feat, your target may 
not choose to avoid you.

\textbf{Mounted Overrun (Trample):} If you attempt an overrun while mounted, your 
mount makes the Strength check to determine the success or failure of the overrun 
attack (and applies its size modifier, rather than yours). If you have the Trample 
feat and attempt an overrun while mounted, your target may not choose to avoid 
you, and if you knock your opponent prone with the overrun, your mount may make 
one hoof attack against your opponent.

%%%%%%%%%%%%%%%%%%%%%%%%%
\subsection{Sunder}\index{Sunder}
%%%%%%%%%%%%%%%%%%%%%%%%%

You can use a melee attack with a slashing or bludgeoning weapon to strike a weapon 
or shield that your opponent is holding. If you're attempting to sunder a weapon 
or shield, follow the steps outlined here. (Attacking held objects other than weapons 
or shields is covered below.)

\textbf{Step 1:} Attack of Opportunity. You provoke an attack of opportunity from 
the target whose weapon or shield you are trying to sunder. (If you have the Improved 
Sunder feat, you don't incur an attack of opportunity for making the attempt.)

\textbf{Step 2:} Opposed Rolls. You and the defender make opposed attack rolls 
with your respective weapons. The wielder of a two-handed weapon on a sunder attempt 
gets a +4 bonus on this roll, and the wielder of a light weapon takes a -4 penalty. 
If the combatants are of different sizes, the larger combatant gets a bonus on 
the attack roll of +4 per difference in size category.

\textbf{Step 3:} Consequences. If you beat the defender, roll damage and deal it 
to the weapon or shield. See Table: Common Armor, Weapon, and Shield Hardness and 
Hit Points to determine how much damage you must deal to destroy the weapon or 
shield.

If you fail the sunder attempt, you don't deal any damage.

\textit{Sundering a Carried or Worn Object:} You don't use an opposed attack roll 
to damage a carried or worn object. Instead, just make an attack roll against the 
object's AC. A carried or worn object's AC is equal to 10 + its size modifier + 
the Dexterity modifier of the carrying or wearing character. Attacking a carried 
or worn object provokes an attack of opportunity just as attacking a held object 
does. To attempt to snatch away an item worn by a defender rather than damage it, 
see Disarm. You can't sunder armor worn by another character.

%%%%%%%%%%%%%%%%%%%%%%%%%
\subsection{Throw Splash Weapon}\index{Splash Weapons}
%%%%%%%%%%%%%%%%%%%%%%%%%

A splash weapon is a ranged weapon that breaks on impact, splashing or scattering 
its contents over its target and nearby creatures or objects. To attack with a 
splash weapon, make a ranged touch attack against the target. Thrown weapons require 
no weapon proficiency, so you don't take the -4 nonproficiency penalty. A hit deals 
direct hit damage to the target, and splash damage to all creatures within 5 feet 
of the target.

You can instead target a specific grid intersection. Treat this as a ranged attack 
against AC 5. However, if you target a grid intersection, creatures in all adjacent 
squares are dealt the splash damage, and the direct hit damage is not dealt to 
any creature. (You can't target a grid intersection occupied by a creature, such 
as a Large or larger creature; in this case, you're aiming at the creature.)

If you miss the target (whether aiming at a creature or a grid intersection), roll 
1d8. This determines the misdirection of the throw, with 1 being straight back 
at you and 2 through 8 counting clockwise around the grid intersection or target 
creature. Then, count a number of squares in the indicated direction equal to the 
range increment of the throw.

After you determine where the weapon landed, it deals splash damage to all creatures 
in adjacent squares.

%%%%%%%%%%%%%%%%%%%%%%%%%
\subsection{Trip}\index{Trip}
%%%%%%%%%%%%%%%%%%%%%%%%%

You can try to trip an opponent as an unarmed melee attack. You can only trip an 
opponent who is one size category larger than you, the same size, or smaller.

\textbf{Making a Trip Attack:} Make an unarmed melee touch attack against your 
target. This provokes an attack of opportunity from your target as normal for unarmed 
attacks.

If your attack succeeds, make a Strength check opposed by the defender's Dexterity 
or Strength check (whichever ability score has the higher modifier). A combatant 
gets a +4 bonus for every size category he is larger than Medium or a -4 penalty 
for every size category he is smaller than Medium. The defender gets a +4 bonus 
on his check if he has more than two legs or is otherwise more stable than a normal 
humanoid. If you win, you trip the defender. If you lose, the defender may immediately 
react and make a Strength check opposed by your Dexterity or Strength check to 
try to trip you.

\textit{Avoiding Attacks of Opportunity:} If you have the Improved Trip feat, or 
if you are tripping with a weapon (see below), you don't provoke an attack of opportunity 
for making a trip attack.

\textbf{Being Tripped (Prone):} A tripped character is prone. Standing up is a 
move action.

\textbf{Tripping a Mounted Opponent:} You may make a trip attack against a mounted 
opponent. The defender may make a Ride check in place of his Dexterity or Strength 
check. If you succeed, you pull the rider from his mount.

\textbf{Tripping with a Weapon:} Some weapons can be used to make trip attacks. 
In this case, you make a melee touch attack with the weapon instead of an unarmed 
melee touch attack, and you don't provoke an attack of opportunity.

If you are tripped during your own trip attempt, you can drop the weapon to avoid 
being tripped.

%%%%%%%%%%%%%%%%%%%%%%%%%
\subsection{Turn or Rebuke Undead}\index{Turn Undead}\index{Rebuke Undead}
%%%%%%%%%%%%%%%%%%%%%%%%%

Good clerics and paladins and some neutral clerics can channel positive energy, 
which can halt, drive off (rout), or destroy undead.

Evil clerics and some neutral clerics can channel negative energy, which can halt, 
awe (rebuke), control (command), or bolster undead.

Regardless of the effect, the general term for the activity is "turning." When 
attempting to exercise their divine control over these creatures, characters make 
turning checks.

%%%
\subsubsection{Turning Checks}
%%%

Turning undead is a supernatural ability that a character can perform as a standard 
action. It does not provoke attacks of opportunity.

You must present your holy symbol to turn undead. Turning is considered an attack.

\textbf{Times per Day:} You may attempt to turn undead a number of times per day 
equal to 3 + your Charisma modifier. You can increase this number by taking the 
Extra Turning feat.

\textbf{Range:} You turn the closest turnable undead first, and you can't turn 
undead that are more than 60 feet away or that have total cover relative to you. 
You don't need line of sight to a target, but you do need line of effect.

\textbf{Turning Check:} The first thing you do is roll a turning check to see how 
powerful an undead creature you can turn. This is a Charisma check (1d20 + your 
Charisma modifier). Table: Turning Undead gives you the Hit Dice of the most powerful 
undead you can affect, relative to your level. On a given turning attempt, you 
can turn no undead creature whose Hit Dice exceed the result on this table.

\textbf{Turning Damage:} If your roll on Table: Turning Undead is high enough to 
let you turn at least some of the undead within 60 feet, roll 2d6 + your cleric 
level + your Charisma modifier for turning damage. That's how many total Hit Dice 
of undead you can turn.

If your Charisma score is average or low, it's possible to roll fewer Hit Dice 
of undead turned than indicated on Table: Turning Undead.

You may skip over already turned undead that are still within range, so that you 
do not waste your turning capacity on them.

\textbf{Effect and Duration of Turning:} Turned undead flee from you by the best 
and fastest means available to them. They flee for 10 rounds (1 minute). If they 
cannot flee, they cower (giving any attack rolls against them a +2 bonus). If you 
approach within 10 feet of them, however, they overcome being turned and act normally. 
(You can stand within 10 feet without breaking the turning effect -- you just can't 
approach them.) You can attack them with ranged attacks (from at least 10 feet 
away), and others can attack them in any fashion, without breaking the turning 
effect.

\textbf{Destroying Undead:} If you have twice as many levels (or more) as the undead 
have Hit Dice, you destroy any that you would normally turn.

\begin{table}[htb]
\rowcolors{1}{white}{offyellow}\mcinherit
\caption{Turning Undead}
\centering
\begin{tabular}{c c}
\multicolumn{1}{p{3.5cm}}{\textbf{Turning Check Result}} & \multicolumn{1}{p{5cm}}{\textbf{Most Powerful Undead Affected (Maximum Hit Dice)}}\\
0 or lower & Cleric's Level -4\\
1-3 & Cleric's Level -3\\
4-6 & Cleric's Level -2\\
7-9 & Cleric's Level -1\\
10-12 & Cleric's Level\\
13-15 & Cleric's Level +1\\
16-18 & Cleric's Level +2\\
19-21 & Cleric's Level +3\\
22 or higher & Cleric's Level +4\\
\end{tabular}
\end{table}

%%%
\subsubsection{Evil Clerics and Undead}
%%%

Evil clerics channel negative energy to rebuke (awe) or command (control) undead 
rather than channeling positive energy to turn or destroy them. An evil cleric 
makes the equivalent of a turning check. Undead that would be turned are rebuked 
instead, and those that would be destroyed are commanded.

\textbf{Rebuked:} A rebuked undead creature cowers as if in awe (attack rolls against 
the creature get a +2 bonus). The effect lasts 10 rounds.

\textbf{Commanded:} A commanded undead creature is under the mental control of 
the evil cleric. The cleric must take a standard action to give mental orders to 
a commanded undead. At any one time, the cleric may command any number of undead 
whose total Hit Dice do not exceed his level. He may voluntarily relinquish command 
on any commanded undead creature or creatures in order to command new ones.

\textbf{Dispelling Turning:} An evil cleric may channel negative energy to dispel 
a good cleric's turning effect. The evil cleric makes a turning check as if attempting 
to rebuke the undead. If the turning check result is equal to or greater than the 
turning check result that the good cleric scored when turning the undead, then 
the undead are no longer turned. The evil cleric rolls turning damage of 2d6 + 
cleric level + Charisma modifier to see how many Hit Dice worth of undead he can 
affect in this way (as if he were rebuking them).

\textbf{Bolstering Undead:} An evil cleric may also bolster undead creatures against 
turning in advance. He makes a turning check as if attempting to rebuke the undead, 
but the Hit Dice result on Table: Turning Undead becomes the undead creatures' 
effective Hit Dice as far as turning is concerned (provided the result is higher 
than the creatures' actual Hit Dice). The bolstering lasts 10 rounds. An evil undead 
cleric can bolster himself in this manner.

%%%
\subsubsection{Neutral Clerics and Undead}
%%%

A cleric of neutral alignment can either turn undead but not rebuke them, or rebuke 
undead but not turn them. See Turn or Rebuke Undead for more information.

Even if a cleric is neutral, channeling positive energy is a good act and channeling 
negative energy is evil.

%%%
\subsubsection{Paladins and Undead}
%%%

Beginning at 4th level, paladins can turn undead as if they were clerics of three 
levels lower than they actually are.

%%%
\subsubsection{Turning Other Creatures}
%%%

Some clerics have the ability to turn creatures other than undead.

The turning check result is determined as normal.

%%%%%%%%%%%%%%%%%%%%%%%%%
\subsection{Two-Weapon Fighting}\index{Two-Weapon Fighting}
%%%%%%%%%%%%%%%%%%%%%%%%%

If you wield a second weapon in your off hand, you can get one extra attack per 
round with that weapon. You suffer a -6 penalty with your regular attack or attacks 
with your primary hand and a -10 penalty to the attack with your off hand when 
you fight this way. You can reduce these penalties in two ways:

\begin{itemize*}
\item If your off-hand weapon is light, the penalties are reduced by 2 each. (An unarmed strike is always considered light.)
\item The \linkfeat{Two-Weapon Fighting} feat lessens the primary hand penalty by 2, and the off-hand penalty by 6.
\end{itemize*}

Table: Two-Weapon Fighting Penalties summarizes the interaction of all these factors.

\begin{table}[htb]
\rowcolors{1}{white}{offyellow}\mcinherit
\caption{Two-Weapon Fighting Penalties}
\centering
\begin{tabular}{l c c}
\textbf{Circumstances} & \textbf{Primary Hand} & \textbf{Off Hand}\\
Normal penalties & -6 & -10\\
Off-hand weapon is light & -4 & -8\\
Two-Weapon Fighting feat & -4 & -4\\
\shortstack{Off-hand weapon is light and\\Two-Weapon Fighting feat} & -2 & -2\\
\end{tabular}
\end{table}

\textbf{Double Weapons}: You can use a double weapon to make an extra attack with 
the off-hand end of the weapon as if you were fighting with two weapons. The penalties 
apply as if the off-hand end of the weapon were a light weapon.

\textbf{Thrown Weapons:} The same rules apply when you throw a weapon from each 
hand. Treat a dart or shuriken as a light weapon when used in this manner, and 
treat a bolas, javelin, net, or sling as a one-handed weapon.

%%%%%%%%%%%%%%%%%%%%%%%%%%%%%%%%%%%%%%%%%%%%%%%%%%
\section{Special Initiative Actions}
%%%%%%%%%%%%%%%%%%%%%%%%%%%%%%%%%%%%%%%%%%%%%%%%%%

Here are ways to change when you act during combat by altering your place in the 
initiative order.

%%%%%%%%%%%%%%%%%%%%%%%%%
\subsection{Delay}\index{Delay Action}\index{Initiative!Delay}
%%%%%%%%%%%%%%%%%%%%%%%%%

By choosing to delay, you take no action and then act normally on whatever initiative 
count you decide to act. When you delay, you voluntarily reduce your own initiative 
result for the rest of the combat. When your new, lower initiative count comes 
up later in the same round, you can act normally. You can specify this new initiative 
result or just wait until some time later in the round and act then, thus fixing 
your new initiative count at that point.

You never get back the time you spend waiting to see what's going to happen. You 
can't, however, interrupt anyone else's action (as you can with a readied action).

\textbf{Initiative Consequences of Delaying:} Your initiative result becomes the 
count on which you took the delayed action. If you come to your next action and 
have not yet performed an action, you don't get to take a delayed action (though 
you can delay again).

If you take a delayed action in the next round, before your regular turn comes 
up, your initiative count rises to that new point in the order of battle, and you 
do not get your regular action that round.

%%%%%%%%%%%%%%%%%%%%%%%%%
\subsection{Ready}\index{Readied Action}\index{Initiative!Readied Action}
%%%%%%%%%%%%%%%%%%%%%%%%%

The ready action lets you prepare to take an action later, after your turn is over 
but before your next one has begun. Readying is a standard action. It does not 
provoke an attack of opportunity (though the action that you ready might do so).

\textbf{Readying an Action:} You can ready a standard action, a move action, or 
a free action. To do so, specify the action you will take and the conditions under 
which you will take it. Then, any time before your next action, you may take the 
readied action in response to that condition. The action occurs just before the 
action that triggers it. If the triggered action is part of another character's 
activities, you interrupt the other character. Assuming he is still capable of 
doing so, he continues his actions once you complete your readied action. Your 
initiative result changes. For the rest of the encounter, your initiative result 
is the count on which you took the readied action, and you act immediately ahead 
of the character whose action triggered your readied action.

You can take a 5-foot step as part of your readied action, but only if you don't 
otherwise move any distance during the round. 

\textbf{Initiative Consequences of Readying:} Your initiative result becomes the 
count on which you took the readied action. If you come to your next action and 
have not yet performed your readied action, you don't get to take the readied action 
(though you can ready the same action again). If you take your readied action in 
the next round, before your regular turn comes up, your initiative count rises 
to that new point in the order of battle, and you do not get your regular action 
that round.

\textbf{Distracting Spellcasters:} You can ready an attack against a spellcaster 
with the trigger "if she starts casting a spell." If you damage the spellcaster, 
she may lose the spell she was trying to cast (as determined by her Concentration 
check result).

\textbf{Readying to Counterspell:} You may ready a counterspell against a spellcaster 
(often with the trigger "if she starts casting a spell"). In this case, when 
the spellcaster starts a spell, you get a chance to identify it with a Spellcraft 
check (DC 15 + spell level). If you do, and if you can cast that same spell (are 
able to cast it and have it prepared, if you prepare spells), you can cast the 
spell as a counterspell and automatically ruin the other spellcaster's spell. Counterspelling 
works even if one spell is divine and the other arcane.

A spellcaster can use \linkspell{Dispel Magic} to counterspell another spellcaster, 
but it doesn't always work.

\textbf{Readying a Weapon against a Charge:} You can ready certain piercing weapons, 
setting them to receive charges. A readied weapon of this type deals double damage 
if you score a hit with it against a charging character.

%%%%%%%%%%%%%%%%%%%%%%%%%%%%%%%%%%%%%%%%%%%%%%%%%%
\section{Conditions}
%%%%%%%%%%%%%%%%%%%%%%%%%%%%%%%%%%%%%%%%%%%%%%%%%%

If more than one condition affects a character, apply them all. If certain effects 
can't combine, apply the most severe effect.

%%%
\subsubsection{Ability Damaged}\index{Ability Damage}
%%%

The character has temporarily lost 1 or more ability 
score points. Lost points return at a rate of 1 per day unless noted otherwise 
by the condition dealing the damage. A character with Strength 0 falls to the ground 
and is helpless. A character with Dexterity 0 is paralyzed. A character with Constitution 
0 is dead. A character with Intelligence, Wisdom, or Charisma 0 is unconscious. 
Ability damage is different from penalties to ability scores, which go away when 
the conditions causing them go away.

%%%
\subsubsection{Ability Drained}\index{Ability Drain}
%%%

The character has permanently lost 1 or more ability 
score points. The character can regain these points only through magical means. 
A character with Strength 0 falls to the ground and is helpless. A character with 
Dexterity 0 is paralyzed. A character with Constitution 0 is dead. A character 
with Intelligence, Wisdom, or Charisma 0 is unconscious. 

%%%
\subsubsection{Blinded}\index{Blind}
%%%

The character cannot see. He takes a -2 penalty to Armor Class, 
loses his Dexterity bonus to AC (if any), moves at half speed, and takes a -4 penalty 
on \linkskill{Search} checks and on most Strength- and Dexterity-based skill checks. All checks 
and activities that rely on vision (such as reading and \linkskill{Spot} checks) automatically 
fail. All opponents are considered to have total concealment (50\% miss chance) 
to the blinded character. Characters who remain blinded for a long time grow accustomed 
to these drawbacks and can overcome some of them.

%%%
\subsubsection{Blown Away}\index{Blown Away}
%%%

Depending on its size, a creature can be blown away by winds 
of high velocity. A creature on the ground that is blown away is knocked down and 
rolls 1d4 x 10 feet, taking 1d4 points of nonlethal damage per 10 feet. A flying 
creature that is blown away is blown back 2d6 x 10 feet and takes 2d6 points of 
nonlethal damage due to battering and buffering. 

%%%
\subsubsection{Checked}\index{Checked}
%%%

Prevented from achieving forward motion by an applied force, 
such as wind. Checked creatures on the ground merely stop. Checked flying creatures 
move back a distance specified in the description of the effect.

%%%
\subsubsection{Confused}\index{Confused}
%%%

A confused character's actions are 
determined by rolling d\% at the beginning of his turn: 01-10, attack caster with 
melee or ranged weapons (or close with caster if attacking is not possible); 11-20, 
act normally; 21-50, do nothing but babble incoherently; 51-70, flee away from 
caster at top possible speed; 71-100, attack nearest creature (for this purpose, 
a familiar counts as part of the subject's self). A confused character 
who can't carry out the indicated action does nothing but babble incoherently. 
Attackers are not at any special advantage when attacking a confused character. 
Any confused character who is attacked automatically attacks its attackers 
on its next turn, as long as it is still confused when its turn comes. 
A confused character does not make attacks of opportunity against any 
creature that it is not already devoted to attacking (either because of its most 
recent action or because it has just been attacked).

%%%
\subsubsection{Cowering}\index{Cowering}
%%%

The character is frozen in fear and can take no actions. A cowering 
character takes a -2 penalty to Armor Class and loses her Dexterity bonus (if any).

%%%
\subsubsection{Dazed}\index{Dazed}
%%%

The creature is unable to act normally. A dazed creature can take 
no actions, but has no penalty to AC.

A dazed condition typically lasts 1 round.

%%%
\subsubsection{Dazzled}\index{Dazzled}
%%%

The creature is unable to see well because of overstimulation 
of the eyes. A dazzled creature takes a -1 penalty on attack rolls, \linkskill{Search} checks, 
and \linkskill{Spot} checks.

%%%
\subsubsection{Dead}\index{Dead}
%%%

The character's hit points are reduced to -10, his Constitution 
drops to 0, or he is killed outright by a spell or effect. The character's soul 
leaves his body. Dead characters cannot benefit from normal or magical healing, 
but they can be restored to life via magic. A dead body decays normally unless 
magically preserved, but magic that restores a dead character to life also restores 
the body either to full health or to its condition at the time of death (depending 
on the spell or device). Either way, resurrected characters need not worry about 
rigor mortis, decomposition, and other conditions that affect dead bodies.

%%%
\subsubsection{Deafened}\index{Deafened}
%%%

A deafened character cannot hear. She takes a -4 penalty on 
initiative checks, automatically fails \linkskill{Listen} checks, and has a 20\% chance of 
spell failure when casting spells with verbal components. Characters who remain 
deafened for a long time grow accustomed to these drawbacks and can overcome some 
of them.

%%%
\subsubsection{Disabled}\index{Disabled}
%%%

A character with 0 hit points, or one who has negative hit points 
but has become stable and conscious, is disabled. A disabled character may take 
a single move action or standard action each round (but not both, nor can she take 
full-round actions). She moves at half speed. Taking move actions doesn't risk 
further injury, but performing any standard action (or any other action the DM 
deems strenuous, including some free actions such as casting a quickened spell) 
deals 1 point of damage after the completion of the act. Unless the action increased 
the disabled character's hit points, she is now in negative hit points and dying.

A disabled character with negative hit points recovers hit points naturally if 
she is being helped. Otherwise, each day she has a 10\% chance to start recovering 
hit points naturally (starting with that day); otherwise, she loses 1 hit point. 
Once an unaided character starts recovering hit points naturally, she is no longer 
in danger of losing hit points (even if her current hit points are negative).

%%%
\subsubsection{Dying}\index{Dying}
%%%

A dying character is unconscious and near death. She has -1 to 
-9 current hit points. A dying character can take no actions and is unconscious. 
At the end of each round (starting with the round in which the character dropped 
below 0 hit points), the character rolls d\% to see whether she becomes stable. 
She has a 10\% chance to become stable. If she does not, she loses 1 hit point. 
If a dying character reaches -10 hit points, she is dead.

%%%
\subsubsection{Energy Drained}\index{Energy Drain}
%%%

The character gains one or more negative levels, which 
might permanently drain the character's levels. If the subject has at least as 
many negative levels as Hit Dice, he dies. Each negative level gives a creature 
the following penalties: -1 penalty on attack rolls, saving throws, skill checks, 
ability checks; loss of 5 hit points; and -1 to effective level (for determining 
the power, duration, DC, and other details of spells or special abilities). In 
addition, a spellcaster loses one spell or spell slot from the highest spell level 
castable.

%%%
\subsubsection{Entangled}\index{Entangled}
%%%

The character is ensnared. Being entangled impedes movement, 
but does not entirely prevent it unless the bonds are anchored to an immobile object 
or tethered by an opposing force. An entangled creature moves at half speed, cannot 
run or charge, and takes a -2 penalty on all attack rolls and a -4 penalty to Dexterity. 
An entangled character who attempts to cast a spell must make a Concentration check 
(DC 15 + the spell's level) or lose the spell. 

%%%
\subsubsection{Exhausted}\index{Exhausted}
%%%

An exhausted character moves at half speed and takes a -6 penalty 
to Strength and Dexterity. After 1 hour of complete rest, an exhausted character 
becomes fatigued. A fatigued character becomes exhausted by doing something else 
that would normally cause fatigue.

%%%
\subsubsection{Fascinated}\index{Fascinated}
%%%

A fascinated creature is entranced by a supernatural or spell 
effect. The creature stands or sits quietly, taking no actions other than to pay 
attention to the fascinating effect, for as long as the effect lasts. It takes 
a -4 penalty on skill checks made as reactions, such as \linkskill{Listen} and \linkskill{Spot} checks. 
Any potential threat, such as a hostile creature approaching, allows the fascinated 
creature a new saving throw against the fascinating effect. Any obvious threat, 
such as someone drawing a weapon, casting a spell, or aiming a ranged weapon at 
the fascinated creature, automatically breaks the effect. A fascinated creature's 
ally may shake it free of the spell as a standard action. 

%%%
\subsubsection{Fatigued}\index{Fatigued}
%%%

A fatigued character can neither run nor charge and takes a 
-2 penalty to Strength and Dexterity. Doing anything that would normally cause 
fatigue causes the fatigued character to become exhausted. After 8 hours of complete 
rest, fatigued characters are no longer fatigued. 

%%%
\subsubsection{Flat-Footed}\index{Flat-Footed}
%%%

A character who has not yet acted during a combat is flat-footed, 
not yet reacting normally to the situation. A flat-footed character loses his Dexterity 
bonus to AC (if any) and cannot make attacks of opportunity. 

%%%
\subsubsection{Frightened}\index{Frightened}
%%%

A frightened creature flees from the source of its fear as 
best it can. If unable to flee, it may fight. A frightened creature takes a -2 
penalty on all attack rolls, saving throws, skill checks, and ability checks. A 
frightened creature can use special abilities, including spells, to flee; indeed, 
the creature must use such means if they are the only way to escape. 

Frightened is like shaken, except that the creature must flee if possible.
\linksec{Panicked} is a more extreme state of fear.

%%%
\subsubsection{Grappling}\index{Grappling}
%%%

Engaged in wrestling or some other form of hand-to-hand struggle 
with one or more attackers. A grappling character can undertake only a limited 
number of actions. He does not threaten any squares, and loses his Dexterity bonus 
to AC (if any) against opponents he isn't grappling.

%%%
\subsubsection{Helpless}\index{Helpless}
%%%

A helpless character is paralyzed, \textit{held}, bound, sleeping, 
unconscious, or otherwise completely at an opponent's mercy. A helpless target 
is treated as having a Dexterity of 0 (-5 modifier). Melee attacks against a helpless 
target get a +4 bonus (equivalent to attacking a prone target). Ranged attacks 
gets no special bonus against helpless targets. Rogues can sneak attack helpless 
targets. 

As a full-round action, an enemy can use a melee weapon to deliver a coup de grace 
to a helpless foe. An enemy can also use a bow or crossbow, provided he is adjacent 
to the target. The attacker automatically hits and scores a critical hit. (A rogue 
also gets her sneak attack damage bonus against a helpless foe when delivering 
a coup de grace.) If the defender survives, he must make a Fortitude save (DC 10 
+ damage dealt) or die. 

Delivering a \gameterm{Coup de Grace} provokes attacks of opportunity. 

Creatures that are immune to critical hits do not take critical damage, nor do 
they need to make Fortitude saves to avoid being killed by a coup de grace.

%%%
\subsubsection{Incorporeal}\index{Incorporeal}
%%%

Having no physical body. Incorporeal creatures are immune 
to all nonmagical attack forms. They can be harmed only by other incorporeal creatures, 
+1 or better magic weapons, spells, spell-like effects, or supernatural effects. 

%%%
\subsubsection{Invisible}\index{Invisible}
%%%

Visually undetectable. An invisible creature gains a +2 bonus 
on attack rolls against sighted opponents, and ignores its opponents' Dexterity 
bonuses to AC (if any). (See \linksec{Invisibility}, under Special Abilities.)

%%%
\subsubsection{Knocked Down}\index{Knocked Down}
%%%

Depending on their size, creatures can be knocked down by 
winds of high velocity. Creatures on the ground are knocked prone by the force 
of the wind. Flying creatures are instead blown back 1d6 x 10 feet.

%%%
\subsubsection{Nauseated}\index{Nauseated}
%%%

Experiencing stomach distress. Nauseated creatures are unable 
to attack, cast spells, concentrate on spells, or do anything else requiring attention. 
The only action such a character can take is a single move action per turn.

%%%
\subsubsection{Panicked}\index{Panicked}
%%%

A panicked creature must drop anything it holds and flee at 
top speed from the source of its fear, as well as any other dangers it encounters, 
along a random path. It can't take any other actions. In addition, the creature 
takes a -2 penalty on all saving throws, skill checks, and ability checks. If cornered, 
a panicked creature cowers and does not attack, typically using the total defense 
action in combat. A panicked creature can use special abilities, including spells, 
to flee; indeed, the creature must use such means if they are the only way to escape.

Panicked is a more extreme state of fear than \linksec{Shaken} or \linksec{Frightened}.

%%%
\subsubsection{Paralyzed}\index{Paralyzed}
%%%

A paralyzed character is frozen in place and unable to move 
or act. A paralyzed character has effective Dexterity and Strength scores of 0 
and is helpless, but can take purely mental actions. A winged creature flying in 
the air at the time that it becomes paralyzed cannot flap its wings and falls. 
A paralyzed swimmer can't swim and may drown. A creature can move through a space 
occupied by a paralyzed creature -- ally or not. Each square occupied by a paralyzed 
creature, however, counts as 2 squares.

%%%
\subsubsection{Petrified}\index{Petrified}
%%%

A petrified character has been turned to stone and is considered 
unconscious. If a petrified character cracks or breaks, but the broken pieces are 
joined with the body as he returns to flesh, he is unharmed. If the character's 
petrified body is incomplete when it returns to flesh, the body is likewise incomplete 
and there is some amount of permanent hit point loss and/or debilitation.

%%%
\subsubsection{Pinned}\index{Pinned}
%%%

Held immobile (but not helpless) in a grapple.

%%%
\subsubsection{Prone}\index{Prone}
%%%

The character is on the ground. An attacker who is prone has a 
-4 penalty on melee attack rolls and cannot use a ranged weapon (except for a crossbow). 
A defender who is prone gains a +4 bonus to Armor Class against ranged attacks, 
but takes a -4 penalty to AC against melee attacks.

Standing up is a move-equivalent action that provokes an attack of opportunity.

%%%
\subsubsection{Shaken}\index{Shaken}
%%%

A shaken character takes a -2 penalty on attack rolls, saving 
throws, skill checks, and ability checks.

Shaken is a less severe state of fear than \linksec{Frightened} or \linksec{Panicked}.

%%%
\subsubsection{Sickened}\index{Sickened}
%%%

The character takes a -2 penalty on all attack rolls, weapon 
damage rolls, saving throws, skill checks, and ability checks.

%%%
\subsubsection{Stable}\index{Stable}
%%%

A character who was dying but who has stopped losing hit points 
and still has negative hit points is stable. The character is no longer dying, 
but is still unconscious. If the character has become stable because of aid from 
another character (such as a \linkskill{Heal} check or magical healing), then the character 
no longer loses hit points. He has a 10\% chance each hour of becoming conscious 
and disabled (even though his hit points are still negative). 

If the character became stable on his own and hasn't had help, he is still at risk 
of losing hit points. Each hour, he has a 10\% chance of becoming conscious and 
disabled. Otherwise he loses 1 hit point.

%%%
\subsubsection{Staggered}\index{Staggered}
%%%

A character whose nonlethal damage exactly equals his current 
hit points is staggered. A staggered character may take a single move action or 
standard action each round (but not both, nor can she take full-round actions).

A character whose current hit points exceed his nonlethal damage is no longer staggered; 
a character whose nonlethal damage exceeds his hit points becomes unconscious.

%%%
\subsubsection{Stunned}\index{Stunned}
%%%

A stunned creature drops everything held, can't take actions, 
takes a -2 penalty to AC, and loses his Dexterity bonus to AC (if any).

%%%
\subsubsection{Turned}\index{Turned}
%%%

Affected by a turn undead attempt. Turned undead flee for 10 rounds 
(1 minute) by the best and fastest means available to them. If they cannot flee, 
they cower.

%%%
\subsubsection{Unconscious}\index{Unconscious}
%%%

Knocked out and helpless. Unconsciousness can result from 
having current hit points between -1 and -9, or from nonlethal damage in excess 
of current hit points.


\chapter{Magic}
\section{Casting Spells}
foo
\section{How To Read A Spell Description}
foo
\section{Arcane Spells}
foo
\section{Divine Spells}
foo
\section{Special Abilities and Spells}
foo
\section{Spell Lists}
foo

\chapter{Magic Items}
\section{Magic Item Basics}
Scaling, 8 Item Limit, etc
\section{Minor Magical Items}
foo
\section{Moderate Magical Items}
foo
\section{Major Magical Items}
foo

\chapter{Running The Game}
\section{Writing GM Advice Is Hard}
foo

%%%%%%%%%%%%%%%%%%%%%%%%%%%%%%%%%%%%%%%%%%%%%%%%%%
%%%%%%%%%%%%%%%%%%%%%%%%%%%%%%%%%%%%%%%%%%%%%%%%%%
\appendix
%%%%%%%%%%%%%%%%%%%%%%%%%%%%%%%%%%%%%%%%%%%%%%%%%%
%%%%%%%%%%%%%%%%%%%%%%%%%%%%%%%%%%%%%%%%%%%%%%%%%%
\appendixpage

\makeatletter
\renewcommand{\@makechapterhead}[1]{%
\vspace*{0 pt}{
\raggedright \normalfont \fontsize{32}{32} \selectfont \bfseries
\ifnum \value{secnumdepth}>-1
  \if@mainmatter \vspace{-8pt} {\fontsize{20}{20} \selectfont Appendix \thechapter:}\\[8pt]
  \fi%
\fi
\hspace{0.65cm} #1\par\nobreak\vspace{20 pt}
}}
\makeatother

\clearpage

%% Appendix Chapters Here

\chapter{Spells}
\section{Spells, A through Z}
foo

\chapter{Prestige Classes}
\section{Prestige Class Basics}
\section{?WhatClasses?}

\chapter{Monsters}
\section{Reading a Monster Entry}
\section{Monsters, A though Z}

\chapter{NPC Classes}
\section{Adept}
\section{Aristocrat}
\section{Commoner}
\section{Expert}
\section{Warrior}

%\input{tome-srd-ogl}

\clearpage
\phantomsection
\listoftables

\clearpage
\phantomsection
\printindex

\end{document}