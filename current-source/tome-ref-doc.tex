\documentclass[10pt,twoside,onecolumn,openany,final]{memoir}
\setstocksize{11in}{8.5in}

\usepackage[toc,lot,lof]{multitoc}
\usepackage[top=.5in, bottom=.5in, left=.75in, right=.75in]{geometry}
\usepackage{graphicx} \graphicspath{{./images/}}
\usepackage{longtable}
\usepackage{mdwlist}
\usepackage{microtype} \DisableLigatures{encoding = *, family = *}
\usepackage{multicol}
\usepackage{textcomp}
\usepackage[normalem]{ulem}
\usepackage{wrapfig}
\usepackage{xtab}
\usepackage{enumerate}
\usepackage{phonetic}
\usepackage{bbding}
\usepackage{linearb}
		
%\usepackage{cypriot}

\usepackage{tipa}
\usepackage{xfrac}
\usepackage{appendix}
\usepackage{xparse}
\usepackage{letltxmacro}
\usepackage{makeidx} \makeindex
\usepackage[table,dvipsnames]{xcolor}
\definecolor{offyellow}{RGB}{255,255,128}
\definecolor{links}{RGB}{200,0,50}
\usepackage{placeins}
\usepackage{floatflt}
\usepackage{anyfontsize}
\usepackage{colortbl}
\usepackage{tabularx}
\usepackage{mdframed}
\usepackage{longtable}
\usepackage{tabu}
\usepackage{afterpage}
\usepackage{caption}

%\usepackage[bf, big, raggedright]{titlesec}

\usepackage{amsmath}

%% Font
\usepackage[T1]{fontenc}
\usepackage[bitstream-charter]{mathdesign}
\usepackage{aurical}

\usepackage[colorlinks=true,linkcolor=blue,urlcolor=links,pdfstartview={XYZ null null 1.00},bookmarksdepth=2]{hyperref}


%%%%%%%%%%%%%%%%%%%%%%%%%
%%%% End of Import %%%%%%
%%%%%%%%%%%%%%%%%%%%%%%%%



%%%%%%%%%%%%%%%%%%%%%%%%%%%%%%%%%%%%%%%%%%%%%%%%%%
%%%%%%%%%%%%%%%%%%%%%%%%%%%%%%%%%%%%%%%%%%%%%%%%%%
%%% General Font Display
%%%%%%%%%%%%%%%%%%%%%%%%%%%%%%%%%%%%%%%%%%%%%%%%%%
%%%%%%%%%%%%%%%%%%%%%%%%%%%%%%%%%%%%%%%%%%%%%%%%%%

\renewcommand*{\familydefault}{\sfdefault}
%% sets default text to sans-serif, so text doesn't flip back to serif in some environments.



%%%%%%%%%%%%%%%%%%%%%%%%%%%%%%%%%%%%%%%%%%%%%%%%%%
%%%%%%%%%%%%%%%%%%%%%%%%%%%%%%%%%%%%%%%%%%%%%%%%%%
%%% Sectioning Display
%%%%%%%%%%%%%%%%%%%%%%%%%%%%%%%%%%%%%%%%%%%%%%%%%%
%%%%%%%%%%%%%%%%%%%%%%%%%%%%%%%%%%%%%%%%%%%%%%%%%%






%%%%%%%%%%%%%%%%%%%%%%%%%%%%%%%%%%%%%%%%%%%%%%%%%%
%%%%%%%%%%%%%%%%%%%%%%%%%%%%%%%%%%%%%%%%%%%%%%%%%%
%%% Revised Commands
%%%%%%%%%%%%%%%%%%%%%%%%%%%%%%%%%%%%%%%%%%%%%%%%%%
%%%%%%%%%%%%%%%%%%%%%%%%%%%%%%%%%%%%%%%%%%%%%%%%%%
\makeatletter

%fiddles with how chapter titles are displayed
\renewcommand{\@makechapterhead}[1]{%
\vspace*{0 pt}{%
\raggedright \fontsize{32}{32} \selectfont \bfseries%
\ifnum \value{secnumdepth}>-1%
  \if@mainmatter \vspace{-8pt} {\fontsize{20}{20} \selectfont Chapter \thechapter:}\\[8pt]%
  \fi%
\fi
\hspace{0.65cm} #1\par\nobreak\vspace{20 pt}%
}}

%makes paragraphs show up closer together
\renewcommand{\paragraph}{%
\@startsection{paragraph}{4}%
{\z@}{1.0ex \@plus 1ex \@minus 0.2ex}{-1em} % wtf is an 'ex' anyways?
{\normalfont\normalsize\bfseries}%
}

%lets multicolumn have the proper background colors as defined by rowcolors
\let\oldmc\multicolumn
\newcommand{\mcinherit}{% Activate \multicolumn inheritance
  \renewcommand{\multicolumn}[3]{%
    \oldmc{##1}{##2}{\ifodd\rownum \@oddrowcolor\else\@evenrowcolor\fi ##3}%
  }}

\makeatother

%add labels within sections, subsections, and subsubsections
\LetLtxMacro{\oldsection}{\section}
\renewcommand{\section}[1]{\oldsection{#1}\label{sec:#1}}

\LetLtxMacro{\oldsubsection}{\subsection}
\renewcommand{\subsection}[1]{\oldsubsection{#1}\label{sec:#1}}

\LetLtxMacro{\oldsubsubsection}{\subsubsection}
\renewcommand{\subsubsection}[1]{\oldsubsubsection{#1}\label{sec:#1}}

%only put chapters and sections into the TOC
\setcounter{secnumdepth}{1}

%makes a subsubsection start off indented.
\setlength{\beforesubsubsecskip}{-\beforesubsubsecskip}



%%%%%%%%%%%%%%%%%%%%%%%%%%%%%%%%%%%%%%%%%%%%%%%%%%
%%%%%%%%%%%%%%%%%%%%%%%%%%%%%%%%%%%%%%%%%%%%%%%%%%
%%% Table Formatting
%%%%%%%%%%%%%%%%%%%%%%%%%%%%%%%%%%%%%%%%%%%%%%%%%%
%%%%%%%%%%%%%%%%%%%%%%%%%%%%%%%%%%%%%%%%%%%%%%%%%%
\newcolumntype{L}[1]{>{\raggedright\let\newline\\\arraybackslash\hspace{0pt}}m{#1}} %New type of column 'L' that is ragged-right, behaves like a paragraph, and allows manual definition of width like a 'p' column.
\newcolumntype{C}[1]{>{\centering\let\newline\\\arraybackslash\hspace{0pt}}m{#1}}  %New type of column 'C' that is centered, behaves like a paragraph, and allows manual definition of width like a 'p' column.
\newcolumntype{R}[1]{>{\raggedleft\let\newline\\\arraybackslash\hspace{0pt}}m{#1}}  %New type of column 'R' that is ragged-left, behaves like a paragraph, and allows manual definition of width like a 'p' column.
\newcommand{\header}{\rowcolor{headercolor}}
%when inserted in a row, makes that row the color headercolor

\global\tabulinesep=1mm


%%%%%%%%%%%%%%%%%%%%%%%%%%%%%%%%%%%%%%%%%%%%%%%%%%
%%%%%%%%%%%%%%%%%%%%%%%%%%%%%%%%%%%%%%%%%%%%%%%%%%
%%% List Formatting
%%%%%%%%%%%%%%%%%%%%%%%%%%%%%%%%%%%%%%%%%%%%%%%%%%
%%%%%%%%%%%%%%%%%%%%%%%%%%%%%%%%%%%%%%%%%%%%%%%%%%

\let\olditemize\itemize
\renewcommand{\itemize}{
  \olditemize
  \setlength{\itemsep}{1pt}
  \setlength{\parskip}{0pt}
  \setlength{\parsep}{0pt}
}
%fixes itemize spacing

\let\olddescription\description
\renewcommand{\description}{
  \olddescription
  \setlength{\itemsep}{1pt}
  \setlength{\parskip}{0pt}
  \setlength{\parsep}{0pt}
}
%fixes description spacing

\let\oldenumerate\enumerate
\renewcommand{\enumerate}{
  \oldenumerate
  \setlength{\itemsep}{1pt}
  \setlength{\parskip}{0pt}
  \setlength{\parsep}{0pt}
}
%fixes enumerate spacing

\newcommand{\descability}[2]{\item[#1:] #2}


%%%%%%%%%%%%%%%%%%%%%%%%%%%%%%%%%%%%%%%%%%%%%%%%%%
%%%%%%%%%%%%%%%%%%%%%%%%%%%%%%%%%%%%%%%%%%%%%%%%%%
%%% New Commands
%%%%%%%%%%%%%%%%%%%%%%%%%%%%%%%%%%%%%%%%%%%%%%%%%%
%%%%%%%%%%%%%%%%%%%%%%%%%%%%%%%%%%%%%%%%%%%%%%%%%%


%%%%%%%%%%%%%%%%%%%%%%%%
%%Basic Formatting
%%%%%%%%%%%%%%%%%%%%%%%%

\newcommand{\originallineskip}{\baselineskip}
 %A command that is equal to the original \baselineskip of the doc, in case we change it for a section and want to change it back later

\newcommand{\ability}[2]{\textbf{#1:} #2} 
%The \ability{#1}{#2} command from legacy-source. Should rarely be directly used, changes to this will cascade into other new commands that use its functionality

\newcommand{\shortability}[2]{\noindent\textbf{#1} #2\\}
%A specialized version of the \ability command

\newcommand{\itemspace}{\setlength{\itemsep}{-1mm}\setlength{\topsep}{-1mm} }
%A command from legacy-source for compatabilty

\newenvironment{awesomelist}{\begin{list}{$\bullet$}{\itemspace}}{\end{list}\vspace{8pt}}

\newcommand{\listone}{\begin{list}{$\bullet$}{\itemspace}}

\newcommand{\listtwo}{\begin{list}{$\triangleright$}{\itemspace}}
%A type of list from legacy sorce

\newcommand{\listnum}{\begin{list}{\textbf{\arabic{counter}}:}{\usecounter{counter}}}

\newcommand{\spell}[1]{\emph{\MakeLowercase{#1}}}
%makes spell name lowercase italics.

\setlength{\parindent}{12pt}
%sets the indentation of all paragraphs in the work

\newcommand{\quot}[1]{
	\vspace{-8pt}
	\noindent\emph{#1}\medskip}
%Displays a flavor quote.}

\newcommand{\half}[0]{\ensuremath{\sfrac{1}{2}} }

\newcommand{\third}[0]{\ensuremath{\sfrac{1}{3}} }

\newcommand{\fourth}[0]{\ensuremath{\sfrac{1}{4}} }

\newcommand{\ex}{(Ex)}
\newcommand{\sla}{(Sp)}
\newcommand{\su}{(Su)}

\newcommand{\condition}[1]{\emph{#1}}

%%%%%%%%%%%%%%%%%%%%%%%%
%%Logic
%%%%%%%%%%%%%%%%%%%%%%%%
\newcommand{\testempty}{\empty}
\newcommand{\isempty}{\empty}
%Two commands that can be compared to one another for \ifx logic tests. \isempty should never be changed. If \testempty holds a value of anything but empty, the test should return false.

\newcounter{counter}




%%%%%%%%%%%%%%%%%%%%%%%%
%%Colors
%%%%%%%%%%%%%%%%%%%%%%%%
\colorlet{colorone}{white}
\colorlet{colortwo}{gray!15}
\colorlet{headercolor}{gray!50}
\colorlet{tablecolorone}{gray!40}
\colorlet{tablecolortwo}{gray!20}


%%%%%%%%%%%%%%%%%%%%%%%%
%%Sectioning
%%%%%%%%%%%%%%%%%%%%%%%%
\newcommand{\classentry}[1]{\section{#1} \label{class:#1} \renewcommand{\class}{#1} \index{#1 (class)} \renewcommand{\testempty}{\isempty}}
%\newcommand{\classentry}[1]{\newpage \section{#1} \label{class:#1} \renewcommand{\class}{#1} \index{#1 (class)} \renewcommand{\testempty}{\isempty}}
%Starts a <new page>, creates a section with the name of the class (#1), sets \class to be the name of the class, indexes the class.

\newcommand{\raceentry}[2]{\newpage\renewcommand{\race}{#1}\section{#1} \label{race:#1}\vspace{-1em}\textit{#2}}
%\newcommand{\raceentry}[1]{\oldsection{#1}\index{#1 (race)}\label{race:#1}}

\newcommand{\Requirements}{\oldsubsubsection*{Requirements}}

\newcommand{\Basics}{\oldsubsubsection*{Basics}}

\newcommand{\ClassFeatures}{\oldsubsubsection*{Class Features}}

\newcommand{\skillentry}[2]{\oldsubsection[#1]{#1 #2}\index{#1 (skill)}\label{skill:#1}}





%%%%%%%%%%%%%%%%%%%%%%%%
%%Unsorted Commands
%%%%%%%%%%%%%%%%%%%%%%%%
\newcommand{\tagline}[1]{\vspace{-6pt} \textit{#1} \medskip}

\newcommand{\gameterm}[1]{#1\index{#1}}

\NewDocumentCommand\featentry{m+g}{%
  \IfNoValueTF{#2}
    {\oldsubsubsection[#1]{#1 [General]}\label{feat:#1}}%no second arg, general feat
    {\oldsubsubsection[#1]{#1 [#2]}\label{feat:#1}}%second arg, special type of feat
}

\newcommand{\spellentry}[1]{\oldsubsubsection{#1}\label{spell:#1}}

\NewDocumentCommand\linkrace{m+g}{%
  \IfNoValueTF{#2}
    {\hyperref[race:#1]{#1}}%no second arg, display is same as link
    {\hyperref[race:#1]{#2}}%second arg, link to first and display second
}
\NewDocumentCommand\linkclass{m+g}{%
  \IfNoValueTF{#2}
    {\hyperref[class:#1]{#1}}%no second arg, display is same as link
    {\hyperref[class:#1]{#2}}%second arg, link to first and display second
}
\NewDocumentCommand\linkskill{m+g}{%
  \IfNoValueTF{#2}
    {\hyperref[skill:#1]{#1}}%no second arg, display is same as link
    {\hyperref[skill:#1]{#2}}%second arg, link to first and display second
}
\NewDocumentCommand\linkfeat{m+g}{%
  \IfNoValueTF{#2}
    {\hyperref[feat:#1]{#1}}%no second arg, display is same as link
    {\hyperref[feat:#1]{#2}}%second arg, link to first and display second
}
\NewDocumentCommand\linkspell{m+g}{%
  \IfNoValueTF{#2}
    {\hyperref[spell:#1]{#1}}%no second arg, display is same as link
    {\hyperref[spell:#1]{#2}}%second arg, link to first and display second
}
\NewDocumentCommand\linkcondition{m+g}{%
  \IfNoValueTF{#2}
    {\hyperref[condition:#1]{#1}}%no second arg, display is same as link
    {\hyperref[condition:#1]{#2}}%second arg, link to first and display second
}
\NewDocumentCommand\linkability{m+g}{%
  \IfNoValueTF{#2}
    {\hyperref[ability:#1]{#1}}%no second arg, display is same as link
    {\hyperref[ability:#1]{#2}}%second arg, link to first and display second
}
\NewDocumentCommand\linksec{m+g}{%
  \IfNoValueTF{#2}
    {\hyperref[sec:#1]{#1}}%no second arg, display is same as link
    {\hyperref[sec:#1]{#2}}%second arg, link to first and display second
}

\begin{document}

%%%%%%%%%%%%%%%%%%%%%%%%%%%%%%%%%%%%%%%%%%%%%%%%%%
%%%%%%%%%%%%%%%%%%%%%%%%%%%%%%%%%%%%%%%%%%%%%%%%%%
%%% Title Page
%%%%%%%%%%%%%%%%%%%%%%%%%%%%%%%%%%%%%%%%%%%%%%%%%%
%%%%%%%%%%%%%%%%%%%%%%%%%%%%%%%%%%%%%%%%%%%%%%%%%%
\thispagestyle{empty}
\begin{center}
\textsc{\Large}\\[0.25cm]
\rule{\linewidth}{0.5mm} \\[0.70cm]
\fontsize{30}{30} \selectfont Tome Reference Document\\[.30cm]
\fontsize{16}{18} \selectfont \guillemotleft{} For that game we all known and love \guillemotright{}\\
\rule{\linewidth}{0.5mm} \\[0.6cm]
%\includegraphics[clip,trim=5cm 2cm 9cm 1cm,width=\linewidth]{OldBookArt--MapImages-173.jpg}
\vfill
{\large \textit{This material is Open Game Content, and is licensed for public use under the terms of the Open Game License v1.0a.}\\
\today}
\end{center}

\pagebreak
\sffamily
\pagestyle{plain}
\raggedbottom

%%%%%%%%%%%%%%%%%%%%%%%%%%%%%%%%%%%%%%%%%%%%%%%%%%
%%%%%%%%%%%%%%%%%%%%%%%%%%%%%%%%%%%%%%%%%%%%%%%%%%
%%% Table of Contents
%%%%%%%%%%%%%%%%%%%%%%%%%%%%%%%%%%%%%%%%%%%%%%%%%%
%%%%%%%%%%%%%%%%%%%%%%%%%%%%%%%%%%%%%%%%%%%%%%%%%%
\renewcommand{\contentsname}{Table of Contents}
\setcounter{tocdepth}{1}
\tableofcontents

%%%%%%%%%%%%%%%%%%%%%%%%%%%%%%%%%%%%%%%%%%%%%%%%%%
%%%%%%%%%%%%%%%%%%%%%%%%%%%%%%%%%%%%%%%%%%%%%%%%%%
%%% Main Content %%%
%%%%%%%%%%%%%%%%%%%%%%%%%%%%%%%%%%%%%%%%%%%%%%%%%%
%%%%%%%%%%%%%%%%%%%%%%%%%%%%%%%%%%%%%%%%%%%%%%%%%%

%% Primary Chapters Here

\clearpage

\input{phb/01-intro}
%%%%%%%%%%%%%%%%%%%%%%%%
%%Race Chapter Formatting
%%%%%%%%%%%%%%%%%%%%%%%%
\newcommand{\race}{placeholder}

\newcommand{\racedescription}[1]{\indent\ability{Physical Description}{#1}}
\newcommand{\racepersonality}[1]{\indent\ability{Personality}{#1}}
\newcommand{\racesociety}[1]{\indent\ability{Society}{#1}}
\newcommand{\racealignment}[1]{\indent\ability{Alignment}{#1}}

\newcommand{\type}[1]{\ability{Type}{#1}\\}
\newcommand{\size}[1]{\ability{Size}{#1}\\}
\newcommand{\speed}[1]{\ability{Speed}{#1 feet}\\}
\newcommand{\scores}[1]{\ability{Racial Ability Score Modifiers}{#1}\\}
\newcommand{\racialtraits}[1]{~\\*\ability{\race ~Special Abilities}{#1}\\}
\newcommand{\racetrait}[2]{\newline\indent\ability{#1}{#2}}
\newcommand{\senses}[1]{\ability{Senses}{#1}\\}
\newcommand{\autolanguages}[1]{\ability{Automatic Languages}{#1}\\}
\newcommand{\bonuslanguages}[1]{\ability{Bonus Languages}{#1}\\}
\newcommand{\favoredclasses}[1]{\ability{Favored Classes}{#1}\\}
\newcommand{\male}[4]{Male &#1 &#2 &#3 &#4\\}
\newcommand{\female}[4]{Female &#1 &#2 &#3 &#4\\}

\newcommand{\racedatastart}{
\noindent
\begin{minipage}[t]{\linewidth}
\vspace{-.5em}
\begin{multicols}{2}
}

\newcommand{\racedataend}{\
\end{multicols}
\end{minipage}
}

\newenvironment{racetable}
{
\tabulinesep=1mm
\noindent
\begin{tabu} to \linewidth {X}
\header\textbf{\race ~Racial Traits} \\ 
\hline
\end{tabu}
\rowcolors{1}{colortwo}{colorone}
\begin{tabu} to \linewidth {X [1, l]}
}{
\hline
\end{tabu}
}

\newcommand{\agetable}[4]{
\columnbreak
\tabulinesep=1mm
\noindent
\begin{tabu} to \linewidth {X}
\vspace{-1em} \\
\header\textbf{\race ~Starting Age} \\ \hline
\end{tabu}
%\vspace{-.15em}
\rowcolors{1}{colortwo}{colorone}
\begin{tabu} to \linewidth {X X X X}
\textbf{Adulthood:} &\textbf{Simple:} &\textbf{Moderate:} &\textbf{Complex:} \\
#1 Years &#2 &#3 &#4 \\ \hline
\end{tabu}
}

\newenvironment{heightweighttable}
{
\tabulinesep=1mm
\noindent
\begin{tabu} to \linewidth {X}
\header\textbf{\race ~Height and Weight} \\ \hline
\end{tabu}
\vspace{-1pt}
\rowcolors{1}{colortwo}{colorone}
\begin{tabu} to \linewidth {X X X X X}
\textbf{Gender} &\textbf{Base Height} &\textbf{Height Mod.} &\textbf{Base Weight} &\textbf{Weight Mod.} \\
}{
\hline
\end{tabu}
}

\newenvironment{raceleft}
{
\vspace{0pt}
\begin{minipage}[t]{0.5\linewidth}
}{
\end{minipage}
}

\newenvironment{raceright}
{\begin{minipage}[t]{0.5\linewidth}
\vspace{0pt}
}{
\end{minipage}
}

\newenvironment{racebox}
{
\vspace{0pt}
%\nointerlineskip
\begin{minipage}{\textwidth}
}{
\end{minipage}
}


\chapter{Races}

%\begin{racebox}
\raceentry{Aasimar}{``My ancestors were more beautiful than you can imagine."}

Aasimar are humans that have a beautiful outsider, usually but not always a celestial, somewhere in their ancestry.

\racedescription{Aasimar look like especially beautiful humans, though they sometimes bear vestiges of their ancestry that denote them as being different (strangely colored eyes, silver-blonder or white hair, slightly `off' facial features).}

\racepersonality{Though mostly human, an aasimar's immortal heritage influences their mental development. Aasimar tend toward more extreme personalities, being especially quiet and introspective or particularly loud and boisterous. Most aasimar are very opinionated, and have strongly held beliefs.}

\racesociety{Aasimar are typically born and raised in human societies, and gain the same customs of that culture}

\racealignment{Most aasimar are the descendants of celestials, and tend towards the good alignments. Rarely, an aasimar might instead have an infernal heritage, being the descendant of an erinyes or succubus. Such aasimar instead tend towards an evil alignment.}

\racedatastart
\begin{racetable}
\type{Outsider (Native and Human Subtype)}
\size{Medium}
\scores{+2 Wisdom, +2 Charisma}
\speed{30}
\senses{Standard}
\autolanguages{Common}
\bonuslanguages{Abyssal, Aquan, Auran, Celestial, Formian, Ignan, Slaad, Sylvan, Terran.}
\favoredclasses{Paladin and Sorcerer}
\end{racetable}

\vspace{\baselineskip}
\agetable{20}{+1d6}{+2d6}{+3d6}

\vspace{\baselineskip}
\begin{heightweighttable}
\male{4' 7"}{+2d8}{90 lb.}{x(2d4)}
\female{4' 5"}{+2d8}{80 lb.}{x(2d4)}
\end{heightweighttable}
\racedataend

\racialtraits{
\racetrait{Inner Light \sla}{An Aasimar with a Charisma of at least 10 may cast \spell{light} once per day, with a caster level equal to their character level.}
\racetrait{Keen Senses}{+2 bonus to Spot, and Listen checks.}
}
%\end{racebox}
%\begin{racebox}
\raceentry{Drow}{``Time to die for the Spider Queen."}

No description until we either obtain or write one.

\racedescription{NYW}

\racepersonality{NYW}

\racesociety{NYW}

\racealignment{NYW}

\racedatastart
\begin{racetable}
\type{Humanoid (Elf Subtype)}
\size{Medium}
\scores{+2 Dexterity, -2 Constitution}
\speed{30}
\senses{Darkvision 120'}
\autolanguages{Elvish}
\bonuslanguages{Abyssal, Beholder, Common, Draconic, Drow Sign Language, Dwarvish, Gnome, Kuo-Toa, Terran, Undercommon}
\favoredclasses{Cleric and Wizard}
\end{racetable}

\vspace{\baselineskip}
\agetable{20}{+1d6}{+2d6}{+3d6}

\vspace{\baselineskip}
\begin{heightweighttable}
\male{4' 7"}{+2d8}{90 lb.}{x(2d4)}
\female{4' 5"}{+2d8}{80 lb.}{x(2d4)}
\end{heightweighttable}
\racedataend

\racialtraits{
\racetrait{Daylight Sensitivity}{While in brightly lit surroundings (such as a daylight spell), a Drow suffers a -2 penalty to attack rolls and precision-based skill checks.}
\racetrait{Innate Magic}{Drow with a Charisma of at least 10 may cast deeper darkness (duration 4 hours), and fairie fire as spell-like abilities with a caster level equal to their character level once per day each.}
\racetrait{Magic Resistant}{+2 bonus to saving throws against spells and spell-like abilities.}
\racetrait{Skill Bonus}{+2 bonus to Spot and Listen checks.}
\racetrait{Elven Trance}{Drow never sleep and are immune to sleep effects. Drow must still perform their 4 hour daily trance to stay coherent and rested.}
\racetrait{Interesting Times}{Drow live an exceedingly interesting life and every Drow has proficiency with the rapier and an exotic ranged weapon of their choice.}
}
%\end{racebox}

\begin{racebox}
\raceentry{Dwarf}{``I remember that..."}
\begin{multicols}{2}

\begin{racetable}
\type{Humanoid (Dwarf subtype)}
\size{Medium}
\scores{+2 Constitution, --2 Charisma}
\speed{20}
\senses{Darkvision}
\autolanguages{Common, Dwarven}
\bonuslanguages{Giant, Gnome, Goblin, Orc, Terran, and Undercommon}
\favoredclasses{Fighter}
\end{racetable}

Dwarves are (unfinished sentence, because I'm guessing that descriptive info is not OGL so this needs to be written wholecloth).

\racedescription{NYW}

\racepersonality{NYW}

\racesociety{NYW}

\racealignment{NYW}

\racialtraits{
\racetrait{Dwarves can move up to their full speed even when wearing medium or heavy armor or when carrying a medium or heavy load.}
\racetrait{Stonecunning}{This ability grants a dwarf a +2 racial bonus on Search checks to notice unusual stonework, such as sliding walls, stonework traps, new construction (even when built to match the old), unsafe stone surfaces, shaky stone ceilings, and the like. Something that isn’t stone but that is disguised as stone also counts as unusual stonework. A dwarf who merely comes within 10 feet of unusual stonework can make a Search check as if he or she were actively searching, and a dwarf can use the Search skill to find stonework traps as a rogue can. A dwarf can also intuit depth, sensing his or her approximate depth underground as naturally as a human can sense which way is up.}
\racetrait{Weapon Familiarity}{Dwarves may treat dwarven waraxes and dwarven urgroshes as martial weapons, rather than exotic weapons.}
\racetrait{Stability}{A dwarf gains a +4 bonus on ability checks made to resist being bull rushed or tripped when standing on the ground (but not when climbing, flying, riding, or otherwise not standing firmly on the ground).}
\racetrait{+2 racial bonus on saving throws against poison.}
\racetrait{+2 racial bonus on saving throws against spells and spell-like effects.}
\racetrait{+1 racial bonus on attack rolls against orcs and goblinoids.}
\racetrait{+4 dodge bonus to Armor Class against monsters of the giant type.}
\racetrait{+2 racial bonus on Appraise checks that are related to stone or metal items.}
\racetrait{+2 racial bonus on Craft checks that are related to stone or metal.}
}


%\columnbreak

\vspace{\baselineskip}
\agetable{40}{+3d6}{+5d6}{+7d6}

\vspace{\baselineskip}
\begin{heightweighttable}
\male{3' 9"}{+2d4}{130 lb.}{x(2d6)}
\female{3' 7"}{+2d4}{100 lb.}{x(2d6)}
\end{heightweighttable}
\end{multicols}
\end{racebox}
\begin{racebox}
\raceentry{Elf}{``You shall never harm my beautiful trees!"}
\begin{multicols}{2}

\begin{racetable}
\type{Humanoid (Elf subtype)}
\size{Medium}
\scores{+2 Dexterity, --2 Constitution}
\speed{30}
\senses{Low Light Vision}
\autolanguages{Common and Elven}
\bonuslanguages{Draconic, Gnoll, Gnome, Goblin, Orc, and Sylvan.}
\favoredclasses{Wizard}
\end{racetable}

No description until we either obtain or write one.

\racedescription{NYW}

\racepersonality{NYW}

\racesociety{NYW}

\racealignment{NYW}

\racialtraits{
\racetrait{Weapon Proficiency}{Elves are proficient with the longsword, rapier, longbow (including composite longbow), and shortbow (including composite shortbow).}
\racetrait{+2 racial bonus on Listen, Search, and Spot checks. An elf who merely passes within 5 feet of a secret or concealed door is entitled to a Search check to notice it as if he or she were actively looking for it.}
}


%\columnbreak

\vspace{\baselineskip}
\agetable{110}{+4d6}{+6d6}{+10d6}

\vspace{\baselineskip}
\begin{heightweighttable}
\male{4' 5"}{+2d6}{85 lb.}{x(1d6)}
\female{4' 5"}{+2d6}{80 lb.}{x(1d6)}
\end{heightweighttable}
\end{multicols}
\end{racebox}
\input{phb/races/feytouched}
%\begin{racebox}
\raceentry{Gnome}{``What's that you say little mole? Kobolds in the well!?"}

Nothing until we find or write it.

\racedescription{NYW}

\racepersonality{NYW}

\racesociety{NYW}

\racealignment{NYW}

\racedatastart
\begin{racetable}
\type{Humanoid (Gnome subtype)}
\size{Small}
\scores{+2 Constitution, --2 Strength}
\speed{20}
\senses{Low Light Vision}
\autolanguages{Common and Gnome}
\bonuslanguages{Draconic, Dwarven, Elven, Giant, Goblin, and Orc. In addition, a gnome can speak with a burrowing mammal (a badger, fox, rabbit, or the like).}
\favoredclasses{Bard}
\end{racetable}

\vspace{\baselineskip}
\agetable{40}{+4d6}{+6d6}{+9d6}

\vspace{\baselineskip}
\begin{heightweighttable}
\male{3' 0"}{+2d4}{40 lb.}{x(1)}
\female{2' 10"}{+2d4}{35 lb.}{x(1)}
\end{heightweighttable}
\racedataend

\racialtraits{
\racetrait{Weapon Familiarity}{Gnomes may treat gnome hooked hammers as martial weapons rather than exotic weapons.}
\racetrait{Desensitized to Illusions}{+2 racial bonus on saving throws against illusions.}
\racetrait{Practiced Illusions}{Add +1 to the Difficulty Class for all saving throws against illusion spells cast by gnomes. This adjustment stacks with those from similar effects.}
\racetrait{Racial Enmity}{+1 racial bonus on attack rolls against kobolds and goblinoids.}
\racetrait{+4 dodge bonus to Armor Class against monsters of the giant type.}
\racetrait{Big Ears}{+2 racial bonus on Listen checks.}
\racetrait{Gnomish Alchemy}{+2 racial bonus on Craft (alchemy) checks.}
\racetrait{Burrowers Tongue \sla}{Gnomes may cast \spell{speak with animals} once per day, though they may only speak with burrowing mammals and only for one minute).}
\racetrait{Innate Illusions \sla}{A gnome with a Charisma score of at least 10 may cast \spell{dancing lights}, \spell{ghost sound}, and \spell{prestidigitation} each once per day, with a caster level equal to their character level.}
}


%\end{racebox}
\input{phb/races/goblin}
\raceentry{Half-Elf}{``I don't fit in anywhere, please, listen to me cry.''}

\listone
		\item Medium Size
		\item 30' Movement
		\item Humanoid Type
		\item Low-Light Vision: Half-Elves can see twice as humans in poor lighting.
		\item Immunity to sleep spells and similar magical effects, and a +2 racial bonus on saving throws against enchantment spells or effects.
		\item +1 racial bonus on Listen, Search, and Spot checks.
		\item +2 racial bonus on Diplomacy and Gather Information checks.
		\item Elven Blood: For all effects related to race, a half-elf is considered an elf.
		\item Favored Class: Any
		\item Automatic Languages: Common and Elven.
		\item Bonus Languages: Any (other than secret languages, such as Druidic).
\end{list}
\raceentry{Halfling}{``Where are we going Mr. Frodo?''}

\listone
		\item Small Size
		\item 20' movement
		\item +2 Dexterity, -2 Strength
		\item +2 racial bonus on Climb, Jump, Listen, and Move Silently checks.
		\item +1 racial bonus on all saving throws.
		\item +2 morale bonus on saving throws against fear: This bonus stacks with the halfling’s +1 bonus on saving throws in general.
		\item +1 racial bonus on attack rolls with thrown weapons and slings.
		\item Favored Class: Rogue
		\item Automatic Languages: Common and Halfling.
		\item Bonus Languages: Dwarven, Elven, Gnome, Goblin, and Orc.
\end{list}		

\input{phb/races/half-orc}
\input{phb/races/hobgoblin}
\input{phb/races/human}
\input{phb/races/kobold}
\input{phb/races/orc}
\input{phb/races/tiefling}
\input{phb/03-classes}
\input{phb/04-skills}
%%%%%%%%%%%%%%%%%%%%%%%%
%%Feats Chapter Formatting
%%%%%%%%%%%%%%%%%%%%%%%%
\newcommand{\combatfeat}[8]{
\belowpdfbookmark{#1}{feat:#1}\paragraph{\Large#1}\normalsize\textbf{#2}

\noindent\textit{#3} \\
%\indent\hypertarget{feat:#1}{}\textbf{#1 [Combat] #2} \\
\indent\ability{+0 BAB}{#4} \\
\indent\ability{+1 BAB}{#5} \\
\indent\ability{+6 BAB}{#6} \\
\indent\ability{+11 BAB}{#7} \\
\indent\ability{+16 BAB}{#8} \\
%\begin{description}
%\item[Benefit:] #4
%\item[BAB +1:] #5
%\item[BAB +6:] #6
%\item[BAB +11:] #7
%\item[BAB +16:] #8
%\end{description}
%~\\*
}

\newcommand{\skillfeat}[8]{
\belowpdfbookmark{#1}{feat:#1}\paragraph{\Large#1}\normalsize\textbf{#2}

\noindent\textit{#3} \\
%\indent\hypertarget{feat:#1}{}\textbf{#1 [Skill:#2] #3} \\
\indent\ability{0 Ranks}{#4} \\
\indent\ability{4 Ranks}{#5} \\
\indent\ability{9 Ranks}{#6} \\
\indent\ability{14 Ranks}{#7} \\
\indent\ability{19 Ranks}{#8} \\
%\begin{description}
%\item[Benefit:] #4
%\item[4 Ranks:] #5
%\item[9 Ranks:] #6
%\item[14 Ranks:] #7
%\item[19 Ranks:] #8
%\end{description}
%~\\*
}

\newcommand{\spellfeat}[8]{
\belowpdfbookmark{#1}{feat:#1}\paragraph{\Large#1}\normalsize\textbf{#2}

\noindent\textit{#3} \\
%\indent\hypertarget{feat:#1}{}\textbf{#1 [Spellcasting] #2} \\
\indent\ability{0th Level}{#4} \\
\indent\ability{1st Level}{#5} \\
\indent\ability{3rd Level}{#6} \\
\indent\ability{6th Level}{#7} \\
\indent\ability{9th Level}{#8} \\
%\begin{description}
%\item[Benefit:] #4
%\item[Level 1:] #5
%\item[Level 3:] #6
%\item[Level 6:] #7
%\item[Level 9:] #8
%\end{description}
%~\\*
}

\newcommand{\genfeat}[3]{
\belowpdfbookmark{#1}{feat:#1}\paragraph{\Large#1}\normalsize\textbf{#2}

\noindent\textit{#3} \\
}

\newcommand{\featprereq}[1]{
\indent\ability{Prerequisite}{#1} \\
}

\newcommand{\featbenefit}[1]{
\indent\ability{Benefit}{#1} \\
}

\newcommand{\featspecial}[1]{
\indent\ability{Special}{#1} \\
}

%%%%%%%%%%%%%%%%%%%%%%%%

\chapter{Feats}
\section{How Feats Work}
foo

\input{phb/feats/generalfeats}
\section{Combat Feats}

%\begin{multicols}{2}

\input{phb/feats/combat/blindfighting}
\input{phb/feats/combat/blitz}
\input{phb/feats/combat/combatlooting}
\input{phb/feats/combat/combatschool}
\input{phb/feats/combat/command}
\input{phb/feats/combat/dangersense}
\input{phb/feats/combat/elusivetarget}
\input{phb/feats/combat/experttactician}
\input{phb/feats/combat/ghosthunter}
\input{phb/feats/combat/giantslayer}
\input{phb/feats/combat/greatfortitude}
\input{phb/feats/combat/hordebreaker}
\input{phb/feats/combat/hunter}
\input{phb/feats/combat/insightfulstrike}
\input{phb/feats/combat/ironwill}
\input{phb/feats/combat/juggernaut}
\input{phb/feats/combat/lightningreflexes}
\input{phb/feats/combat/mageslayer}
\input{phb/feats/combat/murderousintent}
\input{phb/feats/combat/phalanxfighter}
\input{phb/feats/combat/pointblankshot}
\input{phb/feats/combat/sniper}
\input{phb/feats/combat/subtlecut}
\input{phb/feats/combat/twoweaponfighting}
\input{phb/feats/combat/weaponfinesse}
%\input{phb/feats/combat/weaponrighteousdestruction}
\input{phb/feats/combat/whirlwind}
\input{phb/feats/combat/zenarchery}

%\end{multicols}
\section{Skill Feats}

%\begin{multicols}{2}

\input{phb/feats/skill/acquierereye}
\input{phb/feats/skill/acrobatic}
\input{phb/feats/skill/alertness}
\input{phb/feats/skill/animalaffinity}
\input{phb/feats/skill/armyofdemons}
\input{phb/feats/skill/battlefieldsurgeon}
\input{phb/feats/skill/bureaucrat}
\input{phb/feats/skill/combatcasting}
\input{phb/feats/skill/conartist}
\input{phb/feats/skill/cryptographer}
\input{phb/feats/skill/deftfingers}
\input{phb/feats/skill/detective}
\input{phb/feats/skill/dreadfuldemeanor}
\input{phb/feats/skill/expertcounterfeiter}
\input{phb/feats/skill/ghoststep}
\input{phb/feats/skill/investigator}
\input{phb/feats/skill/itemmaster}
\input{phb/feats/skill/leadership}
\input{phb/feats/skill/legendarywrangler}
\input{phb/feats/skill/lordofdeath}
\input{phb/feats/skill/magicalaptitude}
\input{phb/feats/skill/many-faced}
\input{phb/feats/skill/masterofterror}
\input{phb/feats/skill/monsterrancher}
\input{phb/feats/skill/mountedcombat}
\input{phb/feats/skill/naturalempath}
\input{phb/feats/skill/persuasive}
\input{phb/feats/skill/professionalluddite}
\input{phb/feats/skill/sharp-eyed}
\input{phb/feats/skill/slipperycontortionist}
\input{phb/feats/skill/steadystance}
\input{phb/feats/skill/stealthy}
\input{phb/feats/skill/swimlikeafish}
\input{phb/feats/skill/track}
\input{phb/feats/skill/tyrant}

%\end{multicols}
\input{phb/feats/spellcastingfeats}

\input{phb/06-goods-and-services}
\input{phb/07-description}
\input{phb/08-adventuring}
\input{phb/09-your-role-in-the-campaign}
\input{phb/10-combat}
\input{phb/11-magic}
\input{phb/12-magic-items}

%%%%%%%%%%%%%%%%%%%%%%%%%%%%%%%%%%%%%%%%%%%%%%%%%%
%%%%%%%%%%%%%%%%%%%%%%%%%%%%%%%%%%%%%%%%%%%%%%%%%%
\appendix
%%%%%%%%%%%%%%%%%%%%%%%%%%%%%%%%%%%%%%%%%%%%%%%%%%
%%%%%%%%%%%%%%%%%%%%%%%%%%%%%%%%%%%%%%%%%%%%%%%%%%
\appendixpage

\makeatletter
\renewcommand{\@makechapterhead}[1]{%
\vspace*{0 pt}{
\raggedright \normalfont \fontsize{32}{32} \selectfont \bfseries
\ifnum \value{secnumdepth}>-1
  \if@mainmatter \vspace{-8pt} {\fontsize{20}{20} \selectfont Appendix \thechapter:}\\[8pt]
  \fi%
\fi
\hspace{0.65cm} #1\par\nobreak\vspace{20 pt}
}}
\makeatother

\clearpage

%% Appendix Chapters Here

%%%%%%%%%%%%%%%%%%%%%%%%
%%Spell Formatting
%%%%%%%%%%%%%%%%%%%%%%%%

\newenvironment{spellwriteup}{
\vspace{0pt}
\begin{minipage}[h]{\linewidth}
}{
\end{minipage}
}

\newenvironment{spellinformation}{
\begin{wraptable}[10]{R}{.5\linewidth}
\begin{tabular}{|r p{1.75in}|}
\hline
}{
\hline
\end{tabular}
\end{wraptable}
}

\newcommand{\spellstart}{
\begin{wraptable}{L!}{.45\linewidth}
\vspace{-10pt}
\begin{tabular}{|r p{.55\linewidth}|}
}

\newcommand{\spellend}{
\hline
\end{tabular}
\vspace{-12pt}
\end{wraptable}
\vspace{-12pt}
}

\newcommand{\spellhead}[4]{\vspace{12pt}\belowpdfbookmark{#1}{spell:#1}\paragraph{\Large#1}\large\textbf{#2 (#3) [#4]}\normalsize \\}

%\paragraph{\Large#1}\large\textbf{#2}

%\newcommand{\spellschool}[3]{\multicolumn{2}{l}{\textbf{#1 (#2) [#3]}}\\}
% unused

\newcommand{\spelllevel}[1]{\textbf{Level:} & #1\\}
\newcommand{\spellcomp}[1]{\textbf{Components:} & #1\\}
\newcommand{\spellcast}[1]{\textbf{Casting Time:} & #1\\}
\newcommand{\spellrange}[1]{\textbf{Range:} & #1\\}
\newcommand{\spelleffect}[1]{\textbf{Effect:} & #1\\}
\newcommand{\spellarea}[1]{\textbf{Area:} & #1\\}
\newcommand{\spelltarget}[1]{\textbf{Target:} & #1\\}
\newcommand{\spelltargets}[1]{\textbf{Targets:} & #1\\}
\newcommand{\spellduration}[1]{\textbf{Duration:} & #1\\}
\newcommand{\spellsave}[1]{\textbf{Saving Throw:} & #1\\}
\newcommand{\spellsr}[1]{\textbf{Spell Resistance:} & #1\\}

\newcommand{\component}[1]{\textit{Material Component:} #1}
\newcommand{\focus}[1]{\textit{Focus:} #1}

%%%%%%%%%%%%%%%%%%%%%%%%

\chapter{Spells}
\section{Spells, A through Z}
%\begin{spellwriteup}
\spellhead{Acid Arrow}{Conjuration}{Creation}{Acid}
\begin{spellinformation}
\spelllevel{Sor/Wiz 2}
\spellcomp{V, S, M, F}
\spellcast{1 standard action}
\spellrange{Long (400 ft. + 40 ft / level)}
\spelleffect{One arrow of acid}
\spellduration{1 round + 1 round / three levels}
\spellsave{None}
\spellsr{No}
\end{spellinformation}
A magical arrow of acid springs from your hand and speeds to its target. You must succeed on a ranged touch attack to hit your target. The arrow deals 2d4 points of acid damage with no splash damage. For every three caster levels (to a maximum of 18th), the acid, unless somehow neutralized, lasts for another round, dealing another 2d4 points of damage in that round.

\component{Powdered rhubarb leaf and an adder's stomach.}

\focus{A dart.}
%\end{spellwriteup}
%\begin{spellwriteup}
\spellhead{Acid Fog}{Conjuration}{Creation}{Acid}
\begin{spellinformation}
\spelllevel{Sor/Wiz 6, Water 7}
\spellcomp{V, S, M/DF}
\spellcast{1 standard action}
\spellrange{Medium (100 ft. + 10 ft. / level)}
\spelleffect{Fog spreads in 20-ft. radius, 20 ft. high}
\spellduration{1 round / level}
\spellsave{None}
\spellsr{No}
\end{spellinformation}
Acid Fog creates a billowing mass of misty vapors similar to that produced by a \linkspell{Solid Fog} spell. In addition to slowing creatures down and obscuring sight, this spell's vapors are highly acidic. Each round on your turn, starting when you cast the spell, the fog deals 2d6 points of acid damage to each creature and object within it.

\component{A pinch of dried, powdered peas combined with powdered animal hoof.}
%\end{spellwriteup}
%\begin{spellwriteup}
\spellhead{Acid Fog}{Conjuration}{Creation}{Acid}
\begin{spellinformation}
\spelllevel{Sor/Wiz 6, Water 7}
\spellcomp{V, S, M/DF}
\spellcast{1 standard action}
\spellrange{Medium (100 ft. + 10 ft. / level)}
\spelleffect{Fog spreads in 20-ft. radius, 20 ft. high}
\spellduration{1 round / level}
\spellsave{None}
\spellsr{No}
\end{spellinformation}
Acid Fog creates a billowing mass of misty vapors similar to that produced by a \linkspell{Solid Fog} spell. In addition to slowing creatures down and obscuring sight, this spell's vapors are highly acidic. Each round on your turn, starting when you cast the spell, the fog deals 2d6 points of acid damage to each creature and object within it.

\component{A pinch of dried, powdered peas combined with powdered animal hoof.}
%\end{spellwriteup}


\chapter{Prestige Classes}
\section{Prestige Class Basics}
\section{WhatClasses}

\chapter{Monsters}
\section{Reading a Monster Entry}
\section{Monsters, A though Z}

\chapter{NPC Classes}
\section{Adept}
foo
\section{Aristocrat}
foo
\section{Commoner}
foo
\section{Expert}
foo
\section{Warrior}
foo

\input{tome-srd-ogl}

\clearpage
\phantomsection
\listoftables

\clearpage
\phantomsection
\printindex

\end{document}