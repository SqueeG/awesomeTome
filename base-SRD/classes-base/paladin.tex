%%%%%%%%%%%%%%%%%%%%%%%%%%%%%%%%%%%%%%%%%%%%%%%%%%
\classentry{Paladin}
%%%%%%%%%%%%%%%%%%%%%%%%%%%%%%%%%%%%%%%%%%%%%%%%%%

\textbf{Alignment:} Lawful good.

\textbf{Hit Die:} d10.

\textbf{Class Skills}

The paladin's class skills (and the key ability for each skill) are \linkskill{Concentration} 
(Con), \linkskill{Craft} (Int), \linkskill{Diplomacy} (Cha), \linkskill{Handle Animal} (Cha), \linkskill{Heal} (Wis), \linkskill{Knowledge} 
(nobility and royalty) (Int), \linkskill{Knowledge} (religion) (Int), \linkskill{Profession} (Wis), \linkskill{Ride} 
(Dex), and \linkskill{Sense Motive} (Wis).

\textbf{Skill Points at 1st Level:} (2 + Int modifier) x4.

\textbf{Skill Points at Each Additional Level:} 2 + Int modifier.

\begin{table}[htb]
\rowcolors{1}{white}{offyellow}
\caption{The Paladin}
\centering
\begin{tabular}{*{6}{l}*{4}{c}}
\textbf{Level} & \textbf{BAB} & \textbf{Fort} & \textbf{Reflex} & \textbf{Will} & \textbf{Special} & \textbf{1st} & \textbf{2nd} & \textbf{3rd} & \textbf{4th} \\
1st & +1 & +2 & +0 & +0 & Aura of Good, Detect Evil, Smite Evil 1/day & -- & -- & -- & -- \\
2nd & +2 & +3 & +0 & +0 & Divine Grace, Lay On Hands & -- & -- & -- & -- \\
3rd & +3 & +3 & +1 & +1 & Aura of Courage, Divine Health & -- & -- & -- & -- \\
4th & +4 & +4 & +1 & +1 & Turn Undead & 0 & -- & -- & -- \\
5th & +5 & +4 & +1 & +1 & Smite Evil 2/day, Special Mount & 0 & -- & -- & -- \\
6th & +6 & +5 & +2 & +2 & Remove Disease 1/week & 1 & -- & -- & -- \\
7th & +7 & +5 & +2 & +2 & -- & 1 & -- & -- & -- \\
8th & +8 & +6 & +2 & +2 & -- & 1 & 0 & -- & -- \\
9th & +9 & +6 & +3 & +3 & Remove Disease 2/week & 1 & 0 & -- & -- \\
10th & +10 & +7 & +3 & +3 & Smite Evil 3/day & 1 & 1 & -- & -- \\
11th & +11 & +7 & +3 & +3 & -- & 1 & 1 & 0 & -- \\
12th & +12 & +8 & +4 & +4 & Remove Disease 3/week & 1 & 1 & 1 & -- \\
13th & +13 & +8 & +4 & +4 & -- & 1 & 1 & 1 & -- \\
14th & +14 & +9 & +4 & +4 & -- & 2 & 1 & 1 & 0 \\
15th & +15 & +9 & +5 & +5 & Remove Disease 4/week, Smite Evil 4/day & 2 & 1 & 1 & 1 \\
16th & +16 & +10 & +5 & +5 & -- & 2 & 2 & 1 & 1 \\
17th & +17 & +10 & +5 & +5 & -- & 2 & 2 & 2 & 1 \\
18th & +18 & +11 & +6 & +6 & Remove Disease 5/week & 3 & 2 & 2 & 1 \\
19th & +19 & +11 & +6 & +6 & -- & 3 & 3 & 3 & 2 \\
20th & +20 & +12 & +6 & +6 & Smite Evil 5/day & 3 & 3 & 3 & 3 \\
\end{tabular}
\end{table}

%%%%%%%%%%%%%%%%%%%%%%%%%
\ClassFeatures
%%%%%%%%%%%%%%%%%%%%%%%%%

All of the following are class features of the paladin.

\textbf{Weapon and Armor Proficiency:} Paladins are proficient with all simple 
and martial weapons, with all types of armor (heavy, medium, and light), and with 
shields (except tower shields).

\textbf{Aura of Good (Ex):} The power of a paladin's aura of good (see the \linkspell{Detect Good} spell) is equal to her paladin level.

\textbf{Detect Evil (Sp):} At will, a paladin can use \linkspell{Detect Evil}, as the spell.

\textbf{Smite Evil (Su):} Once per day, a paladin may attempt to smite evil with 
one normal melee attack. She adds her Charisma bonus (if any) to her attack roll 
and deals 1 extra point of damage per paladin level. If the paladin accidentally 
smites a creature that is not evil, the smite has no effect, but the ability is 
still used up for that day.

At 5th level, and at every five levels thereafter, the paladin may smite evil one 
additional time per day, as indicated on Table: The Paladin, to a maximum of five 
times per day at 20th level.

\textbf{Divine Grace (Su):} At 2nd level, a paladin gains a bonus equal to her 
Charisma bonus (if any) on all saving throws.

\textbf{Lay on Hands (Su):} Beginning at 2nd level, a paladin with a Charisma score 
of 12 or higher can heal wounds (her own or those of others) by touch. Each day 
she can heal a total number of hit points of damage equal to her paladin level 
x her Charisma bonus. A paladin may choose to divide her healing among multiple 
recipients, and she doesn't have to use it all at once. Using lay on hands is a 
standard action.

Alternatively, a paladin can use any or all of this healing power to deal damage 
to undead creatures. Using lay on hands in this way requires a successful melee 
touch attack and doesn't provoke an attack of opportunity. The paladin decides 
how many of her daily allotment of points to use as damage after successfully touching 
an undead creature.

\textbf{Aura of Courage (Su):} Beginning at 3rd level, a paladin is immune to fear 
(magical or otherwise). Each ally within 10 feet of her gains a +4 morale bonus 
on saving throws against fear effects.

This ability functions while the paladin is conscious, but not if she is unconscious 
or dead.

\textbf{Divine Health (Ex):} At 3rd level, a paladin gains immunity to all diseases, 
including supernatural and magical diseases.

\textbf{Turn Undead (Su):} When a paladin reaches 4th level, she gains the supernatural 
ability to turn undead. She may use this ability a number of times per day equal 
to 3 + her Charisma modifier. She turns undead as a cleric of three levels lower 
would.

\textbf{Spells:} Beginning at 4th level, a paladin gains the ability to cast a 
small number of divine spells, which are drawn from the paladin spell list. A paladin 
must choose and prepare her spells in advance.

To prepare or cast a spell, a paladin must have a Wisdom score equal to at least 
10 + the spell level. The Difficulty Class for a saving throw against a paladin's 
spell is 10 + the spell level + the paladin's Wisdom modifier.

Like other spellcasters, a paladin can cast only a certain number of spells of 
each spell level per day. Her base daily spell allotment is given on Table: The 
Paladin. In addition, she receives bonus spells per day if she has a high Wisdom 
score. When Table: The Paladin indicates that the paladin gets 0 spells per day 
of a given spell level, she gains only the bonus spells she would be entitled to 
based on her Wisdom score for that spell level The paladin does not have access 
to any domain spells or granted powers, as a cleric does.

A paladin prepares and casts spells the way a cleric does, though she cannot lose 
a prepared spell to spontaneously cast a \textit{cure} spell in its place. A paladin 
may prepare and cast any spell on the paladin spell list, provided that she can 
cast spells of that level, but she must choose which spells to prepare during her 
daily meditation.

Through 3rd level, a paladin has no caster level. At 4th level and higher, her 
caster level is one-half her paladin level.

\textbf{Special Mount (Sp):} Upon reaching 5th level, a paladin 
gains the service of an unusually intelligent, strong, and loyal steed to serve 
her in her crusade against evil (see below). This mount is usually a heavy warhorse 
(for a Medium paladin) or a warpony (for a Small paladin).

Once per day, as a full-round action, a paladin may magically call her mount from 
the celestial realms in which it resides. This ability is the equivalent of a spell 
of a level equal to one-third the paladin's level. The mount immediately appears 
adjacent to the paladin and remains for 2 hours per paladin level; it may be dismissed 
at any time as a free action. The mount is the same creature each time it is summoned, 
though the paladin may release a particular mount from service.

Each time the mount is called, it appears in full health, regardless of any damage 
it may have taken previously. The mount also appears wearing or carrying any gear 
it had when it was last dismissed. Calling a mount is a conjuration (calling) effect.

Should the paladin's mount die, it immediately disappears, leaving behind any equipment 
it was carrying. The paladin may not summon another mount for thirty days or until 
she gains a paladin level, whichever comes first, even if the mount is somehow 
returned from the dead. During this thirty-day period, the paladin takes a -1 penalty 
on attack and weapon damage rolls.

\textbf{Remove Disease (Sp):} At 6th level, a paladin can produce 
a \linkspell{Remove Disease} effect, as the spell, once per week. She can use this 
ability one additional time per week for every three levels after 6th (twice per 
week at 9th, three times at 12th, and so forth).

\textbf{Code of Conduct:} A paladin must be of lawful good alignment and loses 
all class abilities if she ever willingly commits an evil act.

Additionally, a paladin's code requires that she respect legitimate authority, 
act with honor (not lying, not cheating, not using poison, and so forth), help 
those in need (provided they do not use the help for evil or chaotic ends), and 
punish those who harm or threaten innocents.

\textbf{Associates:} While she may adventure with characters of any good or neutral 
alignment, a paladin will never knowingly associate with evil characters, nor will 
she continue an association with someone who consistently offends her moral code. 
A paladin may accept only henchmen, followers, or cohorts who are lawful good.

%%%%%%%%%%%%%%%%%%%%%%%%%
\subsection{Ex-Paladin}
%%%%%%%%%%%%%%%%%%%%%%%%%

A paladin who ceases to be lawful good, who willfully commits an evil act, or who 
grossly violates the code of conduct loses all paladin spells and abilities (including 
the service of the paladin's mount, but not weapon, armor, and shield proficiencies). 
She may not progress any farther in levels as a paladin. She regains her abilities 
and advancement potential if she atones for her violations (see the \linkspell{Atonement} spell description), as appropriate.

Like a member of any other class, a paladin may be a multiclass character, but 
multiclass paladins face a special restriction. A paladin who gains a level in 
any class other than paladin may never again raise her paladin level, though she 
retains all her paladin abilities.

%%%%%%%%%%%%%%%%%%%%%%%%%
\subsection{The Paladin's Mount}\index{Paladin's Mount}
%%%%%%%%%%%%%%%%%%%%%%%%%

The paladin's mount is superior to a normal mount of its kind and has special powers, 
as described below. The standard mount for a Medium paladin is a heavy warhorse, 
and the standard mount for a Small paladin is a warpony. Another kind of mount, 
such as a riding dog (for a halfling paladin) or a Large shark (for a paladin in 
an aquatic campaign) may be allowed as well.

A paladin's mount is treated as a magical beast, not an animal, for the purpose 
of all effects that depend on its type (though it retains an animal's HD, base 
attack bonus, saves, skill points, and feats).

\begin{table}[htb]
\rowcolors{1}{white}{offyellow}
\caption{Paladin Mount Progression}
\centering
\begin{tabular}{l c c c c p{5cm}}
\textbf{Level} & \textbf{Bonus HD} & \textbf{Nat Armor} & \textbf{Str} & \textbf{Int} & \textbf{Special}\\
5th-7th & +2 & +4 & +1 & 6 & Empathic Link, Improved Evasion, Share Spells, Share Saving Throws\\
8rd-10th & +4 & +6 & +2 & 7 & Improved Speed\\
11th-14th & +6 & +8 & +3 & 8 & Command creatures of its kind\\
15th-20th & +8 & +10 & +4 & 9 & Spell Resistance\\
\end{tabular}
\end{table}

\textbf{Paladin's Mount Basics:} Use the base statistics for a creature of the 
mount's kind, but make changes to take into account the attributes and 
characteristics summarized on the table and described below.

\textit{Bonus HD:} Extra eight-sided (d8) Hit Dice, each of which gains a Constitution 
modifier, as normal. Extra Hit Dice improve the mount's base attack and base save 
bonuses. A special mount's base attack bonus is equal to that of a cleric of a 
level equal to the mount's HD. A mount has good Fortitude and Reflex saves (treat 
it as a character whose level equals the animal's HD). The mount gains additional 
skill points or feats for bonus HD as normal for advancing a monster's Hit Dice.

\textit{Natural Armor Adj.:} The number on the table is an improvement to the mount's 
existing natural armor bonus.

\textit{Str Adj.:} Add this figure to the mount's Strength score.

\textit{Int:} The mount's Intelligence score.

\textit{Empathic Link (Su):} The paladin has an empathic link with her mount out 
to a distance of up to 1 mile. The paladin cannot see through the mount's eyes, 
but they can communicate empathically.

Note that even intelligent mounts see the world differently from humans, so misunderstandings 
are always possible.

Because of this empathic link, the paladin has the same connection to an item or 
place that her mount does, just as with a master and his familiar (see Familiars).

\textit{Improved Evasion (Ex):} When subjected to an attack that normally allows 
a Reflex saving throw for half damage, a mount takes no damage if it makes a successful 
saving throw and half damage if the saving throw fails.

\textit{Share Spells:} At the paladin's option, she may have any spell (but not 
any spell-like ability) she casts on herself also affect her mount. 

The mount must be within 5 feet at the time of casting to receive the benefit. 
If the spell or effect has a duration other than instantaneous, it stops affecting 
the mount if it moves farther than 5 feet away and will not affect the mount again 
even if it returns to the paladin before the duration expires. Additionally, the 
paladin may cast a spell with a target of "You" on her mount (as a touch range 
spell) instead of on herself. A paladin and her mount can share spells even if 
the spells normally do not affect creatures of the mount's type (magical beast).

\textit{Share Saving Throws:} For each of its saving throws, the mount uses its 
own base save bonus or the paladin's, whichever is higher. The mount applies its 
own ability modifiers to saves, and it doesn't share any other bonuses on saves 
that the master might have.

\textit{Improved Speed (Ex):} The mount's speed increases by 10 feet.

\textit{Command (Sp):} Once per day per two paladin levels of its master, a mount 
can use this ability to command other any normal animal of approximately the same 
kind as itself (for warhorses and warponies, this category includes donkeys, mules, 
and ponies), as long as the target creature has fewer Hit Dice than the mount. 
This ability functions like the \linkspell{Command} spell, but the mount must make 
a DC 21 \linkskill{Concentration} check to succeed if it's being ridden at the time. If the 
check fails, the ability does not work that time, but it still counts against the 
mount's daily uses. Each target may attempt a Will save (DC 10 + 1/2 paladin's 
level + paladin's Cha modifier) to negate the effect.

\textit{Spell Resistance (Ex): }A mount's spell resistance equals its master's 
paladin level + 5. To affect the mount with a spell, a spellcaster must get a result 
on a caster level check (1d20 + caster level) that equals or exceeds the mount's 
spell resistance.
