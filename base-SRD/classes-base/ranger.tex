%%%%%%%%%%%%%%%%%%%%%%%%%%%%%%%%%%%%%%%%%%%%%%%%%%
\classentry{Ranger}
%%%%%%%%%%%%%%%%%%%%%%%%%%%%%%%%%%%%%%%%%%%%%%%%%%

\textbf{Alignment:} Any.

\textbf{Hit Die:} d8.

\textbf{Class Skills}

The ranger's class skills (and the key ability for each skill) are \linkskill{Climb} (Str), 
\linkskill{Concentration} (Con), \linkskill{Craft} (Int), \linkskill{Handle Animal} (Cha), \linkskill{Heal} (Wis), \linkskill{Hide} (Dex), 
\linkskill{Jump} (Str), \linkskill{Knowledge} (dungeoneering) (Int), \linkskill{Knowledge} (geography) (Int), \linkskill{Knowledge} 
(nature) (Int), \linkskill{Listen} (Wis), \linkskill{Move Silently} (Dex), \linkskill{Profession} (Wis), \linkskill{Ride} (Dex), 
\linkskill{Search} (Int), \linkskill{Spot} (Wis), \linkskill{Survival} (Wis), \linkskill{Swim} (Str), and \linkskill{Use Rope} (Dex).

\textbf{Skill Points at 1st Level:} (6 + Int modifier) x4.

\textbf{Skill Points at Each Additional Level:} 6 + Int modifier.

\begin{table}[htb]
\rowcolors{1}{white}{offyellow}
\caption{The Ranger}
\centering
\begin{tabular}{*{6}{l}*{4}{c}}
\textbf{Level} & \textbf{BAB} & \textbf{Fort} & \textbf{Reflex} & \textbf{Will} & \textbf{Special} & \textbf{1st} & \textbf{2nd} & \textbf{3rd} & \textbf{4th} \\
1st & +1 & +2 & +2 & +0 & 1st Favored Enemy, Track, Wild Empathy & -- & -- & -- & -- \\
2nd & +2 & +3 & +3 & +0 & Combat Style & -- & -- & -- & -- \\
3rd & +3 & +3 & +3 & +1 & Endurance & -- & -- & -- & -- \\
4th & +4 & +4 & +4 & +1 & Animal Companion & 0 & -- & -- & -- \\
5th & +5 & +4 & +4 & +1 & 2nd Favored Enemy & 0 & -- & -- & -- \\
6th & +6 & +5 & +5 & +2 & Improved Combat Style & 1 & -- & -- & -- \\
7th & +7 & +5 & +5 & +2 & Woodland Stride & 1 & -- & -- & -- \\
8th & +8 & +6 & +6 & +2 & Swift Tracker & 1 & 0 & -- & -- \\
9th & +9 & +6 & +6 & +3 & Evasion & 1 & 0 & -- & -- \\
10th & +10 & +7 & +7 & +3 & 3rd Favored Enemy & 1 & 1 & -- & -- \\
11th & +11 & +7 & +7 & +3 & Combat Style Mastery & 1 & 1 & 0 & -- \\
12th & +12 & +8 & +8 & +4 & -- & 1 & 1 & 1 & -- \\
13th & +13 & +8 & +8 & +4 & Camouflage & 1 & 1 & 1 & -- \\
14th & +14 & +9 & +9 & +4 & -- & 2 & 1 & 1 & 0 \\
15th & +15 & +9 & +9 & +5 & 4th Favored Enemy & 2 & 1 & 1 & 1 \\
16th & +16 & +10 & +10 & +5 & -- & 2 & 2 & 1 & 1 \\
17th & +17 & +10 & +10 & +5 & Hide In Plain Sight & 2 & 2 & 2 & 1 \\
18th & +18 & +11 & +11 & +6 & -- & 3 & 2 & 2 & 1 \\
19th & +19 & +11 & +11 & +6 & -- & 3 & 3 & 3 & 2 \\
20th & +20 & +12 & +12 & +6 & 5th Favored Enemy & 3 & 3 & 3 & 3 \\
\end{tabular}
\end{table}

%%%%%%%%%%%%%%%%%%%%%%%%%
\ClassFeatures
%%%%%%%%%%%%%%%%%%%%%%%%%

All of the following are class features of the ranger.

\textbf{Weapon and Armor Proficiency:} A ranger is proficient with all simple and 
martial weapons, and with light armor and shields (except tower shields).

\textbf{Favored Enemy (Ex):} At 1st level, a ranger may select a type of creature 
from among those given on Table: Ranger Favored Enemies. The ranger gains a +2 
bonus on \linkskill{Bluff}, \linkskill{Listen}, \linkskill{Sense Motive}, \linkskill{Spot}, and \linkskill{Survival} checks when using these 
skills against creatures of this type. Likewise, he gets a +2 bonus on weapon damage 
rolls against such creatures.

At 5th level and every five levels thereafter (10th, 15th, and 20th level), the 
ranger may select an additional favored enemy from those given on the table. In 
addition, at each such interval, the bonus against any one favored enemy (including 
the one just selected, if so desired) increases by 2. 

If the ranger chooses humanoids or outsiders as a favored enemy, he must also choose 
an associated subtype, as indicated on the table. If a specific creature falls 
into more than one category of favored enemy, the ranger's bonuses do not stack; 
he simply uses whichever bonus is higher.

\begin{table}[htb]
\rowcolors{1}{white}{offyellow}
\caption{Ranger Favored Enemies}
\centering
\begin{tabular}{l l}
\textbf{Type (Subtype)} & \textbf{Type (Subtype)}\\
Abberation & Humanoid (Reptilian)\\
Animal & Magical Beast\\
Construct & Monstrous Humanoid\\
Dragon & Ooze\\
Elemental & Outsider (Air)\\
Fey & Outsider (Chaotic)\\
Giant & Outisder (Earth)\\
Humanoid (Aquatic) & Outsider (Evil)\\
Humanoid (Dwarf) & Outsider (Fire)\\
Humanoid (Elf) & Outsider (Good)\\
Humanoid (Goblinoid) & Outsider (Lawful)\\
Humanoid (Gnoll) & Outsider (Native)\\
Humanoid (Gnome) & Outsider (Water)\\
Humanoid (Halfling) & Plant\\
Humanoid (Human) & Undead\\
Humanoid (Orc) & Vermin\\
\end{tabular}
\end{table}

\textbf{Track:} A ranger gains Track as a bonus feat.

\textbf{Wild Empathy (Ex):} A ranger can improve the attitude of an animal. This 
ability functions just like a \linkskill{Diplomacy} check to improve the attitude of a person. 
The ranger rolls 1d20 and adds his ranger level and his Charisma bonus to determine 
the wild empathy check result. The typical domestic animal has a starting attitude 
of indifferent, while wild animals are usually unfriendly.

To use wild empathy, the ranger and the animal must be able to study each other, 
which means that they must be within 30 feet of one another under normal visibility 
conditions. Generally, influencing an animal in this way takes 1 minute, but, as 
with influencing people, it might take more or less time.

The ranger can also use this ability to influence a magical beast with an Intelligence 
score of 1 or 2, but he takes a -4 penalty on the check.

\textbf{Combat Style (Ex):} At 2nd level, a ranger must select one of two combat 
styles to pursue: archery or two-weapon combat. This choice affects the character's 
class features but does not restrict his selection of feats or special abilities 
in any way.

If the ranger selects archery, he is treated as having the Rapid Shot feat, even 
if he does not have the normal prerequisites for that feat.

If the ranger selects two-weapon combat, he is treated as having the Two-Weapon 
Fighting feat, even if he does not have the normal prerequisites for that feat.

The benefits of the ranger's chosen style apply only when he wears light or no 
armor. He loses all benefits of his combat style when wearing medium or heavy armor.

\textbf{Endurance:} A ranger gains Endurance as a bonus feat at 3rd level.

\textbf{Animal Companion (Ex):} At 4th level, a ranger gains an animal companion 
selected from the following list: badger, camel, dire rat, dog, riding dog, eagle, 
hawk, horse (light or heavy), owl, pony, snake (Small or Medium viper), or wolf. 
If the campaign takes place wholly or partly in an aquatic environment, the following 
creatures may be added to the ranger's list of options: crocodile, porpoise, Medium 
shark, and squid. This animal is a loyal companion that accompanies the ranger 
on his adventures as appropriate for its kind.

This ability functions like the druid ability of the same name, except that the 
ranger's effective druid level is one-half his ranger level. A ranger may select 
from the alternative lists of animal companions just as a druid can, though again 
his effective druid level is half his ranger level. Like a druid, a ranger cannot 
select an alternative animal if the choice would reduce his effective druid level 
below 1st.

\textbf{Spells:} Beginning at 4th level, a ranger gains the ability to cast a small 
number of divine spells, which are drawn from the ranger spell list. A ranger must 
choose and prepare his spells in advance (see below).

To prepare or cast a spell, a ranger must have a Wisdom score equal to at least 
10 + the spell level. The Difficulty Class for a saving throw against a ranger's 
spell is 10 + the spell level + the ranger's Wisdom modifier.

Like other spellcasters, a ranger can cast only a certain number of spells of each 
spell level per day. His base daily spell allotment is given on Table: The Ranger. 
In addition, he receives bonus spells per day if he has a high Wisdom score. When 
Table: The Ranger indicates that the ranger gets 0 spells per day of a given spell 
level, he gains only the bonus spells he would be entitled to based on his Wisdom 
score for that spell level. The ranger does not have access to any domain spells 
or granted powers, as a cleric does.

A ranger prepares and casts spells the way a cleric does, though he cannot lose 
a prepared spell to cast a \textit{cure} spell in its place. A ranger may prepare 
and cast any spell on the ranger spell list, provided that he can cast spells of 
that level, but he must choose which spells to prepare during his daily meditation.

Through 3rd level, a ranger has no caster level. At 4th level and higher, his caster 
level is one-half his ranger level.

\textbf{Improved Combat Style (Ex):} At 6th level, a ranger's aptitude in his chosen 
combat style (archery or two-weapon combat) improves. If he selected archery at 
2nd level, he is treated as having the Manyshot feat, even if he does not have 
the normal prerequisites for that feat.

If the ranger selected two-weapon combat at 2nd level, he is treated as having 
the Improved Two-Weapon Fighting feat, even if he does not have the normal prerequisites 
for that feat.

As before, the benefits of the ranger's chosen style apply only when he wears light 
or no armor. He loses all benefits of his combat style when wearing medium or heavy 
armor.

\textbf{Woodland Stride (Ex):} Starting at 7th level, a ranger may move through 
any sort of undergrowth (such as natural thorns, briars, overgrown areas, and similar 
terrain) at his normal speed and without taking damage or suffering any other impairment.

However, thorns, briars, and overgrown areas that are enchanted or magically manipulated 
to impede motion still affect him.

\textbf{Swift Tracker (Ex):} Beginning at 8th level, a ranger can move at his normal 
speed while following tracks without taking the normal -5 penalty. He takes only 
a -10 penalty (instead of the normal -20) when moving at up to twice normal speed 
while tracking.

\textbf{Evasion (Ex):} At 9th level, a ranger can avoid even magical and unusual 
attacks with great agility. If he makes a successful Reflex saving throw against 
an attack that normally deals half damage on a successful save, he instead takes 
no damage. Evasion can be used only if the ranger is wearing light armor or no 
armor. A helpless ranger does not gain the benefit of evasion.

\textbf{Combat Style Mastery (Ex):} At 11th level, a ranger's aptitude in his chosen 
combat style (archery or two-weapon combat) improves again. If he selected archery 
at 2nd level, he is treated as having the Improved Precise Shot feat, even if he 
does not have the normal prerequisites for that feat.

If the ranger selected two-weapon combat at 2nd level, he is treated as having 
the Greater Two-Weapon Fighting feat, even if he does not have the normal prerequisites 
for that feat.

As before, the benefits of the ranger's chosen style apply only when he wears light 
or no armor. He loses all benefits of his combat style when wearing medium or heavy 
armor.

\textbf{Camouflage (Ex):} A ranger of 13th level or higher can use the \linkskill{Hide} skill 
in any sort of natural terrain, even if the terrain doesn't grant cover or concealment.

\textbf{Hide in Plain Sight (Ex):} While in any sort of natural terrain, a ranger 
of 17th level or higher can use the \linkskill{Hide} skill even while being observed.
