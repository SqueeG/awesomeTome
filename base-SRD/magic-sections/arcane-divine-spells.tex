%%%%%%%%%%%%%%%%%%%%%%%%%%%%%%%%%%%%%%%%%%%%%%%%%%
\section{Arcane Spells}
%%%%%%%%%%%%%%%%%%%%%%%%%%%%%%%%%%%%%%%%%%%%%%%%%%

Wizards, sorcerers, and bards cast arcane spells. Compared to divine spells, arcane 
spells are more likely to produce dramatic results.

%%%%%%%%%%%%%%%%%%%%%%%%%
\subsection{Preparing Wizard Spells}\index{Spell Preperation!Arcane}
%%%%%%%%%%%%%%%%%%%%%%%%%

A wizard's level limits the number of spells she can prepare and cast. Her high 
Intelligence score might allow her to prepare a few extra spells. She can prepare 
the same spell more than once, but each preparation counts as one spell toward 
her daily limit. To prepare a spell the wizard must have an Intelligence score 
of at least 10 + the spell's level.

\textbf{Rest:} To prepare her daily spells, a wizard must first sleep for 8 hours. 
The wizard does not have to slumber for every minute of the time, but she must 
refrain from movement, combat, spellcasting, skill use, conversation, or any other 
fairly demanding physical or mental task during the rest period. If her rest is 
interrupted, each interruption adds 1 hour to the total amount of time she has 
to rest in order to clear her mind, and she must have at least 1 hour of uninterrupted 
rest immediately prior to preparing her spells. If the character does not need 
to sleep for some reason, she still must have 8 hours of restful calm before preparing 
any spells. 

\textbf{Recent Casting Limit/Rest Interruptions:} If a wizard has cast spells recently, 
the drain on her resources reduces her capacity to prepare new spells. When she 
prepares spells for the coming day, all the spells she has cast within the last 
8 hours count against her daily limit.

\textbf{Preparation Environment:} To prepare any spell, a wizard must have enough 
peace, quiet, and comfort to allow for proper concentration. The wizard's surroundings 
need not be luxurious, but they must be free from overt distractions. Exposure 
to inclement weather prevents the necessary concentration, as does any injury or 
failed saving throw the character might experience while studying. Wizards also 
must have access to their spellbooks to study from and sufficient light to read 
them by. There is one major exception: A wizard can prepare a \linkspell{Read Magic} 
spell even without a spellbook. 

\textbf{Spell Preparation Time:} After resting, a wizard must study her spellbook 
to prepare any spells that day. If she wants to prepare all her spells, the process 
takes 1 hour. Preparing some smaller portion of her daily capacity takes a proportionally 
smaller amount of time, but always at least 15 minutes, the minimum time required 
to achieve the proper mental state.

\textbf{Spell Selection and Preparation:} Until she prepares spells from her spellbook, 
the only spells a wizard has available to cast are the ones that she already had 
prepared from the previous day and has not yet used. During the study period, she 
chooses which spells to prepare. If a wizard already has spells prepared (from 
the previous day) that she has not cast, she can abandon some or all of them to 
make room for new spells.

When preparing spells for the day, a wizard can leave some of these spell slots 
open. Later during that day, she can repeat the preparation process as often as 
she likes, time and circumstances permitting. During these extra sessions of preparation, 
the wizard can fill these unused spell slots. She cannot, however, abandon a previously 
prepared spell to replace it with another one or fill a slot that is empty because 
she has cast a spell in the meantime. That sort of preparation requires a mind 
fresh from rest. Like the first session of the day, this preparation takes at least 
15 minutes, and it takes longer if the wizard prepares more than one-quarter of 
her spells.

\textbf{Spell Slots:} The various character class tables show how many spells of 
each level a character can cast per day. These openings for daily spells are called 
spell slots. A spellcaster always has the option to fill a higher-level spell slot 
with a lower-level spell. A spellcaster who lacks a high enough ability score to 
cast spells that would otherwise be his or her due still gets the slots but must 
fill them with spells of lower level.

\textbf{Prepared Spell Retention:} Once a wizard prepares a spell, it remains in 
her mind as a nearly cast spell until she uses the prescribed components to complete 
and trigger it or until she abandons it. Certain other events, such as the effects 
of magic items or special attacks from monsters, can wipe a prepared spell from 
a character's mind.

\textbf{Death and Prepared Spell Retention:} If a spellcaster dies, all prepared 
spells stored in his or her mind are wiped away. Potent magic (such as \linkspell{Raise Dead},
\linkspell{Resurrection}, or \linkspell{True Resurrection}) can recover the lost energy 
when it recovers the character.

%%%%%%%%%%%%%%%%%%%%%%%%%
\subsection{Arcane Magical Writings}
%%%%%%%%%%%%%%%%%%%%%%%%%

To record an arcane spell in written form, a character uses complex notation that 
describes the magical forces involved in the spell. The writer uses the same system 
no matter what her native language or culture. However, each character uses the 
system in her own way. Another person's magical writing remains incomprehensible 
to even the most powerful wizard until she takes time to study and decipher it.

To decipher an arcane magical writing (such as a single spell in written form in 
another's spellbook or on a scroll), a character must make a Spellcraft check (DC 
20 + the spell's level). If the skill check fails, the character cannot attempt 
to read that particular spell again until the next day. A \linkspell{Read Magic} spell 
automatically deciphers a magical writing without a skill check. If the person 
who created the magical writing is on hand to help the reader, success is also 
automatic.

Once a character deciphers a particular magical writing, she does not need to decipher 
it again. Deciphering a magical writing allows the reader to identify the spell 
and gives some idea of its effects (as explained in the spell description). If 
the magical writing was a scroll and the reader can cast arcane spells, she can 
attempt to use the scroll.

%%%
\subsubsection{Wizard Spells and Borrowed Spellbooks}
%%%

A wizard can use a borrowed spellbook to prepare a spell she already knows and 
has recorded in her own spellbook, but preparation success is not assured. First, 
the wizard must decipher the writing in the book (see Arcane Magical Writings, 
above). Once a spell from another spellcaster's book is deciphered, the reader 
must make a \linkskill{Spellcraft} check (DC 15 + spell's level) to prepare the spell. If the 
check succeeds, the wizard can prepare the spell. She must repeat the check to 
prepare the spell again, no matter how many times she has prepared it before. If 
the check fails, she cannot try to prepare the spell from the same source again 
until the next day. (However, as explained above, she does not need to repeat a 
check to decipher the writing.)

%%%
\subsubsection{Adding Spells to a Wizard's Spellbook}
%%%

Wizards can add new spells to their spellbooks through several methods. If a wizard 
has chosen to specialize in a school of magic, she can learn spells only from schools 
whose spells she can cast.

\textbf{Spells Gained at a New Level:} Wizards perform a certain amount of spell 
research between adventures. Each time a character attains a new wizard level, 
she gains two spells of her choice to add to her spellbook. The two free spells 
must be of spell levels she can cast. If she has chosen to specialize in a school 
of magic, one of the two free spells must be from her specialty school.

\textbf{Spells Copied from Another's Spellbook or a Scroll:} A wizard can also 
add a spell to her book whenever she encounters one on a magic scroll or in another 
wizard's spellbook. No matter what the spell's source, the wizard must first decipher 
the magical writing (see \linksec{Arcane Magical Writings}, above). Next, she must spend 
a day studying the spell. At the end of the day, she must make a Spellcraft check 
(DC 15 + spell's level). A wizard who has specialized in a school of spells gains 
a +2 bonus on the Spellcraft check if the new spell is from her specialty school. 
She cannot, however, learn any spells from her prohibited schools. If the check 
succeeds, the wizard understands the spell and can copy it into her spellbook (see 
Writing a New Spell into a Spellbook, below). The process leaves a spellbook that 
was copied from unharmed, but a spell successfully copied from a magic scroll disappears 
from the parchment.

If the check fails, the wizard cannot understand or copy the spell. She cannot 
attempt to learn or copy that spell again until she gains another rank in Spellcraft. 
A spell that was being copied from a scroll does not vanish from the scroll.

In most cases, wizards charge a fee for the privilege of copying spells from their 
spellbooks. This fee is usually equal to the spell's level x50 gp.

\textbf{Independent Research:} A wizard also can research a spell independently, 
duplicating an existing spell or creating an entirely new one.

%%%
\subsubsection{Writing a New Spell into a Spellbook}
%%%

Once a wizard understands a new spell, she can record it into her spellbook.

\textbf{Time:} The process takes 24 hours, regardless of the spell's level.

\textbf{Space in the Spellbook:} A spell takes up one page of the spellbook per 
spell level. Even a 0-level spell (cantrip) takes one page. A spellbook has one 
hundred pages.

\textbf{Materials and Costs:} Materials for writing the spell cost 100 gp per page.

Note that a wizard does not have to pay these costs in time or gold for the spells 
she gains for free at each new level. 

%%%
\subsubsection{Replacing and Copying Spellbooks}
%%%

A wizard can use the procedure for learning a spell to reconstruct a lost spellbook. 
If she already has a particular spell prepared, she can write it directly into 
a new book at a cost of 100 gp per page (as noted in Writing a New Spell into a 
Spellbook, above). The process wipes the prepared spell from her mind, just as 
casting it would. If she does not have the spell prepared, she can prepare it from 
a borrowed spellbook and then write it into a new book.

Duplicating an existing spellbook uses the same procedure as replacing it, but 
the task is much easier. The time requirement and cost per page are halved.

%%%
\subsubsection{Selling a Spellbook}
%%%

Captured spellbooks can be sold for a gp amount equal to one-half the cost of purchasing 
and inscribing the spells within (that is, 50 gp per page of spells). 
A spellbook entirely filled with spells (that is, with one hundred pages of spells 
inscribed in it) is worth 5,000 gp.

%%%%%%%%%%%%%%%%%%%%%%%%%
\subsection{Sorcerers and Bards}
%%%%%%%%%%%%%%%%%%%%%%%%%

Sorcerers and bards cast arcane spells, but they do not have spellbooks and do 
not prepare their spells. A sorcerer's or bard's class level limits the number 
of spells he can cast (see these class descriptions). His high Charisma score might 
allow him to cast a few extra spells. A member of either class must have a Charisma 
score of at least 10 + a spell's level to cast the spell.

\textbf{Daily Readying of Spells:} Each day, sorcerers and bards must focus their 
minds on the task of casting their spells. A sorcerer or bard needs 8 hours of 
rest (just like a wizard), after which he spends 15 minutes concentrating. (A bard 
must sing, recite, or play an instrument of some kind while concentrating.) During 
this period, the sorcerer or bard readies his mind to cast his daily allotment 
of spells. Without such a period to refresh himself, the character does not regain 
the spell slots he used up the day before.

\textbf{Recent Casting Limit:} As with wizards, any spells cast within the last 
8 hours count against the sorcerer's or bard's daily limit.

\textbf{Adding Spells to a Sorcerer's or Bard's Repertoire:} A sorcerer or bard 
gains spells each time he attains a new level in his class and never gains spells 
any other way. When your sorcerer or bard gains a new level, consult Table: Bard 
Spells Known or Table: Sorcerer Spells Known to learn how many spells from the 
appropriate spell list he now knows. With permission, sorcerers and bards can also 
select the spells they gain from new and unusual spells that they have gained some 
understanding of.

%%%%%%%%%%%%%%%%%%%%%%%%%%%%%%%%%%%%%%%%%%%%%%%%%%
\section{Divine Spells}
%%%%%%%%%%%%%%%%%%%%%%%%%%%%%%%%%%%%%%%%%%%%%%%%%%

Clerics, druids, experienced paladins, and experienced rangers can cast divine 
spells. Unlike arcane spells, divine spells draw power from a divine source. Clerics 
gain spell power from deities or from divine forces. The divine force of nature 
powers druid and ranger spells. The divine forces of law and good power paladin 
spells. Divine spells tend to focus on healing and protection and are less flashy, 
destructive, and disruptive than arcane spells.

%%%%%%%%%%%%%%%%%%%%%%%%%
\subsection{Preparing Divine Spells}\index{Spell Preperation!Divine}
%%%%%%%%%%%%%%%%%%%%%%%%%

Divine spellcasters prepare their spells in largely the same manner as wizards 
do, but with a few differences. The relevant ability for divine spells is Wisdom. 
To prepare a divine spell, a character must have a Wisdom score of 10 + the spell's 
level. Likewise, bonus spells are based on Wisdom.

\textbf{Time of Day:} A divine spellcaster chooses and prepares spells ahead of 
time, just as a wizard does. However, a divine spellcaster does not require a period 
of rest to prepare spells. Instead, the character chooses a particular part of 
the day to pray and receive spells. The time is usually associated with some daily 
event. If some event prevents a character from praying at the proper time, he must 
do so as soon as possible. If the character does not stop to pray for spells at 
the first opportunity, he must wait until the next day to prepare spells.

\textbf{Spell Selection and Preparation:} A divine spellcaster selects and prepares 
spells ahead of time through prayer and meditation at a particular time of day. 
The time required to prepare spells is the same as it is for a wizard (1 hour), 
as is the requirement for a relatively peaceful environment. A divine spellcaster 
does not have to prepare all his spells at once. However, the character's mind 
is considered fresh only during his or her first daily spell preparation, so a 
divine spellcaster cannot fill a slot that is empty because he or she has cast 
a spell or abandoned a previously prepared spell.

Divine spellcasters do not require spellbooks. However, such a character's spell 
selection is limited to the spells on the list for his or her class. Clerics, druids, 
paladins, and rangers have separate spell lists. A cleric also has access to two 
domains determined during his character creation. Each domain gives him access 
to a domain spell at each spell level from 1st to 9th, as well as a special granted 
power. With access to two domain spells at each spell level -- one from each of 
his two domains -- a cleric must prepare, as an extra domain spell, one or the other 
each day for each level of spell he can cast. If a domain spell is not on the cleric 
spell list, it can be prepared only in a domain spell slot.

\textbf{Spell Slots:} The character class tables show how many spells of each level 
a character can cast per day.

These openings for daily spells are called spell slots. A spellcaster always has 
the option to fill a higher-level spell slot with a lower level spell. A spellcaster 
who lacks a high enough ability score to cast spells that would otherwise be his 
or her due still gets the slots but must fill them with spells of lower level. 

\textbf{Recent Casting Limit:} As with arcane spells, at the time of preparation 
any spells cast within the previous 8 hours count against the number of spells 
that can be prepared.

\textbf{Spontaneous Casting of Cure and Inflict Spells:}
A good cleric (or a cleric of a good deity) can spontaneously 
cast a \textit{cure} spell in place of a prepared spell of the same level or higher, 
but not in place of a domain spell. An evil cleric (or a cleric of an evil deity) 
can spontaneously cast an \textit{inflict} spell in place of a prepared spell (one 
that is not a domain spell) of the same level or higher. Each neutral cleric of 
a neutral deity either spontaneously casts \textit{cure} spells like a good cleric 
or \textit{inflict} spells like an evil one, depending on which option the player 
chooses when creating the character. The divine energy of the spell that the
\textit{cure}or \textit{inflict} spell substitutes for is converted into the \textit{cure} or 
\textit{inflict} spell as if that spell had been prepared all along.

\textbf{Spontaneous Casting of Summon Nature's Ally Spells:} 
A druid can spontaneously cast a \textit{summon nature's ally} spell in place of 
a prepared spell of the same level or higher. The divine energy of the spell that 
the \textit{summon nature's ally} spell substitutes for is converted into the \textit{summon}
spell as if that spell had been prepared all along.

%%%%%%%%%%%%%%%%%%%%%%%%%
\subsection{Divine Magical Writings}
%%%%%%%%%%%%%%%%%%%%%%%%%

Divine spells can be written down and deciphered just as arcane spells can (see 
\linksec{Arcane Magical Writings}, above). Any character with the Spellcraft skill can attempt 
to decipher the divine magical writing and identify it. However, only characters 
who have the spell in question (in its divine form) on their class spell list can 
cast a divine spell from a scroll.

%%%%%%%%%%%%%%%%%%%%%%%%%
\subsection{New Divine Spells}
%%%%%%%%%%%%%%%%%%%%%%%%%

Divine spellcasters most frequently gain new spells in one of the following two 
ways.

\textbf{Spells Gained at a New Level:} Characters who can cast divine spells undertake 
a certain amount of study between adventures. Each time such a character receives 
a new level of divine spells, he or she learns new spells from that level automatically.

\textbf{Independent Research:} A divine spellcaster also can research a spell independently, 
much as an arcane spellcaster can. Only the creator of such a spell can prepare 
and cast it, unless he decides to share it with others.

%%%%%%%%%%%%%%%%%%%%%%%%%%%%%%%%%%%%%%%%%%%%%%%%%%
\section{Special Ability Rules}
%%%%%%%%%%%%%%%%%%%%%%%%%%%%%%%%%%%%%%%%%%%%%%%%%%

\textbf{Spell-Like Abilities:}\index{Spell-Like Ability} Usually, a spell-like ability works just like the 
spell of that name. A few spell-like abilities are unique; these are explained 
in the text where they are described.

A spell-like ability has no verbal, somatic, or material component, nor does it 
require a focus or have an XP cost. The user activates it mentally. Armor never 
affects a spell-like ability's use, even if the ability resembles an arcane spell 
with a somatic component.

A spell-like ability has a casting time of 1 standard action unless noted otherwise 
in the ability or spell description. In all other ways, a spell-like ability functions 
just like a spell.

Spell-like abilities are subject to spell resistance and to being dispelled by 
\linkspell{Dispel Magic}. They do not function in areas where magic is suppressed 
or negated. Spell-like abilities cannot be used to counterspell, nor can they be 
counterspelled.

Some creatures are actually sorcerers of a sort. They cast arcane spells as sorcerers 
do, using components when required. In fact, an individual creature could have 
some spell-like abilities and also cast other spells as a sorcerer.

\textbf{Supernatural Abilities:}\index{Supernatural Ability} These abilities cannot be disrupted in combat, 
as spells can, and they generally do not provoke attacks of opportunity. Supernatural 
abilities are not subject to spell resistance, counterspells, or to being dispelled 
by \textit{dispel magic}, and do not function in areas where magic is suppressed 
or negated.

\textbf{Extraordinary Abilities:}\index{Extraordinary Ability} These abilities cannot be disrupted in combat, 
as spells can, and they generally do not provoke attacks of opportunity. Effects 
or areas that negate or disrupt magic have no effect on extraordinary abilities. 
They are not subject to dispelling, and they function normally in an \linkspell{Antimagic Field}.
Indeed, extraordinary abilities do not qualify as magical, though they 
may break the laws of physics.

\textbf{Natural Abilities:}\index{Natural Ability} This category includes abilities a creature has because 
of its physical nature. Natural abilities are those not otherwise designated as 
extraordinary, supernatural, or spell-like.
