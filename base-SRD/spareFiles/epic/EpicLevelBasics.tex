%&LaTeX
% !TEX encoding = UTF-8 Unicode
\documentclass{article}
\usepackage[utf8x]{inputenc}
\usepackage[T1]{fontenc}
\usepackage{textcomp}

\usepackage{longtable}

\newcommand{\tab}{\hspace{5mm}}

nt, and is licensed for public use under the terms 
of the Open Game License v1.0a.

\subsection*{{\LARGE EPIC LEVEL BASICS}}

\vspace{12pt}
Epic characters---those whose character level is 21st or higher---are handled slightly 
differently from nonepic characters. While epic characters continue to receive 
most of the benefits of gaining levels, some benefits are replaced by alternative 
gains. A class can be advanced beyond 20\textsuperscript{th} level.  A ten-level 
prestige class can progress beyond 10\textsuperscript{th} level, but only if the 
character level is already 20\textsuperscript{th} or higher.  A class with fewer 
than ten levels cannot progress beyond the maximum for that class, regardless of 
character level.

\vspace{12pt}
\textbf{Epic Save Bonus:} A character's base save bonus does not increase after 
character level reaches 20th. However, the character does receive a cumulative 
+1 epic bonus on all saving throws at every even-numbered level beyond 20th, as 
shown on Table: Epic Save and Epic Attack Bonuses. Any time a feat, prestige class, 
or other rule refers to your base save bonus, use the sum of your base save bonus 
and epic save bonus.

\vspace{12pt}
\textbf{Epic Attack Bonus:} Similarly, the character's base attack bonus does not 
increase after character level reaches 20th. However, the character does receive 
a cumulative +1 epic bonus on all attacks at every odd-numbered level beyond 20th, 
as shown on Table: Epic Save and Epic Attack Bonuses. Any time a feat, prestige 
class, or other rule refers to your base attack bonus, use the sum of your base 
attack bonus and epic attack bonus.

\vspace{12pt}
\textbf{Class Skill Max Ranks:} The maximum number of ranks a character can have 
in a class skill is equal to his or her character level +3. 

\vspace{12pt}
\textbf{Cross-Class Skill Max Ranks:} For cross-class skills, the maximum number 
of ranks is one-half the maximum for a class skill. 

Feats: Every character gains one feat (which may be an epic or nonepic feat at 
the player's choice) at every level divisible by three. These feats are in addition 
to any bonus feats granted in the class descriptions. 

Ability Increases: Upon gaining any level divisible by four, a character increases 
one of his or her ability scores by 1 point. The player chooses which ability score 
to improve. For multiclass characters, feats and ability increases are gained according 
to character level, not class level. 

\vspace{12pt}
\textbf{Table: Epic Save and Epic Attack Bonuses }

\begin{longtable}{llllll}
\hline
% ROW 1
\multicolumn{1}{|p{0.825in}|}{\begin{minipage}[t]{0.825in}\centering
\textbf{Character Level}\end{minipage}} & \multicolumn{1}{p{0.875in}|}{\begin{minipage}[t]{0.875in}\raggedright
\textbf{Epic Save Bonus}\end{minipage}} & \multicolumn{1}{p{1.000in}|}{\begin{minipage}[t]{1.000in}\raggedright
\textbf{Epic Attack Bonus}\end{minipage}}\\
\hline
% ROW 2
\multicolumn{1}{p{0.069in}|}{\begin{minipage}[t]{0.069in}\centering
21st\end{minipage}} & \multicolumn{1}{p{0.069in}|}{\begin{minipage}[t]{0.069in}\raggedright
+0\end{minipage}} & \multicolumn{1}{p{0.069in}|}{\begin{minipage}[t]{0.069in}\raggedright
+1\end{minipage}}\\
\hline
% ROW 3
\multicolumn{1}{|p{0.825in}|}{\begin{minipage}[t]{0.825in}\centering
22nd\end{minipage}} & \multicolumn{1}{p{0.875in}|}{\begin{minipage}[t]{0.875in}\raggedright
+1\end{minipage}} & \multicolumn{1}{p{1.000in}|}{\begin{minipage}[t]{1.000in}\raggedright
+1\end{minipage}}\\
\hline
% ROW 4
\multicolumn{1}{p{0.069in}|}{\begin{minipage}[t]{0.069in}\centering
23rd\end{minipage}} & \multicolumn{1}{p{0.069in}|}{\begin{minipage}[t]{0.069in}\raggedright
+1\end{minipage}} & \multicolumn{1}{p{0.069in}|}{\begin{minipage}[t]{0.069in}\raggedright
+2\end{minipage}}\\
\hline
% ROW 5
\multicolumn{1}{|p{0.825in}|}{\begin{minipage}[t]{0.825in}\centering
24th\end{minipage}} & \multicolumn{1}{p{0.875in}|}{\begin{minipage}[t]{0.875in}\raggedright
+2\end{minipage}} & \multicolumn{4}{p{1.208in}|}{\begin{minipage}[t]{1.208in}\raggedright
+2\end{minipage}}\\
\hline
% ROW 6
\multicolumn{1}{p{0.069in}|}{\begin{minipage}[t]{0.069in}\centering
25th\end{minipage}} & \multicolumn{1}{p{0.069in}|}{\begin{minipage}[t]{0.069in}\raggedright
+2\end{minipage}} & \multicolumn{1}{p{0.069in}|}{\begin{minipage}[t]{0.069in}\raggedright
+3\end{minipage}}\\
\hline
% ROW 7
\multicolumn{1}{|p{0.825in}|}{\begin{minipage}[t]{0.825in}\centering
26th\end{minipage}} & \multicolumn{1}{p{0.875in}|}{\begin{minipage}[t]{0.875in}\raggedright
+3\end{minipage}} & \multicolumn{4}{p{1.208in}|}{\begin{minipage}[t]{1.208in}\raggedright
+3\end{minipage}}\\
\hline
% ROW 8
\multicolumn{1}{|p{0.825in}|}{\begin{minipage}[t]{0.825in}\centering
27th\end{minipage}} & \multicolumn{1}{p{0.875in}|}{\begin{minipage}[t]{0.875in}\raggedright
+3\end{minipage}} & \multicolumn{4}{p{1.208in}|}{\begin{minipage}[t]{1.208in}\raggedright
+4\end{minipage}}\\
\hline
% ROW 9
\multicolumn{1}{|p{0.825in}|}{\begin{minipage}[t]{0.825in}\centering
28th\end{minipage}} & \multicolumn{1}{p{0.875in}|}{\begin{minipage}[t]{0.875in}\raggedright
+4\end{minipage}} & \multicolumn{4}{p{1.208in}|}{\begin{minipage}[t]{1.208in}\raggedright
+4\end{minipage}}\\
\hline
% ROW 10
\multicolumn{1}{|p{0.825in}|}{\begin{minipage}[t]{0.825in}\centering
29th\end{minipage}} & \multicolumn{1}{p{0.875in}|}{\begin{minipage}[t]{0.875in}\raggedright
+4\end{minipage}} & \multicolumn{4}{p{1.208in}|}{\begin{minipage}[t]{1.208in}\raggedright
+5\end{minipage}}\\
\hline
% ROW 11
\multicolumn{1}{|p{0.825in}|}{\begin{minipage}[t]{0.825in}\centering
30th\end{minipage}} & \multicolumn{1}{p{0.875in}|}{\begin{minipage}[t]{0.875in}\raggedright
+5\end{minipage}} & \multicolumn{4}{p{1.208in}|}{\begin{minipage}[t]{1.208in}\raggedright
+5\end{minipage}}\\
\hline
\end{longtable}

\vspace{12pt}
Although most of the tables only show information up to a certain level (often 
30th), that level is by no means the limit of a character's advancement. It can 
be generally assumed that any patterns on a particular table continue infinitely.

\vspace{12pt}
\subsubsection*{CLASS FEATURES }

Many, but not all, class features continue to accumulate after 20th level. The 
following guidelines describe how the epic class progressions. •

\vspace{12pt}
A character's base save bonuses and base attack bonus don't increase after 20th 
level. Use Table: Epic Save and Epic Attack Bonuses to determine the character's 
epic bonus on saving throws and attacks. •

Characters continue to gain Hit Dice and skill points as normal beyond 20th level.•

Generally, any class feature that uses class level as part of a mathematical formula 
continues to increase using the character's class level in the formula.  Any prestige 
class feature that calculates a save DC using the class level should add only half 
the character's class levels above 10th. •

For spellcasters, caster level continues to increase after 20th level. However, 
spells per day don't increase after 20th level. The only way to gain additional 
spells per day (other than the bonus spells gained from a high ability score) is 
to select the Improved Spell Capacity epic feat.•

The powers of familiars, special mounts, and fiendish servants continue to increase 
as their masters gain levels. •

Any class features that increase or accumulate as part of a repeated pattern also 
continues to increase or accumulate after 20th level at the same rate.  An exception 
to this rule is any bonus feat granted as a class feature. If a character gets 
bonus feats as part of a class feature, these do not increase with epic levels. 
Instead, these classes get bonus feats at a different rate (described in each epic 
class description). •

In addition to the class features retained from nonepic levels, each class gains 
a bonus feat every two, three, four, or five levels after 20th. This augments each 
class's progression of class features, because not all classes otherwise improve 
class features after 20th level. A character must select these feats from the list 
of bonus feats for that class. These bonus feats are in addition to the feat that 
every character gets every three levels. The character isn't limited to selecting 
from the class list when selecting these feats. •

Characters don't gain any new class features, because there aren't any new class 
features described for these levels. Class features with a progression that slows 
or stops before 20th level and features that have a limited list of options do 
not improve as a character gains epic levels. Likewise, class features that are 
gained only at a single level do not improve. 

\vspace{12pt}
\subsubsection*{Adding a Second Class }

When a single-class epic character gains a level, he or she may choose to increase 
the level of his or her current class or pick up a new class at 1st level. The 
standard rules for multiclass characters still apply, but epic characters must 
keep in mind the rules for epic advancement. The epic character gains all the 1st-level 
class skills, weapon proficiency, armor proficiency, spells, and other class features 
of the new class, as well as a Hit Die of the appropriate type. In addition, the 
character gets the usual skill points from the new class. Just as with standard 
multiclassing, adding the second class does not confer some of the benefits for 
a 1st-level character, including maximum hit points from the first Hit Die, quadruple 
the per-level skill points, starting equipment, starting gold, or an animal companion. 
An epic character does not gain the base attack bonuses and base save bonuses normally 
gained when adding a second class. Instead, an epic character uses the epic attack 
bonus and epic save bonus progression shown on Table: Epic Save and Epic Attack 
Bonus.

\end{document}
