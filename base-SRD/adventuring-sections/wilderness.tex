%%%%%%%%%%%%%%%%%%%%%%%%%%%%%%%%%%%%%%%%%%%%%%%%%%
\section{Wilderness}
%%%%%%%%%%%%%%%%%%%%%%%%%%%%%%%%%%%%%%%%%%%%%%%%%%

%%%%%%%%%%%%%%%%%%%%%%%%%
\subsection{Getting Lost}\index{Getting Lost}
%%%%%%%%%%%%%%%%%%%%%%%%%

There are many ways to get lost in the wilderness. Following an obvious road, trail, 
or feature such as a stream or shoreline prevents any possibility of becoming lost, 
but travelers striking off cross-country may become disoriented -- especially in 
conditions of poor visibility or in difficult terrain. 

\textbf{Poor Visibility:} Any time characters cannot see at least 60 feet in the 
prevailing conditions of visibility, they may become lost. Characters traveling 
through fog, snow, or a downpour might easily lose the ability to see any landmarks 
not in their immediate vicinity. Similarly, characters traveling at night may be 
at risk, too, depending on the quality of their light sources, the amount of moonlight, 
and whether they have darkvision or lowlight vision.

\textbf{Difficult Terrain:} Any character in forest, moor, hill, or mountain terrain 
may become lost if he or she moves away from a trail, road, stream, or other obvious 
path or track. Forests are especially dangerous because they obscure far-off landmarks 
and make it hard to see the sun or stars.

\textbf{Chance to Get Lost:} If conditions exist that make getting lost a possibility, 
the character leading the way must succeed on a \linkskill{Survival} check or become lost. 
The difficulty of this check varies based on the terrain, the visibility conditions, 
and whether or not the character has a map of the area being traveled through. 
Refer to the table below and use the highest DC that applies.

\begin{table}[htb]
\rowcolors{1}{white}{offyellow}
\caption{Survival DCs to avoid getting Lost}
\centering
\begin{tabular}{l c l c}
 & \textbf{Survival DC} & & \textbf{Survival DC}\\
Moor or hill (map) & 6 & Moor or hill (no map) & 10\\
Mountain (map) & 8 & Mountain (no map) & 12\\
Poor visibility & 12 & Forest & 15\\
\end{tabular}
\end{table}

A character with at least 5 ranks in \linkskill{Knowledge} (geography) or Knowledge (local) 
pertaining to the area being traveled through gains a +2 bonus on this check.

Check once per hour (or portion of an hour) spent in local or overland movement 
to see if travelers have become lost. In the case of a party moving together, only 
the character leading the way makes the check.

\textbf{Effects of Being Lost:} If a party becomes lost, it is no longer certain 
of moving in the direction it intended to travel. Randomly determine the direction 
in which the party actually travels during each hour of local or overland movement. 
The characters' movement continues to be random until they blunder into a landmark 
they can't miss, or until they recognize that they are lost and make an effort 
to regain their bearings.

\textit{Recognizing that You're Lost:} Once per hour of random travel, each character 
in the party may attempt a Survival check (DC 20, -1 per hour of random travel) 
to recognize that they are no longer certain of their direction of travel. Some 
circumstances may make it obvious that the characters are lost.

\textit{Setting a New Course:} A lost party is also uncertain of determining in 
which direction it should travel in order to reach a desired objective. Determining 
the correct direction of travel once a party has become lost requires a Survival 
check (DC 15, +2 per hour of random travel). If a character fails this check, he 
chooses a random direction as the "correct" direction for resuming travel.

Once the characters are traveling along their new course, correct or incorrect, 
they may get lost again. If the conditions still make it possible for travelers 
to become lost, check once per hour of travel as described in Chance to Get Lost, 
above, to see if the party maintains its new course or begins to move at random 
again.

\textit{Conflicting Directions:} It's possible that several characters may attempt 
to determine the right direction to proceed after becoming lost. Make a Survival 
check for each character in secret, then tell the players whose characters succeeded 
the correct direction in which to travel, and tell the players whose characters 
failed a random direction they think is right. 

\textbf{Regaining Your Bearings:} There are several ways to become un-lost. First, 
if the characters successfully set a new course and follow it to the destination 
they're trying to reach, they're not lost anymore. Second, the characters through 
random movement might run into an unmistakable landmark. Third, if conditions suddenly 
improve -- the fog lifts or the sun comes up -- lost characters may attempt to set 
a new course, as described above, with a +4 bonus on the Survival check. Finally, 
magic may make their course clear.

%%%%%%%%%%%%%%%%%%%%%%%%%
\subsection{Forest Terrain}
%%%%%%%%%%%%%%%%%%%%%%%%%

Forest terrain can be divided into three categories: sparse, medium, and dense. 
An immense forest could have all three categories within its borders, with more 
sparse terrain at the outer edge of the forest and dense forest at its heart. 

The table below describes in general terms how likely it is that a given square 
has a terrain element in it.

%%%
\subsubsection{Forest Terrain Features}
%%%

\begin{table}[htb]
\rowcolors{1}{white}{offyellow}
\caption{Random Forest Features}
\centering
\begin{tabular}{l c c c}
 & \multicolumn{3}{c}{\textbf{Forest Category}}\\
\textbf{Feature} & \textbf{Sparse} & \textbf{Medium} & \textbf{Dense}\\
Typical Trees & 50\% & 70\% & 80\%\\
Massive Trees & -- & 10\% & 20\%\\
Light Undergrowth & 50\% & 70\% & 50\%\\
Heavy Undergrowth& -- & 20\% & 50\%\\
\end{tabular}
\end{table}

\textbf{Trees:} The most important terrain element in a forest is the trees, obviously. 
A creature standing in the same square as a tree gains a +2 bonus to Armor Class 
and a +1 bonus on Reflex saves (these bonuses don't stack with cover bonuses from 
other sources). The presence of a tree doesn't otherwise affect a creature's fighting 
space, because it's assumed that the creature is using the tree to its advantage 
when it can. The trunk of a typical tree has AC 4, hardness 5, and 150 hp. A DC 
15 \linkskill{Climb} check is sufficient to climb a tree. Medium and dense forests have massive 
trees as well. These trees take up an entire square and provide cover to anyone 
behind them. They have AC 3, hardness 5, and 600 hp. Like their smaller counterparts, 
it takes a DC 15 Climb check to climb them.

\textbf{Undergrowth:} Vines, roots, and short bushes cover much of the ground in 
a forest. A space covered with light undergrowth costs 2 squares of movement to 
move into, and it provides concealment. Undergrowth increases the DC of \linkskill{Tumble} 
and \linkskill{Move Silently} checks by 2 because the leaves and branches get in the way. Heavy 
undergrowth costs 4 squares of movement to move into, and it provides concealment 
with a 30\% miss chance (instead of the usual 20\%). It increases the DC of Tumble 
and Move Silently checks by 5. Heavy undergrowth is easy to hide in, granting a 
+5 circumstance bonus on Hide checks. Running and charging are impossible. Squares 
with undergrowth are often clustered together. Undergrowth and trees aren't mutually 
exclusive; it's common for a 5-foot square to have both a tree and undergrowth.

\textbf{Forest Canopy:} It's common for elves and other forest dwellers to live 
on raised platforms far above the surface floor. These wooden platforms generally 
have rope bridges between them. To get to the treehouses, characters generally 
ascend the trees' branches (Climb DC 15), use rope ladders (Climb DC 0), or take 
pulley elevators (which can be made to rise a number of feet equal to a Strength 
check, made each round as a full-round action). Creatures on platforms or branches 
in a forest canopy are considered to have cover when fighting creatures on the 
ground, and in medium or dense forests they have concealment as well.

\textbf{Other Forest Terrain Elements:} Fallen logs generally stand about 3 feet 
high and provide cover just as low walls do. They cost 5 feet of movement to cross. 
Forest streams are generally 5 to 10 feet wide and no more than 5 feet deep. Pathways 
wind through most forests, allowing normal movement and providing neither cover 
nor concealment. These paths are less common in dense forests, but even unexplored 
forests will have occasional game trails.

\textbf{Stealth and Detection in a Forest:} In a sparse forest, the maximum distance 
at which a \linkskill{Spot} check for detecting the nearby presence of others can succeed is 
3d6x10 feet. In a medium forest, this distance is 2d8x10 
feet, and in a dense forest it is 2d6x10 feet.

Because any square with undergrowth provides concealment, it's usually easy for 
a creature to use the Hide skill in the forest. Logs and massive trees provide 
cover, which also makes hiding possible.

The background noise in the forest makes \linkskill{Listen} checks more difficult, increasing 
the DC of the check by 2 per 10 feet, not 1 (but note that Move Silently is also 
more difficult in undergrowth). 

%%%
\subsubsection{Forest Fires (CR 6)}
%%%

Most campfire sparks ignite nothing, but if conditions are dry, winds are strong, 
or the forest floor is dried out and flammable, a forest fire can result. Lightning 
strikes often set trees afire and start forest fires in this way. Whatever the 
cause of the fire, travelers can get caught in the conflagration.

A forest fire can be spotted from as far away as 2d6x100 feet 
by a character who makes a \linkskill{Spot} check, treating the fire as a Colossal creature 
(reducing the DC by 16). If all characters fail their Spot checks, the fire moves 
closer to them. They automatically see it when it closes to half the original distance.

Characters who are blinded or otherwise unable to make Spot checks can feel the 
heat of the fire (and thus automatically "spot" it) when it is 100 feet away.

The leading edge of a fire (the downwind side) can advance faster than a human 
can run (assume 120 feet per round for winds of moderate strength). Once a particular 
portion of the forest is ablaze, it remains so for 2d4x10 minutes 
before dying to a smoking smolder. Characters overtaken by a forest fire may find 
the leading edge of the fire advancing away from them faster than they can keep 
up, trapping them deeper and deeper in its grasp.

Within the bounds of a forest fire, a character faces three dangers: heat damage, 
catching on fire, and smoke inhalation. 

\textbf{Heat Damage:} Getting caught within a forest fire is even worse than being 
exposed to extreme heat (see Heat Dangers). Breathing the air causes a character 
to take 1d6 points of damage per round (no save). In addition, a character must 
make a Fortitude save every 5 rounds (DC 15, +1 per previous check) or take 1d4 
points of nonlethal damage. A character who holds his breath can avoid the lethal 
damage, but not the nonlethal damage. Those wearing heavy clothing or any sort 
of armor take a -4 penalty on their saving throws. In addition, those wearing metal 
armor or coming into contact with very hot metal are affected as if by a \linkspell{Heat Metal} spell.

\textbf{Catching on Fire:} Characters engulfed in a forest fire are at risk of 
catching on fire when the leading edge of the fire overtakes them, and are then 
at risk once per minute thereafter (see \linksec{Catching on Fire}).

\textbf{Smoke Inhalation:} Forest fires naturally produce a great deal of smoke. 
A character who breathes heavy smoke must make a Fortitude save each round (DC 
15, +1 per previous check) or spend that round choking and coughing. A character 
who chokes for 2 consecutive rounds takes 1d6 points of nonlethal damage. Also, 
smoke obscures vision, providing concealment to characters within it.

%%%%%%%%%%%%%%%%%%%%%%%%%
\subsection{Marsh Terrain}
%%%%%%%%%%%%%%%%%%%%%%%%%

Two categories of marsh exist: relatively dry moors and watery swamps. Both are 
often bordered by lakes (described in Aquatic Terrain, below), which effectively 
are a third category of terrain found in marshes.

The table below describes terrain features found in marshes.

%%%
\subsubsection{Marsh Terrain Features}
%%%

\begin{table}[htb]
\rowcolors{1}{white}{offyellow}
\caption{Random Marsh Features}
\centering
\begin{tabular}{l c c c}
 & \multicolumn{2}{c}{\textbf{Marsh Category}}\\
\textbf{Feature} & \textbf{Moor} & \textbf{Swamp}\\
Shallow bog & 20\% & 40\%\\
Deep bog & 5\% & 20\%\\
Light undergrowth & 30\% & 20\%\\
Heavy Undergrowth & 10\% & 20\%\\
\end{tabular}
\end{table}

\textbf{Bogs:} If a square is part of a shallow bog, it has deep mud or standing 
water of about 1 foot in depth. It costs 2 squares of movement to move into a square 
with a shallow bog, and the DC of \linkskill{Tumble} checks in such a square increases by 2. 

A square that is part of a deep bog has roughly 4 feet of standing water. It costs 
Medium or larger creatures 4 squares of movement to move into a square with a deep 
bog, or characters can swim if they wish. Small or smaller creatures must swim 
to move through a deep bog. Tumbling is impossible in a deep bog.

The water in a deep bog provides cover for Medium or larger creatures. Smaller 
creatures gain improved cover (+8 bonus to AC, +4 bonus on Reflex saves). Medium 
or larger creatures can crouch as a move action to gain this improved cover. Creatures 
with this improved cover take a -10 penalty on attacks against creatures that aren't 
underwater.

Deep bog squares are usually clustered together and surrounded by an irregular 
ring of shallow bog squares.

Both shallow and deep bogs increase the DC of \linkskill{Move Silently} checks by 2.

\textbf{Undergrowth:} The bushes, rushes, and other tall grasses in marshes function 
as undergrowth does in a forest (see above). A square that is part of a bog does 
not also have undergrowth. 

\textbf{Quicksand:} Patches of quicksand present a deceptively solid appearance 
(appearing as undergrowth or open land) that may trap careless characters. A character 
approaching a patch of quicksand at a normal pace is entitled to a DC 8 \linkskill{Survival} 
check to spot the danger before stepping in, but charging or running characters 
don't have a chance to detect a hidden bog before blundering in. A typical patch 
of quicksand is 20 feet in diameter; the momentum of a charging or running character 
carries him or her 1d2x5 feet into the quicksand.

\textit{Effects of Quicksand:} Characters in quicksand must make a DC 10 \linkskill{Swim} check 
every round to simply tread water in place, or a DC 15 Swim check to move 5 feet 
in whatever direction is desired. If a trapped character fails this check by 5 
or more, he sinks below the surface and begins to drown whenever he can no longer 
hold his breath (see the Swim skill description).

Characters below the surface of a bog may swim back to the surface with a successful 
Swim check (DC 15, +1 per consecutive round of being under the surface).

\textit{Rescue:} Pulling out a character trapped in quicksand can be difficult. 
A rescuer needs a branch, spear haft, rope, or similar tool that enables him to 
reach the victim with one end of it. Then he must make a DC 15 Strength check to 
successfully pull the victim, and the victim must make a DC 10 Strength check to 
hold onto the branch, pole, or rope. If the victim fails to hold on, he must make 
a DC 15 Swim check immediately to stay above the surface. If both checks succeed, 
the victim is pulled 5 feet closer to safety.

\textbf{Hedgerows:} Common in moors, hedgerows are tangles of stones, soil, and 
thorny bushes. Narrow hedgerows function as low walls, and it takes 15 feet of 
movement to cross them. Wide hedgerows are more than 5 feet tall and take up entire 
squares. They provide total cover, just as a wall does. It takes 4 squares of movement 
to move through a square with a wide hedgerow; creatures that succeed on a DC 10 
\linkskill{Climb} check need only 2 squares of movement to move through the square.

\textbf{Other Marsh Terrain Elements:} Some marshes, particularly swamps, have 
trees just as forests do, usually clustered in small stands. Paths lead across 
many marshes, winding to avoid bog areas. As in forests, paths allow normal movement 
and don't provide the concealment that undergrowth does.

\textbf{Stealth and Detection in a Marsh:} In a moor, the maximum distance at which 
a \linkskill{Spot} check for detecting the nearby presence of others can succeed is 6d6x10 
feet. In a swamp, this distance is 2d8x10 feet.

Undergrowth and deep bogs provide plentiful concealment, so it's easy to hide in 
a marsh.

A marsh imposes no penalties on \linkskill{Listen} checks, and using the \linkskill{Move Silently} skill 
is more difficult in both undergrowth and bogs.

%%%%%%%%%%%%%%%%%%%%%%%%%
\subsection{Hills Terrain}
%%%%%%%%%%%%%%%%%%%%%%%%%

A hill can exist in most other types of terrain, but hills can also dominate the 
landscape. Hills terrain is divided into two categories: gentle hills and rugged 
hills. Hills terrain often serves as a transition zone between rugged terrain such 
as mountains and flat terrain such as plains.

%%%
\subsubsection{Hills Terrain Features}
%%%

\begin{table}[htb]
\rowcolors{1}{white}{offyellow}
\caption{Random Hills Features}
\centering
\begin{tabular}{l c c c}
 & \multicolumn{2}{c}{\textbf{Hills Category}}\\
\textbf{Feature} & \textbf{Gentle Hill} & \textbf{Rugged Hill}\\
Gradual slope & 75\% & 40\%\\
Steep slope & 20\% & 50\%\\
Cliff & 5\% & 10\%\\
Light undergrowth & 15\% & 15\%\\
\end{tabular}
\end{table}

\textbf{Gradual Slope:} This incline isn't steep enough to affect movement, but 
characters gain a +1 bonus on melee attacks against foes downhill from them.

\textbf{Steep Slope:} Characters moving uphill (to an adjacent square of higher 
elevation) must spend 2 squares of movement to enter each square of steep slope. 
Characters running or charging downhill (moving to an adjacent square of lower 
elevation) must succeed on a DC 10 \linkskill{Balance} check upon entering the first steep 
slope square. Mounted characters make a DC 10 \linkskill{Ride} check instead. Characters who 
fail this check stumble and must end their movement 1d2x5 feet 
later. Characters who fail by 5 or more fall prone in the square where they end 
their movement. A steep slope increases the DC of Tumble checks by 2.

\textbf{Cliff:} A cliff typically requires a DC 15 \linkskill{Climb} check to scale and is 
1d4x10 feet tall, although the needs of your map may mandate 
a taller cliff. A cliff isn't perfectly vertical, taking up 5-foot squares if it's 
less than 30 feet tall and 10-foot squares if it's 30 feet or taller. 

\textbf{Light Undergrowth:} Sagebrush and other scrubby bushes grow on hills, athough 
they rarely cover the landscape as they do in forests and marshes. Light undergrowth 
provides concealment and increases the DC of \linkskill{Tumble} and \linkskill{Move Silently} checks by 
2. 

\textbf{Other Hills Terrain Elements:} Trees aren't out of place in hills terrain, 
and valleys often have active streams (5 to 10 feet wide and no more than 5 feet 
deep) or dry streambeds (treat as a trench 5 to 10 feet across) in them. If you 
add a stream or streambed, remember that water always flows downhill.

\textbf{Stealth and Detection in Hills:} In gentle hills, the maximum distance 
at which a \linkskill{Spot} check for detecting the nearby presence of others can succeed is 
2d10x10 feet. In rugged hills, this distance is 2d6x10 
feet.

Hiding in hills terrain can be difficult if there isn't undergrowth around. A hilltop 
or ridge provides enough cover to hide from anyone below the hilltop or ridge.

Hills don't affect \linkskill{Listen} or Move Silently checks. 

%%%%%%%%%%%%%%%%%%%%%%%%%
\subsection{Mountain Terrain}
%%%%%%%%%%%%%%%%%%%%%%%%%

The three mountain terrain categories are alpine meadows, rugged mountains, and 
forbidding mountains. As characters ascend into a mountainous area, they're likely 
to face each terrain category in turn, beginning with alpine meadows, extending 
through rugged mountains, and reaching forbidding mountains near the summit.

Mountains have an important terrain element, the rock wall, that is marked on the 
border between squares rather than taking up squares itself. 

%%%
\subsubsection{Mountain Terrain Features}
%%%

\begin{table}[htb]
\rowcolors{1}{white}{offyellow}
\caption{Random Mountain Features}
\centering
\begin{tabular}{l c c c}
 & \multicolumn{3}{c}{\textbf{Mountain Category}}\\
\textbf{Feature} & \textbf{Alpine Meadow} & \textbf{Rugged} & \textbf{Forbidding}\\
Gradual slope & 50\% & 25\% & 15\%\\
Steep slope & 40\% & 55\% & 55\%\\
Cliff & 10\% & 15\% & 20\%\\
Chasm & -- & 5\% & 10\%\\
Light undergrowth & 20\% & 10\% & --\\
Scree & -- & 20\% & 30\%\\
Dense rubble & -- & 20\% & 30\%\\
\end{tabular}
\end{table}

\textbf{Gradual and Steep Slopes:} These function as described in Hills Terrain, 
above.

\textbf{Cliff:} These terrain elements also function like their hills terrain counterparts, 
but they're typically 2d6x10 feet tall. Cliffs taller than 80 
feet take up 20 feet of horizontal space.

\textbf{Chasm:} Usually formed by natural geological processes, chasms function 
like pits in a dungeon setting. Chasms aren't hidden, so characters won't fall 
into them by accident (although bull rushes are another story). A typical chasm 
is 2d4x10 feet deep, at least 20 feet long, and anywhere from 
5 feet to 20 feet wide. It takes a DC 15 \linkskill{Climb} check to climb out of a chasm. In 
forbidding mountain terrain, chasms are typically 2d8x10 feet 
deep.

\textbf{Light Undergrowth:} This functions as described in Forest Terrain, above.

\textbf{Scree:} A field of shifting gravel, scree doesn't affect speed, but it 
can be treacherous on a slope. The DC of \linkskill{Balance} and \linkskill{Tumble} checks increases by 
2 if there's scree on a gradual slope and by 5 if there's scree on a steep slope. 
The DC of \linkskill{Move Silently} checks increases by 2 if the scree is on a slope of any 
kind.

\textbf{Dense Rubble:} The ground is covered with rocks of all sizes. It costs 
2 squares of movement to enter a square with dense rubble. The DC of Balance and 
Tumble checks on dense rubble increases by 5, and the DC of Move Silently checks 
increases by +2. 

\textbf{Rock Wall:} A vertical plane of stone, rock walls require DC 25 Climb checks 
to ascend. A typical rock wall is 2d4x10 feet tall in rugged 
mountains and 2d8x10 feet tall in forbidding mountains. Rock 
walls are drawn on the edges of squares, not in the squares themselves.

\textbf{Cave Entrance:} Found in cliff and steep slope squares and next to rock 
walls, cave entrances are typically between 5 and 20 feet wide and 5 feet deep. 
Beyond the entrance, a cave could be anything from a simple chamber to the entrance 
to an elaborate dungeon. Caves used as monster lairs typically have 1d3 rooms that 
are 1d4x10 feet across. 

\textbf{Other Mountain Terrain Features:} Most alpine meadows begin above the tree 
line, so trees and other forest elements are rare in the mountains. Mountain terrain 
can include active streams (5 to 10 feet wide and no more than 5 feet deep) and 
dry streambeds (treat as a trench 5 to 10 feet across). Particularly high-altitude 
areas tend to be colder than the lowland areas that surround them, so they may 
be covered in ice sheets (described below).

\textbf{Stealth and Detection in Mountains:} As a guideline, the maximum distance 
in mountain terrain at which a \linkskill{Spot} check for detecting the nearby presence of 
others can succeed is 4d10x10 feet. Certain peaks and ridgelines 
afford much better vantage points, of course, and twisting valleys and canyons 
have much shorter spotting distances. Because there's little vegetation to obstruct 
line of sight, the specifics on your map are your best guide for the range at which 
an encounter could begin. As in hills terrain, a ridge or peak provides enough 
cover to hide from anyone below the high point.

It's easier to hear faraway sounds in the mountains. The DC of \linkskill{Listen} checks increases 
by 1 per 20 feet between listener and source, not per 10 feet.

%%%
\subsubsection{Avalanches (CR 7)}
%%%

The combination of high peaks and heavy snowfalls means that avalanches are a deadly 
peril in many mountainous areas. While avalanches of snow and ice are common, it's 
also possible to have an avalanche of rock and soil.

An avalanche can be spotted from as far away as 1d10x500 feet 
downslope by a character who makes a DC 20 Spot check, treating the avalanche as 
a Colossal creature. If all characters fail their Spot checks to determine the 
encounter distance, the avalanche moves closer to them, and they automatically 
become aware of it when it closes to half the original distance. It's possible 
to hear an avalanche coming even if you can't see it. Under optimum conditions 
(no other loud noises occurring), a character who makes a DC 15 Listen check can 
hear the avalanche or landslide when it is 1d6x500 feet away. 
This check might have a DC of 20, 25, or higher in conditions where hearing is 
difficult (such as in the middle of a thunderstorm). 

A landslide or avalanche consists of two distinct areas: the bury zone (in the 
direct path of the falling debris) and the slide zone (the area the debris spreads 
out to encompass). Characters in the bury zone always take damage from the avalanche; 
characters in the slide zone may be able to get out of the way. Characters in the 
bury zone take 8d6 points of damage, or half that amount if they make a DC 15 Reflex 
save. They are subsequently buried (see below). Characters in the slide zone take 
3d6 points of damage, or no damage if they make a DC 15 Reflex save. Those who 
fail their saves are buried. 

Buried characters take 1d6 points of nonlethal damage per minute. If a buried character 
falls unconscious, he or she must make a DC 15 Constitution check or take 1d6 points 
of lethal damage each minute thereafter until freed or dead.

The typical avalanche has a width of 1d6x100 feet, from one edge 
of the slide zone to the opposite edge. The bury zone in the center of the avalanche 
is half as wide as the avalanche's full width.

To determine the precise location of characters in the path of an avalanche, roll 
1d6x20; the result is the number of feet from the center of the 
path taken by the bury zone to the center of the party's location. Avalanches of 
snow and ice advance at a speed of 500 feet per round, and rock avalanches travel 
at a speed of 250 feet per round.

%%%
\subsubsection{Mountain Travel}
%%%

High altitude can be extremely fatiguing -- or sometimes deadly -- to creatures that 
aren't used to it. Cold becomes extreme, and the lack of oxygen in the air can 
wear down even the most hardy of warriors.

\textbf{Acclimated Characters:} Creatures accustomed to high altitude generally 
fare better than lowlanders. Any creature with an Environment entry that includes 
mountains is considered native to the area, and acclimated to the high altitude. 
Characters can also acclimate themselves by living at high altitude for a month. 
Characters who spend more than two months away from the mountains must reacclimate 
themselves when they return. Undead, constructs, and other creatures that do not 
breathe are immune to altitude effects.

\textbf{Altitude Zones:} In general, mountains present three possible altitude 
bands: low pass, low peak/high pass, and high peak. 

\textit{Low Pass (lower than 5,000 feet):} Most travel in low mountains takes place 
in low passes, a zone consisting largely of alpine meadows and forests. Travelers 
may find the going difficult (which is reflected in the movement modifiers for 
traveling through mountains), but the altitude itself has no game effect.

\textit{Low Peak or High Pass (5,000 to 15,000 feet):} Ascending to the highest 
slopes of low mountains, or most normal travel through high mountains, falls into 
this category. All nonacclimated creatures labor to breathe in the thin air at 
this altitude. Characters must succeed on a Fortitude save each hour (DC 15, +1 
per previous check) or be fatigued. The fatigue ends when the character descends 
to an altitude with more air. Acclimated characters do not have to attempt the 
Fortitude save. 

\textit{High Peak (more than 15,000 feet):} The highest mountains exceed 20,000 
feet in height. At these elevations, creatures are subject to both high altitude 
fatigue (as described above) and altitude sickness, whether or not they're acclimated 
to high altitudes. Altitude sickness represents long-term oxygen deprivation, 
and it affects mental and physical ability scores. After each 6-hour period a character 
spends at an altitude of over 15,000 feet, he must succeed on a Fortitude save 
(DC 15, +1 per previous check) or take 1 point of damage to all ability scores. 
Creatures acclimated to high altitude receive a +4 competence bonus on their saving 
throws to resist high altitude effects and altitude sickness, but eventually even 
seasoned mountaineers must abandon these dangerous elevations. 

%%%%%%%%%%%%%%%%%%%%%%%%%
\subsection{Desert Terrain}
%%%%%%%%%%%%%%%%%%%%%%%%%

Desert terrain exists in warm, temperate, and cold climates, but all deserts share 
one common trait: little rain. The three categories of desert terrain are tundra 
(cold deserts), rocky desert (often temperate), and sandy desert (often warm).

Tundra differs from the other desert categories in two important ways. Because 
snow and ice cover much of the landscape, it's easy to find water. And during the 
height of summer, the permafrost thaws to a depth of a foot or so, turning the 
landscape into a vast field of mud. The muddy tundra affects movement and skill 
use as the shallow bogs described in marsh terrain, although there's little standing 
water.

The table above describes terrain elements found in each of the three desert categories. 
The terrain elements on this table are mutually exclusive; for instance, a square 
of tundra may contain either light undergrowth or an ice sheet, but not both.

%%%
\subsubsection{Desert Terrain Features}
%%%

\begin{table}[htb]
\rowcolors{1}{white}{offyellow}
\caption{Random Desert Features}
\centering
\begin{tabular}{l c c c}
 & \multicolumn{3}{c}{\textbf{Desert Category}}\\
\textbf{Feature} & \textbf{Tundra} & \textbf{Rocky} & \textbf{Sandy}\\
Light undergrowth & 15\% & 5\% & 5\%\\
Ice sheet & 25\% & -- & --\\
Light rubble & 5\% & 30\% & 10\%\\
Dense rubble & -- & 30\% & 5\%\\
Sand dunes & -- & -- & 50\%\\
\end{tabular}
\end{table}

\textbf{Light Undergrowth:} Consisting of scrubby, hardy bushes and cacti, light 
undergrowth functions as described for other terrain types.

\textbf{Ice Sheet:} The ground is covered with slippery ice. It costs 2 squares 
of movement to enter a square covered by an ice sheet, and the DC of \linkskill{Balance} and 
Tumble checks there increases by 5. A DC 10 Balance check is required to run or 
charge across an ice sheet. 

\textbf{Light Rubble:} Small rocks are strewn across the ground, making nimble 
movement more difficult more difficult. The DC of Balance and \linkskill{Tumble} checks increases 
by 2. 

\textbf{Dense Rubble:} This terrain feature consists of more and larger stones. 
It costs 2 squares of movement to enter a square with dense rubble. The DC of Balance 
and Tumble checks increases by 5, and the DC of \linkskill{Move Silently} checks increases 
by 2.

\textbf{Sand Dunes:} Created by the action of wind on sand, sand dunes function 
as hills that move. If the wind is strong and consistent, a sand dune can move 
several hundred feet in a week's time. Sand dunes can cover hundreds of squares. 
They always have a gentle slope pointing in the direction of the prevailing wind 
and a steep slope on the leeward side.

\textbf{Other Desert Terrain Features:} Tundra is sometimes bordered by forests, 
and the occasional tree isn't out of place in the cold wastes. Rocky deserts have 
towers and mesas consisting of flat ground surrounded on all sides by cliffs and 
steep slopes (described in Mountain Terrain, above). Sandy deserts sometimes have 
quicksand; this functions as described in Marsh Terrain, above, although desert 
quicksand is a waterless mixture of fine sand and dust. All desert terrain is crisscrossed 
with dry streambeds (treat as trenches 5 to 15 feet wide) that fill with water 
on the rare occasions when rain falls.

\textbf{Stealth and Detection in the Desert:} In general, the maximum distance 
in desert terrain at which a \linkskill{Spot} check for detecting the nearby presence of others 
can succeed is 6d6x20 feet; beyond this distance, elevation changes 
and heat distortion in warm deserts makes spotting impossible. The presence of 
dunes in sandy deserts limits spotting distance to 6d6x10 feet. 

The desert imposes neither bonuses nor penalties on \linkskill{Listen} or Spot checks. The 
scarcity of undergrowth or other elements that offer concealment or cover makes 
hiding more difficult.

%%%
\subsubsection{Sandstorms}
%%%

A sandstorm reduces visibility to 1d10x5 feet and provides a 
-4 penalty on Listen, Search, and Spot checks. A sandstorm deals 1d3 points of 
nonlethal damage per hour to any creatures caught in the open, and leaves a thin 
coating of sand in its wake. Driving sand creeps in through all but the most secure 
seals and seams, to chafe skin and contaminate carried gear. 

%%%%%%%%%%%%%%%%%%%%%%%%%
\subsection{Plains Terrain}
%%%%%%%%%%%%%%%%%%%%%%%%%

Plains come in three categories: farms, grasslands, and battlefields. Farms are 
common in settled areas, of course, while grasslands represent untamed plains. 
The battlefields where large armies clash are temporary places, usually reclaimed 
by natural vegetation or the farmer's plow. Battlefields represent a third terrain 
category because adventurers tend to spend a lot of time there, not because they're 
particularly prevalent.

The table below shows the proportions of terrain elements in the different categories 
of plains. On a farm, light undergrowth represents most mature grain crops, so 
farms growing vegetable crops will have less light undergrowth, as will all farms 
during the time between harvest and a few months after planting.

The terrain elements in the table below are mutually exclusive.

%%%
\subsubsection{Plains Terrain Features}
%%%

\begin{table}[htb]
\rowcolors{1}{white}{offyellow}
\caption{Random Plains Features}
\centering
\begin{tabular}{l c c c}
 & \multicolumn{3}{c}{\textbf{Plains Category}}\\
\textbf{Feature} & \textbf{Farm} & \textbf{Grassland} & \textbf{Battlefield}\\
Light undergrowth & 40\% & 20\% & 10\%\\
Heavy undergrowth & -- & 10\% & --\\
Light rubble & -- & -- & 10\%\\
Trench & 5\% & -- & 5\%\\
Berm & -- & -- & 5\%\\
\end{tabular}
\end{table}

\textbf{Undergrowth:} Whether they're crops or natural vegetation, the tall grasses 
of the plains function like light undergrowth in a forest. Particularly thick bushes 
form patches of heavy undergrowth that dot the landscape in grasslands.

\textbf{Light Rubble:} On the battlefield, light rubble usually represents something 
that was destroyed: the ruins of a building or the scattered remnants of a stone 
wall, for example. It functions as described in the desert terrain section above.

\textbf{Trench:} Often dug before a battle to protect soldiers, a trench functions 
as a low wall, except that it provides no cover against adjacent foes. It costs 
2 squares of movement to leave a trench, but it costs nothing extra to enter one. 
Creatures outside a trench who make a melee attack against a creature inside the 
trench gain a +1 bonus on melee attacks because they have higher ground. In farm 
terrain, trenches are generally irrigation ditches.

\textbf{Berm:} A common defensive structure, a berm is a low, earthen wall that 
slows movement and provides a measure of cover. Put a berm on the map by drawing 
two adjacent rows of steep slope (described in Hills Terrain, above), with the 
edges of the berm on the downhill side. Thus, a character crossing a two-square 
berm will travel uphill for 1 square, then downhill for 1 square. Two square berms 
provide cover as low walls for anyone standing behind them. Larger berms provide 
the low wall benefit for anyone standing 1 square downhill from the top of the 
berm. 

\textbf{Fences:} Wooden fences are generally used to contain livestock or impede 
oncoming soldiers. It costs an extra square of movement to cross a wooden fence. 
A stone fence provides a measure of cover as well, functioning as low walls. Mounted 
characters can cross a fence without slowing their movement if they succeed on 
a DC 15 Ride check. If the check fails, the steed crosses the fence, but the rider 
falls out of the saddle.

\textbf{Other Plains Terrain Features:} Occasional trees dot the landscape in many 
plains, although on battlefields they're often felled to provide raw material for 
siege engines (described in Urban Features). Hedgerows (described in Marsh Terrain) 
are found in plains as well. Streams, generally 5 to 20 feet wide and 5 to 10 feet 
deep, are commonplace.

\textbf{Stealth and Detection in Plains:} In plains terrain, the maximum distance 
at which a \linkskill{Spot} check for detecting the nearby presence of others can succeed is 
6d6x40 feet, although the specifics of your map may restrict 
line of sight. Plains terrain provides no bonuses or penalties on \linkskill{Listen} and Spot 
checks. Cover and concealment are not uncommon, so a good place of refuge is often 
nearby, if not right at hand.

%%%%%%%%%%%%%%%%%%%%%%%%%
\subsection{Aquatic Terrain}
%%%%%%%%%%%%%%%%%%%%%%%%%

Aquatic terrain is the least hospitable to most PCs, because they can't breathe 
there. Aquatic terrain doesn't offer the variety that land terrain does. The ocean 
floor holds many marvels, including undersea analogues of any of the terrain elements 
described earlier in this section. But if characters find themselves in the water 
because they were bull rushed off the deck of a pirate ship, the tall kelp beds 
hundreds of feet below them don't matter. Accordingly, these rules simply divide 
aquatic terrain into two categories: flowing water (such as streams and rivers) 
and nonflowing water (such as lakes and oceans).

\textbf{Flowing Water:} Large, placid rivers move at only a few miles per hour, 
so they function as still water for most purposes. But some rivers and streams 
are swifter; anything floating in them moves downstream at a speed of 10 to 40 
feet per round. The fastest rapids send swimmers bobbing downstream at 60 to 90 
feet per round. Fast rivers are always at least rough water (\linkskill{Swim} DC 15), and whitewater 
rapids are stormy water (Swim DC 20). If a character is in moving water, move her 
downstream the indicated distance at the end of her turn. A character trying to 
maintain her position relative to the riverbank can spend some or all of her turn 
swimming upstream.

\textit{Swept Away:} Characters swept away by a river moving 60 feet per round 
or faster must make DC 20 Swim checks every round to avoid going under. If a character 
gets a check result of 5 or more over the minimum necessary, he arrests his motion 
by catching a rock, tree limb, or bottom snag -- he is no longer being carried along 
by the flow of the water. Escaping the rapids by reaching the bank requires three 
DC 20 Swim checks in a row. Characters arrested by a rock, limb, or snag can't 
escape under their own power unless they strike out into the water and attempt 
to swim their way clear. Other characters can rescue them as if they were trapped 
in quicksand (described in Marsh Terrain, above). 

\textbf{Non-Flowing Water:} Lakes and oceans simply require a swim speed or successful 
Swim checks to move through (DC 10 in calm water, DC 15 in rough water, DC 20 in 
stormy water). Characters need a way to breathe if they're underwater; failing 
that, they risk drowning. When underwater, characters can move in any direction 
as if they were flying with perfect maneuverability.

\textbf{Stealth and Detection Underwater:} How far you can see underwater depends 
on the water's clarity. As a guideline, creatures can see 4d8x10 
feet if the water is clear, and 1d8x10 feet if it's murky. Moving 
water is always murky, unless it's in a particularly large, slow-moving river.

It's hard to find cover or concealment to hide underwater (except along the seafloor). 
Listen and Move Silently checks function normally underwater.

\textit{Invisibility:} An invisible creature displaces water and leaves a visible, 
body-shaped "bubble" where the water was displaced. The creature still has concealment 
(20\% miss chance), but not total concealment (50\% miss chance).

%%%
\subsubsection{Underwater Combat}
%%%

Land-based creatures can have considerable difficulty when fighting in water. Water 
affects a creature's Armor Class, attack rolls, damage, and movement. In some cases 
a creature's opponents may get a bonus on attacks. The effects are summarized in 
the accompanying table. They apply whenever a character is swimming, walking in 
chest-deep water, or walking along the bottom. 

\textbf{Ranged Attacks Underwater:} Thrown weapons are ineffective underwater, 
even when launched from land. Attacks with other ranged weapons take a -2 penalty 
on attack rolls for every 5 feet of water they pass through, in addition to the 
normal penalties for range. 

\textbf{Attacks from Land:} Characters swimming, floating, or treading water on 
the surface, or wading in water at least chest deep, have improved cover (+8 bonus 
to AC, +4 bonus on Reflex saves) from opponents on land. Landbound opponents who 
have \linkspell{Freedom of Movement} effects ignore this cover when making melee attacks 
against targets in the water. A completely submerged creature has total cover against 
opponents on land unless those opponents have \textit{Freedom of Movement} effects. 
Magical effects are unaffected except for those that require attack rolls (which 
are treated like any other effects) and fire effects.

\textbf{Fire:} Nonmagical fire (including alchemist's fire) does not burn underwater. 
Spells or spell-like effects with the fire descriptor are ineffective underwater 
unless the caster makes a \linkskill{Spellcraft} check (DC 20 + spell level). If the check 
succeeds, the spell creates a bubble of steam instead of its usual fiery effect, 
but otherwise the spell works as described. A supernatural fire effect is ineffective 
underwater unless its description states otherwise. The surface of a body of water 
blocks line of effect for any fire spell. If the caster has made a Spellcraft check 
to make the fire spell usable underwater, the surface still blocks the spell's 
line of effect.

\begin{table}[htb]
\rowcolors{1}{white}{offyellow}
\caption{Combat Adjustments Underwater}
\centering
\begin{tabular}{l c c c c}
& \multicolumn{2}{c}{\textbf{Attack / Damage}} & &\\
\textbf{Condition} & \textbf{Slashing or Bludgeoning} & \textbf{Tail} & \textbf{Movement} & \textbf{Off Balance?\textsuperscript{4}}\\
\textit{Freedom of Movement} & normal/normal & normal/normal & normal & No\\
Has a Swim Speed & -2 / half & normal & normal & No\\
Successful Swim Check & -2 / half\textsuperscript{1} & -2 / half & quarter or half\textsuperscript{2} & No\\
Firm footing\textsuperscript{3} & -2 / half & -2 / half & half & No\\
None of the above & -2 / half & -2 / half & normal & Yes\\
\multicolumn{5}{p{16cm}}{\textsuperscript{1} A creature without a \textit{Freedom of Movement} effect or a swim speed makes grapple checks underwater at a -2 penalty, but deals damage normally when grappling.}\\
\multicolumn{5}{p{16cm}}{\textsuperscript{2} A successful Swim check lets a creature move one-quarter its speed as a move action, or one-half its speed as a full-round action.}\\
\multicolumn{5}{p{16cm}}{\textsuperscript{3} Creatures have firm footing when walking along the bottom, braced againat a ship's hull, or the like. A Creature can only walk along the bottom of it wears or carried enough gear to weigh itself down -- at least 16 pounds for Medium creatures, twice that for each Size larger, half that for each Size smaller.}\\
\multicolumn{5}{p{16cm}}{\textsuperscript{4} Creatures flailing about in the water (usually because they failed their Swim checks) have a hard time fighting effectively. An off-balance creature loses its Dexterity bonus to AC, and opponents gain a +2 bonus on atacks against it.}\\
\end{tabular}
\end{table}

%%%
\subsubsection{Floods}
%%%

In many wilderness areas, river floods are a common occurrence.

In spring, an enormous snowmelt can engorge the streams and rivers it feeds. Other 
catastrophic events such as massive rainstorms or the destruction of a dam can 
create floods as well.

During a flood, rivers become wider, deeper, and swifter. Assume that a river rises 
by 1d10+10 feet during the spring flood, and its width increases by a factor of 
1d4x50\%. Fords may disappear for days, bridges may be swept 
away, and even ferries might not be able to manage the crossing of a flooded river. 
A river in flood makes Swim checks one category harder (calm water becomes rough, 
and rough water becomes stormy). Rivers also become 50\% swifter.
