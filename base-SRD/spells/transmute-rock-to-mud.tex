\spellentry{Transmute Rock to Mud}

Transmutation [Earth]

\textbf{Level:} Drd 5, Sor/Wiz 5

\textbf{Components:} V, S, M/DF

\textbf{Casting Time:} 1 standard action

\textbf{Range:} Medium (100 ft. + 10 ft./level)

\textbf{Area:} Up to two 10-ft. cubes/level (S)

\textbf{Duration:} Permanent; see text

\textbf{Saving Throw:} See text

\textbf{Spell Resistance:} No

This spell turns natural, uncut or unworked rock of any sort into an equal volume 
of mud. Magical stone is not affected by the spell. The depth of the mud created 
cannot exceed 10 feet. A creature unable to levitate, fly, or otherwise free itself 
from the mud sinks until hip- or chest-deep, reducing its speed to 5 feet and causing 
a -2 penalty on attack rolls and AC. Brush thrown atop the mud can support creatures 
able to climb on top of it. Creatures large enough to walk on the bottom can wade 
through the area at a speed of 5 feet.

If \textit{transmute rock to mud} is cast upon the ceiling of a cavern or tunnel, 
the mud falls to the floor and spreads out in a pool at a depth of 5 feet. The 
falling mud and the ensuing cave-in deal 8d6 points of bludgeoning damage to anyone 
caught directly beneath the area, or half damage to those who succeed on Reflex 
saves.

Castles and large stone buildings are generally immune to the effect of the spell, 
since \textit{transmute rock to mud} can't affect worked stone and doesn't reach 
deep enough to undermine such buildings' foundations. However, small buildings 
or structures often rest upon foundations shallow enough to be damaged or even 
partially toppled by this spell.

The mud remains until a successful \textit{dispel magic} or \textit{transmute mud 
to rock} spell restores its substance -- but not necessarily its form. Evaporation 
turns the mud to normal dirt over a period of days. The exact time depends on exposure 
to the sun, wind, and normal drainage.

\textit{Arcane Material Component:} Clay and water.

