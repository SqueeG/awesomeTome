\spellentry{Control Weather}

Transmutation

\textbf{Level:} Air 7, Clr 7, Drd 7, Sor/Wiz 7

\textbf{Components:} V, S

\textbf{Casting Time:} 10 minutes; see text

\textbf{Range:} 2 miles

\textbf{Area:} 2-mile-radius circle, centered on you; see text

\textbf{Duration:} 4d12 hours; see text

\textbf{Saving Throw:} None

\textbf{Spell Resistance:} No

You change the weather in the local area. It takes 10 minutes to cast the spell 
and an additional 10 minutes for the effects to manifest. You can call 
forth weather appropriate to the climate and season of the area you are in.

\begin{description*}
\item[Spring] Tornado, thunderstorm, sleet storm, or hot weather
\item[Summer] Torrential rain, heat wave, or hailstorm
\item[Fall] Hot or cold weather, fog, or sleet
\item[Winter] Frigid cold, blizzard, or thaw
\item[Late Winter] Hurricane-force winds or early spring (coastal area)
\end{description*}

You control the general tendencies of the weather, such as the direction and intensity 
of the wind. You cannot control specific applications of the weather -- where lightning 
strikes, for example, or the exact path of a tornado. When you select a certain 
weather condition to occur, the weather assumes that condition 10 minutes later 
(changing gradually, not abruptly). The weather continues as you left it for the 
duration, or until you use a standard action to designate a new kind of weather 
(which fully manifests itself 10 minutes later). Contradictory conditions are not 
possible simultaneously.

\textit{Control weather} can do away with atmospheric phenomena (naturally occurring 
or otherwise) as well as create them.

A druid casting this spell doubles the duration and affects a circle with a 3-mile 
radius.

