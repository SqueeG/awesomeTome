\spellentry{Gate}

Conjuration (Creation or Calling)

\textbf{Level:} Clr 9, Sor/Wiz 9

\textbf{Components:} V, S, XP; see text

\textbf{Casting Time:} 1 standard action

\textbf{Range:} Medium (100 ft. + 10 ft./level)

\textbf{Effect:} See text

\textbf{Duration:} Instantaneous or concentration (up to 1 round/level); see text

\textbf{Saving Throw:} None

\textbf{Spell Resistance:} No

Casting a \textit{gate} spell has two effects. First, it creates an interdimensional 
connection between your plane of existence and a plane you specify, allowing travel 
between those two planes in either direction.

Second, you may then call a particular individual or kind of being through the 
\textit{gate}.

The \textit{gate} itself is a circular hoop or disk from 5 to 20 feet in diameter 
(caster's choice), oriented in the direction you desire when it comes into existence 
(typically vertical and facing you). It is a two-dimensional window looking into 
the plane you specified when casting the spell, and anyone or anything that moves 
through is shunted instantly to the other side.

A \textit{gate} has a front and a back. Creatures moving through the \textit{gate 
}from the front are transported to the other plane; creatures moving through it 
from the back are not.

\textit{Planar Travel:} As a mode of planar travel, a \textit{gate} spell functions 
much like a \textit{plane shift} spell, except that the \textit{gate} opens precisely 
at the point you desire (a creation effect). Deities and other beings who rule 
a planar realm can prevent a \textit{gate} from opening in their presence or personal 
demesnes if they so desire. Travelers need not join hands with you -- anyone who 
chooses to step through the portal is transported. A \textit{gate} cannot be opened 
to another point on the same plane; the spell works only for interplanar travel.

You may hold the \textit{gate} open only for a brief time (no more than 1 round 
per caster level), and you must concentrate on doing so, or else the interplanar 
connection is severed.

\textit{Calling Creatures:} The second effect of the \textit{gate} spell is to 
call an extraplanar creature to your aid (a calling effect). By naming a particular 
being or kind of being as you cast the spell, you cause the \textit{gate} to open 
in the immediate vicinity of the desired creature and pull the subject through, 
willing or unwilling. Deities and unique beings are under no compulsion to come 
through the \textit{gate}, although they may choose to do so of their own accord. 
This use of the spell creates a \textit{gate} that remains open just long enough 
to transport the called creatures. This use of the spell has an XP cost (see below).

If you choose to call a kind of creature instead of a known individual you may 
call either a single creature (of any HD) or several creatures. You can call and 
control several creatures as long as their HD total does not exceed your caster 
level. In the case of a single creature, you can control it if its HD do not exceed 
twice your caster level. A single creature with more HD than twice your caster 
level can't be controlled. Deities and unique beings cannot be controlled in any 
event. An uncontrolled being acts as it pleases, making the calling of such creatures 
rather dangerous. An uncontrolled being may return to its home plane at any time.

A controlled creature can be commanded to perform a service for you. Such services 
fall into two categories: immediate tasks and contractual service. Fighting for 
you in a single battle or taking any other actions that can be accomplished within 
1 round per caster level counts as an immediate task; you need not make any agreement 
or pay any reward for the creature's help. The creature departs at the end of the 
spell.

If you choose to exact a longer or more involved form of service from a called 
creature, you must offer some fair trade in return for that service. The service 
exacted must be reasonable with respect to the promised favor or reward; see the 
\textit{lesser planar ally} spell for appropriate rewards. (Some creatures may 
want their payment in "livestock" rather than in coin, which could involve complications.) 
Immediately upon completion of the service, the being is transported to your vicinity, 
and you must then and there turn over the promised reward. After this is done, 
the creature is instantly freed to return to its own plane.

Failure to fulfill the promise to the letter results in your being subjected to 
service by the creature or by its liege and master, at the very least. At worst, 
the creature or its kin may attack you.

\textit{Note:} When you use a calling spell such as \textit{gate} to call an air, 
chaotic, earth, evil, fire, good, lawful, or water creature, it becomes a spell 
of that type.

\textit{XP Cost:} 1,000 XP (only for the \textit{calling creatures} function).

