%%%%%%%%%%%%%%%%%%%%%%%%%%%%%%%%%%%%%%%%%%%%%%%%%%
%%%%%%%%%%%%%%%%%%%%%%%%%%%%%%%%%%%%%%%%%%%%%%%%%%
\chapter{The Basics}\label{chapter:Basics}
%%%%%%%%%%%%%%%%%%%%%%%%%%%%%%%%%%%%%%%%%%%%%%%%%%
%%%%%%%%%%%%%%%%%%%%%%%%%%%%%%%%%%%%%%%%%%%%%%%%%%

Wherever possible, I have tried to stick to simply presenting the SRD as is. However, for a document produced by a professional corporation, the SRD has a lot of typos and word omissions that needed to be fixed. I've fixed those sorts of minor problems whenever I've seen them, and I've also organized the various sections a little better, which includes moving some auxiliary feats, domains, and spells into the main section.

The Psionic, Epic, and Divine parts of the SRD have been left out entirely; I don't plan to include them ever. For the Monster section, the current plan is to only include the monsters that have a listed purchase price or appear on a \textit{Summon} list. This makes it a little more player oriented I guess, but the MC should be able to find suitable monster entries all over the web with a \href{https://www.google.com/search?q=SRD monsters}{Simple Search}.

I've also added some entirely new \linksec{Character Creation} and \linksec{Character Advancement} rules so that players can play the game using just this PDF if they want. These aren't in the SRD, but they're pretty self-evident things if you've played the game before. You only really need to refer to them if you're new to the game.

%%%%%%%%%%%%%%%%%%%%%%%%%%%%%%%%%%%%%%%%%%%%%%%%%%
\section{The Core Mechanic}
%%%%%%%%%%%%%%%%%%%%%%%%%%%%%%%%%%%%%%%%%%%%%%%%%%
Whenever you attempt an action that has some chance 
of failure, you roll a twenty-sided die (d20). To determine if your character succeeds 
at a task you do this:
\begin{itemize*}
\item Roll a d20.
\item Add any relevant modifiers.
\item Compare the result to a target number.
\end{itemize*}
If the result equals or exceeds the target number, your character succeeds. If 
the result is lower than the target number, you fail.

%%%%%%%%%%%%%%%%%%%%%%%%%%%%%%%%%%%%%%%%%%%%%%%%%%
\section{Dice}
%%%%%%%%%%%%%%%%%%%%%%%%%%%%%%%%%%%%%%%%%%%%%%%%%%

Dice rolls are described with expressions such as "3d4+3", which means "roll 
three four-sided dice and add 3" (resulting in a number between 6 and 15). The 
first number tells you how many dice to roll (adding the results together). The 
number immediately after the "d" tells you the type of die to use. Any number 
after that indicates a quantity that is added or subtracted from the result.

\textbf{d\%:} Percentile dice work a little differently. You generate a number 
between 1 and 100 by rolling two different ten-sided dice. One (designated before 
you roll) is the tens digit. The other is the ones digit. Two 0s represent 100.

%%%%%%%%%%%%%%%%%%%%%%%%%%%%%%%%%%%%%%%%%%%%%%%%%%
\section{Rounding Fractions}
%%%%%%%%%%%%%%%%%%%%%%%%%%%%%%%%%%%%%%%%%%%%%%%%%%

In general, if you wind up with a fraction, round down, even if the fraction is 
one-half or larger.

\textit{Exception:} Certain rolls, such as damage and hit points, have a minimum 
of 1.

%%%%%%%%%%%%%%%%%%%%%%%%%%%%%%%%%%%%%%%%%%%%%%%%%%
\section{Multiplying}
%%%%%%%%%%%%%%%%%%%%%%%%%%%%%%%%%%%%%%%%%%%%%%%%%%

Sometimes a rule makes you multiply a number or a die roll. As long as you're applying 
a single multiplier, multiply the number normally. When two or more multipliers 
apply to any abstract value (such as a modifier or a die roll), however, combine 
them into a single multiple, with each extra multiple adding 1 less than its value 
to the first multiple. Thus, a double (x2) and a double (x2) 
applied to the same number results in a triple (x3, because 2 
+ 1 = 3).


When applying multipliers to real-world values (such as weight or distance), normal 
rules of math apply instead. A creature whose size doubles (thus multiplying its 
weight by 8) and then is turned to stone (which would multiply its weight by a 
factor of roughly 3) now weighs about 24 times normal, not 10 times normal. Similarly, 
a blinded creature attempting to negotiate difficult terrain would count each square 
as 4 squares (doubling the cost twice, for a total multiplier of x4), 
rather than as 3 squares (adding 100\% twice). 

%%%%%%%%%%%%%%%%%%%%%%%%%%%%%%%%%%%%%%%%%%%%%%%%%%
\section{Ability Scores}
%%%%%%%%%%%%%%%%%%%%%%%%%%%%%%%%%%%%%%%%%%%%%%%%%%

The table below shows the ability modifier associated with an ability score of a given value, as well as the bonus spells that a caster with that ability score as their spellcasting stat gains. Regardless of ability score, a spellcaster never gains bonus 0th level spell slots.

\begin{table}[htb]
\rowcolors{1}{white}{offyellow}
\caption{Ability Modifiers and Bonus Spells}
\centering
\begin{tabular}{*{11}{c}}
\textbf{Stat} & \textbf{Mod} & \textbf{1st} & \textbf{2nd} & \textbf{3rd} & \textbf{4th} & \textbf{5th} & \textbf{6th} & \textbf{7th} & \textbf{8th} & \textbf{9th}\\
1 & -5 & -- & -- & -- & -- & -- & -- & -- & -- & -- \\
2-3 & -4 & -- & -- & -- & -- & -- & -- & -- & -- & -- \\
4-5 & -3 & -- & -- & -- & -- & -- & -- & -- & -- & -- \\
6-7 & -2 & -- & -- & -- & -- & -- & -- & -- & -- & -- \\
8-9 & -1 & -- & -- & -- & -- & -- & -- & -- & -- & -- \\
10-11 & +0 & 0 & 0 & 0 & 0 & 0 & 0 & 0 & 0 & 0\\
12-13 & +1 & 1 & 0 & 0 & 0 & 0 & 0 & 0 & 0 & 0\\
14-15 & +2 & 1 & 1 & 0 & 0 & 0 & 0 & 0 & 0 & 0\\
16-17 & +3 & 1 & 1 & 1 & 0 & 0 & 0 & 0 & 0 & 0\\
18-19 & +4 & 1 & 1 & 1 & 1 & 0 & 0 & 0 & 0 & 0\\
20-21 & +5 & 2 & 1 & 1 & 1 & 1 & 0 & 0 & 0 & 0\\
22-23 & +6 & 2 & 2 & 1 & 1 & 1 & 1 & 0 & 0 & 0\\
24-25 & +7 & 2 & 2 & 2 & 1 & 1 & 1 & 1 & 0 & 0\\
26-27 & +8 & 2 & 2 & 2 & 2 & 1 & 1 & 1 & 1 & 0\\
28-29 & +9 & 3 & 2 & 2 & 2 & 2 & 1 & 1 & 1 & 1\\
30-31 & +10 & 3 & 3 & 2 & 2 & 2 & 2 & 1 & 1 & 1\\
32-33 & +11 & 3 & 3 & 3 & 2 & 2 & 2 & 2 & 1 & 1\\
34-35 & +12 & 3 & 3 & 3 & 3 & 2 & 2 & 2 & 2 & 1\\
36-37 & +13 & 4 & 3 & 3 & 3 & 3 & 2 & 2 & 2 & 2\\
38-39 & +14 & 4 & 4 & 3 & 3 & 3 & 3 & 2 & 2 & 2\\
40-41 & +15 & 4 & 4 & 4 & 3 & 3 & 3 & 3 & 2 & 2\\
42-43 & +16 & 4 & 4 & 4 & 4 & 3 & 3 & 3 & 3 & 2\\
44-45 & +17 & 5 & 4 & 4 & 4 & 4 & 3 & 3 & 3 & 3\\
46-47 & +18 & 5 & 5 & 4 & 4 & 4 & 4 & 3 & 3 & 3\\
48-49 & +19 & 5 & 5 & 5 & 4 & 4 & 4 & 4 & 3 & 3\\
50-51 & +20 & 5 & 5 & 5 & 5 & 4 & 4 & 4 & 4 & 3\\
\end{tabular}
\end{table}

%%%%%%%%%%%%%%%%%%%%%%%%%
\subsection{Ability Modifiers}\index{Ability Modifer}
%%%%%%%%%%%%%%%%%%%%%%%%%

Each ability, after changes made because of race, has a modifier ranging from -5 
to +5. Table 1.1: Ability Modifiers and Bonus Spells shows the modifier for each score. 
It also shows bonus spells, which you'll need to know about if your character is 
a spellcaster.

The modifier is the number you apply to the die roll when your character tries 
to do something related to that ability. You also use the modifier with some numbers that aren't die rolls. 
A positive modifier is called a bonus, and a negative modifier is called a penalty.

%%%%%%%%%%%%%%%%%%%%%%%%%
\subsection{Abilities and Spellcasters}\index{Bonus Spells}
%%%%%%%%%%%%%%%%%%%%%%%%%

The ability that governs bonus spells depends on what type of spellcaster your 
character is: Intelligence for wizards; Wisdom for clerics, druids, paladins, and 
rangers; or Charisma for sorcerers and bards. In addition to having a high ability 
score, a spellcaster must be of high enough class level to be able to cast spells 
of a given spell level. (See the class descriptions for details.)

%%%%%%%%%%%%%%%%%%%%%%%%%
\subsection{The Abilities}
%%%%%%%%%%%%%%%%%%%%%%%%%

Each ability partially describes your character and affects some of his or her 
actions.

When an ability score changes, all attributes associated with that score change 
accordingly. A character does not retroactively get additional skill points for 
previous levels if she increases her intelligence.

%%%
\subsubsection{Strength (Str)}\index{Strength}
%%%

Strength measures your character's muscle and physical power. This ability is especially 
important for fighters, barbarians, paladins, rangers, and monks because it helps 
them prevail in combat. Strength also limits the amount of equipment your character 
can carry.

You apply your character's Strength modifier to:
\begin{itemize}
\item Melee attack rolls.
\item Damage rolls when using a melee weapon or a thrown weapon (including a sling). 
(\textit{Exceptions:} Off-hand attacks receive only one-half the character's Strength 
bonus, while two-handed attacks receive one and a half times the Strength bonus. 
A Strength penalty, but not a bonus, applies to attacks made with a bow that is 
not a composite bow.)
\item \linkskill{Climb}, \linkskill{Jump}, and \linkskill{Swim} checks. These are the skills that have Strength as their 
key ability.
\item Strength checks (for breaking down doors and the like).
\end{itemize}

%%%
\subsubsection{Dexterity (Dex)}\index{Dexterity}
%%%

Dexterity measures hand-eye coordination, agility, reflexes, and balance. This 
ability is the most important one for rogues, but it's also high on the list for 
characters who typically wear light or medium armor (rangers and barbarians) or 
no armor at all (monks, wizards, and sorcerers), and for anyone who wants to be 
a skilled archer.

You apply your character's Dexterity modifier to:
\begin{itemize}
\item Ranged attack rolls, including those for attacks made with bows, crossbows, throwing 
axes, and other ranged weapons.
\item Armor Class (AC), provided that the character can react to the attack.
\item Reflex saving throws, for avoiding fireballs and other attacks that you can escape 
by moving quickly.
\item \linkskill{Balance}, \linkskill{Escape Artist}, \linkskill{Hide}, \linkskill{Move Silently}, \linkskill{Open Lock}, \linkskill{Ride}, \linkskill{Sleight of Hand}, 
\linkskill{Tumble}, and \linkskill{Use Rope} checks. These are the skills that have Dexterity as their 
key ability.
\end{itemize}

%%%
\subsubsection{Constitution (Con)}\index{Constitution}
%%%

Constitution represents your character's health and stamina. A Constitution bonus 
increases a character's hit points, so the ability is important for all classes.

You apply your character's Constitution modifier to:
\begin{itemize}
\item Each roll of a Hit Die (though a penalty can never drop a result below 1, that 
is, a character always gains at least 1 hit point each time he or she advances 
in level).
\item Fortitude saving throws, for resisting poison and similar threats.
\item \linkskill{Concentration} checks. Concentration is a skill, important to spellcasters, that 
has Constitution as its key ability.
\end{itemize}
If a character's Constitution score changes enough to alter his or her Constitution 
modifier, the character's hit points also increase or decrease accordingly.

%%%
\subsubsection{Intelligence (Int)}\index{Intelligence}
%%%

Intelligence determines how well your character learns and reasons. This ability 
is important for wizards because it affects how many spells they can cast, how 
hard their spells are to resist, and how powerful their spells can be. It's also 
important for any character who wants to have a wide assortment of skills.

You apply your character's Intelligence modifier to:
\begin{itemize}
\item The number of languages your character knows at the start of the game.
\item The number of skill points gained each level. (But your character always gets at 
least 1 skill point per level.)
\item \linkskill{Appraise}, \linkskill{Craft}, \linkskill{Decipher Script}, \linkskill{Disable Device}, \linkskill{Forgery}, \linkskill{Knowledge}, \linkskill{Search}, and 
\linkskill{Spellcraft} checks. These are the skills that have Intelligence as their key ability.
\end{itemize}
A wizard gains bonus spells based on her Intelligence score. The minimum Intelligence 
score needed to cast a wizard spell is 10 + the spell's level. 

An animal has an Intelligence score of 1 or 2. A creature of humanlike intelligence 
has a score of at least 3.

%%%
\subsubsection{Wisdom (Wis)}\index{Wisdom}
%%%

Wisdom describes a character's willpower, common sense, perception, and intuition. 
While Intelligence represents one's ability to analyze information, Wisdom represents 
being in tune with and aware of one's surroundings. Wisdom is the most important 
ability for clerics and druids, and it is also important for paladins and rangers. 
If you want your character to have acute senses, put a high score in Wisdom. Every 
creature has a Wisdom score.

You apply your character's Wisdom modifier to:
\begin{itemize}
\item Will saving throws (for negating the effect of charm person and other spells).
\item \linkskill{Heal}, \linkskill{Listen}, \linkskill{Profession}, \linkskill{Sense Motive}, \linkskill{Spot}, and \linkskill{Survival} checks. These are the 
skills that have Wisdom as their key ability.
\end{itemize}
Clerics, druids, paladins, and rangers get bonus spells based on their Wisdom scores. 
The minimum Wisdom score needed to cast a cleric, druid, paladin, or ranger spell 
is 10 + the spell's level.

%%%
\subsubsection{Charisma (Cha)}\index{Charisma}
%%%

Charisma measures a character's force of personality, persuasiveness, personal 
magnetism, ability to lead, and physical attractiveness. This ability represents 
actual strength of personality, not merely how one is perceived by others in a 
social setting. Charisma is most important for paladins, sorcerers, and bards. 
It is also important for clerics, since it affects their ability to turn undead. 
Every creature has a Charisma score.

You apply your character's Charisma modifier to:
\begin{itemize}
\item \linkskill{Bluff}, \linkskill{Diplomacy}, \linkskill{Disguise}, \linkskill{Gather Information}, \linkskill{Handle Animal}, \linkskill{Intimidate}, \linkskill{Perform}, 
and \linkskill{Use Magic Device} checks. These are the skills that have Charisma as their key 
ability.
\item Checks that represent attempts to influence others. 
\item Turning checks for clerics and paladins attempting to turn zombies, vampires, and 
other undead.
\end{itemize}
Sorcerers and bards get bonus spells based on their Charisma scores. The minimum 
Charisma score needed to cast a sorcerer or bard spell is 10 + the spell's level.

%%%%%%%%%%%%%%%%%%%%%%%%%%%%%%%%%%%%%%%%%%%%%%%%%%
\section{Ability Score Generation}
%%%%%%%%%%%%%%%%%%%%%%%%%%%%%%%%%%%%%%%%%%%%%%%%%%

Your ability scores are generated randomly at the start of the game so that they range from 3 to 18 before applying racial modifiers (an average human has a 10 or 11). Many ways exist to do this. The problem with rolling is that one player might get stats that are simply better than another player's, which isn't fair. The problem with some sort of point-buy system is that players will always max out their primary stat first at the expense of other stats, and the resulting stats look very inorganic. Spellcasters particularly benefit from point-buy systems, and they're the ones who generally shoot off the charts in terms of power level in the first place, and so we don't want that.

As a result, the suggested method for stat generation is as follows:
\begin{itemize}
\item Each player rolls 4d6 dice for each stat, dropping the lowest die from the total. (Player Characters are above average, so we give them slightly better odds than just 3d6.) Do this six times to get a number for each stat.
\item If the resulting stat set doesn't have a single stat of at least 13, reroll it.
\item Also, If the resulting stat set doesn't have a total ability modifier of at least +1, reroll it.
\item Each player does this so that they have a stat set, then all players can pick any of the stat sets that were rolled.
\item Assign each stat one of the numbers from the set and proceed with the rest of Character Creation.
\end{itemize}
This way, you get "organic" style stat sets that aren't always just 18 and then a few 14s, but you also don't give any single player an unfair advantage.

%%%%%%%%%%%%%%%%%%%%%%%%%%%%%%%%%%%%%%%%%%%%%%%%%%
\section{Character Creation}
%%%%%%%%%%%%%%%%%%%%%%%%%%%%%%%%%%%%%%%%%%%%%%%%%%

To create a complete character there are several steps that you need to follow:

\begin{itemize}
\item Select Race (\hyperref[chapter:Races]{Chapter 2})
\item Select Class (\hyperref[chapter:Classes]{Chapter 3})
\item Determine Ability Scores (\linksec{Ability Score Generation}{above})
\item Select Class Features (not all classes have feature options at first level)
\item Assign Skill Points (\hyperref[chapter:Skills]{Chapter 4})
\item Select Feat (\hyperref[chapter:Feats]{Chapter 5})
\item Purchase Starting Equipment (\hyperref[chapter:Equipment]{Chapter 6})
\item Note Character Details (\hyperref[chapter:Description]{Chapter 7})
\end{itemize}

%%%%%%%%%%%%%%%%%%%%%%%%%%%%%%%%%%%%%%%%%%%%%%%%%%
\section{Character Advancement}
%%%%%%%%%%%%%%%%%%%%%%%%%%%%%%%%%%%%%%%%%%%%%%%%%%

As you adventure and such your \gameterm{MC} (short for "\gameterm{Mister Cavern}") will describe the world around the characters of the other players, and also play the role of all the Non-Player Characters (NPCs\index{NPC}) that you meet. At the end of each play session your characters will earn \gameterm{Experience Points} (\gameterm{XP}) based on what you've done. These are used to gain additional levels, which lets you take on stronger threats and generally have larger scale adventures.

You need 1,000 experience points times your current \gameterm{Character Level} to advance to the next level. Each time you gain a level, spend the appropriate number of points and then follow these steps:

\begin{itemize}
\item Choose the class you want to level up in. One of your existing classes goes up by one level, or you can add a new class at level 1 by following the rules for Multiclassing.
\item Increase Base Attack Bonus and Base Save Bonus based on the new class level you gained.
\item Roll for additional hit points based on the class level you gained. Remember that you add your Constitution Modifier to the roll.
\item If your total level is now a multiple of 4 (4th, 8th, 12th, etc), increase a stat of your choosing by 1 point.
\item Assign additional skill points from your new class level. If you gained a stat point during the previous step and used it to increase Intelligence, then you gain additional skill points (if any) during this level.
\item If your total level is now a multiple of 3 (3rd, 6th, 9th, etc) then you gain a new Feat.
\item Add any new class features that you gained to your sheet, and update your old class features that may have improved (such as gaining additional spells per day, or increased sneak attack damage)
\end{itemize}

Characters earn experience points for getting things done. Things that are level-appropriate, and usually things of a questly nature. This does include fighting foes, but it also includes stealing items, swaying minds, and generally having an important effect on the world around you. If the players take on threats and challenges that are above their level they get more experience, and if they take on things beneath them then they get less experience. 

Generally, a threat of CR X will give 75*X experience points to each player that participated. For each level a challenge is above or below the level of a character, increase or decrease the amount of experience that that character earns by 10\%. Whenever possible, the group as a whole should be kept at the same level. It sucks being even one level behind the people around you for an extended period. Remember that a Cohort from the Leadership feat is only 2 levels behind their leader, each level is important in this game. If a player is lower level than the rest of the group, they won't be able to do as much and they can easily become frustrated at the fact that everyone around them is doing bigger and better things.

Groups can usually face threats below their level with ease. Increasing the number of foes will usually keep things even (double the number for each 2 levels below the group level). A group generally can't take on a threat more than 2 levels above their own if they want anything more than a pyrrhic victory. A threat more than 4 levels above their own might quickly turn into a Total Party Kill.